\documentclass[aspectratio=169]{beamer}
\usepackage{lmodern}
\usepackage{multirow}
\usepackage{romannum}

\newcommand{\diff}{\operatorname{d}}
\title{PhD Application Presentation for IMT}
\author{Jianyu MA}
\date{\today}
\setbeamercovered{transparent}

\begin{document}

\begin{frame}
	\titlepage
\end{frame}

\section{Academical curriculum vitae}

\begin{frame}
	\frametitle{About Jianyu MA}
	\begin{table}
		\begin{tabular}{c | c }
			Institution                   & Internship Topic                       \\
			\hline
			Wuhan University BS           & \multirow{2}{8em}{Stochastic calculus} \\
			UPS Toulouse \Romannum{3}  M1                                          \\
			\pause
			UPS Toulouse \Romannum{3}  M2 & Optimal transport                      \\
		\end{tabular}
	\end{table}
	\setbeamercovered{invisible}
	\pause
	\begin{description}
		\item[Research interest] Measure theory and Riemannian geometry \pause
		\item[Currently reading] Optimal transportation and convex analysis \pause
		\item[Why choose IMT] Common research interest with advisor J.Bertrand and enjoyable environment in Toulouse
	\end{description}
\end{frame}

\section{Master Internship}
\begin{frame}
	\frametitle{Research topic}
	Existence and uniquness for \alert{barycenter} of probability measure
\end{frame}

\section{Definition of barycenter}
\begin{frame}
	\frametitle{Heuristic definition}
	In mechanics, for a system of particles $m_\nu$ with coordinates $\boldsymbol{r}_\nu$, the barycenter $C$ has coordinate
	\[
		\boldsymbol{r}_C = \sum_ { \nu } \frac{m_{\nu}}{M} \boldsymbol { r }_ { \nu } =\arg \min_{\boldsymbol{r}} \sum_{\nu} \frac{m_\nu}{M} \Vert \boldsymbol{r}_{\nu} - \boldsymbol{r}\Vert^2
	\]
	where $ M := \sum _ { \nu } m _ { \nu } $.

	\pause
	\begin{definition}
		A barycenter $x \in E$ of probability measure $\mu$ on a metric space $(E,d)$ is
		\[
			x \in \arg\min_{z \in E} \int_{E} d^2(z,y) \diff \mu(y)
		\]
	\end{definition}
\end{frame}

\subsection{Examples ad my work}
\begin{frame}
	\frametitle{Examples revealing geometric structure}
	\setbeamercovered{invisible}
	\begin{example}
		For $x,y \in E$, consider meausre $\frac{1}{2}(\delta_x + \delta_y)$, barycenter $z$ is exactly the midpoint if exists
		\[
			d(x, z) = d(z, y) = \frac{1}{2} d(x,y).
		\]
	\end{example}
	\pause
	\begin{block}{Remark}
		\begin{itemize}
			\item Points on equator are midpoints for north and south poles of a sphere.

			      Hence, barycenter is not unique on the sphere. This is because sphere has positive curvature.
			      \pause
			\item Existence of midpoint is equivalent to existence of shortest path in complete intrinsic metric space.
			      Hence, barycenter fails to always exist on infinite dimensional ellipse with decreasing axes.
		\end{itemize}
	\end{block}
\end{frame}

\subsection{Internship overview}
\begin{frame}
	\setbeamercovered{transparent}
	\frametitle{Overview on internship topic \textit{barycenter}}
	Aim to find connection between geometric or meausre-theoretic structures and properties of barycenter, such as existence and uniqueness.
	\pause
	\begin{block}{Promising prospect and motivation}
		This topic relates deeply to fruitful fields where we can borrow inspirations:
		\begin{itemize}
			\item Measure theory and metric geometry for geometric implication
			      \pause
			\item Optimal transportation for study on abstract space of measures
		\end{itemize}
	\end{block}
	\pause
	\begin{block}{Main difficulties}
		\begin{itemize}
			\item Firm mastery of calculation on Remiannian manifolds
			      \pause
			\item Detailed investigation of results from convex analysis and optimal transportation
			      \pause
			\item No linear structure, functional analysis not applicable
		\end{itemize}
	\end{block}
\end{frame}

\section{PhD proposal}
\begin{frame}
	\frametitle{PhD research topic}
	Barycenter and Central Limit Theorem (abbr. as CLT)
\end{frame}

\subsection{Introduction to PhD proposal}
\begin{frame}
	\frametitle{Barycenter and CLT}
	Barycneter has application in statistics, serving as an estimation of mean.

	It is consistent w.r.t. convergence of measures. We already have \textit{law of large numbers} for barycenter, and aim to study \textit{CLT} for this convergence. \pause
	\setbeamercovered{transparent}
	\begin{block}{Related works}
		\begin{itemize}
			\item On Remannian manifold, CLT holds under complicated conditions, likely relating to cut locus. Aim: \alert{Clarify this realtion}. \pause
			\item On Wasserstein space of measures, by reducing to classical CLT, we can have CLT for some special measures. Aim: \alert{Find general condition}. \pause
			\item A CLT holds for the singular space open book. Aim: \alert{Investigate it and provide more interesting examples}.
		\end{itemize}
	\end{block}
\end{frame}
\end{document}
