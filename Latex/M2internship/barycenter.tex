\documentclass{report}

\usepackage{amsthm, amsmath, amsfonts}
\newtheorem{thm}{Theorem}
\newtheorem{prop}{Proposition}
\newtheorem{coro}{Corollary}
\newtheorem{rmk}{Remark}

\usepackage[backend=biber]{biblatex}
\addbibresource{Bibliography.bib}

\usepackage{hyperref}
\hypersetup{
    colorlinks=true,
    linkcolor=blue,
    filecolor=magenta,      
    urlcolor=cyan,
}

\newcommand{\diff}{\mathrm{d}}

\begin{document}
\author{Jianyu MA}
\date{\today}
\title{Existence and Uniqueness of Barycenters in various settings}
\maketitle

\begin{abstract}
	Barycenters, as a geometric concept, has easy generaliztion in various abstract spaces. Further applications of such a concept rely heavily on its existence and uniqueness. In the report, we study this problem in geometrical and measure-theoretical settings.

	Current working plan is:
	\begin{itemize}
		\item Verfify exmaple of non-uniquness in relation of non-branching: four pod by M.Bertrand
		\item Hilbert space geometric properties
		\item Counter examples of non-uniquess and reasons
	\end{itemize}
\end{abstract}

\tableofcontents

\chapter{Reading Notes}
\section{Barycenters in Alexandrov spaces of curvature bounded below}
\begin{prop}[Example 3.1 (a)]
	Let \( X \) be the infinite dimensional ellipsoid of axes of lengths \( c _ { n } = ( n + 1 ) / 2 n \) with \( n \in \mathbb { N } , \) namely \[ X = \left\{ \left( x _ { 1 } , x _ { 2 } , \ldots \right) \in \mathbb { R } ^ { \infty } \mid \sum _ { n \in \mathbb { N } } \frac { x _ { n } ^ { 2 } } { c _ { n } ^ { 2 } } = 1 \right\} \]

	Then \( X \) is complete, but \( \mu = \left( \delta _ { ( 1,0,0 , \ldots ) } + \delta _ { ( - 1,0,0 , \ldots ) } \right) / 2 \) has no barycenter in \( X \).
\end{prop}

\begin{proof}
	As $ \frac{1}{2} \leq c_{n} \leq 1$, we know $X$ is closed subspace of Hilbert space $\ell^2$. Hence we can calculate distance through inner product. Pick $ x \in X$ and set $ e=(1,0,0\ldots)$, we have:
	\begin{align*}
		W_p(\delta_x, \mu) &= \int_{y} d^2(x, y) \diff \mu(y) = \frac{1}{2}(d^2(x,e)+d^2(x,-e))\\
											 &=\frac{1}{2}(\Vert x - e \Vert^2 + \Vert x + e \Vert^2) \\
											 &=\Vert x \Vert^2 + 1
\end{align*}

Hence a barycenter of $\mu$ in $X$ should minimize its length, which is impossible.
\end{proof}

\begin{rmk}
	\begin{enumerate}
		\item Hilbert space is not locally compact, as unit closed ball in infinite dimensional normed space is not sequential compact.
		\item $X$ is the unit sphere of $\ell^2$ with a different inner product: \[\langle x, y \rangle:=\sum_{i \in \mathbb{N}^*}\frac{x_i}{c_i}\cdot \frac{y_i}{c_i}.\] And we see that $X$ is hoemeomorphic to the unit sphere in $\ell^2$. From previous ponit, $X$ is then not compact. However, this conclusion is almost trivial as we don't have barycenter for $\mu$ in $X$.
		\item $X$ is in fact not locally compact, as proved in \cite{bessaga1975selected} Chapter IV §2 that unit sphere in $\ell^2$ is hoemeomorphic to $\ell^2$.
		\item $X$ is not locally compact with respect to the induced intrinsic metric topology. By compatibility of length structre with topology of base space $X$ (see Exercise 2.1.5 in \cite{burago2001course}), open set in origin topology is again open in the induced intrinsic metric topology. Cover an otherwise compact set of $X$ in new topology by open sets in original topology, we then get a contradiction.
		\item Unit sphere in $\ell^2$ is geodesic. By Theorem 2.4.16 in \cite{burago2001course}, we only need to show that midpoint in $X$ always exits, i.e. $\forall x,y \in X$, $\exists z \in X$ such that $d(x,z)=d(y,z)=\frac{1}{2}d(x,y)$. For such two points $x,y$, $z$ can be selected from the intersection of $X$ and the hypersurface passing the origin in $\ell^2$ that evenly seperates $x$ and $y$.
		\item $X$ is geodesic. Recall that shortest paths are closed under pointwise convergence (Proposition 2.5.17 in \cite{burago2001course}, proved by lower semicontinuity of length function). To begin with, any two points in $X$ with finite coordinate components can be connected by a shortest path. Then pass a point to have infinite coordinate components and later do the same for another point; we thus proved $X$ is geodesic. We use that points with finite coordinate components are dense in $\ell^2$ and so are they in $X$.
	\end{enumerate}
\end{rmk}

\begin{rmk}[Exmaple 2.1 (c)]
	This example is in contradiction with \href{https://en.wikipedia.org/wiki/Hadamard_space}{Wikipedia: Hadamard space}. The author claims Hilbert space is of nonnegative curvature, while Wikipedia says ``a normed space is an Hadamard space if and only if it is a Hilbert space".
\end{rmk}

Then we discuss the infinitesimal structure of an Alexandrov space \( ( X , d ) . \) Fix \( z \in X \)
and let \( \hat { \Sigma } _ { z } \) be the set of all unit speed geodesics emanating from \( z \). For \( \gamma , \eta \in \hat { \Sigma } _ { z } , \) by virtue of the curvature bound, the limit
\[ \angle _ { z } ( \gamma , \eta ): = \arccos \left( \lim _ { s , t \downarrow 0 } \frac { s ^ { 2 } + t ^ { 2 } - d _ { X } ( \gamma ( s ) , \eta ( t ) ) ^ { 2 } } { 2 s t } \right) \]
exists and is regarded as the angle (pseudo-)distance of \( \hat { \Sigma } _ { z } . \) We define the space of directions \( \left( \Sigma _ { z } , \angle _ { z } \right) \) at \( z \) as the completion of \( \Sigma _ { z } / \sim \) with respect to \( \angle _ { z } , \) where \( \gamma \sim \eta \) if \( \angle _ { z } ( \gamma , \eta ) = 0 . \) The tangent cone \( \left( C _ { z } , d _ { C _ { z } } \right) \) is defined as the Euclidean cone over \( \left( \Sigma _ { z } , \angle _ { z } \right) , \)
that is to say,
\begin{align*}
	C _ { z }: &= \Sigma _ { z } \times [ 0 , \infty ) / \Sigma _ { z } \times \{ 0 \} \\
	d _ { C _ { z } } ( ( \gamma , s ) , ( \eta , t ) ): &= \sqrt { s ^ { 2 } + t ^ { 2 } - 2 s t \cos \angle _ { z } ( \gamma , \eta ) }
\end{align*}

\begin{thm}[Computation of distance in tangent cone]
	Let $\gamma$ and  $\eta$ be two geodesics from unit interval into $X$ starting at $z$.  Then $\gamma^\prime(0)$ and $\eta^\prime(0)$ are naturally elements in $C_z$.  We can calculate their distance as:
	\[  d_{C_z}(\gamma^\prime(0),\eta^\prime(0))=\lim _ { t \downarrow 0 } \frac  {d ( \eta ( t ) , \gamma ( t ) )  }{ t }\]
\end{thm}
\begin{proof}
	This is a calculation of angle between $\gamma^\prime(0),\eta^\prime(0)$ with a specific selected convergening process.
\end{proof}
\chapter{Introduction and preparation}
\section{Definition of barycenter}
To begin with, recall in physics we define barycenter as a virtual point where we can image gravity acts at. Afetr choosing a coordinate, we formulate barycenter as:

Put this definition in mathematical context, we consider Euclidean space with usual square root distance $d:=$.

\section{Geometry and probability background}
We should introduce necessary concepts and recall classic results to help our discussion; for the sake of instruction, we present some classic proofs.
\subsection{Metric geometry}
\subsection{Measure theory}

\section{Optimal transportation}

\part{Barycenters in Geometrical Spaces}

\chapter{Locally compact complete metric space}

\section{Riemannian manifold}
\subsection{Euclidean spaces}

\chapter{Alexandrov spaces}
\section{Curvature bounded above}
\section{Curvature bounded below}
\subsection{Hilbert spaces}

\part{Barycenters in Wasserstein Spaces}

\printbibliography
\end{document}
