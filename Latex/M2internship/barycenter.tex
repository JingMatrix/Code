\documentclass{report}

\usepackage{amsthm, amsmath, amsfonts}
\newtheorem{thm}{Theorem}
\newtheorem{prop}{Proposition}
\newtheorem{coro}{Corollary}
\theoremstyle{remark}
\newtheorem{rmk}{Remark}
\theoremstyle{definition}
\newtheorem{defn}{Definition}

% \usepackage{siunitx}
% \sisetup{detect-family,detect-inline-family=math}

\usepackage[backend=biber]{biblatex}
\addbibresource{Bibliography.bib}

\usepackage{hyperref}
\hypersetup{
    colorlinks=true,
    linkcolor=blue,
    filecolor=magenta,
    urlcolor=cyan,
}

\newcommand{\diff}{\mathrm{d}}

\begin{document}
\author{Jianyu MA}
\date{\today}
\title{Existence and Uniqueness of Barycenters in various settings}
\maketitle

\begin{abstract}
	Barycenter, as a geometric concept, has easy generalization in various abstract spaces. Further applications of such a concept rely heavily on its existence and uniqueness. In the report, we study this problem in geometrical and measure-theoretical settings.

	Current working plan is:
	\begin{itemize}
		\item Verify example of non-uniqueness in relation of non-branching: four points by M.Bertrand
		\item Hilbert space geometric properties
		\item Counter examples of non-uniqueness and reasons
	\end{itemize}
\end{abstract}

\tableofcontents

\chapter{Reading Notes}
\section{Barycenters in Alexandrov spaces of curvature bounded below}
\begin{prop}[Example 3.1 a]
	Let \( X \) be the infinite dimensional ellipsoid of axes of lengths \( c _ { n } = ( n + 1 ) / 2 n \) with \( n \in \mathbb { N } , \) namely \[ X = \left\{ \left( x _ { 1 } , x _ { 2 } , \ldots \right) \in \mathbb { R } ^ { \infty } \mid \sum _ { n \in \mathbb { N } } \frac { x _ { n } ^ { 2 } } { c _ { n } ^ { 2 } } = 1 \right\} \]

	Then \( X \) is complete, but \( \mu = \left( \delta _ { ( 1,0,0 , \ldots ) } + \delta _ { ( - 1,0,0 , \ldots ) } \right) / 2 \) has no barycenter in \( X \).
\end{prop}

\begin{proof}
	As $ 1 / 2 \leq c_{n} \leq 1$, we know $X$ is closed subspace of Hilbert space $\ell^2$. Hence we can calculate distance through inner product. Pick $ x \in X$ and set $ e=(1,0,0\ldots)$, we have:
	\begin{align*}
		W_2(\delta_x, \mu)^2 & = \int_{X} d(x, \cdot)^2 \diff \mu = \frac{d(x,e)^2+d(x,-e)^2 }{2} \\
		                     & =\frac{\Vert x - e \Vert^2 + \Vert x + e \Vert^2}{2}               \\
		                     & =\Vert x \Vert^2 + 1
	\end{align*}

	Hence a barycenter of $\mu$ in $X$ should minimize its length. For a vector in $X$ to attain minimum length, if restricting to the first $n$ coordinate components, $X$ cannot have nonzero components except the last one (minimum length axes). Otherwise we can keep other not considered coordinate components unchanged but vanish first $n-1$ coordinate components to get a strictly shorter vector in $X$. This indicates that no such $X$ could exist.
\end{proof}

\begin{rmk}
	\begin{enumerate}
		\item Hilbert space is not locally compact, as unit closed ball in infinite dimensional normed space is not sequential compact. Recall that the last claim can be proved using \href{https://en.wikipedia.org/wiki/Riesz%27s_lemma}{Wikipedia: Riesz's lemma}.
		\item % $X$ is the unit sphere of $\ell^2$ with a different inner product: \[\langle x, y \rangle:=\sum_{i \in \mathbb{N}^*}\frac{x_i}{c_i}\cdot \frac{y_i}{c_i}.\]
		      We see that $X$ is homeomorphic to the unit sphere in $\ell^2$. From previous point, $X$ is then not compact. However, this conclusion is almost trivial as we don't have barycenter for $\mu$ in $X$.
		\item $X$ is in fact not locally compact as a subspace of $\ell^2$, as proved in \cite{bessaga1975selected} Chapter IV §2 that unit sphere in $\ell^2$ is homeomorphic to $\ell^2$.
		\item $X$ with metric inherited form $\ell^2$ is apparently not a length space. To explore more of this example, we should start to consider induced intrinsic metric $\hat{d}$ on $X$ by inherited metric from $\ell^2$.
		\item $X$ is not locally compact with respect to the induced intrinsic metric topology. By compatibility of induced length structure with the topology of base space $X$ (see Exercise 2.1.5 in \cite{burago2001course}), open set in original topology is again open in the induced intrinsic metric topology. Cover an otherwise compact set of $X$ in new topology by open sets in original topology, we get a contradiction.
		      % \item Unit sphere in $\ell^2$ is geodesic with respect to induced intrinsic metric. By Theorem 2.4.16 in \cite{burago2001course}, we only need to show that midpoint in $X$ always exits, i.e., $\forall x,y \in X$, $\exists z \in X$ such that $d(x,z)=d(y,z)=\frac{1}{2}d(x,y)$. For such two points $x$ and $y$, $z$ can be selected from the intersection of $X$ and the hypersurface passing the origin in $\ell^2$ that evenly separates $x$ and $y$.
		\item % $X$ is geodesic with respect to induced intrinsic metric $\hat{d}$.
		      Induced length structure in $X$. %Recall that shortest paths are closed under point-wise convergence (Proposition 2.5.17 in \cite{burago2001course}, proved by lower semi-continuity of length function).
		      We show that distance in $X$ is approximate by curves with finite coordinate components. To begin with, any two points in $X$ with finite coordinate components can be connected by a shortest path. Then pass a point to have infinite coordinate components and later do the same for another point.
		      % We use the fact that points with finite coordinate components are dense in $\ell^2$ and so are they in $X$.
		\item A geodesic space $(E, \hat{d})$ always admits midpoint $z$ of two points $x$ and $y$ in $E$ as the barycenter of $\mu : = \frac{1}{2} (\delta_x + \delta_y)$, since $z$ attain equality in following general inequality:
		      \[
			      \hat{d}(x,y)^2 \leq \left(\hat{d}(x,z) + \hat{d}(z,y)\right)^2 \leq 2\left(\hat{d}(x,z)^2+ \hat{d}(z,y)^2\right)
		      \]
		\item An interesting question could be: consider $X$ with induced intrinsic metric $\hat{d}$, does the barycenter of $\mu$ exist?
		      Basically, we need to answer whether $\hat{d}(e,-e)$ is realized by a rectifiable curve in $X$.

		      Assume now we have $L_{d}(\gamma)=\hat{d}(e,-e)$.
		      Restricting to the first $n$ coordinate components, barycenters always exist and are exactly endponits of shortest axes. Existence is a priori guaranteed by compactness of finite dimensional ellipse.
		      % And those two endponits attain equality in following inequality:

		      % 		            \[
		      % 		      	      \hat{d}(e,-e)^2 \leq \left(\hat{d}(e,z) + \hat{d}(z,-e)\right)^2 \leq 2\left(\hat{d}(e,z)^2+ \hat{d}(z,-e)^2\right)
		      % 		            \]
		      The restricted induced intrinsic metric $\hat{d}_n(e,-e)$ depends on $n$ as we are considering minimal geodesics in $n-1$ dimensional ellipse. Actually, set $k=\sqrt{1-c_n^2}$ as the eccentricity, we have
		      \[
			      \hat{d}_n(e,-e)= 2\int _{0}^{\tfrac {\pi }{2}}{\sqrt {1-k^{2}\sin ^{2}\theta }}\,\mathrm {d} \theta = 2\int _{0}^{1}{\frac {\sqrt {1-k^{2}t^{2}}}{\sqrt {1-t^{2}}}}\,\mathrm {d} t
		      \]
		      This is a decreasing function in $n$.	To apply the same idea in the proof of previous proposition, we should decompose distance $d$ of $\ell^2$ into two parts, involving first $n$ coordinate components or not, $d^2=d_n^2+d_r^2$, i.e., $\Vert \cdot \Vert^2 = \Vert \cdot \Vert_n^2 + \Vert \cdot \Vert_r^2$ with $d_n=\Vert \cdot \Vert_n$ the distance in $\mathbb{R} ^n $ . For $\Delta=[\delta_0=0,\delta_1, \ldots, \delta_m=1]$ all possible finite partition of $\gamma$, we have:
		      \[
			      \hat{d}(e,-e)=\sup_{\Delta} \sum_{i=1}^{m} \Vert\gamma(\delta_{i-1}) - \gamma(\delta_{i})\Vert
		      \]
		      % We use that square function is continuous in the second equality, and $\gamma_i$ is the value of $\gamma$ at $i$th partition.:w
		      We \textbf{would like to} have following energy variation hold:
		      \begin{equation}
			      \label{energy_variation_in_X}
			      \gamma = \mathrm{arg} \inf_{\eta} \sup_{\Delta} \sum_{i=1}^{m} \Vert\eta(\delta_{i-1})- \eta(\delta_{i})\Vert^2
		      \end{equation}
		      Then this formula \ref{energy_variation_in_X} will contradict our assumption. Write $\gamma$ in coordinate as $(\gamma_1, \gamma_2\ldots\gamma_n\ldots)$, to keep $\gamma_k$ for $ k > n$ unchanged. Up to choose a bigger integer $n$ we can assume $ \exists t \in [0,1]$, such that $\gamma_n(t) \neq 0$. Consider $\tilde{\gamma}^n = (\gamma_1, \gamma_2 \ldots \gamma_n)$ a function with $\tilde\gamma^n(t)$ in $E_{\gamma(t)}$ where
		      \begin{align*}
			      E_{\gamma(t)} : & = \left\{ (x_1,x_2 \ldots x_n) \in \mathbb{R}^n \mid \sum_{i=1}^n \frac{x_i^2}{ c_i^2} = c_{\gamma(t)}^2\right\} \\
			      c_{\gamma(t)}:  & =\sqrt{ 1- \sum_{i > n} \frac{\gamma_j(t)^2}{c_j^2}} \geq 0
		      \end{align*}
		      Finally $\eta := (0,0,\ldots, c_n c_{\gamma(\cdot)},\gamma_{n+1},\ldots)$ will violate \ref{energy_variation_in_X} since energy variation coincide with arc-length variation in complete Riemannian manifold and we have $\int_{[0,1]} c_n c_{\gamma(\cdot)} \diff \lambda < L_d(\tilde{\gamma}^n)$.
				\item Use theory from Hilbert Riemannian manifold, as Example 5.1 in \cite{grossman1965hilbert} (I suspect this is the origin of our example), we can actually show that $e$ and $-e$ cannot be connected by minimal geodesic.
		      Define \( T: X \rightarrow X \) by \( T x = y \), where
		      \[
			      y _ { 1 } = x _ { 1 } , y _ { 2 } = 0 , y _ { i } = \frac{c_i}{c_{i-1}} x_{i-1} \text { for } i \geq 3 . \]

						Then \( T \) is a smooth map with only \( e \) and \(- e \) as fixed points. It is easily seen that any smooth curve from \( e\) to \( -e \) is taken by \( T \) into another such curve which is strictly shorter than the original ($c_i$ is decreasing). Therefore, there can be no minimal geodesic from \( e \) to \( -e . \)


	\end{enumerate}
\end{rmk}

\begin{rmk}[Exmaple 2.1 c]
	This example is in contradiction with \href{https://en.wikipedia.org/wiki/Hadamard_space}{Wikipedia: Hadamard space}. The author claims Hilbert space is of non-negative curvature, while Wikipedia says ``a normed space is an Hadamard space if and only if it is a Hilbert space".
\end{rmk}

Then we discuss the infinitesimal structure of an Alexandrov space \( ( X , d ) . \)

\begin{defn}[space of directions]
	Fix \( z \in X \) and let \( \hat { \Sigma } _ { z } \) be the set of all unit speed geodesics emanating from \( z \). For \( \gamma , \eta \in \hat { \Sigma } _ { z } , \) by virtue of the curvature bound, the limit
	\[ \angle _ { z } ( \gamma , \eta ): = \arccos \left( \lim _ { s , t \downarrow 0 } \frac { s ^ { 2 } + t ^ { 2 } - d _ { X }\left( \gamma ( s ) , \eta ( t ) \right) ^ { 2 } } { 2 s t } \right) \]
	exists and is regarded as the angle (pseudo-)distance of \( \hat { \Sigma } _ { z } . \) We define the space of directions \( \left( \Sigma _ { z } , \angle _ { z } \right) \) at \( z \) as the completion of \( \Sigma _ { z } / \sim \) with respect to \( \angle _ { z } , \) where \( \gamma \sim \eta \) if \( \angle _ { z } ( \gamma , \eta ) = 0 . \)
\end{defn}

\begin{defn}[tangent cone]
	The tangent cone \( \left( C _ { z } , d _ { C _ { z } } \right) \) is defined as the Euclidean cone over \( \left( \Sigma _ { z } , \angle _ { z } \right) , \)
	that is to say,
	\begin{align*}
		C _ { z }:                                           & = \Sigma _ { z } \times [ 0 , \infty ) / \Sigma _ { z } \times \{ 0 \}          \\
		d _ { C _ { z } } ( ( \gamma , s ) , ( \eta , t ) ): & = \sqrt { s ^ { 2 } + t ^ { 2 } - 2 s t \cos \angle _ { z } ( \gamma , \eta ) }
	\end{align*}
\end{defn}

\begin{thm}[Computation of distance in tangent cone]
	Let $\gamma$ and  $\eta$ be two geodesics from unit interval into $X$ starting at $z$.  Then $\gamma^\prime(0)$ and $\eta^\prime(0)$ are naturally elements in $C_z$.  We can calculate their distance as:
	\[  d_{C_z}(\gamma^\prime(0),\eta^\prime(0))=\lim _ { t \downarrow 0 } \frac  {d ( \eta ( t ) , \gamma ( t ) )  }{ t }\]
\end{thm}

\begin{proof}
	This is a calculation of angle between $\gamma^\prime(0),\eta^\prime(0)$ with a specific selected converging process.
\end{proof}
\chapter{Introduction and preparation}
\section{Definition of barycenter}
To begin with, recall in physics we define barycenter as a virtual point where we can image gravity acts at. After choosing a coordinate, we formulate barycenter as:

Put this definition in mathematical context, we consider Euclidean space with usual square root distance $d:=$.

\section{Geometry and probability background}
We should introduce necessary concepts and recall classic results to help our discussion; for the sake of instruction, we present some classic proofs.
\subsection{Metric geometry}
\subsection{Measure theory}

\section{Optimal transportation}

\part{Barycenters in Geometrical Spaces}

\chapter{Locally compact complete metric space}

\section{Riemannian manifold}
\subsection{Euclidean spaces}

\chapter{Alexandrov spaces}
\section{Curvature bounded above}
\subsection{Hilbert spaces}
\section{Curvature bounded below}

\chapter{Hilbert Riemannian manifold}

\part{Barycenters in Wasserstein Spaces}

\printbibliography
\end{document}
