\documentclass{report}
\usepackage{amsthm, amsmath}
\usepackage[backend=biber]{biblatex}
\addbibresource{Bibliography.bib}
\usepackage{hyperref}
\hypersetup{bookmarks=true, hidelinks}
\begin{document}
\author{Jianyu MA}
\date{\today}
\title{Existence and Uniqueness of Barycenters in various settings}
\maketitle

\begin{abstract}
	Barycenters, as a geometric concept, has easy generaliztion in various abstract spaces. Further applications of such a concept rely heavily on its existence and uniqueness. In the report, we study this problem in geometrical and measure-theoretical settings.
\end{abstract}

\tableofcontents

\chapter{Reading Notes}
\section{Working plan}
\begin{itemize}
	\item Verfify exmaple of non-uniquness in relation of non-branching: four pod by M.Bertrand.
	\item Hilbert space geometric properties
	\item Counter examples of non-uniquess and reasons
\end{itemize}
\section{Barycenters in Alexandrov spaces of curvature bounded below}
See \cite{ohta2012barycenters}.


\chapter{Introduction and Preparation}
\section{Definition of Barycenter}
To begin with, recall in physics we define barycenter as a virtual point where we can image gravity acts at. Afetr choosing a coordinate, we formulate barycenter as:

Put this definition in mathematical context, we consider Euclidean space with usual square root distance $d:=$.

\section{Geometry and Probability Background}
We should introduce necessary concepts and recall classic results to help our discussion; for the sake of instruction, we present some classic proofs.
\subsection{Metric Geometry}
\subsection{Probability Theory}

\section{Optimal Transpoatation}

\part{Barycenters in Geometrical spaces}

\chapter{Localy compact complete metric space}

\section{Riemannian manidfold}
\subsection{Euclidean spaces}

\chapter{Alexandrov Spaces}
\section{Curvature bounded above}
\section{Curvature bounde below}
\subsection{Hilbert spaces}

\chapter{Barycenters in Wasserstein spaces}

\printbibliography
\end{document}
