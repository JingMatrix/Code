\documentclass{report}

\usepackage{hyperref}
\hypersetup{
	colorlinks=true,
	linkcolor=blue,
	filecolor=magenta,
	urlcolor=cyan,
}

\usepackage{amsthm, amsmath, amsfonts, mathrsfs, amssymb, bbm}
\usepackage[]{cancel} 
\usepackage{cleveref}
\newtheorem{thm}{Theorem}[section]
\newtheorem{prop}[thm]{Proposition}
\newtheorem{coro}[thm]{Corollary}
\newtheorem{lem}[thm]{Lemma}
\theoremstyle{remark}
\newtheorem{rmk}{Remark}[chapter]
\newtheorem{example}{Example}[chapter]
\theoremstyle{definition}
\newtheorem{defn}{Definition}[chapter]

\usepackage{biblatex}
\addbibresource{Bibliography.bib}


\newcommand{\diff}{\operatorname{d}}

\title{Existence and Uniqueness of Barycenters in various settings}
\author{Jianyu MA}
\date{\today}

\begin{document}
\pagenumbering{Alph}
% \begin{titlepage}
\maketitle

\begin{abstract}
	Barycenter, as a geometric concept, has easy generalization in various abstract spaces. Further applications of such a concept rely heavily on its existence and uniqueness. In the report, we study this problem in geometrical and measure-theoretical settings.

	Current working plan is:
	\begin{itemize}
		% \item Verify example of non-uniqueness in relation of non-branching: four points by M.Bertrand
		% \item Hilbert manifold geometric properties
		% \item Counter examples of non-uniqueness and reasons
		% \item Familiar with measure theory of metric space
		% \item Study convex analysis
		% \item Paper review of \cite{KIM2017640}
		% \item Formulate a concise proof of main results in \cite{KIM2017640}
		% \item Measure theory on Riemannian manifold
		\item Write final report
	\end{itemize}
\end{abstract}

\tableofcontents
\pagenumbering{arabic}
%! TEX root = ../barycenter.tex
\chapter{Papers Review}
% In this chapter, I shall review in order papers: \cite{ohta2012barycenters}, \cite{le2017existence}.
\section{Barycenters in Alexandrov spaces of curvature bounded below}
\begin{prop}[Example 3.1 a]
	Let \( X \) be the infinite dimensional ellipsoid of axes of lengths \( c _ { n } = ( n + 1 ) / 2 n \) with \( n \in \mathbb { N } , \) namely 
	\[
	X = \left\{ \left( x _ { 1 } , x _ { 2 } , \ldots \right) \in \mathbb { R } ^ { \infty } \mid \sum _ { n \in \mathbb { N } } \frac { x _ { n } ^ { 2 } } { c _ { n } ^ { 2 } } = 1 \right\} 
\]

	Then \( X \) is complete, but \( \mu = \left( \delta _ { ( 1,0,0 , \ldots ) } + \delta _ { ( - 1,0,0 , \ldots ) } \right) / 2 \) has no barycenter in \( X \).
\end{prop}

\begin{proof}
	As $ 1 / 2 \leq c_{n} \leq 1$, we know $X$ is closed subspace of Hilbert space $\ell^2$. Hence we can calculate distance through inner product. Pick $ x \in X$ and set $ e=(1,0,0\ldots)$, we have:
	\begin{align*}
		W_2(\delta_x, \mu)^2 & = \int_{X} d(x, \cdot)^2 \diff \mu = \frac{d(x,e)^2+d(x,-e)^2 }{2} \\
		                     & =\frac{\Vert x - e \Vert^2 + \Vert x + e \Vert^2}{2}               \\
		                     & =\Vert x \Vert^2 + 1
	\end{align*}

	Hence a barycenter of $\mu$ in $X$ should minimize its length. For a vector $x$ in $X$ to attain minimum length, if restricting to the first $n$ coordinate components, $x$ cannot have nonzero components except the last one (minimum length axes), i.e., $x_i =0$ for $i<n$. Otherwise we can keep other not considered coordinate components, $x_j$ for $j>n$,  unchanged but vanish first $n-1$ coordinate components to get a strictly shorter vector in $X$. This indicates that no such barycenter $x$ could exist in $X$.
\end{proof}

\begin{rmk}
	\begin{enumerate}
		% \item Hilbert space is not locally compact, as unit closed ball in infinite dimensional normed space is not sequential compact. Recall that the last claim can be proved using \href{https://en.wikipedia.org/wiki/Riesz%27s_lemma}{Wikipedia: Riesz's lemma}.
		% \item  $X$ is the unit sphere of $\ell^2$ with a different inner product: \[\langle x, y \rangle:=\sum_{i \in \mathbb{N}^*}\frac{x_i}{c_i}\cdot \frac{y_i}{c_i}.\]
		% We see that $X$ is homeomorphic to the unit sphere in $\ell^2$. From previous point, $X$ is then not compact. However, this conclusion is almost trivial as we don't have barycenter for $\mu$ in $X$.
		\item $X$ is in fact not locally compact as a subspace of $\ell^2$. Consider the sequence with only one nonzero coordinate component, it has no converging subsequence.
		      % as proved in \cite{bessaga1975selected} Chapter IV §2 Theorem 2.1, that unit sphere in $\ell^2$ is homeomorphic to $\ell^2$.
		\item $X$ with metric inherited form $\ell^2$ is apparently not a length space. To explore more of this example, we should start to consider induced length metric $\hat{d}$ on $X$ by inherited metric from $\ell^2$.
		\item $X$ is not locally compact with respect to the induced length metric topology. By compatibility of induced length structure with the topology of base space $X$ (see Exercise 2.1.5 in \cite{burago2001course}), open set in original topology is again open in the induced length metric topology. Cover an otherwise compact set of $X$ in new topology by open sets in original topology, we get a contradiction.
		      % \item Unit sphere in $\ell^2$ is geodesic with respect to induced length metric. By Theorem 2.4.16 in \cite{burago2001course}, we only need to show that midpoint in $X$ always exists, i.e., $\forall x,y \in X$, $\exists z \in X$ such that $d(x,z)=d(y,z)=\frac{1}{2}d(x,y)$. For such two points $x$ and $y$, $z$ can be selected from the intersection of $X$ and the hypersurface passing the origin in $\ell^2$ that evenly separates $x$ and $y$.

		      % \item  $X$ is geodesic with respect to induced length metric $\hat{d}$.
		      % Induced length structure in $X$.
		      %Recall that shortest paths are closed under point-wise convergence (Proposition 2.5.17 in \cite{burago2001course}, proved by lower semi-continuity of length function).
		      % We show that distance in $X$ is approximate by curves with finite coordinate components. To begin with, any two points in $X$ with finite coordinate components can be connected by a shortest path. Then pass a point to have infinite coordinate components and later do the same for another point.
		      % We use the fact that points with finite coordinate components are dense in $\ell^2$ and so are they in $X$.
	\end{enumerate}
\end{rmk}


\begin{rmk}[Example 2.1 c]
	% This example is in contradiction with \href{https://en.wikipedia.org/wiki/Hadamard_space}{Wikipedia: Hadamard space}. The author claims Hilbert space is of non-negative curvature, while Wikipedia says ``a normed space is an Hadamard space if and only if it is a Hilbert space".
	Hilbert space satisfies equality in triangle comparison with Euclidean space, hence can be regarded ``flat".
\end{rmk}

Then we discuss the infinitesimal structure of an Alexandrov space \( ( X , d )\).

\begin{defn}[space of directions]
	Fix \( z \in X \) and let \( \hat { \Sigma } _ { z } \) be the set of all unit speed geodesics emanating from \( z \). For \( \gamma , \eta \in \hat { \Sigma } _ { z } , \) by virtue of the curvature bound, the limit
	\begin{equation}
		\label{angle_def}
		\angle _ { z } ( \gamma , \eta ): = \arccos \left( \lim _ { s , t \downarrow 0 } \frac { s ^ { 2 } + t ^ { 2 } - d _ { X }\left( \gamma ( s ) , \eta ( t ) \right) ^ { 2 } } { 2 s t } \right)
	\end{equation}
	exists and is regarded as the angle (pseudo-)distance of \( \hat { \Sigma } _ { z } . \) We define the space of directions \( \left( \Sigma _ { z } , \angle _ { z } \right) \) at \( z \) as the completion of \( \Sigma _ { z } / \sim \) with respect to \( \angle _ { z } , \) where \( \gamma \sim \eta \) if \( \angle _ { z } ( \gamma , \eta ) = 0 . \)
\end{defn}

\begin{defn}[tangent cone]
	The tangent cone \( \left( C _ { z } , d _ { C _ { z } } \right) \) is defined as the Euclidean cone over \( \left( \Sigma _ { z } , \angle _ { z } \right) , \)
	that is to say,
	\begin{align*}
		C _ { z }:                                           & = \Sigma _ { z } \times [ 0 , \infty ) / \Sigma _ { z } \times \{ 0 \}          \\
		d _ { C _ { z } } ( ( \gamma , s ) , ( \eta , t ) ): & = \sqrt { s ^ { 2 } + t ^ { 2 } - 2 s t \cos \angle _ { z } ( \gamma , \eta ) }
	\end{align*}
\end{defn}

\begin{thm}[Computation of distance in tangent cone]
	Let $\gamma$ and  $\eta$ be two geodesics from unit interval into $X$ starting at $z$. Then $\gamma^\prime(0)$ and $\eta^\prime(0)$ are naturally elements in $C_z$. We can calculate their distance as:
	\[  d_{C_z}(\gamma^\prime(0),\eta^\prime(0))=\lim _ { t \downarrow 0 } \frac  {d ( \eta ( t ) , \gamma ( t ) )  }{ t }\]
\end{thm}

\begin{proof}
	This is a calculation of angle between $\gamma^\prime(0)$ and $\eta^\prime(0)$ with a specific selected converging process. To apply angle definition in \cref{angle_def}, normalize $\gamma$ and $\eta$ as $\gamma\left(\vert\gamma^\prime(0)\vert \cdot\right)$ and $\eta\left(\vert\eta^\prime(0)\vert \cdot\right)$
	\[
		\cos\angle _ { z } ( \gamma , \eta ): = \left( \lim _ { s , t \downarrow 0 } \frac { s ^ { 2 } + t ^ { 2 } - d _ { X }\left( \gamma (\vert \gamma^\prime(0)\vert s ) , \eta (\vert \eta^\prime(0)\vert  t ) \right) ^ { 2 } } { 2 s t } \right)
	\]
	To get desired formula, put $ x := \vert  \gamma^\prime(0)\vert s =\vert \eta^\prime(0)\vert t $ and pass $x \downarrow 0$.
\end{proof}

\begin{rmk}[Lemma 4.5, Lang and Schroeder's inequality]
	When passing from finite support measures to general ones, the author should use Wasserstein metric $W_p$ convergence but not only weak convergence. Otherwise, he cannot justify convergence of integrals of distance function $d_{C_z}$. Here we use the fact that finite supported measures is a dense subspace of Wasserstein space $\mathcal{W}_2(X)$ if $X$ is a Polish space (see \cite{villani2008optimal} Theorem 6.18), recalled in this report as \cref{topology_Wasserstein}.
\end{rmk}


%! TEX root = ../barycenter.tex
\chapter{Introduction and preparation}
\section{Definition of barycenter}
To begin with, recall in physics we define barycenter (or written as barycentre), i.e.,  center inertial mass, as a geometic point where we can image gravity acts at. In the case of a system of particles $\boldsymbol{P}_\nu$, $\nu = 1, \ldots , N$, each with mass $m_\nu$ that are located in space with coordinates $\boldsymbol{r}_\nu$, $\nu = 1, \ldots , N$, the coordinate of the center of mass $C$ satisfies condition:
\[
	\sum _{\nu=1}^{N} m_{\nu}\boldsymbol {r}_{\nu} -	\sum _{\nu=1}^{N}m_{\nu}\boldsymbol {r}_C = \sum _{\nu=1}^{N}m_{\nu}(\boldsymbol {r} _{\nu}-\boldsymbol {r}_C )=\boldsymbol {0}.
\]
Then set \( M := \sum _ { \nu = 1 } ^ { N } m _ { \nu } \), we should have
\[
	\boldsymbol { r } _ { C } = \left( \sum _ { \nu = 1 } ^ { N } m _ { \nu } \boldsymbol { r } _ { \nu } \right) / M .
\]
We can characterize barycenter as the unique solution of minimizing problem
\begin{equation}
	\label{barycenter_def}
	\boldsymbol{r}_C = \operatorname{arg} \min_{\boldsymbol{r}} \sum_{\nu= 1}^N \frac{m_\nu}{M} \Vert \boldsymbol{r}_{\nu} - \boldsymbol{r}\Vert^2
\end{equation}
as norm is calculated as inner product on Eculidean space.

To generalize \ref{barycenter_def}, there are two core concepts to keep: metric distance and measure integral.

\section{Geometry and probability background}
We should introduce necessary concepts and recall classic results to help our discussion; for the sake of instruction, we present some classic proofs.
\subsection{Metric geometry}
\subsection{Measure theory}

Let \( X \) be a metric space and \( \mathscr { M } ( X ) \) the space of all measures defined on \( \mathscr { B } _ { X } . \) An element \( \mu \in \mathscr { H } ( X ) \) is a nonnegative, countably additive set function defined on \( \mathscr { B } _ { X } \) with the property \( \mu ( X ) = 1 . \quad C ( X ) \) stands for the space of all bounded real valued continuous functions on \( X \).  We shall now topologize the space \( \mathscr { M } ( X ) \) by defining a base of open neighborhoods for any point \( \mu . \) Consider the family of sets of the form
\( V _ { \mu } \left( f _ { 1 } , f _ { 2 } , \ldots , f _ { k } ; \varepsilon _ { 1 } , \ldots , \varepsilon _ { k } \right) \)
\[ \left\{\nu \in \mathscr { M } ( X ) \, \mid  \left| \int f _ { i } d v - \int f _ { i } d \mu \right| < \varepsilon _ { i } , \quad i = 1,2 , \ldots , k \right\} \]

where \( f _ { 1 } , f _ { 2 } , \ldots , f _ { k } \) are elements from \( C ( X ) \) and \( \varepsilon _ { 1 } , \varepsilon _ { 2 } , \ldots , \varepsilon _ { k } , \) are positive numbers. It is easy to verify that the family of sets obtained by varying \( k , f _ { 1 } , f _ { 2 } , \ldots , f _ { k } , \varepsilon _ { 1 } , \ldots , \varepsilon _ { k } \) satisfies the axioms of a basis for a topology.

\begin{defn}
	We shall refer to this as the weak topology  in \( \mathscr { M } ( X ) \).
\end{defn}

It is then clear that a net \( \left\{ \mu _ { \alpha } \right\} \) of measures converges in the weak topology to a measure \( \mu \) if and only if \( \int f d \mu _ { \alpha } \rightarrow \int f d \mu \) for every \( f \in C ( X ) . \) In such a case we shall say that \( \mu _ { \alpha } \) converges "weakly" to \( \mu \) or \( \mu _ { \alpha } \Rightarrow \mu \) in symbols. Unless otherwise stated, \( \mathscr { M } ( X ) \) will always be considered as a topological space with the weak topology.

We shall first recall a theorem which yields several useful equivalent definitions of the weak topology.

\begin{thm}
	Let \( \mu _ { \alpha } \) be a net in \( \mathscr { M } ( X ) . \) Then the following statements are equivalent:
	\begin{itemize}
		\label{weak_convergence_thm}
		\item \( \mu _ { \alpha } \Rightarrow \mu \)
		\item \( \lim _ { \alpha } \int g d \mu _ { \alpha } = \int g d \mu \) for all \( g \in U ( X ) \) where \( U ( X ) \) is the space of all bounded real valued uniformly continuous functions;
		\item \( \overline { \lim } _ { \alpha } \mu _ { \alpha } ( C ) \leqslant \mu ( C ) \) for every closed set \( C \)
		\item \( \lim _ { \alpha } \mu _ { \alpha } ( G ) \geqslant \mu ( G ) \) for every open set \( G \)
		\item \( \lim _ { \alpha } \mu _ { \alpha } ( A ) = \mu ( A ) \) for every Borel set \( A \) whose boundary has \( \mu \) -measure \( 0 . \)
	\end{itemize}
\end{thm}

For each point \( x \in X \) we shall denote by \( p _ { x } \) the measure degenerate at the point \( x \). Denote \( D = \left\{ p _ { x }: x \in X \right\} \).

\begin{lem}
	\label{dirac_measure_weak_homeomeorphic}
	\( X \) is homeomorphic to the (topological) subset $D$.
\end{lem}

\begin{proof}
	For any point \( x \) and \( g \in C ( X ) , \) we have \( \int g d p _ { x } = g ( x ) \). If \( x _ { \alpha } \rightarrow x _ { 0 } \) then \( g \left( x _ { \alpha } \right) \rightarrow g \left( x _ { 0 } \right) . \) Hence \( p _ { x _ { \alpha } } \Rightarrow p _ { x _ { 0 } } \). Conversely, let \( p _ { x _ { \alpha } } \Rightarrow p _ { x _ { 0 } } \) If \( x _ { \alpha } \) does not converge to \( x _ { 0 } \), there is an open set \( G \) and a subnet \( \left\{ x _ { \beta } \right\} \) such that \( x _ { 0 } \in G \) and \( x _ { \beta } \in X - G \) for all \( \beta . \) Let \( g \) be a continuous function such that \( 0 \leqslant g \leqslant 1 , g \left( x _ { 0 } \right) = 0 \) and \( g ( x ) = 1 \) for \( x \in X - G \). Then \( \int g d p _ { x _ { \beta } } = 1 , \) while \( \int g d p _ { x _ { 0 } } = 0 . \) This is a contradiction. This completes the proof.
\end{proof}

\begin{lem}
	\( D \) is a sequentially closed subset of \( \mathscr { M } ( X ) \).
\end{lem}

\begin{proof}
	Let \( \left\{ x _ { n } \right\} \) be a sequence of points in \( X \) such that \( p _ { x _ { n } } \Rightarrow q \) Suppose \( \left\{ x _ { n } \right\} \) does not have any convergent subsequence. Then the set \( S = \left\{ x _ { 1 } , x _ { 2 } , \ldots \right\} \) is closed and thus is any subset \( C \) of \( S . \) Since \( p _ { x _ { n } } \Rightarrow q \) we have by Theorem \ref{weak_convergence_thm}, \( q ( C ) \geqslant \overline { \lim } p _ { x _ { n } } ( C ) \) for \( C \subseteq S \). It follows that for every infinite subset \( S _ { 1 } \subseteq S , q \left( S _ { 1 } \right) = 1 \). This is a contradiction since \( q \) is a measure.

	Thus there is a subsequence \( \left\{ x _ { n _ { k } } \right\} , x _ { n _ { k } } \rightarrow x . \) By Lemma \ref{dirac_measure_weak_homeomeorphic}, \( q = p _ { x } \). Hence \( D \) is sequentially closed.
\end{proof}


\begin{thm}
	\label{finite_support_approximation}
	Let \( X \) be a separable metric space and \( E \subseteq X \) dense in \( X  \). Then the set of all measures whose supports are finite subsets of \( E \) is dense in \( \mathscr{ M } ( X ) \), the set of all Borel probability measures on $X$.
\end{thm}

\begin{proof}
	This proof is copy from Theorem 6.3 in \cite{parthasarathy2005probability}.

	It is obviously enough to prove that the set of all measures whose supports are finite subsets of \( X \) is dense in \( \mathscr { M } ( X ) . \) Let us denote the class of such measures by \( \mathscr { F } ( X ) . \) It is clear that any measure concentrated in a countable subset of \( X \) is a weak limit of measures from \( \mathscr { F } ( X ) . \) Thus it is sufficient to prove that any measure is a weak limit of measures vanishing outside countable subsets of \( X \).

	Choose and fix \( \mu \in \mathscr { M } ( X ) \). Since \( X \) is separable we can, for each integer \( n , \) write \( X \) as \( \bigcup _ { j } A _ { n j } , A _ { n j } \cap A _ { n k } = \phi \) if \( j \neq k , A _ { n j } \in \mathscr { B } _ { X } \) for all \( n \) and \( j \) and the diameter of \( A _ { n j } \) is \( \leq 1 / n \) for all \( j \). Let \( x _ { n j } \in A _ { n j } \) be arbitrary. Let \( \mu _ { n } \) be the measure with masses \( \mu \left( A _ { n j } \right) \) at the points \( x _ { n j } , \) respectively. Let $g$ be an arbitrary uniformly continuous (we can even assume Lipschitz continous here) bounded function, and let \[ \alpha _ { n j } = \inf _ { x \in A _ { n j } } g ( x ) , \quad \beta _ { n j } = \sup _ { x \in A _ { n j } } g ( x ) \]
	Since \( g \) is uniformly continuous and since the diameter of \( A _ { n j } \rightarrow 0 \) as \( n \rightarrow \infty \) uniformly in \( j , \sup _ { j } \left( \beta _ { n j } - \alpha _ { n j } \right) \rightarrow 0 \) as \( n \rightarrow \infty \). Now
	\begin{align*}
		\left| \int g d \mu _ { n } - \int g d \mu \right| & = \left| \sum \int _ { A _ { n j } } \left( g - g \left( x _ { n j } \right) \right) d \mu \right|     \\
		                                                   & \leq \sup_{j} \left( \beta _ { n j } - \alpha _ { n j } \right) \xrightarrow{ n \rightarrow \infty} 0.
	\end{align*}
\end{proof}

\begin{thm}[Prokhorov's theorem]
	If \( \mathcal { X } \) is a Polish space, then a set \( \mathcal { P } \subset P ( \mathcal { X } ) \) is precompact for the weak topology if and only if it is tight, i.e. for any \( \varepsilon > 0 \) there is a compact set \( K _ { \varepsilon } \) such that \( \mu \left[ \mathcal { X } \backslash K _ { \varepsilon } \right] \leq \varepsilon \) for all \( \mu \in \mathcal { P } \).
\end{thm}

\section{Optimal transportation}
We start with a basic definition in probability theory.

\begin{defn}[Coupling]
	Let \( ( \mathcal { X } , \mu ) \) and \( ( \mathcal { Y } , \nu ) \) be two probability spaces. Coupling \( \mu \) and \( \nu \) means constructing two random variables \( X \) and \( Y \) on some probability space \( ( \Omega , \mathbb { P } ) , \) such that law \( ( X ) = \mu \), law \( ( Y ) = \nu . \) The couple \( ( X , Y ) \) is called a coupling of \( ( \mu , \nu ) . \) By abuse of language, the law of \( ( X , Y ) \) is also called a coupling of \( ( \mu , \nu ) \).
\end{defn}

If \( \mu \) and \( \nu \) are the only laws in the problem, then without loss of generality one may choose \( \Omega = \mathcal { X } \times \mathcal { Y } . \) In a more measure-theoretical formulation, coupling \( \mu \) and \( \nu \) means constructing a measure \( \pi \) on \( \mathcal { X } \times \mathcal { Y } \) such that \( \pi \) admits \( \mu \) and \( \nu \) as marginals on \( \mathcal { X } \) and \( \mathcal { Y } \) respectively.


We then define the optimal coupling or optimal transport. Here one introduces a cost function \( c ( x , y ) \) on \( \mathcal { X } \times \mathcal { Y } , \) that can be interpreted as the work needed to move one unit of mass from location \( x \) to location \( y . \) Then one considers the Monge-Kantorovich minimization problem
\[
	\operatorname{inf}  \mathbb { E } c ( X , Y )
\]
where the pair \( ( X , Y ) \) runs over all possible couplings of \( ( \mu , \nu ) ; \) or equivalently, in terms of measures,
\[ \inf \int _ { \mathcal { X } \times \mathcal { Y } } c ( x , y ) d \pi ( x , y ) \]
where the infimum runs over all joint probability measures \( \pi \) on \( \mathcal { X } \times \mathcal { Y } \) with marginals \( \mu \) and \( \nu . \) Such joint measures are called trans- ference plans (or transport plans, or transportation plans); those achieving the infimum are called optimal transference plans.

The first good thing about optimal couplings is that they exist:
\begin{thm}[Existence of an optimal coupling]
	Let \( ( \mathcal { X } , \mu ) \) and \( ( \mathcal { Y } , \nu ) \) be two Polish probability spaces; let \( a: \mathcal { X } \rightarrow \mathbb { R } \cup \{ - \infty \} \) and \( b: \mathcal { Y } \rightarrow \mathbb { R } \cup \{ - \infty \} \) be two upper semicontinuous functions such that \( a \in L ^ { 1 } ( \mu ) , b \in L ^ { 1 } ( \nu ) . \) Let \( c: \mathcal { X } \times \mathcal { Y } \rightarrow \mathbb { R } \cup \{ + \infty \} \) be a lower semicontinuous cost function, such that \( c ( x , y ) \geq a ( x ) + b ( y ) \) for all \( x , y . \) Then there is a coupling of \( ( \mu , \nu ) \) which minimizes the total cost \( \mathbb { E } c ( X , Y ) \) among all possible couplings \( ( X , Y ) \).
\end{thm}
The proof relies on basic variational arguments involving the topology of weak convergence (i.e. imposed by bounded continuous test functions). There are two key properties to check: (a) lower semicontinuity, (b) compactness. These issues are taken care of respectively in two lemmas below.

\begin{lem}[Lower semicontinuity of the cost functional]
	\label{lower_semicontinuity_of_the_cost_functional}
	Let \( \mathcal { X } \) and \( \mathcal { Y } \) be two Polish spaces, and \( c: \mathcal { X } \times \mathcal { Y } \rightarrow \mathbb { R } \cup \{ + \infty \} \) a lower semicontinuous cost function. Let \( h: \mathcal { X } \times \mathcal { Y } \rightarrow \mathbb { R } \cup \{ - \infty \} \) be an upper semicontinuous function such that \( c \geq h . \) Let \( \left( \pi _ { k } \right) _ { k \in \mathbb { N } } \) be a sequence of
	probability measures on \( \mathcal { X } \times \mathcal { Y } , \) converging weakly to some \( \pi \in P ( \mathcal { X } \times \mathcal { Y } ) \), in such a way that \( h \in L ^ { 1 } \left( \pi _ { k } \right) , h \in L ^ { 1 } ( \pi ) , \) and
	\[ \int _ { \mathcal { X } \times \mathcal { Y } } h d \pi _ { k } \underset { k \rightarrow \infty } { \longrightarrow } \int _ { \mathcal { X } \times \mathcal { Y } } h d \pi \]
	\[ \int _ { \mathcal { X } \times \mathcal { Y } } c d \pi \leq \liminf _ { k \rightarrow \infty } \int _ { \mathcal { X } \times \mathcal { Y } } c d \pi _ { k } \]
	In particular, if \( c \) is nonnegative, then \( F: \pi \rightarrow \int c d \pi \) is lower semicontinuous on \( P ( \mathcal { X } \times \mathcal { Y } ) , \) equipped with the topology of weak convergence.
\end{lem}

\begin{proof}
 Replacing \( c \) by \( c - h , \) we may assume that \( c \) is a nonnegative lower semicontinuous function. Then \( c \) can be written as the pointwise limit of a nondecreasing family \( \left( c _ { \ell } \right) _ { \ell \in \mathbb { N } } \) of continuous real-valued functions. By monotone convergence,
	\[ \int c d \pi = \lim _ { \ell \rightarrow \infty } \int c _ { \ell } d \pi = \lim _ { \ell \rightarrow \infty } \lim _ { k \rightarrow \infty } \int c _ { \ell } d \pi _ { k } \leq \liminf _ { k \rightarrow \infty } \int c d \pi _ { k } \]
\end{proof}
\begin{lem}[Tightness of transference plans]
	Let \( \mathcal { X } \) and \( \mathcal { Y } \) be two Polish spaces. Let \( \mathcal { P } \subset P ( \mathcal { X } ) \) and \( \mathcal { Q } \subset P ( \mathcal { Y } ) \) be tight subsets of \( P ( \mathcal { X } ) \) and \( P ( \mathcal { Y } ) \) respectively. Then the set \( \Pi ( \mathcal { P } , \mathcal { Q } ) \) of all transference plans whose marginals lie in \( \mathcal { P } \) and \( \mathcal { Q } \) respectively, is itself tight in \( P ( \mathcal { X } \times \mathcal { Y } ) \).
\end{lem}

\begin{proof}
 Let \( \mu \in \mathcal { P } , \nu \in \mathcal { Q } , \) and \( \pi \in \Pi ( \mu , \nu ) . \) By assump- tion, for any \( \varepsilon > 0 \) there is a compact set \( K _ { \varepsilon } \subset \mathcal { X } , \) independent of the choice of \( \mu \) in \( \mathcal { P } \), such that \( \mu \left[ \mathcal { X } \backslash K _ { \varepsilon } \right] \leq \varepsilon ; \) and similarly there is a compact set \( L _ { \varepsilon } \subset \mathcal { Y } , \) independent of the choice of \( \nu \) in \( \mathcal { Q } , \) such that \( \nu \left[ \mathcal { Y } \backslash L _ { \varepsilon } \right] \leq \varepsilon . \) Then for any coupling \( ( X , Y ) \) of \( ( \mu , \nu ) \)
	\[ \mathbb { P } \left[ ( X , Y ) \notin K _ { \varepsilon } \times L _ { \varepsilon } \right] \leq \mathbb { P } \left[ X \notin K _ { \varepsilon } \right] + \mathbb { P } \left[ Y \notin L _ { \varepsilon } \right] \leq 2 \varepsilon. \]

	The desired result follows since this bound is independent of the coupling, and \( K _ { \varepsilon } \times L _ { \varepsilon } \) is compact in \( \mathcal { X } \times \mathcal { Y }  \).
\end{proof}

By passing to the limit in the equation for marginals, we see that \( \Pi ( \mu , \nu ) \) is closed, so it is in fact compact. Then let \( \left( \pi _ { k } \right) _ { k \in \mathbb { N } } \) be a sequence of probability measures on \( \mathcal { X } \times \mathcal { Y } \), such that \( \int c d \pi _ { k } \) converges to the infimum transport cost. Extracting a subsequence if necessary, we may assume that \( \pi _ { k } \) converges to some \( \pi \in \Pi ( \mu , \nu ) . \) The function \( h: ( x , y ) \longmapsto a ( x ) + b ( y ) \) lies in \( L ^ { 1 } \left( \pi _ { k } \right) \) and in \( L ^ { 1 } ( \pi ) \), and \( c \geq h \) by assumption; moreover, \( \int h d \pi _ { k } = \int h d \pi = \) \( \int a d \mu + \int b d \nu \); so Lemma 4.3 implies
\[ \int c d \pi \leq \liminf _ { k \rightarrow \infty } \int c d \pi _ { k }. \]

Thus $\pi$ is minimizing.
\subsection{Wasserstein space}

\begin{defn}[Wasserstein distances]
	\label{Wasserstein_distance}
	Let  \(( \mathcal { X } , d ) \) be a Polish metric space, and let \( p \in [ 1 , \infty ) . \) For any two probability measures \( \mu , \nu \) on \( \mathcal { X } , \) the Wasserstein distance of order \( p \) between \( \mu \) and \( \nu \) is defined by the formula
	\begin{align*}
		W _ { p } ( \mu , \nu ) & = \left( \inf _ { \pi \in \Pi ( \mu , \nu ) } \int _ { \mathcal { X } } d ( x , y ) ^ { p } d \pi ( x , y ) \right) ^ { 1 / p }                                                      \\
		                        & = \inf \left\{ \left[ \mathbb { E } d ( X , Y ) ^ { p } \right] ^ { \frac { 1 } { p } } , \quad \operatorname { law } ( X ) = \mu , \quad \operatorname { law } ( Y ) = \nu \right\}
	\end{align*}
\end{defn}

\begin{lem}
	\label{lower_semicontinous_Wasserstein_distance}
	Wasserstein distance is lower semi-continous with respect to weakly convergence of measures on $P(\mathscr{X})$.
\end{lem}

\begin{proof}
	This is a direct consequence of lower semicontinuity of the cost functional, see lemma \ref{lower_semicontinuity_of_the_cost_functional}.
\end{proof}

\begin{defn}[Wasserstein space]
	\label{Wasserstein_space}
	With the same conventions as in Definition \ref{Wasserstein_distance} , the Wasserstein space of order \( p \) is defined as
	\[
		P _ { p } ( \mathcal { X } ): = \left\{ \mu \in P ( \mathcal { X } ) ; \quad \int _ { \mathcal { X } } d \left( x _ { 0 } , x \right) ^ { p } \mu ( d x ) < + \infty \right\}
	\]
	where \( x _ { 0 } \in \mathcal { X } \) is arbitrary. This space does not depend on the choice of the point \( x _ { 0 } \). Then \( W _ { p } \) defines a (finite) distance on \( P _ { p } ( \mathcal { X } ) \).
\end{defn}

\begin{defn}[Weak convergence in \( P _ { p } \) ]
	Let \( ( \mathcal { X } , d ) \) be a Polish space, and \( p \in [ 1 , \infty ) \). Let \( \left( \mu _ { k } \right) _ { k \in \mathbb { N } } \) be a sequence of probability measures in \( P _ { p } ( X ) \) and let \( \mu \) be another element of \( P _ { p } ( \mathcal { X } ) . \) Then \( \left( \mu _ { k } \right) \) is said to converge weakly in \( P _ { p } ( \mathcal { X } ) \) if any one of the following equivalent properties is satisfied for some (and then any) \( x _ { 0 } \in \mathcal { X }: \)
	\begin{enumerate}
		\item \( \mu _ { k } \longrightarrow \mu \) and \( \int d \left( x _ { 0 } , x \right) ^ { p } d \mu _ { k } ( x ) \longrightarrow \int d \left( x _ { 0 } , x \right) ^ { p } d \mu ( x ) \)
		\item \( \mu _ { k } \longrightarrow \mu \) and \( \limsup _ { k \rightarrow \infty } \int d \left( x _ { 0 } , x \right) ^ { p } d \mu _ { k } ( x ) \leq \int d \left( x _ { 0 } , x \right) ^ { p } d \mu ( x ) \)
		\item \( \mu _ { k } \longrightarrow \mu \) and \( \lim _ { R \rightarrow \infty } \limsup _ { k \rightarrow \infty } \int _ { d \left( x _ { 0 } , x \right) \geq R } d \left( x _ { 0 } , x \right) ^ { p } d \mu _ { k } ( x ) = 0 \)
		\item For all continuous functions \( \varphi \) with \( | \varphi ( x ) | \leq C \left( 1 + d \left( x _ { 0 } , x \right) ^ { p } \right) \)
		      \( C \in \mathbb { R } \), one has
		      \[ \int \varphi ( x ) d \mu _ { k } ( x ) \longrightarrow \int \varphi ( x ) d \mu ( x ) \]
	\end{enumerate}
\end{defn}

\begin{thm}[$W _ { p }$  metrizes \( P _ { p } \)]
	Let \( ( \mathcal { X } , d ) \) be a Polish space, and \( p \in [ 1 , \infty ) ; \) then the Wasserstein distance \( W _ { p } \) metrizes the weak convergence in \( P _ { p } ( \mathcal { X } ) . \) In other words, if \( \left( \mu _ { k } \right) _ { k \in \mathbb { N } } \) is a sequence of measures in \( P _ { p } ( \mathcal { X } ) \) and \( \mu \) is another measure in \( P ( \mathcal { X } ) , \) then the statements
	\[ \mu _ { k } \text { converges weakly in } P _ { p } ( \mathcal { X } ) \text { to } \mu \]
	and
	\[ W _ { p } \left( \mu _ { k } , \mu \right) \longrightarrow 0 \]
	are equivalent.
\end{thm}

For topology property of Wasserstein space, we refer to Theorem 6.18 in \cite{villani2008optimal}. This result is also a nature extension of general finite support measure approximation theorem \ref{finite_support_approximation}.
\begin{thm}[Toplogy of Wasserstein space]
	\label{topology_Wasserstein}
	Let \( \mathcal { X } \) be a complete separable metric space and \( p \in [ 1 , \infty ) \). Then the Wasserstein space \( P _ { p } ( \mathcal { X } ) , \) metrized by the Wasserstein distance \( W _ { p } , \) is also a complete separable metric space. In short: The Wasserstein space over a Polish space is itself a Polish space. Moreover, any probability measure can be approximated by a sequence of probability measures with finite support.
\end{thm}

\begin{thm}[Displacement interpolation as geodesics]
	\label{geodesic_Wasserstein_space}
	Let \( ( \mathcal { X } , d ) \) be a complete separable, locally compact length space. Let \( p > 1 \) and let \( P _ { p } ( \mathcal { X } ) \) be the space of probability measures on \( \mathcal { X } \) with finite moment of order \( p , \) metrized by the Wasserstein distance \( W _ { p } . \) Then, given any two \( \mu _ { 0 } , \mu _ { 1 } \in P _ { p } ( \mathcal { X } ) , \) and a continuous curve \( \left( \mu _ { t } \right) _ { 0 \leq t \leq 1 } , \) valued in \( P ( \mathcal { X } ) , \) the following properties are equivalent:
	\begin{enumerate}
		\item \( \mu _ { t } \) is the law of \(\gamma _ { t } \) where \( \gamma \) is a random (minimizing, constantspeed) geodesic such that \( \left( \gamma _ { 0 } , \gamma _ { 1 } \right) \) is an optimal coupling;
		\item \( \left( \mu _ { t } \right) _ { 0 \leq t \leq 1 } \) is a geodesic curve in the space \( P _ { p } ( \mathcal { X } ) \)
	\end{enumerate}
	Moreover, if \( \mu _ { 0 } \) and \( \mu _ { 1 } \) are given, there exists at least one such curve.
	More generally, if \( \mathcal { K } _ { 0 } \subset P _ { p } ( \mathcal { X } ) \) and \( \mathcal { K } _ { 1 } \subset \operatorname { P } _ { p } ( \mathcal { X } ) \) are compact subsets of
	\( P ( \mathcal { X } ) , \) then the set of geodesic curves \( \left( \mu _ { t } \right) _ { 0 < t < 1 } \) such that \( \mu _ { 0 } \in \mathcal { K } _ { 0 } \) and
	\( \mu _ { 1 } \in \mathcal { K } _ { 1 } \) is compact and nonempty; and also the set of dynamical opti-
	mal transference plans \( \Pi \) with \( \left( e _ { 0 } \right) _ { \# } \Pi \in \mathcal { K } _ { 0 } , \left( e _ { 1 } \right) _ { \# } \Pi \in \mathcal { K } _ { 1 } \) is compact
	and nonempty.
\end{thm}

\part{Barycenters in Geometrical Spaces}

%! TEX root = ../barycenter.tex
\chapter{Proper metric space}

Recall that a metric space is said to be proper is every bounded closed set is compact. A proper metric space is complete and separabale, bacause every compact metric space is so and a proper metric space is a countable union of compact closed balls. With this assumption we can easily get the existence of barycenter.

\begin{lem}
	\label{lem:existence_proper_space}
	If \( ( X , d ) \) is a proper metric space, then any \( \mu \in \mathcal { P } _ { 2 } ( X ) \) has a barycenter.
\end{lem}

\begin{proof}
	Fix \( z _ { 0 } \in X \) and take \( r > 1 \) large enough to satisfy
	\[ \mu \left( B \left( z _ { 0 } , r \right) \right) \geq \frac { 1 } { 2 } , \quad \int _ { X \backslash B \left( z _ { 0 } , r \right) } d \left( z _ { 0 } , x \right) ^ { 2 } d \mu ( x ) \leq 1 \]
	Then we have
	\[ \int _ { X } d \left( z _ { 0 } , x \right) ^ { 2 } d \mu ( x ) \leq r ^ { 2 } \cdot \mu \left( B \left( z _ { 0 } , r \right) \right) + 1 \leq r ^ { 2 } + 1 \]
	while for every \( w \in X \backslash B \left( z _ { 0 } , 3 r \right) \)
	\[ \int _ { X } d ( w , x ) ^ { 2 } d \mu ( x ) \geq \int _ { B \left( z _ { 0 } , r \right) } d ( w , x ) ^ { 2 } d \mu ( x ) > ( 2 r ) ^ { 2 } \cdot \mu \left( B \left( z _ { 0 } , r \right) \right) \geq 2 r ^ { 2 } \]
	holds. Therefore it is sufficient to consider the infimum of
	\[ w \longmapsto \int _ { X } d ( w , x ) ^ { 2 } d \mu ( x ) \]
	only for \( w \in B \left( z _ { 0 } , 3 r \right) , \) and it is achieved at some point due to the compactness of the closure of \( B \left( z _ { 0 } , 3 r \right) . \)
\end{proof}

Recall that loaclly compact complete length space are proper. To see that properness is a suitable assumption, we discuss several counter examples for the existence of barycenter.

Every Riemannian manifold is an intrinsic metric space. An example of locally compact, seperable and complete but not proper metric space is the real line with metric $d(x,y)=\min(|x-y|,1)$. Note that this example in fact the Prokhorov metric, which metrizes weak convergence of Borel measures on $\mathbb{R}$, restricted to Dirac measures on $\mathbb{R}$. Interesting examples would be space of measures over metric space with some assumptions, we should study them in later chapters.
% It is not a good idea here to contruce a pathological metric that is not intrinsic for illustrating purpose.

By the contrary, we study length space $(X,d)$ in following discussion.

\begin{prop}
	For two points in a length space $(X,d)$, the barycenter of $\mu:=\frac{\delta_x + \delta_y}{2}$ must be a midpoint between $x$ and $y$. That is to say, a barycenter $z$ satisfies $d(x,z)=d(z,y)=\frac{d(x,y)}{2}$.
\end{prop}

\begin{proof}
	There are two parts to prove.

	If a midpoint $z$ for $x$ and $y$ exits, it will attain minimum of following inequlity and hence is a barycenter
	\[
		d(x,y)^2 \leq \left(d(x,z) + d(z,y)\right)^2 \leq 2\left(d(x,z)^2+ d(z,y)^2\right).
	\]

	Now assume the barycenter $z$ of $\mu:=\frac{\delta_x + \delta_y}{2}$ exists, denote $\Gamma$ the set of all rectifiable curves $\gamma: [0,1] \rightarrow X$ from $x$ to $y$ with arc-length proportional parametrization. Denote $L\gamma$ for the length of rectifiable curve $\gamma$, in our case we have $L(\gamma_{[x, \gamma_\frac{1}{2}]}) = L(\gamma_{[\gamma_\frac{1}{2}, y]})$ for $\gamma \in \Gamma$,
	\[
		d(x,z)^2 + d(z,y)^2 \leq {L(\gamma_{[x, \gamma_\frac{1}{2}]})}^2 + {L(\gamma_{[\gamma_\frac{1}{2}, y]})}^2=\frac{1}{2} {L(\gamma)}^2.
	\]
	Take inf over all $\gamma$ on the right hand side, we finally get $d(x,z)^2 + d(z,y)^2 \leq \frac{1}{2}d(x,y)^2$. This is equivalent to that $z$ is the midpoint of $x$ and $y$.
\end{proof}

\begin{rmk}[Counter-examples]
	\begin{enumerate}
		% \item Locally compact separable complete metric space is not sufficient for the existence of barycenter. For the example we gave above, $\mathbb{R}$ with metric $d(x,y)=\min (|x-y|,1)$, consider a 
		\item Locally compact length space is not sufficient for the existence of barycenter, consider unit disk without origin and uniform measure on it.
		      % \item Complete length space is not sufficient. That is the interest of last remark.
		      % \item A geodesic space $(E, \hat{d})$ always admits midpoint $z$ of two points $x$ and $y$ in $E$ as the barycenter of $\mu : = \frac{1}{2} (\delta_x + \delta_y)$, since $z$ attain equality in following general inequality:
		      % \item We have answered partially the existence of barycenter of $\mu$ in $X$ with induced length metric. Notice that barycenter of $ \mu: = 1/2 {\delta_x + \delta_y } $ for $x$ and $y$ in a length space is always the midpoints between them. Denote $\Gamma{u,v}$ the set of admissible paths join two points $u$ and $v$ in $X$
		\item Complete length space is not sufficient for the existence of barycenter. As we saw in previous propsoition, if barycenters always exists then midpoint always exists. However, in complete length space this is the same as claiming shortest path always exists, see Theorem 2.4.16 in \cite{burago2001course}. We know in addition that there exists complete but not geodesic manifold in infinite dimension.
	\end{enumerate}
\end{rmk}

\begin{lem}
	In infinite dimension, there exists a complete but not geodesic Rimannian Hilbert manifold.
\end{lem}

One such example is the infinite dimensional ellipse in Hilbert space $\ell^2$, let $c_n$ be a strictly decreasing sequence with a positive lower bound, define
\[
	X = \left\{ \left( x _ { 1 } , x _ { 2 } , \ldots \right) \in \ell^2 \mid \sum _ { n \in \mathbb { N } } \frac { x _ { n } ^ { 2 } } { c _ { n } ^ { 2 } } = 1 \right\}.
\]
This example could be found as Example 5.1 in \cite{grossman1965hilbert}.

\begin{proof}

	Take point $e=(c_1, 0,0,\ldots)$ and $-e=(-c_1, 0,0,\ldots)$. And we can actually show that $e$ and $-e$ cannot be connected by minimizing geodesic. Hilbert Riemannian manifold  theory is needed to justify the term smooth.
	% we can safely replace it with rectifiable in this example.

	Define \( T: X \rightarrow X \) by \( T x = y \), where
	\[
		y _ { 1 } = x _ { 1 } , y _ { 2 } = 0 , y _ { i } = \frac{c_i}{c_{i-1}} x_{i-1} \text { for } i \geq 3 . \]

	We aim to show that any smooth curve from \( e\) to \( -e \) is taken by \( T \) into another such curve which is strictly shorter than the original.

	To justify it, notice that length structure is defined as arc-length integral, for a smooth curve $\gamma: [0,1] \rightarrow X$ with arc-length proportional parametrization,
	\[
		L(\gamma) := \int_{0}^{1} \Vert \gamma^\prime \Vert \diff \lambda \geq \int_{0}^{1} \Vert T^\prime(\gamma^\prime) \gamma^\prime \Vert \diff \lambda =: L(T\gamma)
	\]

	Because the tangent map $T^\prime := (1,0,{c_3}/{c_2}, \ldots)$ has $\ell^\infty$ norm 1 as $c_i$ is a strictly decreasing sequence. Since $\gamma^\prime$ is continous, we remain to show that at least in an open interval we have strict inequality $\Vert \gamma^\prime \Vert > \Vert T^\prime(\gamma^\prime) \gamma^\prime \Vert$. Otherwise, $\gamma^\prime$ has only first coordinate on every open interval. In this case intergrate $\gamma^\prime$ we should get $\gamma \in \{ e, -e\}$, this is a contradiction.
\end{proof}

We can in fact prove a non-smooth version for this example, hence get rid of Riemmanian Hilbert manifold theory.

\begin{lem}
	In previous example, if consider $X$ with induced intrinsic metric from $(\ell^2, d)$, where $d$ is the standard metric on $\ell^2$, then there is again no geodesic between two endpoints $e$ and $-e$.
\end{lem}

\begin{proof}
	All curves in this proof are from $e$ to $-e$ and take arc-length parametrization on $[0,1]$.
	% Assume now we have $L_{d}(\tilde\gamma)=\hat{d}(e,-e)$, that is to say $\tilde\gamma$ is a minimizing geodesic, i.e., shortest path between $e$ and $-e$.

	We claim that arc-length variation has the same solution as energy variation:
	\begin{equation}
		\label{energy_variation_in_X}
		\arg \min_{\gamma} L\gamma := \arg \min_{\gamma} \int \Vert \gamma^\prime \Vert \diff \lambda = \arg \min_{\gamma} \int \Vert \gamma^\prime \Vert ^2 \diff \lambda
	\end{equation}
	This comes from Cauchy-Schwatz inequlity and the fact that $\Vert \gamma \Vert$ is a constant for shortest path.

	We decompose the norm $\Vert \cdot \Vert$ of $\ell^2$ into two parts, involving first $n$ coordinate components or not, $\Vert \cdot \Vert^2 = \Vert \cdot \Vert_n^2 + \Vert \cdot \Vert_r^2$ with $\Vert \cdot \Vert_n$ the norm in $\mathbb{R} ^n $. This helps us decomposite energe variation into two independent parts. Write $\gamma$ in coordinate as $(\gamma_1, \gamma_2, \ldots, \gamma_n, \ldots)$, we modify $\gamma$ leaving $\gamma_k$ for $ k > n$ unchanged. Up to choosing a bigger integer $n$ we can assume $ \exists t \in [0,1]$, such that $\gamma_n(t) \neq 0$. The first $n$ coordinate components $\gamma^n := (\gamma_1, \gamma_2 \ldots \gamma_n)$ of $\gamma$ is a function with $\gamma^n(t)$ in $E_{\gamma(t)}$ where
	\begin{align*}
		E_{\gamma(t)} : & = \left\{ (x_1,x_2 \ldots x_n) \in \mathbb{R}^n \mid \sum_{i=1}^n \frac{x_i^2}{ c_i^2} = c_{\gamma(t)}^2\right\} \\
		c_{\gamma(t)}:  & =\sqrt{ 1- \sum_{i > n} \frac{\gamma_j(t)^2}{c_j^2}} \geq 0
	\end{align*}
	Then the curve $\eta := (0,0,\ldots, c_n c_{\gamma(\cdot)},\gamma_{n+1},\ldots)$ will have lower energy variation in \cref{energy_variation_in_X} than $\gamma$ since energy variation coincides with arc-length variation on $E_\gamma(t)$ and we have $L_d(\eta^n)< L_d({\gamma}^n)$.
\end{proof}

\begin{rmk}
	% Continue the discussion of example in the review part of \cite{ohta2012barycenters}.
	Here we should discuss two different length metrics appeared, one is induced from $\ell^2$ norm and the other one is smooth Riemannian length structure. Are they two coincided (on common admissible curves)? Two ways to think about this relation
	\begin{enumerate}
		\item Apply Theorem 2.4.3 in \cite{burago2001course}, we then need to show Riemannian length structure is lower semi-continuous with respect to $\ell^2$ metric.
		\item Consider $X$ as a Hilbert submainifold of $\ell^2$, the norm for tangent space is the restriction of canonical norm of $\ell^2$.
		      % \item Now there is another way to show that $X$ with metric from $\ell^2$ is not locally compact. $X$ is complete and there is rectifiable curve connecting $e$ and $-e$, by Exercise 2.5.25 in \cite{burago2001course} there is shortest path connecting $e$ and $-e$ if $X$ is locally compact. This proposition is proofed by showing that balls of induced length metric are pre-compact in the topology of original topology.
	\end{enumerate}
\end{rmk}


% \chapter{Alexandrov spaces}
% \section{Curvature bounded above}
% \section{Curvature bounded below}
% \section{Hilbert spaces}

% \chapter{Hilbert Riemannian manifold}

\part{Barycenters in Measurable Spaces}
%! TEX root = ../barycenter.tex
\chapter{Wasserstein space over proper space}

Paper review for \emph{Existence and consistency of Wasserstein barycenters}\cite{le2017existence}.

By \cref{topology_Wasserstein}, if $(E,d)$ is a Polish space, then so is $\mathcal{W}_p(E)$. Recall that we only need Polish space to define Wasserstein metric, hence we can talk about $\mathcal{W}_p(\mathcal{W}_p(E))$, and hence study barycenter in Wasserstein space.

% If assume further $X$ is a locally compact length space, then $\mathcal{W}_p(E)$ is a strictly intrinsic length space (see Corollary 7.22 in \cite{villani2008optimal}, recalled in this report as \cref{geodesic_Wasserstein_space}) with respect to $\mathcal{W}_p(E)$. That is the case in this paper, we assume in this section that $X$ is geodesic locally compact space. Now $\mathscr{W}_p(E)$ is a Polish geodesic space, however it is not necessarily a locally compact space.

To take care of notation, we use following convention; we will recall them when necessary,
\begin{itemize}
	\item $x$ and $y$, ponts in Polish space $(E,d)$.
	\item $\mu$ and $\nu$, Borel probability measures on $(E,d)$ with finite $p$-moment, i.e., elements in $\mathcal{W}_p(X)$.
	\item $X$ and $Y$, random variables in $(E, d)$, i.e., Borel measurable functions with value in $E$.
	\item $\mathbb{P}$, Borel measures on Wasserstein space $\mathcal{W}_p(E)$.
	\item \( \hat{\mu} \), random probability measure in $\mathcal{W}_p(E)$, i.e., Borel measurable functions with value in $\mathcal{W}_p(E)$.
	\item $(\Omega, \mathcal{B}, P)$, an abstract measure space which we use as base space for random variables or measures. That is to say, we may write $X_{\#}(P)=\mu$ and $\hat{\mu}_{\#}(P)=\mathbb{P}$, where \# means push-forward of measures.
	\item $W_p$, Wasserstein distance of $\mathcal{W}_p(E)$ and $\mathcal{W}_p(\mathcal{W}_p(E))$.
	\item $W_p(\mu, E)$, distance between $\mu$ and $E$ in $\mathcal{W}_p(E)$.

	      If barycenter $\delta_x$ of $\mu$ exists, then one has $W_p(\mu, \delta_x)=W_p(\mu, E)$.
	\item $W_p(\mathbb{P}, \mathcal{W}_p(E))$, distance between $\mathbb{P}$ and $\mathcal{W}_p(E)$ in $\mathcal{W}_p(\mathcal{W}_p(E))$.

	      If barycenter $\delta_\mu$ of $\mathbb{P}$ exists, then one has $W_p(\mathbb{P}, \delta_\mu)=W_p(\mathbb{P}, \mathcal{W}_p(E))$.
	\item $\mu_n \Rightarrow \mu$ stands for weak convergence of measures.
	\item $\mu_n \rightarrow \mu$ stands for convergence in Wasserstein metric.

\end{itemize}

Compactness is crucial for the existence of barycenter. Bad news is that Wasserstein space is not locally compact unless the base space is compact (see Remark 7.1.9 in\cite{Ambrosio2005}). Good news is that weak convergence behaves well with Wasserstein metric and we can get weak compactness more easily.

Properness of base space gives properness of Wasserstein space for weak convergence topology. We remark that a proper metric space is Polish.

\begin{prop}
	For a proper space $(E,d)$, every bounded set in $\mathcal{W}(E)$ is tight, hence pre-compact in weak convergence topology.
\end{prop}
\begin{proof}
	By Markov inequality, $\mu(E\backslash B(x,r)) \leq W_p(\mu, \delta_x)^p/r^p$, measure of closed ball (compact by assumption) of sufficient large radius can be arbitrarily close to $1$.
\end{proof}

% Apply the definition of barycenter, condition in this proposition will be satisfied by lower semi-continuity of Wasserstein distance.
This simple proposition is enough to build existence of barycenter.

By abuse of language, we drop $\delta$ symbol in following discussion when there is no ambiguity.
\begin{thm}
	Let $(E,d)$ be a proper space. Assume $\mathbb{P}_n \in \mathcal{W}_p(\mathcal{W}_p(E))$ is a sequence with barycenter $\mu_n$ and $\mathbb{P}_n \rightarrow \mathbb{P}$ with respect to Wasserstein metric. Then $\mu_n$ is tight and any weak limit $\mu$ of $\mu_n$ will be a barycenter of $\mathbb{P}$.
\end{thm}

\begin{proof}
	Existence of $\mu$ as a weak limit comes from boundness of $\mu_n$.
	To show that barycenters of bounded set is bounded, by abuse of notation, for a $x$ in $E$,
	\[
		W_p(\mu_n, x) \leq W_p(\mu_n, \mathbb{P}_n)  + W_p(\mathbb{P}_n, x) \leq 2 W_p(\mathbb{P}_n , x),
	\]
	we used definition of barycenter in the second inequality.
	And the last item is bounded as $\mathbb{P}_n$ is a converging sequence.

	By continuity of Wasserstein distance, we have
	\[
		W_p(\mathbb{P}, \mathcal{W}_p(E)) = \lim_{n \rightarrow \infty}W_p(\mathbb{P}_n, \mathcal{W}_p(E))=\lim_{n \rightarrow \infty}W_p(\mathbb{P}_n, \mu_n)
	\]
	By lower semi-continuity of Wasserstein distance, we have
	\[
		\lim_{n \rightarrow \infty}W_p(\mathbb{P}_n, \mu_n)=\lim_{n \rightarrow \infty}W_p(\mathbb{P}, \mu_n)\geq W_p(\mathbb{P}, \mu)
	\]
	Hence, $W_p(\mathbb{P}, \mathcal{W}_p(E)) =W_p(\mathbb{P}, \mu)$.
\end{proof}

As weak convergence is near to Wasserstein convergence, we can see from following proposition that previous weak convergence is in fact convergence in Wasserstein metric.

\begin{prop}
	For a weak convergence sequence $\mu_n \Rightarrow \mu$, the following condition are sufficient and necessary (hence equivalent) to have $\mu_n \rightarrow \mu$ convergence in Wasserstein metric.
	\begin{enumerate}
		\item $W_p(\mu_n, \delta_x) \rightarrow W_p(\mu, \delta_x)$ for an element $x$ in $E$.
		\item $W_p(\mu_n, \nu) \rightarrow W_p(\mu, \nu)$ for an element $\nu$ in $\mathcal{W}_p(E)$.
		\item $W_p(\delta_{\mu_n}, \mathbb{P}) \rightarrow W_p(\delta_\mu, \mathbb{P})$ for an element $\mathbb{P}$ in $\mathcal{W}_p(\mathcal{W}_p(E))$.
	\end{enumerate}
\end{prop}

That is say, what we need in addition is the convergence of Wasserstein distance with respect to a point either in $E$, $\mathcal{W}_p(E)$ or $\mathcal{W}_p(\mathcal{W}_p(E))$.

\begin{proof}
	The first point is stated in \cref{thm:Wp_metrizes_weak_convergence}. The second point is proved in this paper as Lemma 14. The last one is from the second one.

	Take $\hat{\mu}$ a random probability measure with law $\mathbb{P}$, by lower semi-continuity of Wasserstein metric with respect to weak convergence we have
	\begin{align*}
		\mathbb{E}W_p(\mu, \hat{\mu}) & =W_p(\mu, \mathbb{P})=\lim_{n \rightarrow \infty} W_p(\mu_n, \mathbb{P}) \\
		                              & \geq \mathbb{E}\liminf_{n \rightarrow \infty} W_p(\mu_n, \hat{\mu})      \\
		                              & \geq \mathbb{E}W_p(\mu, \hat{\mu}).
	\end{align*}
	The first inequality comes from Fatou's lemma. Now these two inequalities are in fact equalities. An subsequence of $\mu_n$ will satisfy this equality as well. Hence we can prove by contradiction that $\liminf$ above can be safely substitued by $\lim$.
	Therefore, we have almost everywhere that $\lim_{n \rightarrow \infty} W_p(\mu_n, \hat{\mu})=W_p(\mu, \hat{\mu})$.
\end{proof}

As finite support measure is dense in Wasserstein space, the final goal should be the existence of barycenter for finite support measure in $\mathcal{W}_p(E)$.

\begin{thm}
	\label{thm:barycenter_finite_support_measure}
	Let $(E,d)$ be a proper space. Then barycenter of finite support measure on $\mathcal{W}_p(E)$ always exists.
\end{thm}

This is a standard convex optimization problem as $\mathcal{W}_p(E)$ is a convex space (with linear convex structure) and distance function is convex.

\begin{proof}
	Let $\lambda_i$ be $n$ given positive coefficients with sum $1$, $\mu_i$ be $n$ given measures on $(E,d)$. We aims to find solution for optimization problem
	\[
		\min_{\nu} \sum_{i=1}^{n}\lambda_i W_p(\nu, \mu_i)^p.
	\]
	For a given $\nu$, let $(X, X_1,\ldots,X_n)$ be a choice of random variable with value in $E^{n+1}$ such that $(X,X_i)$ is an optimal coupling for $\nu$ and $\mu_i$. We thus have $\mathbb{E}d(X,X_i)^p = W_p(\nu, \mu_i)^p$.

	Denote $\boldsymbol{x}=(x_1, x_2, \ldots, x_n) \in E^n$ and define $f(\boldsymbol{x})= W_p(\eta, E)^p$, where $\eta = \sum_{i=1}^{n} \lambda_i \delta_{x_i}$. We use ``min'' in this definition as barycenter of measures in $E$ always exists. Denote $\Gamma$ the set of all possible choice of $(X_1, \ldots, X_n)$, we have
	\begin{align*}
		\sum_{i=1}^{n}\lambda_i W_p(\nu, \mu_i)^p & = \mathbb{E} \sum_{i=1}^{n}\lambda_i d(X,X_i)^p \\
		                                          & \geq \mathbb{E} f(X_1, \ldots, X_n)             \\
		                                          & \geq \inf_\Gamma \mathbb{E} f(X_1, \ldots, X_n)
	\end{align*}
	These two inequalities can be in fact equalities.

	For the second one, the existence of solution to this minimization problem on $\Gamma$ is a multi-marginal optimal transportation problem with cost funtion $f$.

	Inspired by the proof of existence of optimal plan, we should first show that $\Gamma$ is weakly compact. $\Gamma$ is tight as elements in it have pre-fixed marginal distriutions. $\Gamma$ is weakly closed by stability of optimal plans, which is guaranteed by lower semi-continuity of transport cost with respect to weak convergence.
	% To prove stability, we need the dual formulation of optimal transportation and the fact that a plan is optimal if and only if it is concentrated on a cyclical monotone set, see Theorem 5.12 in \cite{villani2008optimal}.

	What we need to show in addition is that $f$ is indeed lower semi-continous. Set $\eta:= \sum_i^{n}\lambda_i \delta_{x_i} \in \mathcal{W}_p(E)$, observe that sequence convergence of $\boldsymbol{x}$ in $E^n$ corresponds to weak convergence of $\eta$. And we see that $f(\boldsymbol{x}) = W_p(\eta, E)^p=W_p(\eta, y)^p = \sum_{1}^{n} \lambda_i d(y, x_i)^p$ for some barycenter $y$ of $\eta$ in $E$. Let $\eta_j$ be a sequence of measures of such form corresponding to $\boldsymbol{x}_j$ and let $y_j$ be a barycenter of $\eta_j$.

	If we can have $y_j \rightarrow y$ up to a subsequence for some $y$, then by continuity of distance fucntion and $\boldsymbol{x}_j \rightarrow \boldsymbol{x}$
	\[
		\lim W_p(\eta_j, E) = \lim W_p(\eta_j, y_j) =  W_p(\eta, y).
	\]

	It is necessary that $y$ is a barycenter of $\eta$ as a limit of $y_j$, since for a barycenter $z$ of $\eta$ we can pass to the limit in $W_p(\eta_i, z) \geq W_p(\eta_i, E)$.

	% As $E$ is proper, our primary aim is to show that $y_i$ is bounded in $E$.

	Recall that bounded set in Wasserstein space is weakly pre-compact and weakly convergence of Dirac measures has the same topology as base space. Hence we only need to show that $y_i$ is bounded is the Wasserstein space $\mathcal{W}_p(E)$. This comes from that $\eta_j$ is bounded and barycenters of bounded set is bounded as we proved before.
	This shows that $f(\boldsymbol{x})$ is continous with respect to $\boldsymbol{x}$ since
	any sequence of $f(\boldsymbol{x_i})$ has a subsequence converging to $f(\boldsymbol{x})$.

	For the first inequality, we need to show the existence of a measurable function $B(\boldsymbol{x}) \in \arg \min_{x \in E} \sum_{i=1}^{n} \lambda_i d(x, x_i)^p$. This is guaranteed by measurable selection theorem. The set
	\[
	\Gamma:=	\{
		(y,\boldsymbol{x}) \in E^{n+1}\mid  f(\boldsymbol{x}) - \sum_{i=1}^{n} \lambda_i d(y,x_i)^p = 0
		\}
	\]
	is closed as a pre-image of continous function.
	% is measurable as lower semi-continous function $f$ is a limit of continous function hence Borel measurable.
% In fact, this set is closed because lower level sets for lower semi-continous function are closed
% and all arguments that attains minimun vaule is a level set.
We claim that the sliced set $\Gamma_{\boldsymbol{x}}:=\{y \mid (y, \boldsymbol{x}) \in \Gamma\}$ is compact,
% is a closed subset of $E$, so it is $\sigma$-compact.
because barycenters are located in the union of $n$ bounded balls with centers $x_i$.
We conculde the existence of measurable seletion function $B$ by \cref{thm:measurale_selection}.

	For our proof, to be concrete, let $\boldsymbol \gamma$ be a solution to the second inequality.
	Then $\mu:= B_{\#}\boldsymbol \gamma$ will be a solution to our optimization problem, i.e., barycenter of $\sum_{i=1}^{n}\lambda_i \delta_{\mu_i}$.
\end{proof}

\begin{rmk}
	If one $\mu_i$ is absolutely continous respect to volume measure on complete manifold $E$, then the barycenter $\nu$ is unqiue.
	We can argue the uniquness of barycenter $\mu:= B_{\#}\boldsymbol \gamma$ by classical Brenier theorem in the case of $\mathbb{R}^n$.
	We need to use duality argument, see Definition 3.6 in \cite{agueh2011barycenters} and this is done in \cite{kim2015multi} for the case of compact Riemannian manifold.
\end{rmk}

%! TEX root = ../barycenter.tex
\chapter{Wasserstein space over Riemannian manifold}
\section{Convex analysis on Riemannian manifold}
\subsection{Polar factorization of maps on manifolds}

\begin{lem} [Lipschitz cost]
	\label{lem:Lipschitz_cost}
	Let \( ( M , d ) \) be a metric space whose diameter \( | M | : = \sup \{ d ( x , z ) \mid x , z \in M \} \) is finite.
	For each \( y \in M \), the function \( \psi ( x ) = d ^ { 2 } ( x , y ) / 2 \) is Lipschitz continuous:
	\begin{equation}
		\label{equa:Lipschitz}
		| \psi ( x ) - \psi ( z ) | \leq | M | d ( x , z ).
	\end{equation}
\end{lem}

\begin{proof}
	The triangle inequality shows that \( \phi ( x ) : = d ( x , y ) \) has Lipschitz constant one:
	\[ \phi ( x ) - \phi ( z ) = d ( x , y ) - d ( z , y ) \leq d ( x , z ) \]
	for all \( x , z \in M\).
	Also, \( \phi ( x ) = d ( x , y ) \leq | M | < \infty \) is bounded.
	The desired estimate \cref{equa:Lipschitz} then follows easily for \( \psi ( x ) = \phi ^ { 2 } ( x ) / 2 \) :
	\begin{align*}
		2 | \psi ( x ) - \psi ( z ) | & = | \phi ( x ) ( \phi ( x ) - \phi ( z ) ) + \phi ( z ) ( \phi ( x ) - \phi ( z ) ) | \\
		                              & \leq | M | d ( x , z ) + | M | d ( x , z )
	\end{align*}
\end{proof}

\begin{lem} [Infimal convolutions are Lipschitz]
	\label{lem:infimal_convolution_Lipschitz}
	Fix a metric space \( ( M , d ) \) having finite diameter.
	Any \( \psi : M \rightarrow \mathbf { R } \cup \{ \pm \infty \} \) given by an infimal convolution \( \psi = \psi ^ { c c } \) with \( c ( x , y ) = d ^ { 2 } ( x , y ) / 2 \) is either identically infinite \( \psi = \pm \infty \) or Lipschitz continuous throughout \( M\).
	Indeed, it satisfies \cref{equa:Lipschitz}.
\end{lem}

\begin{proof}
	More generally, suppose \( \psi = \phi ^ { c } \) for some \( \phi : M \rightarrow \mathbf { R } \cup \{ \pm \infty \} \) meaning
	\begin{equation}
		\label{equa:infimal_convolution}
		\psi ( x ) = \inf _ { y \in M } c ( x , y ) - \phi ( y )
	\end{equation}
	Observe \( 0 \leq c ( x , y ) \leq | M | ^ { 2 } / 2 \) is bounded.
	Either \( \phi \) is unbounded above, in which case \( ( 11 ) \) yields \( \psi = - \infty \) and the lemma holds trivially, or else \( \psi \) is bounded below.
	Fix \( z \in M \), and note \( \psi ( z ) = + \infty \) in \cref{equa:infimal_convolution} occurs only if \( \phi : = - \infty \) everywhere, in which case \( \psi = + \infty \) again holds trivially.
	Thus we may assume that \( \psi \) is finite everywhere.
	Given any \( \epsilon > 0 \), there exists \( y \in M \)
	such that \( \psi ( z ) + \epsilon \geq c ( z , y ) - \phi ( y ) \), while \( \psi ( x ) \leq c ( x , y ) - \phi ( y ) \) holds because of \cref{equa:infimal_convolution}.
	Subtracting these two inequalities yields
	\begin{align*}
		\psi ( x ) - \psi ( z ) & \leq c ( x , y ) - c ( z , y ) + \epsilon \\
		                        & \leq | M | d ( x , z ) + \epsilon
	\end{align*}
	by \cref{lem:Lipschitz_cost}. Since the last inequality holds for all \( \epsilon > 0 \), the Lipschitz
	estimate \cref{equa:Lipschitz} has been proved.
\end{proof}

One reason to study square distance on manifold could be that under Taylor expansion, it is directlt related to inner product of Riemannian metric.
An example is to prove superdifferentiability of geodesic distance squared in \cite{mccann2001polar}.

\begin{prop}
	[Superdifferentiability of geodesic distance squared]
	Let \( ( M , g ) \) be a \( C ^ { 3 } \)-smooth Riemannian manifold, possibly with boundary.
	Suppose \( \sigma : [ 0,1 ] \rightarrow M \) has minimal length among piecewise \( C ^ { 1 } \) curves joining \( y = \sigma ( 0 ) \) to \( x = \sigma ( 1 ) \notin \partial M \), parameterized with constant speed.
	Then \( \psi ( \cdot ) = d ^ { 2 } ( \cdot , y ) / 2 \) has supergradient \( \dot { \sigma } ( 1 ) \in \bar { \partial } \psi _ { x } \) at \( x \).
\end{prop}

\begin{proof}
	Since \( x \) lies in the interior of \( M \), there is some \( \epsilon > 0 \) and neighbourhood \( X \subset M \) of \( x \) such that: at each \( z \in X \), the exponential map exp \( _ { z } \) maps
	the ball \( \mathbf { B } ( \mathbf { 0 } , \epsilon ) \subset T M _ { z } \) diffeomorphically onto some open set \( U _ { z } \supset X \), as in Milnor.
	The proposition will first be established when
	\( y = \sigma ( 0 ) \in X \), in which case \( \psi \) is actually differentiable at \( \exp _ { y } \dot { \sigma } ( 0 ) = x \)
	We compute its derivative by linearizing exp \( _ { x } \mathbf { v } \in X \) around the origin and $\exp _ { y }$ around $ \dot { \sigma } ( 0 )$:
	\begin{align*}
		\psi \left( \exp _ { x } \mathbf { v } \right) & = d ^ { 2 } \left( y , \exp _ { y } \left( \exp _ { y } ^ { - 1 } \exp _ { x } \mathbf { v } \right) \right) / 2 \\ & = \left| \exp _ { y } ^ { - 1 } \left( \exp _ { x } \mathbf { v } \right) \right| _ { y } ^ { 2 } / 2 \\ & = \left| \dot { \sigma } ( 0 ) + D \left( \exp _ { y } ^ { - 1 } \right) _ { x } D \left( \exp _ { x } \right) _ { 0 } \mathbf { v } + o \left( | \mathbf { v } | _ { x } \right) \right| _ { y } ^ { 2 } / 2 \\ & = | \dot { \sigma } ( 0 ) | _ { y } ^ { 2 } / 2 + g \left\langle \dot { \sigma } ( 0 ) , \left( D \exp _ { y } \right) _ { \dot { \sigma } ( 0 ) } ^ { - 1 } I \mathbf { v } \right\rangle _ { y } + o \left( | \mathbf { v } | _ { x } \right) \\ & = d ^ { 2 } ( x , y ) / 2 + g \langle \dot { \sigma } ( 1 ) , \mathbf { v } \rangle _ { x } + o \left( | \mathbf { v } | _ { x } \right)
	\end{align*}
	so that \( \nabla \psi ( x ) = \dot { \sigma } ( 1 )\).

	Here the last equation follows from \( \dot { \sigma } ( 1 ) = \)
	\( D \left( \exp _ { y } \right) _ { \dot { \sigma } ( 0 ) } \dot { \sigma } ( 0 ) \) and Gauss' lemma.
\end{proof}

\subsection{A Riemannian interpolation inequality à la Borell, Brascamp and Lieb}

We copy following results (tex code) from \cite{cordero2001riemannian}.

\begin{prop}
	[Distances fail to be semiconvex at the cut locus]
	\label{prop:distance_cut_locus}
	At each \( x \in \operatorname { cut } ( y ) \), the square distance \( \psi : = d _ { y } ^ { 2 } / 2 \) satisfies:
	\[ \inf _ { 0 < | v | < 1 } \frac { \psi \left( \exp _ { x } v \right) + \psi \left( \exp _ { x } - v \right) - 2 \psi ( x ) } { | v | ^ { 2 } } = - \infty \]
\end{prop}
\begin{defn} [$c$-transforms and the subset \( \mathcal { I } ^ { c } ( X , Y ) \) of \( c \)-concave functions]
	Let \( X \) and \( Y \) be two compact subsets of \( M \). The set \( \mathcal{I} ^ { c } ( X , Y ) \) of \( c \) -concave functions (relative to \( X \) and \( Y \) ) is the set of functions \( \phi \) : \( X \rightarrow \mathbf { R } \cup \{ - \infty \} \) not identically \( - \infty \), for which there exists a function \( \psi : Y \rightarrow \mathbf { R } \cup \{ - \infty \} \) such that
	\begin{equation}
		\label{defn:c_transform}
		\phi ( x ) = \inf _ { y \in Y } c ( x , y ) - \psi ( y ) \quad \forall x \in X.
	\end{equation}
	We refer to \( \phi \) as the \( c \)-transform of \( \psi \) and abbreviate \cref{defn:c_transform} by writing \( \phi = \psi ^ { c } \)
	Similarly, given \( \phi \in I ^ { c } ( X , Y ) \), we define its \( c \) -transform \( \phi ^ { c } \in \mathcal{I} ^ { c } ( Y , X ) \) by
	\[ \phi ^ { c } ( y ) : = \inf _ { x \in X } c ( x , y ) - \phi ( x ) \quad \forall y \in Y. \]
\end{defn}

\begin{thm}[Optimal mass transport on manifolds]
	Let \( M \) be a complete, continuously curved Riemannian manifold. Fix two Borel probability measures \( \mu \ll \) vol and \( v \) on \( M \), and two compact subsets \( X \) and \( Y \subset M \) containing the support of \( \mu \) and \( v \), respectively. Then there exists \( \phi \in \mathcal { I } ^ { c } ( X , Y ) \) such that the map
	\begin{equation}
		\label{equa:transform_map}
		F ( x ) : = \exp _ { x } ( - \nabla \phi ( x ) )
	\end{equation}
	pushes \( \mu \) forward to \( v \). This map is uniquely characterized among all maps pushing \( \mu \) forward \( v \) by formula \cref{equa:transform_map} with \( \phi \in \mathcal{I} ^ { c } ( X , Y ) . \) Furthermore \( F \) is the unique minimizer of the quadratic cost \( \int d ^ { 2 } ( x , G ( x ) ) d \mu ( x ) \) among all Borel maps \( G : M \rightarrow M \) pushing \( \mu \) forward to \( v \) (apart from variations on sets of \( \mu \)-measure zero).
\end{thm}

The map \( F \) may be referred to either as the optimal map or optimal mass
transport between \( \mu \) and \( v \).

Let us also recall one of the basic lemmas from its proof, which illumi-
nates the structure of the map \( F \). Given two compact subsets \( X \) and \( Y \subset M \)
with \( \phi \in \mathcal { I } ^ { c } ( X , Y ) \), one sees every \( ( x , y ) \in X \times Y \) satisfy
\begin{equation}
	\label{equa:c-concave_conjugate}
	c ( x , y ) - \phi ( x ) - \phi ^ { c } ( y ) \geq 0
\end{equation}
with equality when \( \phi ( x ) = \inf _ { y ^ { \prime } \in Y } c \left( x , y ^ { \prime } \right) - \phi ^ { c } \left( y ^ { \prime } \right) = c ( x , y ) - \phi ^ { c } ( y ) . \)

\begin{lem}
	[Elementary properties of \( c \) -concave functions]
	\label{lem:minimizer_differentiable}
	Fix \( x \subset \subset M \) open and \( Y \subset M \) compact. For \( \phi \in I ^ { c } ( \bar { X } , Y ) \) define \( F ( x ) : = \)
	\( \exp _ { x } ( - \nabla \phi ( x ) ) \).
	\begin{enumerate}
		\item The function \( \phi \) is Lipschitz on \( \bar { X } \) and hence differentiable almost every-
		      where on \( X \).
		\item Fix any point \( x \in X \) where \( \phi \) is differentiable. Then \( y = F ( x ) \) if and
		      only if \( y \) minimizes \cref{equa:c-concave_conjugate} among \( y ^ { \prime } \in Y . \) In the latter case one has
		      \( \nabla \phi ( x ) = \nabla d _ { y } ^ { 2 } ( x ) / 2 \).
	\end{enumerate}
\end{lem}

\begin{defn}
	[$c$-superdifferential \( \partial ^ { c } \phi \)]
	Let \( X , Y \) be two compact sets of M. For \( \phi \in \mathcal { I } ^ { c } ( X , Y ) \)
	and \( x \in X \), the \( c \)-superdifferential of \( \phi \) at \( x \) is the non-empty set
	\begin{align}
		\partial ^ { c } \phi ( x ) & : = \left\{ y \in Y \mid \phi ( x ) + \phi ^ { c } ( y ) = c ( x , y ) \right\}                   \\
		                            & = \{ y \in Y \mid \phi ( z ) \leq \phi ( x ) + c ( z , y ) - c ( x , y ) \quad \forall z \in X \}
		\label{equa:c-superdifferential}
	\end{align}
\end{defn}

\begin{example} [Multivalued extension]
	\label{example:minimizer_differentiable}
	If \( \phi \in \mathcal { I } ^ { c } ( \bar { X } , Y ) \) is differentiable at
	\( x \in \mathcal { X } \subset \subset \),
	then \( \partial ^ { c } \phi ( x ) = \{ F ( x ) \} = \left\{ \exp _ { x } ( - \nabla \phi ( x ) ) \right\} \) according to \cref{lem:minimizer_differentiable}.
\end{example}

First recall that a geodesic ball \( B _ { r } ( x ) \) of radius \( r \) around \( x \in M \) is said to be embedded if the exponential map \( \exp _ { x } : \tilde { B } _ { r } ^ { x } ( 0 ) \rightarrow B _ { r } ( x ) \) defines a diffeomorphism from the open ball \( \tilde { B } _ { r } ^ { x } ( 0 ) \subset T _ { x } M \) onto \( B _ { r } ( x ) \subset M \).
A geodesic ball \( B _ { r } ( x ) \) around \( x \) is a convex embedded ball if it is embedded and geodesically convex---meaning every pair of points \( y , z \in B _ { r } ( x ) \) are joined by a unique geodesic of length less than \( 2 r \), and this geodesic is contained in \( B _ { r } ( x )\).
Small enough balls are always convex embedded balls.

\begin{defn}[Semi-concavity]
	\label{defn:semi-concavity}
	Fix \( \Omega \subset M \) open. A function \( \phi : \Omega \rightarrow \mathbf { R } \) is semi-concave at \( x _ { 0 } \in \Omega \) if there exists a convex embedded ball \( B _ { r } \left( x _ { 0 } \right) \) and a smooth function \( V : B _ { r } \left( x _ { 0 } \right) \rightarrow \mathbf { R } \) such that \( \phi + V \) is geodesically concave throughout \( B _ { r } \left( x _ { 0 } \right) . \) The function \( \phi \) is semi-concave on \( \Omega \) if it is semi-concave at each point of \( \Omega \).
\end{defn}

\begin{defn}[Hessian]
	Let \( \phi : \Omega \rightarrow \mathbf { R } \) be semi-concave on an open set \( \Omega \subset M . \) We say that \( \phi \) has \( a \) Hessian \( H \) at \( x \in \Omega \) if \( \phi \) is differentiable at \( x \)
	and there exists a self-adjoint operator \( H : T _ { x } M \rightarrow T _ { x } M \) satisfying
	\begin{equation}
		\label{defn:hessian}
		\sup _ { v \in \partial \phi \left( \exp _ { x } u \right) } \left| \Pi _ { x , u } v - \nabla \phi ( x ) - H u \right| = o ( | u | )
	\end{equation}
	as \( u \rightarrow 0 \) in \( T _ { x } \) M. Here \( \Pi _ { x , u } : T _ { \exp _ { x } u } M \rightarrow T _ { x } M \) denotes parallel translation to \( x \) along \( \gamma ( t ) : = \exp _ { x } ( t u ) . \) The Hessian of \( \phi \) at \( x \) may also be denoted
	by \( \operatorname { Hess } _ { x } \phi : = H \).
\end{defn}

This definition coincides with the usual one for smooth functions. A more intuitive understanding of the Hessian follows from the fact that existence of a Hessian \( H \) at \( x \) for \( \phi \) implies a second order Taylor expansion for \( \phi \) around \( x : \) as \( u \rightarrow 0 \in T _ { x } M \),
\begin{equation}
	\label{equa:hessian_expan}
	\phi \left( \exp _ { x } u \right) = \phi ( x ) + \langle \nabla \phi ( x ) , u \rangle + \frac { 1 } { 2 } \langle H u , u \rangle + o \left( | u | ^ { 2 } \right)
\end{equation}

It is remarkable that the converse also holds true: if \( \psi \) is semi-concave around \( x \) then \cref{defn:hessian} follows from \cref{equa:hessian_expan}.

\begin{prop}[\( c \)-concave functions are semi-concave]
	\( F i x X \subset \subset M \) open and \( Y \subset M \) compact. A \( c \)-concave function \( \phi \in \mathcal{I} ^ { c } ( \bar { X } , Y ) \) is semi-concave on \( X \) (and hence admits a Hessian \cref{defn:hessian} almost everywhere in \( X \)).
\end{prop}

\begin{thm}[Aleksandrov-Bangert, \cite{bangert1979analytiche} in German]
	Let \( \phi : \Omega \rightarrow M \) be semi-concave function on an open set \( \Omega \subset M . \) Then \( \phi \) admits a Hessian almost everywhere on \( \Omega \).
\end{thm}

This theorem is also proved as Theorem 14.1 in \cite{villani2008optimal}.
The notion of local semiconvexity
with quadratic modulus is invariant by $C^2$ diffeomorphism, so it suffices to prove for $\mathbb{R}^n$.


\begin{prop} [Differentiating optimal transport]
	\label{prop:differentiate_optimal_transport}
	Fix \( X \subset \subset M \) be openand \( Y \subset M \) compact. Let \( \phi \in I ^ { c } ( \bar { X } , Y ) \) and set \( F ( z ) : = \exp _ { z } ( - \nabla \phi ( z ) ) . \)
	Fix a point \( x \in X \) where \( \phi \) admits a Hessian \cref{defn:hessian}.
	Then:
	\begin{enumerate}
		\item \( y : = F ( x ) \notin \operatorname { cut } ( x ) \) and setting \( H : = \operatorname { Hess } _ { x } d _ { y } ^ { 2 } / 2 \), one has \( H - \operatorname { Hess } _ { x } \phi \) \( \geq 0 \)
		\item Introduce \( Y : = d \left( \exp _ { x } \right) _ { - \nabla \phi ( x ) } \) and define \( d F _ { x } : T _ { x } M \longrightarrow T _ { y } M \) by
\( d F _ { x } : = Y \left( H - \operatorname { Hess } _ { x } \phi \right) \). Then as \( u \rightarrow 0 \) in \( T _ { x } M \),
\begin{equation} 
	\sup _ {\substack {\exp _ { y } v \in \partial ^ { c } \phi \left( \exp _ { x } u \right) \\ | v | = d \left( y , \exp _ { y } v \right)} } \left| v - d F _ { x } ( u ) \right| = o ( | u | ) 
\end{equation}

	\end{enumerate}
\end{prop}

We only copy proof for the first conclusion, see Proposition 4.1 in \cite{cordero2001riemannian} for full detail.
\begin{proof}
	Suppose \( \phi \) admits a Hessian \cref{defn:hessian} at \( x \in X \).
	Then \( \phi \) is differentiable at \( x \) and
	\cref{example:minimizer_differentiable} shows that \( \partial ^ { c } \phi ( x ) = \{ F ( x ) \} = \{ y \} \).
	Thus for every \( z \in \mathcal { X } \), \cref{equa:c-superdifferential} yields
	\begin{equation}
		\label{equa:c-concave_distance_compare}
		\phi ( z ) \leq \phi ( x ) + d _ { y } ^ { 2 } ( z ) / 2 - d _ { y } ^ { 2 } ( x ) / 2 \end{equation}
	Taking \( z = \exp _ { x } ( \pm u ) \) and \( \psi : = d _ { y } ^ { 2 } / 2 \) gives
	\[ \frac { \phi \left( \exp _ { x } u \right) + \phi \left( \exp _ { x } - u \right) - 2 \phi ( x ) } { | u | ^ { 2 } } \leq \frac { \psi \left( \exp _ { x } u \right) + \psi \left( \exp _ { x } - u \right) - 2 \psi ( x ) } { | u | ^ { 2 } } \]
	As \( | u | \rightarrow 0 \) the left hand side tends to \( \left\langle \operatorname { Hess } _ { x } \phi ( u ) , u \right\rangle \) by hypothesis, so the right hand side is bounded below.
	Then \cref{prop:distance_cut_locus} ensures that
	\( x \notin \operatorname { cut } ( y ) \), or equivalently \( y \notin c u t ( x ) \).
	From \cref{equa:c-concave_distance_compare} we also observe that the function
	\[ h ( z ) : = d _ { y } ^ { 2 } ( z ) / 2 - \phi ( z ) \]
	has a minimum at \( z = x \). The Taylor expansion \cref{equa:hessian_expan} then implies the
	existence and non-negativity of its Hessian: Hess\(_{ x } h = H - \) Hess\(_ { x } \phi \geq 0\).
\end{proof}

From the proof of Lemma 3.11 in \cite{cordero2001riemannian}, one can actually choose the local smooth function in \cref{defn:semi-concavity} as square distance function.
\section{Paper review}

This is review of \cite{KIM2017640},
Wasserstein barycenters over Riemannian manifolds.

\subsection{Remind of basics in Riemannian geometry}

For \( c ( x , y ) = \frac{ 1 } { 2 } d ^ { 2 } ( x , y ) \), to show \( - D _ { x } c ( x , y ) = \exp _ { x } ^ { - 1 } ( y ) \), we should use exponetial coordinate at $T_yM$.
By Gauss lemma, we have \( \nabla r = \partial _ { r } \).
As $d c = r dr$, $\nabla c = r \partial r$ and our conculsion follows from that $ r \partial r $ is of length $r$.

It is instructive to recall proof of Gauss lemma in \cite{Petersen2016}.
On \( U = \exp _ { p } ( B ( 0 , \varepsilon ) ) \) we define the function \( r ( x ) = \left| \exp _ { p } ^ { - 1 } ( x ) \right| . \)
That is, \( r \) is simply the Euclidean distance function from the origin on \( B ( 0 , \varepsilon ) \subset T _ { p } M \) in exponential coordinates.
This function can be continuously extended to \( U \) by defining \( r ( \partial U ) = \varepsilon . \)

We know that \( \nabla r = \partial _ { r } = \frac { 1 } { r } x ^ { i } \partial _ { i } \) in Cartesian coordinates
on \( T _ { p } M . \)
We show that this is also the gradient with respect to the general metric \( g . \)

\begin{lem}
	[The Gauss Lemma]
	On \( ( U , g ) \) the function \( r \) has gradient \( \nabla r = \partial _ { r } \), where \( \partial _ { r } = D \exp _ { p } \left( \partial _ { r } \right) \).
\end{lem}

\begin{proof}
	We select an orthonormal basis for \( T _ { p } M \) and introduce
	Cartesian coordinates.
	These coordinates are then also used on \( U \) via the exponential map.
	Denote these coordinates by \( \left( x ^ { 1 } , \ldots , x ^ { n } \right) \) and the coordinate vector fields by
	\( \partial _ { 1 } , \ldots , \partial _ { n } . \)
	Then
	\begin{align*}
		r ^ { 2 }        & = \left( x ^ { 1 } \right) ^ { 2 } + \cdots + \left( x ^ { n } \right) ^ { 2 } , \\
		\partial _ { r } & = \frac { 1 } { r } x ^ { i } \partial _ { i }.
	\end{align*}
	For this, take a function $f: M \mapsto \mathbb{R}$, we have $ \frac{\partial f}{\partial x_i}=\frac{\partial f}{\partial r}\cdot \frac{\partial r}{\partial x_i}$.
	Differentiate $ r ^ { 2 } = \left( x ^ { 1 } \right) ^ { 2 } + \cdots + \left( x ^ { n } \right) ^ { 2 } $, we get $\frac{\partial r}{\partial x_i} = \frac{x_i}{r} $.
	Apply this equality agian, we can solve $\frac{\partial f}{\partial r}$ from $\frac{\partial f}{\partial x_i}$.

	To show that this is the gradient field for \( r ( x ) \) on \( ( M , g ) \), we must prove that \( d r ( v ) = \)
	\( g \left( \partial _ { r } , v \right) . \)
	We already know that
	\[
		d r = \frac { 1 } { r } \left( x ^ { 1 } d x ^ { 1 } + \cdots + x ^ { n } d x ^ { n } \right),
	\]
	but have no knowledge of $g$, since it is just some abstract metric.

	One can show that \( d r ( v ) = g \left( \partial _ { r } , v \right) \) by using suitable Jacobi fields for \( r \) in place of \( v \). Let us start with \( v = \partial _ { r } . \)
	The right-hand side is 1 as the integral curves for \( \partial _ { r } \) are unit speed geodesics.
	The left-hand side can be computed directly and is also $1$.
	Next, take a rotational field \( J = - x ^ { i } \partial _ { j } + x ^ { j } \partial _ { i } , i , j = 1 , \ldots , n , i < j . \)
	In dimension $2$ this is simply the angular field \( \partial _ { \theta } \).
	An immediate calculation shows that the left-hand side vanishes: \( d r ( J ) = 0 \).
	For the right-hand side we first note that \( J \) really is a Jacobi field as \( L _ { \partial _ { r } } J = \left[ \partial _ { r } , J \right] = 0 . \)
	Using that \( \nabla _ { \partial _ { r } } \partial _ { r } = 0 \) we obtain
	\[ \begin{aligned}
			\partial _ { r } g \left( \partial _ { r } , J \right) & = g \left( \nabla _ { \partial _ { r } } \partial _ { r } , J \right) + g \left( \partial _ { r } , \nabla _ { \partial _ { r } } J \right) \\
			                                                       & = 0 + g \left( \partial _ { r } , \nabla _ { \partial _ { r } } J \right)                                                                   \\
			                                                       & = g \left( \partial _ { r } , \nabla _ { J } \partial _ { r } \right)                                                                       \\
			                                                       & = \frac { 1 } { 2 } D _ { J } g \left( \partial _ { r } , \partial _ { r } \right)                                                          \\
			                                                       & = 0
		\end{aligned} \]
	Thus \( g \left( \partial _ { r } , J \right) \) is constant along geodesics emanating from \( p \).
	To show that it vanishes first observe that
	\[ \begin{aligned} \left| g \left( \partial _ { r } , J \right) \right| & \leq \left| \partial _ { r } \right| | J |                                                                               \\
                                                                     & = | J |                                                                                                                  \\
                                                                     & \leq \left| x ^ { i } \right| \left| \partial _ { j } \right| + \left| x ^ { j } \right| \left| \partial _ { i } \right| \\
                                                                     & \leq r ( x ) \left( \left| \partial _ { i } \right| + \left| \partial _ { j } \right| \right)\end{aligned} \]
	Continuity of \( D \exp _ { p } \) shows that \( \partial _ { i } , \partial _ { j } \) are bounded near \( p .  \)
	Thus \( \left| g \left( \partial _ { r } , J \right) \right| \rightarrow 0 \) as \( r \rightarrow 0 .  \)
	This forces \( g \left( \partial _ { r } , J \right) = 0 .\)
	Finally, observe that any vector \( v \) is a linear combination of \( \partial _ { r } \) and rotational fields.
	This proves the claim.
\end{proof}

\subsection{Fix typos and correct statements}

We copy original statement and put reference number directly after it.

\begin{rmk}[Remark 2.2]
	Inspection of the proof above shows that \( ( M \), vol \( ) \) can be replaced with a \cancel{compact separable} metric space \( ( X , \nu ) \) equipped with a reference Borel measure \( \nu \).
\end{rmk}

We only need an outer regular reference measure. And Borel measure on metric space is regular, see Theorem 7.1.7 in \cite{Bogachev2007}.

\begin{prop}[Proposition 2.9 Distortion under Ric \( \geq 0 \)]
	Suppose the Ricci curvature of \( M \) is everywhere nonnegative, i.e., Ric \( \geq 0 . \) Then, for any \( x \in M \) and \( \lambda \in P ( M ) \), \textcolor{blue}{if $\lambda$ gives no mass to the cut-locus of its baryceter $\bar{x}$}, we have
	\[ \alpha _ { \lambda } ( x ) \geq 1 \]
\end{prop}

\begin{proof}[Proof of Prop 2.9]
	Minimality of \( z \mapsto \int _ { M } c ( x , z ) \diff \lambda ( x ) \) at the barycenter \( \bar { x } \), combined with semi-concavity of \( z \mapsto c ( x , z ) \) and Fatou's lemma yields
	\[
		\int _ { M } \left. D _ { z z } ^ { 2 } \right| _ { z = \bar { x } } c ( x , z ) \diff \lambda ( x ) \geq 0
	\] as a matrix.
\end{proof}

Locally $c(x,z)=:d^2_x(z)$ near $\bar{x}$ is a geodesically convex function, by linear intergration so is the integral  \( z \mapsto \int _ { M } d^2_x(z) \diff \lambda ( x ) \).
As convexity implies local Lipschitz, $d^2_x(z)$ is also differentiable near $\bar{x}$ and so is its integral by compactness of $M$.
Thus we have locally for $\lambda$-almost all $x$ (not in the cut-locus of $\bar{x}$, arguing with semi-concavity is not strong as this),
% \textcolor{red}{why differentiability of $d^2_x(z)$ implies smoothness?}
\begin{equation}
	\label{equa:convex_distance_inequality}
	d^2_x \left( \exp _ {\bar{x}} u \right) - d^2_x (\bar{x}) - \langle \nabla d^2_x (\bar{x}) , u \rangle = \frac { 1 } { 2 } \langle \left. D _ { z z } ^ { 2 } \right| _ { z = \bar { x } } d^2_x(z) \, u , u \rangle + o \left( | u | ^ { 2 } \right) > 0
\end{equation}

To show positivity of matrix, we take a vector $\nu$ and set $\mu = h \nu$ with $h > 0$ small enough.
Divide by $h$ in \cref{equa:convex_distance_inequality} and then take integral,
\[
	\langle \left. D _ { z z } ^ { 2 } \right| _ { z = \bar { x } } \int_{M} d^2_x(z) \diff \lambda ( x )\, v , v \rangle + \frac{o (h^2)}{h^2} = \int_{M} \langle \left. D _ { z z } ^ { 2 } \right| _ { z = \bar { x } } d^2_x(z)\, v , v \rangle + \frac{o (h^2)}{h^2} \diff \lambda ( x ).
\]
Note that here we need compactness of $M$, hence boundness of $ \nabla d^2_x(\bar{x}) $ for $x \in M$, to get inter-change of integral and gradient:
\[
	\int_M \langle \nabla d^2_x (\bar{x}) , u \rangle \diff \lambda (x) = \langle \int_M \nabla d^2_x (\bar{x}) \diff \lambda (x) , u \rangle
\]
Apply Fatou's lemma, let $f\downarrow 0$,
\[
	\langle \left. D _ { z z } ^ { 2 } \right| _ { z = \bar { x } } \int_{M} d^2_x(z) \diff \lambda ( x )\, v , v \rangle \leq \langle \int_{M} \left. D _ { z z } ^ { 2 } \right| _ { z = \bar { x } } d^2_x(z) \diff \lambda(x)\, v, v \rangle.
\]
And by minimallity of $\bar{x}$, we know left hand side is non-negative.

This argument is also used in Prop 4.2.

\begin{rmk}[Remark 3.2]
	... In fact, it holds for any (compact) metric
	space on which the optimal \cancel{maps} \textcolor{blue}{plans}, \( T _ { \# } \mu = \nu \), exist uniquely for any arbitrary absolutely continuous source measure \( \mu \).
\end{rmk}

\begin{lem}[Lemma  4.1  a.e. \( x \) and $ \Omega$-a.e. $\mu$ ]
	Let \( \bar { \mu } \in \cancel{P ( M )} \textcolor{blue}{P_{ac}(M)}\) and for each \( \mu \in P ( M ) \), let \( u _ { \mu } \) be the dual potential (whose gradient is uniquely determined \( \bar { \mu } \) almost everywhere) for the optimal transport problem between \( \bar { \mu } \) and \( \mu . \) Let \( \Omega \) be a Borel probability on \( P ( M ) . \) For volume almost all \( x , x \mapsto u _ { \mu } ( x ) \) is twice differentiable for \( \Omega \) -almost all \( \mu \in P ( M ) . \)
\end{lem}

\begin{proof}
	$\forall \mu, x \mapsto \mu_u(x)$ is twice differentiable for $\bar{\mu}$-a.e. $x$. Apply Fubini's theorem then.
\end{proof}

\begin{prop} [Prop 4.2 Derivatives inside the integral $\int _ { P ( M ) } \diff \Omega$]
	\begin{equation}
		\label{equa:first_order}
		\nabla _ { x } \int _ { P ( M ) } u _ { \mu } ( x ) \diff \Omega ( \mu ) = \int _ { P ( M ) } \nabla _ { x } u _ { \mu } ( x ) \diff \Omega ( \mu )
	\end{equation}
	\begin{equation}
		\label{equa:second_order}
		\nabla _ { x } ^ { 2 } \int _ { P ( M ) } u _ { \mu } ( x ) \diff \Omega ( \mu ) \geq \int _ { P ( M ) } \nabla _ { x } ^ { 2 } u _ { \mu } ( x ) \diff \Omega ( \mu )
	\end{equation}
\end{prop}

\begin{proof}
	This can be seen by applying the dominated convergence theorem for \cref{equa:first_order} due to uniform Lipschitzness \textcolor{blue}{by \cref{lem:infimal_convolution_Lipschitz}} of \( u _ { \mu } \) and Fatou's lemma for \cref{equa:second_order} due to the semi-convexity
	of \( u _ { \mu } . \)
\end{proof}

% This shouldn't be an independent proposition, rather we can only condsider it in the case of Theorem 4.4.
% $u_{\mu}$ is uniformly Lipschitz on $x\in M$, this shows that $\nabla_{x}u_{\mu}$ is finite but not necessarily bounded (dominated).
To get rid of measurable selection problem, we consider
\[ y \mapsto \int_{P(M)} c(y, T_{\mu}(x)) \diff \Omega ( \mu)\]
at point $x$.
Here we assume $\bar{\mu}$ is barycenter of $\Omega$ and $T_{\mu}$ is the transfer map from $\bar{\mu}$ to $\mu$.
Then this integral valued map is well-defined $\mu$-a.e. by Fubini's theorem.
By stability of Kantorobich potentials (Theorem 1.52 in \cite{Santambrogio2015}), $c(y, T_{\mu}(x))$ is continous with respect to $\mu \in P(M)$.
Fix a $\mu$, $c(y, T_{\mu}(x))$ is locally Lipschitz with respect to $y$.
By continuity Lipschitz inequality will hold as well in a neighbourhood of $\mu$.
Consider these local neighbourhoods cover on compact set $ M \times P(M)$,
we can then get a uniform boundness on
\[\left. D_y\right|_{y=x} c(y, T_{\mu}(x)) = - \nabla_x u_{\mu}(x).\]


\begin{lem} [Lem 4.3 Riemannian barycenter from Wasserstein barycenter]
	\label{lem:inverse_barycenter}
	Let \( \bar { \mu } \) be a Wasserstein barycenter of the measure \( \Omega \) on \( P ( M ) \) and assume \( \bar { \mu } \) is absolutely continuous with respect to volume;
	let \( T _ { \mu } \) be an optimal map from \( \bar { \mu } \) to \( \mu . \)
	Let \( \lambda _ { z } = \left( \mu \mapsto T _ { \mu } ( z ) \right) _ { \# } \Omega . \)
	Then, for \( \bar { \mu } \) almost every \( z , z \) is a barycenter of \( \lambda _ { z }\).
	\textcolor{blue}{And $\lambda_z$ gives no mass to the cut-locus of $z$} by \cref{prop:differentiate_optimal_transport}.

	If, in addition, \( \Omega \left( P _ { a c } ( M ) \right) > 0 \), then for \( \bar { \mu } \) almost every \( z , z \) is the unique barycenter of $\lambda _ { z }$.

\end{lem}

There is an explanation behind this lemma. Consider we have random images, and we want to approch a best representative of them by simulation. We process this by compose all barycenters of simulated images into an average image. On the other hand, for each grid point in our average image, we can simulate a new image by transfering that grid point optimally. This lemma claims that that choosen grid point should be a barycenter of the new generated image. This could be related to ergodic theory as there are two kinds of averages involved.


\begin{proof}[Proof of the 1st order balance]
	...
	% On the other hand, by Lemma  4.3, for \( \bar { \mu } \) almost every \( x \), we have that \( x \) is the
	% barycenter of \( \lambda _ { z } = \left( \mu \mapsto T _ { \mu } ( z ) \right) _ { \# } \Omega ; \) that is, a minimizer of
	% \[ f _ { x } : y \mapsto \int _ { P ( M ) } d ^ { 2 } \left( y , T _ { \mu } ( x ) \right) d \Omega ( \mu ) .\]

	Therefore, the latter function \( f _ { x } \), which is semi-concave is differentiable at \( x : \) due to semi-concavity, there is \( C > 0 \) such that the function \( f _ { x } ( y ) - C \operatorname { dist } ^ { 2 } ( x , y ) \) is locally geodesically concave near \( x \).
	Minimality at \( x \) implies \( f _ { x } ( y ) - C \operatorname { dist } ^ { 2 } ( x , y ) \geq f _ { x } ( x ) - \) \( C \operatorname { dist } ^ { 2 } ( x , \cancel{x}\, \textcolor{blue}{y} ) \).
	Since \( y \mapsto f _ { x } ( x ) - C \operatorname { dist } ^ { 2 } ( x , y ) \) has vanishing derivative at \( x \), concavity of \( f _ { x } ( y ) - C \operatorname { dist } ^ { 2 } ( x , y ) \) implies that locally the function \( y \mapsto f _ { x } ( y ) - C \operatorname { dist } ^ { 2 } ( x , y ) \) is also
	locally bounded from above by the constant \( f _ { x } ( x ) . \) This implies the differentiability of
	\( f _ { x } \) at \( x \).
\end{proof}

% \textcolor{cyan}{If we have} that $\lambda_{z}$ is absolutely continous, then we prove first order balance by taking gradient under integral.
% For the sencond paragraph of argument,
To simplify, we should prove:
\begin{lem}
	For $f, g$ two convex functions near $0$, assume $f(0)=g(0)=g^\prime(0)=0$. If $f \leq g$ and $0$ is a local minimum of $g$, we then have $f \geq 0$ and $f$ is differentiable at $0$.
\end{lem}

\begin{proof}
	We need to show that subdifferential $\partial f(0) $ of $f$ is a single point set $ \{ 0 \}$. $\forall u \in \partial f( 0 )$,
	\[
		\langle x, u \rangle \leq f(x) - f(0) = f(x) \leq g(x) = g(x) - g(0),
	\]
	that is say, $ u \in \partial g(0)$. Hence $ \partial f(0) = \partial g(0) = \{0\}$.

\end{proof}

\begin{proof}[Proof of Theorem 4.6]
	...\\
	Note that each \( - D _ { x y } ^ { 2 } c \left( x , T _ { \mu } ( x ) \right) D T _ { \mu } ( x ) = D _ { x x } ^ { 2 } u _ { \mu } ( x ) + D _ { x x } ^ { 2 } c \left( x , T _ { \mu } ( x ) \right) \) is positive semi-definite by the \( c \)-convexity of \( u _ { \mu } \), and hence so is their integral...
\end{proof}

By definition, we have
\[
	\inf_{\textcolor{cyan}{z}} u_\mu(\textcolor{green}{z}) + c(\textcolor{green}{z}, T_\mu(x))= u_\mu(x) + c(x, T_\mu(x)) = - u^c ( T_\mu(x)).
\]
If consider derivative with respect to first variable,
then we have first differential vanishes and Hessian semi-positive.


% \section{Multi-marginal optimal transport on Riemannian manifolds}
% We will prove uniqueness and Monge solution results for the multi-marginal problem on a compact Riemannian manifold,
% with cost function
% \begin{equation}
% 	\label{equa:mult-imarginal_problem}
% 	c \left( x _ { 1 } , x _ { 2 } , \ldots , x _ { m } \right) = \inf _ { y \in M } \sum _ { i = 1 } ^ { m } \frac { d ^ { 2 } } { 2 } \left( x _ { i } , y \right)
% \end{equation}

% \begin{lem}
% 	Fix \( \left( x _ { 1 } , \ldots , x _ { m } \right)\).
% 	Then any \( y \) which minimizes \( y \mapsto \sum _ { i = 1 } ^ { m } d ^ { 2 } \left( x _ { i } , y \right) \) is not in the cut locus of \( x _ { i } \) for any \( i . \)
% \end{lem}

% \begin{proof}
% 	Choose a point \( y \) in the cut locus of \( x _ { i } \) for some \( i \);
% 	we will show that \( y \) cannot minimize \( y \mapsto \sum _ { i = 1 } ^ { m } d ^ { 2 } \left( x _ { i } , y \right) . \)
% 	By Lemma 3.12 in \cite{cordero2001riemannian}, we can find a constant \( K \) such that, for all \( u \in T _ { y } M \), and \( j = 1,2 , \ldots m \), we have
% 	\[ \frac { d ^ { 2 } \left( x _ { j } , \exp _ { y } u \right) + d ^ { 2 } \left( x _ { j } , \exp _ { y } ( - u ) \right) - 2 d ^ { 2 } \left( x _ { j } , y \right) } { | u | ^ { 2 } } \leq K \]
% 	On the other hand, by Proposition \( 2.5 \) in the same paper, we can find some non
% 	zero \( u \in T _ { y } M \) such that
% 	\[ \frac { d ^ { 2 } \left( x _ { i } , \exp _ { y } u \right) + d ^ { 2 } \left( x _ { i } , \exp _ { y } ( - u ) \right) - 2 d ^ { 2 } \left( x _ { i } , y \right) } { | u | ^ { 2 } } \leq - m K \]
% 	Therefore, we have
% 	\[ \begin{aligned} \sum _ { j = 1 } ^ { m } d ^ { 2 } \left( x _ { j } , y \right) & = \sum _ { j \neq i } ^ { m } d ^ { 2 } \left( x _ { j } , y \right) + d ^ { 2 } \left( x _ { i } , y \right) \\ & \geq \frac { - ( m - 1 ) K | u | ^ { 2 } } { 2 } + \frac { 1 } { 2 } \sum _ { j \neq i } ^ { m } \left( d ^ { 2 } \left( x _ { j } , \exp _ { y } u \right) + d ^ { 2 } \left( x _ { j } , \exp _ { y } ( - u ) \right) \right) \\ & + \frac { m K | u | ^ { 2 } } { 2 } + \frac { 1 } { 2 } d ^ { 2 } \left( x _ { i } , \exp _ { y } u \right) + d ^ { 2 } \left( x _ { i } , \exp _ { y } ( - u ) \right) \\ & > \frac { 1 } { 2 } \sum _ { j = 1 } ^ { m } d ^ { 2 } \left( x _ { j } , \exp _ { y } u \right) + \frac { 1 } { 2 } \sum _ { j = 1 } ^ { m } d ^ { 2 } \left( x _ { j } , \exp _ { y } ( - u ) \right) \end{aligned} \]
% 	Therefore, either
% 	\[ \sum _ { j = 1 } ^ { m } d ^ { 2 } \left( x _ { j } , \exp _ { y } u \right) < \sum _ { j = 1 } ^ { m } d ^ { 2 } \left( x _ { j } , y \right) \]
% 	or
% 	\[ \sum _ { j = 1 } ^ { m } d ^ { 2 } \left( x _ { j } , \exp _ { y } - u \right) < \sum _ { j = 1 } ^ { m } d ^ { 2 } \left( x _ { j } , y \right) \]
% 	in either case, \( y \) cannot minimize \( \sum _ { j = 1 } ^ { m } d ^ { 2 } \left( x _ { j } , y \right) \)
% \end{proof}

% \begin{lem}
% 	The cost function \( c \) is everywhere superdifferentiable with respect to \( x _ { 1 } \).
% 	That is, for all \( \left( x _ { 1 } , x _ { 2 } , \ldots , x _ { m } \right) \in M ^ { m } \) there exist \( p \in T _ { x _ { 1 } } M \) (the super-gradient) such that, for small \( v \in T _ { x _ { 1 } } M \), we have
% 	\[ c \left( \exp _ { x _ { 1 } } v , x _ { 2 } , \ldots , x _ { m } \right) \leq c \left( x _ { 1 } , x _ { 2 } , \ldots , x _ { m } \right) + g ( p , v ) + o ( | v | ) \]
% \end{lem}

% \begin{lem}
% 	At any point \( \left( x _ { 1 } , \ldots , x _ { m } \right) \) where \( c \) is differentiable with respect to
% 	\( x _ { 1 } \), there is a unique minimizing \( y \) in \cref{equa:mult-imarginal_problem}, and moreover,
% 	\[ y = \exp _ { x _ { 1 } } \left( \nabla _ { x _ { 1 } } c \left( x _ { 1 } , \ldots , x _ { m } \right) \right) \]
% \end{lem}

% \begin{proof}
% 	For any minimizing \( y \) in \cref{equa:mult-imarginal_problem},
% 	\( d ^ { 2 } \left( x _ { 1 } , y \right) \) is differentiable as \( y \notin \operatorname { cut } \left( x _ { 1 } \right) \).
% 	We then have
% 	\[
% 		\nabla _ { x _ { 1 } } c \left( x _ { 1 } , \ldots , x _ { m } \right) = \nabla _ { x _ { 1 } } \left( \frac { 1 } { 2 } d ^ { 2 } \left( x _ { 1 } , y \right) \right)
% 	\]
% 	This equation implies that \( y \) must equal \( \exp _ { x _ { 1 } } \left( \nabla _ { x _ { 1 } } c \left( x _ { 1 } , \ldots , x _ { m } \right) \right) \);
% 	uniqueness follows immediately.
% \end{proof}


\section{Barycenter on manifold}

% \subsubsection{A perspective from optimal transportation}

% Make use of previous discussions, we have following nice property

% \textcolor{red}{ALL THESE ARE WRONG! So I finally change orignal $i$ into $j$.
% That is to say, we fix $i$ in following lemma, everything is so trivial now.}

% \begin{lem}
% Let $\mu_j, j=1,2,\ldots,n$ be a $n$ measures on compact Riemanninan manifold $M$.
% Assume $\mu_i$ absolutely continuous, and other $\mu_j, j \neq i$ are discrete.
% % absolutely continous with respect to volume measure on $M$.
% Then the measure $\sum_{j=1}^n \lambda_j \delta_{\mu_j} \in P(P(M))$ has a unique barycenter $\bar{\mu}$ on $M$.
% $\bar{\mu}$ is absolutely continous with respect to volume measure on $M$.
For $n$ points $x_i \in M, i=1,2,\ldots,n$, we define a function on $M^n$,
\[
	f(x_1, x_2, \ldots, x_n) := \min_{z} \sum_{i=1}^n \lambda_i c(z, x_i)^2 =: \frac{1}{2}W^2(\sum_{i=1}^n \lambda_i \delta_{x_i}, M).
\]
Note that $f_i / \lambda_i $ is a $c$-concave function with a global Lipschitz constant bounded by diameter of $M$.
This conclusion relies heavily on compactness of manifold $M$.

% \end{lem}
Denote by $B$ a measurable selection of barycenter of $n$-points supported measures on $M$,
\[
	B(\boldsymbol{x}) \in \arg \min_{x \in M} \sum_{i=1}^{n} \lambda_i d(x, x_i)^p.
\]
% Let $T_i$ be the unique optimal transfer map from $\bar{\mu}$ to $\mu_i$,
% and define $\gamma := (T_1, T_2, \ldots, T_n)_{\#} \bar{\mu}$ to be the push-forward measure of $\bar{\mu}$ on $M^n$.

% Then for $(x_1, x_2, \ldots, x_n)$ in the support of $\gamma$,
% there is $\gamma$-a.e. a unique barycenter $B(x_1, x_2, \ldots, x_n)$ of measure $\sum_{j=1}^n \lambda_j \delta_{x_j}$.
% Both map $B$ and $f$ are $\gamma$-a.e. differentiable.
% The map $B(T_1, T_2, \ldots, T_n)$ is $\bar{\mu}$-a.e. the identity map on $M$.

We have following barycenter formula holds almost everywhere for volume measure,
% if $\mu_i$ is absolutey continous
\begin{equation}
	\label{formula_barycenter}
	B = \exp (- \frac{1}{\lambda_i} \nabla_i f),
\end{equation}
where $\nabla_i f$ means partial gradient of $f$,
i.e., $\nabla_i f = \nabla f_i$ if we define $f_i$ as function on $M$ through $f$ by fixing $x_j$ for $j \neq i$.

% \begin{proof}
% We intend to apply discussion of conditional probability in \cref{discussion_conditional_prob} for the case
% that $\gamma$ is absolutely continous.
% Firstly we should prove that $\gamma$ is absolutely continous,
% and we may get inspiration from \cite{santambrogio2009absolute} for the case of $n=2$.
% If this is done, we can apply previous discussion.

% \end{proof}

\begin{rmk}
	Some comments about this formula \cref{formula_barycenter}:
	\begin{itemize}
		\item The barycenter formula \cref{formula_barycenter} expresses the fact that,
		      from point $x_i$,
		      we can reach barycenter of $\sum_{j=1}^n \lambda_i \delta_{x_j}$ if we follow
		      the direction $-\nabla_i f$
		      and advance $\| \nabla_i f / \lambda_i \| $.
		\item In the case of Euclidean space, $\nabla_i (f/\lambda_i)(x_i)$ reduces to
		      $x_i - \sum_{j=1}^n \lambda_j x_j$.
		% \item $\exp(-\nabla f_i / \lambda_i )$ is the optimal transfer map from $\mu_i$ to $\bar{\mu}$.
		\item The $c$-conjugate function of $f_i / \lambda_i$ is $ (f_i / \lambda_i )^c(z) = -\sum_{j\neq i}^n \lambda_j c(z, x_j) / \lambda_i$.
		\item In the case of Eculidean space, $\nabla (f_i / \lambda_i)^c (z)$ reduces to

		      $(z - \sum_{j=1}^n \lambda_j x_j) / \lambda_i -(z-x_i)$.
	\end{itemize}
\end{rmk}

% \textcolor{red}{We have to fix $i$ and assume discrete measure for others,
% 	since first order balance holds in the case of Euclidean space only
% 	if $n=1$.}
% It is possible to get a inverse to $B$.
% without absolutely continuity of $\bar{\mu}$.
% To do so, we need to fix $n-1$ variables of $B$.
% We define a barycenter map from $M$ to $M$, 
% $br(x_1) := B(x_1, x^\prime) = \exp(-\frac{1}{\lambda_1}\nabla_1 f),$
% where $x^\prime := (x_2, x_3, \ldots, x_n)$;
% and we define
% $
% g ( z ) = -(f_1 / \lambda_1)^c=
% 	-\frac { 1 } { \lambda _ { 1 } } \sum _ { i = 2 } ^ { n } \lambda _ { i } \, c\left( x _ { i } , z \right),
% $
% Recall that for each \( x \),
% every \( z \in br\left( x , x _ { 2 } , \ldots . . x _ { n } \right) \) is not in the cut-locus of any \( x _ { i } \).
% Therefore, \( g ( z ) \) is twice differentiable at each \( z \in br\left( M , x _ { 2 } , \ldots , x _ { n } \right) . \)

% Now, for any point \( y \) such that
% \[ z \in br\left( y , x _ { 2 } , \ldots , x _ { n } \right) \]
% from the definition of \( g ( z ) \), we have \( \left. \nabla _ { w } \right| _ { w = z } c( y , w ) = - \nabla _ { z } g ( z ) \) or, equivalently,
% \( y = \exp _ { z } \nabla g ( z ) . \)
% Therefore, \( \exp _ { z } \nabla g ( z ) \) is the only point with the desired property.

Now we aim to prove that \( \exp(- \nabla (f_i /\lambda_i)^c) \) is a Lipschitz function with Lipschitz constant depending only on $M$ and $\lambda_i$.
As $M$ is compact, though tangent bundle $TM$ is not compact, local Lipschitz plus bounded diameter of $M$ implies global Lipschitz of $\exp$.
Function $(f_i /\lambda_i)^c$ has hessian upper bound as a $c$-concave function,
and has hessian lower bound as a negative linear combination of square distance function.
We then have
\begin{equation}
	\label{equa:hessian_bound_f}
	-\frac{1-\lambda_i}{\lambda_i} H \leq \nabla^2 \left(\frac{f_i}{\lambda_i}\right)^c \leq H,
\end{equation}
where we denote by $H$ the hessian of square distance funtion.
Recall that $H$ is bounded from above.
% Squared distance function $c$ has bounded Hessian from above.
% Hence $ \| \nabla^2 g \|$ is bounded from above by taking the second derivative of definition and also applying minimallity of cost at barycenter.

\section{Absolutely continuity of barycenter}

We discuss in the context of proof of \cref{thm:barycenter_finite_support_measure}.

\subsection{Barycenter of finite supported measure}

Consider a measure $\sum_{i}^{n} \lambda_{i} \delta_{\mu_i}$ on $P(M)$ for $M$ a compact Riemannian manifold.
Assume from now on $ \lambda_1 \neq 0$ and $\mu_1$ absolutely continous.

The barycenter $\bar{\mu}$ of measure $\sum_{i}^{n} \lambda_{i} \delta_{\mu_i}$ is unique,
because this meause gives mass to element in $P_{ac}(M)$.

If we \textcolor{cyan}{assume $\bar{\mu}$ is absolutely continous},
then there is only one element in the set $\Gamma$ by uniqueness of optimal plans.
We can get every $\mu_i$ from $\mu_1$ if first push $\mu_1$ to $\bar{\mu}$ through $T_1^{-1}$ then push $\bar{\mu}$ to $\mu_i$ through $T_i$. Hence,
\[B \circ (T_1, T_2, \ldots, T_n) \circ T_1^{-1} = T_1^{-1}\]
is the unique transfer map from $\mu_1$ to $\bar{\mu}$. That is to say, $\bar{\mu}$-a.e., $B \circ (T_1, T_2, \ldots, T_n) $ is the identity map.
This is already included in \cref{lem:inverse_barycenter}.

\subsubsection{One measure absolutely continous and others Dirac}
Here we \textcolor{cyan}{assume $\mu_i = \delta_{x_i}, i \geq 2$ are Dirac measures}.

As an inverse to $\exp(-\nabla f_1/\lambda_1)$,
\(\exp(-\nabla(f_1/\lambda_1)^c)\) pushes $\bar{\mu}$ to $\mu_1$.
% because $br$ pushes $\mu_1$ to $\bar{\mu}$ by construction of $\bar{\mu}$.
Denote by $C$ the Lipschitz constant of map \(\exp(-\nabla(f_1/\lambda_1)^c)\).
If we have $ \text{Vol}(E) < \delta \implies \mu_1(E) < \epsilon$,
then by $\text{Vol}(br^{-1}(E)) < C^n \text{Vol}(E)$
we have $\text{Vol}(E) < \delta / C^n \implies \bar{\mu}(E)=\mu_1(br^{-1}(E)) < \epsilon$.
Thus $\bar{\mu}$ is absolutely continous.

% \begin{rmk}
% 	Once after we prove that $\bar{\mu}$ is indeed absolutely continous.
% 	$\nabla g$ is invertable and $br(x_1) = \exp_{x_1} \nabla g^c(x_1)$.
% 	It is not surprised that
% 	\[
		% g^c(x_1) = -c(x_1 ,z) - \frac{1}{\lambda_1} \sum_{i=2}^{n} \lambda_i\, c(x_i, z) = 
		% - \frac{1}{2 \lambda_1} W^2( \sum_{i=1}^n \lambda_i \delta_{x_i}, M),
		% - f_1/\lambda_i.
	% \]
	% we can then take derivative with respect to $x_1$.
% \end{rmk}

\subsubsection{To a more general case by conditional probability}
% \label{discussion_conditional_prob}
To attack general case when \textcolor{cyan}{$\mu_i, i \geq 2$ are not assumed Dirac measures},
we should consider conditonal measures of optimal multi-marginal transfer plan $\gamma$:
\[
	\diff \gamma(x_1, x^\prime)= \gamma(\diff x_1 \mid x^\prime)\, \diff \pi(x^\prime) ,
\]
where $\pi$ is the projection of $\gamma$ from $x = (x_1, x^\prime)$ to $x^\prime$,
by abuse of language we also denote by it the push-forward measure from $\gamma$.
For $\pi$-a.e $x^\prime$., $\gamma(\cdot \mid x^\prime)$ is a probability measure
concentrated on $M \times \{x^\prime\}$.

One has from definition of conditional measures that
\[
	\bar{\mu} = B_{\#} \gamma = \int_{M^{n-1}} B_{\#} \gamma(\cdot \mid x^\prime)\, \diff \pi(x^\prime).
\]


If we have that for $\gamma$-a.e. $x^\prime$,
$B_{\#} \gamma(\cdot \mid x^\prime)$
is absolutely continous,
then $\bar{\mu}(E)=0$ for volume measure zero set $E$ by integration above.
% \[
% 	\bar{\mu}(E)
% =\int_{M^{n-1}}
% \frac{ \mu_1( br^{-1}(E) \cap A_{x^\prime} )}{\mu_1(A_{x^\prime})}
% \diff \pi(x^\prime)
% =\int_{M^{n-1}}
% \bar{\mu}^{x^\prime}(E)
% \diff \pi(x^\prime)
% \leq \int_{M^{n-1}} C\,\mu_1^{\prime}(E) \diff \pi(x^\prime) \leq C\, \mu_1(E).
% = 0.
% \]

Generally speaking, computation of $\gamma(\cdot \mid x^\prime)$ is only possible when
\begin{itemize}
	\item $\pi$ is has countable support.
	      This is equivalent to that $\mu_i, i \geq 2$ has countable support.
	      Fix any $x^\prime$ in the support of $\pi$, by direct verification
	      \[
		      \gamma(\cdot \mid x^\prime) =
		      \frac{
			      \mathbbm{1}_{M \times \{x^\prime\}}
			      \gamma
		      }{\pi(x^\prime)}.
	      \]
	\item $\gamma$ is absolutely continous with density function $f: M^n \rightarrow \mathbb{R} $.
	      \[
		      \gamma(\diff x_1 \mid x^\prime) =
		      \frac{
			      f(x_1, x^\prime) \diff \text{Vol}(x_1)
			      % \diff \gamma (x_1, x^\prime)
		      }
		      {\int_{M} f(x_1, x^\prime) \diff \text{Vol}(x_1)
		      }
	      \]
	      where we set the right hand side zero if it is undeterminated
	      and we remove this $x^\prime$ from consideration.
\end{itemize}
Note in both cases,
$\gamma(\cdot \mid x^\prime)$
has absolutely continous push-forward measure
$\mu_1^{x^\prime} := \text{proj}^1_{\#}\gamma(\cdot \mid x^\prime)$
to the first coordinate.
But only in the first case that
measure $\gamma(\cdot \mid x^\prime) \leq \pi(x^\prime) \, \gamma$ is controled by $\gamma$.
Or we hope $f$ has positive lower bound,
for example \textcolor{cyan}{continous and strictly positive}, like volume measure.
In the case $\mu_1=\text{Vol}$? Every point in $\text{spt}\mu_1$ runs over all $\times_i \text{spt}\mu_i$.

\textcolor{red}{Unfortunately, this control is required to apply conditioning optimal plan.}
One may consider the case when at least one marginal is discrete,
but no absolutely continuity can be derived without barycenter push-forward.

It is true that we have $\gamma$ is an optimal plan for all its marginals
% \[
% 	B_{\#} \gamma              = \bar{\mu} \text{ and }
% 	\text{proj}^1_{\#} \gamma  = \mu_1,
% \]
and conditioning of (multi-marginal) optimal plan is still optimal from \cref{thm:restriction_optimal_plan}.
Hence, $\gamma(\cdot \mid x^\prime)$ is an optimal plan of its marginals.
% \textcolor{cyan}{in the case of discrete measure}.
Apply previous discussion, we get absolutely continous measure
\[
	\exp(-\frac{1}{\lambda_1}\nabla f_1)_{\#}  \mu_1^{x^\prime}=
	B_{\#}\gamma(\cdot, x^\prime) =
	\bar{\mu}^{x^\prime},
\]
where $\bar{\mu}^{x^\prime}$ is a barycenter measure.

Since conditioning of optimal maps are still optimal,
barycenter formula \cref{formula_barycenter} still holds.
We define $T_1 := \exp(-\nabla (f_1/\lambda_1)^c)$ to be the optimal transfer map from $\bar{\mu}^{x^\prime}$ to $\mu_1^{x^\prime}$,
\begin{align*}
	\bar{\mu} = B_{\#} \gamma & =
	\int_{M^{n-1}}\bar{\mu}^{x^\prime} \, \diff \pi(x^\prime)                         \\
	                          & = \int_{M^{n-1}}
	\det D T_1  \circ T_1^{-1}\, \gamma(\cdot \mid x^\prime) \diff \pi(x^\prime)      \\
	                          & = \det D T_1  \circ T_1^{-1}\, \gamma                 \\
	                          & = \exp (- \frac{1}{\lambda_1} \nabla_1 f)_{\#} \gamma
														.
\end{align*}
% One possible \textbf{investigation}: Conditional optimal plan, not just the simple case of restriction.
% Hence, the push-forward measure
% $\bar{\mu}^{x^\prime} = br_{\#} \diff \mu_1^{x^\prime}$ is absolutely continous as well.
% We thus have
% \[
% 	B_{\#} \diff \gamma = \int_{M^{n-1}} f(\cdot, x^\prime)  \diff \pi(x^\prime) \, \diff \text{Vol}(\cdot).
% \]
% Moreover, the previous integral is in fact a finite sum.
% For a measurable set $E \subset \text{spt}(\bar{\mu})$ with volume measure $0$,

% So $\bar{\mu} = B_{\#} \diff \gamma$ is absolutely continous with respect to $\mu_1$.
% \begin{rmk}
% 	\textcolor{red}{NEED further investigation}, maybe one can use \cref{formula_barycenter}.

% 	We may guess that	the density of $\bar{\mu}$,
% 	$\int_{M^{n-1}} f(\cdot, x^\prime)  \diff \pi(x^\prime)$
% 	is dominated by density of $\mu_1$ up to a constant coefficient.
% 	% Lipschitz constant of \( \exp _ { z } \nabla g ( z ) \).
% \end{rmk}

\subsubsection{To a more general case by consistency of barycenter}

Now we consider measure $\mathbb{P} = \lambda_1 \delta_{\mu_1} + \lambda_2 \delta_{\mu_2}$
without assuming $\mu_2$ discrete.
Note here it is for simplicity that we only consider the case $n=2$.
Approxiamte $\mu_2$ in Wasserstein metric by a sequence of measures $\mu_2^{m}$,
then $\mathbb{P}_m := \lambda_1 \delta_{\mu_1} + \lambda_2 \delta_{\mu_2^m}$ converges to $\mathbb{P}$
in $P(P(M))$.

By the consistency of barycenters, the unique barycenter $\bar{\mu}_m$ of $\mathbb{P}_m$
converges in Wasserstein metric (or just weakly) to the unique barycenter $\bar{\mu}$ of $\mathbb{P}$.
Recall that there is no duality between $L^{\infty}(\bar{\mu})$ (possible non-separable) and $L^1 (\bar{\mu})$
in functional analysis even when $M$ is compact with Lebesgue measure.
According to Proposition 4.4.2 in \cite{Bogachev2007} below, we need to show that $\bar{\mu}$ is a linear functional on $L^{\infty}(\bar{\mu})$.
\begin{prop}
	Let \( \mu \) be a finite nonnegative measure.
	A continuous linear function \( \Psi \) on \( L ^ { \infty } ( \mu ) \) has the form
	\( \Psi ( f ) = \int _ { X } f g d \mu \)
	, where \( g \in L ^ { 1 } ( \mu ) \),
	precisely when the set function \( A \mapsto \Psi \left( I _ { A } \right) \) is countably additive.
\end{prop}

We define $\Psi(f):= \int_M f \diff \bar{\mu} = \lim_m \int_M f \diff \bar{\mu}_m$ for $f$ continous.
Extend $\Psi$ to be a continous linear functional on $L^{\infty}(\bar{\mu})$ by Hahn-Banach theorem.
Do we still have $\Psi(I_A) = \bar{\mu}(A)$?

A related discussion is available on
\href{https://math.stackexchange.com/questions/574130/does-weak-convergence-with-uniformly-bounded-densities-imply-absolute-continuity/574888#574888}{math Stack Exchange}.
Think about this example carefully.
Let $\lambda$ be the arglength measure and $\phi_n \ge 0$ a continuous function on $\mathbb T$ with $\int_{\mathbb T} \phi_n\, d\lambda = 1$ and $\phi_n(x) = 0$ if $|x-1| \ge \frac 1n$. Then for each continuous function $f\colon \mathbb T \to \mathbb R$ we have $\int_{\mathbb T} f\phi_n d\lambda \to f(1)$, that is $\phi_n \lambda \to \delta_1$ weakly. But $\delta_1$ is not $\lambda$-continuous.

A digression, generally for question if density of $\bar{\mu}_m$ converges to density of $\bar{\mu}$,
we possibly need asymptotically equicontinuous in \cite{Sweeting1986Converse}.

We don't need to find out density explicitly.
Instead, let prove the absolutely continuity by showing that
for any $\epsilon > 0$ there is a $\delta > 0$ such that
\begin{equation}
	\label{equa:absolutely_continous}
	\forall E \subset M \, \text{measurable, } \text{Vol}(E) < \delta
	\implies \bar{\mu}(E) < \epsilon
\end{equation}
In our situation, a \textcolor{red}{uniform $\delta$} can be chosen for all $\bar{\mu}_m$
because of a discrete situation of previous conditioning optimal plan calculation.
See the now ``trivial'' \cref{formula_barycenter},
we can use it to conclude uniform Lipschitz constant for $m$.

Recall for open set $E \subset M$, $\bar{\mu}(E) \leq \liminf \bar{\mu}_m(E)$,
as indicator function of open set is lower semi-continuous.
Hence \cref{equa:absolutely_continous} holds for all open sets.
Es Borel measure is outer regular, for general measurable $E$ with $\text{Vol}(E) < \frac{\delta}{2}$
select an open set $E^\prime$ such that
$ E \subset E^\prime$ and $ \text{Vol}(E^\prime) < \delta$,
then $\bar{\mu}(E) \leq \bar{\mu}(E^\prime) < \epsilon$.

Previous argument works for general $\lambda_1 \delta_{\mu_1} + \lambda_2 \mathbb{P}$ for any $\mathbb{P} \in P(P(M))$
if we approxiamte $\mathbb{P}$ by finite supported measures.
To sum up, we should be able to single out a part of $\mathbb{P}$ in the form $\lambda_1 \delta_{\mu_1}$
with lower bound on $\lambda_1$ and dominated $\mu_1$.
One way to do so is to assume $\mu_1$ has bounded density.
\begin{defn}
	[The set $ \mathcal { A } _ { L }$]
	For \( 0 < L < \infty \), let \( \mathcal { A } _ { L } \) be the set of Borel probability
	measures on \( M \), absolutely continuous with respect to volume, whose densities have \( L ^ { \infty } \)
	norm less than or equal to \( L\).
\end{defn}

Note that, since the bound on the \( L ^ { \infty } \) norm is preserved under weak-* convergence,
\( \mathcal { A } _ { L } \) is a weakly-* closed, and thus Borel measurable, subset of \( P ( M ) \).

If $\mu_1 \in \mathcal{A}_L$, then by duality between $L^p(\bar{\mu})$ and $L^q(\bar{\mu})$
and that they are separable spaces,
we have that $\bar{\mu}$ has density in $L^p(\bar{\mu})$ for
$p < \infty$ with a upper norm bound depending only on constat $L$ (, $\lambda_1$ and $M$).
We pass $p$ to infity to get $\bar{\mu} \in \mathcal{A}_L$.
Hence, our conclusion holds for any $\mathbb{P}$ that is not atomless on $\mathcal{A}_L$.

Finally, if we are given $\mathbb{P}$ with only assumption that $\mathbb{P}(\mathcal{A}_L(M)) > 0$,
we need more control on the density function of $\bar{\mu}_m$
to ``remove'' the dependency of its upper bound on a singled-out coefficient $\lambda_1$.
One hope is that $\lambda_1$ is replaced by $\mathbb{P}(\mathcal{A}_L(M))$.

\subsection{Calculate density function}

Every measure $\mu$ on $M$ in following discussion is absolutely continous.
We use the convention that denote by $g$ the density function for absolutely measure $\mu$.
One principle of differential geometry is to differentiate everything we could.

% If we only apply general bound on $c$-concave function
Recall change of variable (Theorem 4.2 in \cite{cordero2001riemannian}) and notations in \cref{prop:differentiate_optimal_transport},
\[
	\bar{g} = g_i \circ T_i \det D T_i :
	= g_i \circ T_i \det[Y(H-\text{Hess} u_i]
\]
where $T_i = \exp(-\nabla u_i)$ is the unique optimal maps from $\bar{\mu}$ to $\mu_i$.

For general $c$-concave function $\mu_i$, we have only upper hessian bound.
And this is not enough to get an estimation on the absolute value of Jacobian determinants.
% For instance, in Euclidean space, $DT_i = (\lambda_i -1)/\lambda_i \leq \text{Id}$.

We differentiate the equality $B(T_1(x), \ldots, T_n(x))=x$ for $\bar{\mu}$-a.e. $x$,
\begin{align*}
	\text{Id} =\sum_{i=1}^n \partial_i B\, DT_i
&=\sum_{i=1}^n D \exp(-\frac{1}{\lambda_i}\nabla f_i) \, D \exp(-\nabla u_i)\\
&=\sum_{i=1}^n D \exp^{-1}(-\nabla \left( \frac{f_i}{\lambda_i}\right)^c) \, D \exp(-\nabla u_i)\\
&=\sum_{i=1}^n(H-\text{Hess}(f_i / \lambda_i)^c)^{-1}\,Y_i^{-1}\,
Y_i\,(H-\text{Hess} u_i)\\
&=\sum_{i=1}^n(H-\text{Hess}(f_i / \lambda_i)^c)^{-1}\,
(H-\text{Hess}u_i) .
\end{align*}

Then by Minkowski's determinant inequality, we get
\begin{align*}
	1 &\geq \sum_{i=1}^{n} \det [H-\text{Hess}(f_i/\lambda_i)^c]^{-1/n}\,\det[H-\text{Hess}u_i]^{1/n}\\
		&=\sum_{i=1}^n \det[\partial_i B]^{1/n}\,\det[DT_i]^{1/n}
\end{align*}
Recall that we know $(f_i / \lambda_i)^c$ has hessian bound from both sides, see \cref{equa:hessian_bound_f}.
From it we get $\det[\partial_i B]^{1/n} \geq \min \{1, \lambda_i / (1 - \lambda_i)\} > \lambda_i$,
where $C > 0$ depends only on hessian bound of square distance function and Lipschitz constant of exponential map.

Combine these two inequalities, and we then apply Jensen inequlity
\[
	\bar{g} \leq 
	\left[ \sum_{i=1}^n \frac{\det[\partial_i B]^{1/n}}
	{g_i^{1/n} \circ T_i}\right]^{-n}
	< \left[ \sum_{i=1}^n \frac{C \, \lambda_i}
	{g_i^{1/n} \circ T_i}\right]^{-n}
	\leq C^{-n} \sum_{i=1}^n \lambda_i g_i \circ T_i.
\]
With this estimation in hand,
one shows easily that if measure $\mathbb{P} \in P(P(M))$ on $P(M)$ give mass to closed set $\mathcal{A}_L$,
then it has a unique absolutely continous barycenter.

\subsubsection{Jacobian determinant inequality for the Wasserstein barycenter}

This is done by Kim and Pass.
\begin{defn}[Volume distortion]
	Let \( \lambda \) be a Borel probability measure on \( M \) with a
	unique barycenter \( \bar { x } \) (that is, such that \( B C ( \lambda ) \) is a singleton). We define the generalized,
	or barycentric, volume distortion coefficients at \( y \notin \operatorname { cut } ( \bar { x } ) \)

	\[ \alpha _ { \lambda } ( y ) : = \frac { \operatorname { det } \left[ - \left. D _ { y z } ^ { 2 } \right| _ { z = \bar { x } } c ( y , z ) \right] } { \operatorname { det } \left[ \left. \int _ { M } D _ { z z } ^ { 2 } \right| _ { z = \bar { x } } c ( x , z ) d \lambda ( x ) \right] } \]
	where \( D _ { z z } ^ { 2 } c ( x , z ) \) denotes the Hessian of the function \( z \mapsto c ( x , z ) \), and the determinants
	are computed in exponential local coordinates at \( \bar { x } \) and \( y . \)
\end{defn}

\begin{thm}
	[Jacobian determinant inequality for the Wasserstein barycenter]
	Assume that the Wasserstein barycenter \( \bar { \mu } \) of the measure \( \Omega \) on \( P ( M ) \) is absolutely continuous.
	Letting \( T _ { \mu } \) denote the optimal map from \( \bar { \mu } \) to \( \mu \), consider the measure on \( M \) given by
	\[ \lambda _ { x } : = \int _ { P ( M ) } \delta _ { T _ { \mu } ( x ) } d \Omega ( \mu ) \]
	which is defined with respect to a.e. $x$.
	Then, for \( \bar { \mu } \)-a.e. \(x\),
	\[ 1 \geq \int _ { P ( M ) } \alpha _ { \lambda _ { x } } ^ { 1 / n } \left( T _ { \mu } ( x ) \right) \operatorname { det } ^ { 1 / n } D T _ { \mu } ( x ) d \Omega ( \mu ) \]
\end{thm}

% \subsubsection{Use local coordinate}
% Need to work on it.

\subsubsection{Use Skorohod representation}
\textcolor{red}{This is not possible!}
One may consider to construct absolutely continous random variables.
For example, to use the Skorohod representation (see section 8.5 in \cite{Bogachev2007}),
\begin{defn}
	We shall say that a topological space \( X \) has the strong
	Skorohod property for Radon measures if to every Radon probability measure
	\( \mu \) on \( X \),
	one can associate a Borel mapping \( \xi _ { \mu } : [ 0,1 ] \rightarrow X \) such that \( \mu \) is
	the image of Lebesgue measure under the mapping \( \xi _ { \mu } \) and \( \xi _ { \mu _ { n } } ( t ) \rightarrow \xi _ { \mu } ( t ) \) a.e.
	whenever the measures \( \mu _ { n } \) converge weakly to \( \mu . \)
\end{defn}

However, even if $\mu$ is absolutely continous respect to Lebesgue measure on $\mathbb{R}$,
we don't have necessarily that $\xi_{\mu}$ is a absolutely continous function.
In fact, the Housdorff dimension of the image of $\xi_{\mu}$ is not likely to be greater than 2.
See \cite{Besicov1937Sets} for discussions on $\delta$-Lipschitz curves,
their Housdorff dimensions are bounded by $2-\delta$.

% \subsection{Control on density function}

%! TEX root = ../barycenter.tex
\section{Uniqueness of barycenter over Wasserstein space}

For optimal transportation on Riemannian manifold,
\cref{thm:optimal_transport_manifold} gives a detailed description.
However, this theorem requires measures in consideration to have compact supports.

In our discussion,
we have no information in advance
whether a barycenter measure has compact support.
We can get following uniqueness on transfer map
without explicit infomation of it.
See Theorem 10.41 in \cite{villani2008optimal} for full details.
\begin{thm}[Uniqueness of Solution of the Monge problem for the square distance]
	\label{thm:uniquness_monge_problem_manifold}
	Let \( M \) be a Riemannian manifold, and \( c ( x , y ) = d ( x , y ) ^ { 2 } \).
	Let \( \mu , \nu \) be two probability measures on \( M \), such that the optimal cost
	between \( \mu \) and \( \nu \) is finite. If \( \mu \) is absolutely continuous,
	% then there is a
	% unique solution of the Monge problem between \( \mu \) and \( \nu \).
	% volume measure on \( M . \) 
	then the Monge-Kantorovich mass transportation
	problem between \( \mu \) and \( \nu \) admits a unique optimal transference plan, and it
	has the form \( d \pi ( x , y ) = d \mu ( x ) \delta [ y = T ( x ) ] \), or equivalently
	\[ \pi = ( \mathrm { Id } \times T ) \# \mu , \]
	where \( T \) is uniquely determined, \( \mu \)-almost everywhere, by the requirements
	that \( T \# \mu = \nu \).
\end{thm}

With the help of this theorem, we are ready to prove
\begin{prop}
	For measure $\mathbb{P} \in \mathcal{W}_2(\mathcal{W}_2(M))$ on $\mathcal{W}_2(M)$,
	assume $\mathbb{P}$ gives mass to the set of absolutely continous measures $P_{ac}(M)$, $\mathbb{P}(P_{ac}(M)) \geq 0 $.
	Then $\mathbb{P}$ has a unique barycenter measure on $M$.
\end{prop}

\begin{proof}
	We consider space $\mathcal{W}_2(\mathcal{W}_2(M))$, probability measures on $\mathcal{W}_2(M)$,
	any convex combination of elements in this space is still inside it.
	The squared Wasserstein distance function $W^2_2(\mu, \cdot)$ is convex by definition of optimal plan,
	for $0 \leq \lambda_1, \lambda_2 \leq 1, \gamma_1 + \gamma_2 =1 $ and $ \nu_1,\nu_2 \in \mathcal{W}_2(M)$,
	\begin{equation}
		\label{equa:convexity_Wassersetein_distance}
		W_2^2(\mu, \lambda_1 \nu_2 + \lambda_2 \nu_2) \leq \lambda_1 W_2^2(\mu, \nu_1) + \lambda_2 W_2^2(\mu, \nu_2).
	\end{equation}

	And when $\mu \in \mathcal{W}_2(M)$ is absolutely continuous, convexity above becomes strict convexity.
	It means inequality \cref{equa:convexity_Wassersetein_distance} becomes equality only when
	$\lambda_1=0$ or $\lambda_2=0$.
	To prove this claim,
	through \cref{thm:uniquness_monge_problem_manifold} we write
	$\gamma_1 : = (\text{Id}  \times T_1)_{\#}\mu$ the optimal plan from $\mu$ to $\nu_1$ and
	$\gamma_2 : = (\text{Id}  \times T_2)_{\#}\mu$ the optimal plan from $\mu$ to $\nu_2$.

	Set $\gamma := \lambda_1 \gamma_1 + \lambda_2 \gamma_2$ for $\gamma_1, \gamma_2$ that turn
	\cref{equa:convexity_Wassersetein_distance} into equality,
	we have
	\begin{align*}
		\lambda_1 W_2^2(\mu, \nu_1) + \lambda_2 W_2^2(\mu, \nu_2) & = W_2^2(\mu, \lambda_1 \nu_2 + \lambda_2 \nu_2)            \\
																															& \leq \int_{M \times M} d(x,y)^2 \diff \gamma(x,y)                    \\
		                                                          & =	\lambda_1 W_2^2(\mu, \nu_1) + \lambda_2 W_2^2(\mu, \nu_2)
	\end{align*}
	Then $\gamma$ is an optimal plan from $ \mu$ to $\lambda_1 \nu_1 + \lambda_2 \nu_2$,
	but it is not in the form of transform map unless $\lambda_1 =0$ or $\lambda_2 =0$.

	After integration with respect to $\mathbb{P} \in \mathcal{W}_2(\mathcal{W}_2(M))$,
	we get a convex function $\int_{\mathcal{W}_2(M)} W_2^2(\mu, \cdot) \diff \mathbb{P}(\mu)$.
	And it is strictly convex if $\mathbb{P}$ gives mass to absolutely continuous measures.
	Hence, barycenter of $\mathbb{P}$ is unique.
\end{proof}

\section{Absolutely continuity of barycenter}

We discuss in the context of proof of \cref{thm:barycenter_finite_support_measure}.

\subsection{Barycenter of finite supported measure}

Consider a measure $\sum_{i}^{n} \lambda_{i} \delta_{\mu_i}$ on $P_{2}(M)$ for $M$ a complete Riemannian manifold.
Assume from now on \textcolor{cyan}{$ \lambda_1 \neq 0$ and
	that $\mu_1$ is absolutely continous with compact support}.

% \textcolor{red}{Later the assumption of compact support could be removed}?

The barycenter $\bar{\mu}$ of measure $\sum_{i}^{n} \lambda_{i} \delta_{\mu_i}$ is unique,
because this meause gives mass to element in $P_{ac}(M)$.

In the case that \textcolor{cyan}{$\bar{\mu}$ is absolutely continous},
there is only one element in the set $\Gamma$ by uniqueness of optimal plans.
We can get every $\mu_i$ from $\mu_1$ if first push $\mu_1$ to $\bar{\mu}$ through $T_1^{-1}$ then push $\bar{\mu}$ to $\mu_i$ through $T_i$. Hence,
\[B \circ (T_1, T_2, \ldots, T_n) \circ T_1^{-1} = T_1^{-1}\]
is the unique transfer map from $\mu_1$ to $\bar{\mu}$. That is to say, $\bar{\mu}$-a.e., $B \circ (T_1, T_2, \ldots, T_n) $ is the identity map.
This is already included in \cref{lem:inverse_barycenter}.

\subsubsection{One measure absolutely continous and others Dirac}
We firstly consider the case when \textcolor{cyan}{$\mu_i = \delta_{x_i}, i \geq 2$ are Dirac measures}.

In this case, there is only one element in the set $\Gamma$.
As a consequence, $\exp(-\nabla f_1 / \lambda_1)$ pushes $\mu_1$ to $\bar{\mu}$.
As a left inverse to $\exp(-\nabla f_1/\lambda_1)$,
\(\exp(-\nabla g_1)\) pushes $\bar{\mu}$ to $\mu_1$.

We have in general that the $c$-conjugate of $f_1 / \lambda_1$ satisfies
$(f_1 / \lambda_1)^c = g_1^{cc} \geq g_1$,
and thus $c(w, x_1) \geq (f_1 / \lambda_1)^c(w) + f_1/\lambda_1 (x_1) \geq g_1(w) + f_1/\lambda_1 (x_1)$.
When $w$ is barycenter for some $x_1$,
we have $g_1(w) = (f_1 / \lambda_1)^c(w)$.
Hence, from the definition of $\bar{\mu}$ we have
$g_1 = (f_1 /\lambda_1)^c$ for $\bar{\mu}$ alomost everywhere.
% This long inequality implies that any $w$ in the equality
% $c(x_i, w) = f_i / \lambda_i (x_i) + (f_i / \lambda_i)^c(w)$ must be a barycenter of $\sum_{i=1}^{n} \lambda_i \delta_{x_i}$.
% Moreover, we have following equality holds whenever one of gradients exits,
% \[\exp(-\nabla g_i) = \exp(-\nabla (f_i / \lambda_i)^c) = \exp^{-1}(-\nabla f_i / \lambda_i).\]
% Note cut-locus are excluded from consideration in above equality,
% and as a corollary $\nabla g_i = \nabla (f_i / \lambda_i)^c$.
% Therefore, two locally Lipschitz function $g_i$ and $f_i /\lambda_i$ coincide almost everywhere.
% If follows $g_i=f_i /\lambda_i$ on $M$.

We discuss following in the sense of $\bar{\mu}$ almost everywhere. The function $g_1 = (f_1 / \lambda_1)^c $ is by definition just a sum of squared distance functions.
Now we aim to find out when \( \exp(- \nabla g_1 ) = \exp(- \nabla (f_1 / \lambda_1)^c) \) is a Lipschitz function.
% with Lipschitz constant depending only on $M$ and $\lambda_1$.
% As $M$ is compact, though tangent bundle $TM$ is not compact, local Lipschitz plus bounded diameter of $M$ implies global Lipschitz of $\exp$.
% Function $g_1$ has hessian upper bound by \cref{prop:differentiate_optimal_transport},
% here we treat it as an infimal convolution over a fixed sigleton set $\{x_1\}$.
% Note that from $ \nabla g_1 = \nabla (f_1 / \lambda_1)^c$,
% $g_1$ and $(f_1 / \lambda_1)^c$ have the same hessian.
Function $(f_1 / \lambda_1)^c$ has hessian upper bound as a $c$-conjugate function;
function $g_1$ has hessian lower bound as a negative linear combination of square distance function.
We then have
\begin{equation}
	\label{equa:hessian_bound_f}
	-\frac{1-\lambda_1}{\lambda_1} H \leq \nabla^2 (f_1 / \lambda_1)^c \leq H,
\end{equation}
where we denote by $H$ a possible bound from above for the hessian of square distance funtion.
% Recall that $H$ is bounded from above.
% Squared distance function $c$ has bounded Hessian from above.
% Hence $ \| \nabla^2 g \|$ is bounded from above by taking the second derivative of definition and also applying minimallity of cost at barycenter.

Hence, we need two properties to have Lipschitzness of $\exp(-\nabla (f_1 / \lambda_1)^c)$:
\begin{itemize}
	\item $ \exp $ is Lipschitz on the domain we are interested in.
	      % This would be the case when we are in the support of a compact supported function.
	      For example, in a compact domain.
	\item The hessian of square distance function has upper bound $H$.
	      This is the case when we have lower bound of sectional curvature.
\end{itemize}

We always take the second condition in following discussion.
When $x_1$ runs through a bounded set,
all possible barycenters are included in a bounded set.
Hence $\bar{\mu}$ has compact support.
As we actually only consider measures with compact support, the first assumption is fullfilled as well.

% because $br$ pushes $\mu_1$ to $\bar{\mu}$ by construction of $\bar{\mu}$.

Denote by $C$ the Lipschitz constant of map \(\exp(-\nabla(f_1/\lambda_1)^c)\).
If we have $ \text{Vol}(E) < \delta \implies \mu_1(E) < \epsilon$,
then by $\text{Vol}(B^{-1}( E\times \{x^\prime\})) < C^n \text{Vol}(E)$,
where we write $x^\prime = (x_2, \ldots, x_n)$,
we have
\begin{equation}
	\label{equa:absolutely_continuity_estimation}
	\text{Vol}(E) < \delta / C^n \implies \bar{\mu}(E)=\mu_1(B^{-1}(E \times \{x^\prime\})) < \epsilon.
\end{equation}
Thus $\bar{\mu}$ is absolutely continous.
Here the Lipschitz constant $C$ depends on the support of $\mu_1$.
% Since $\exp(-\nabla(f_1 / \lambda_1)^c) = \exp(-\nabla g_1) $ is continous for Vol-a.e. $x_1$
% (outside of cut-locus of $x_j, j \ne 1$),
% we deduce from absolutely continuity of $\bar{\mu}$ and compact support of $\mu_1$ that
% $\bar{\mu}$ has also compact support.

% \begin{rmk}
% 	Once after we prove that $\bar{\mu}$ is indeed absolutely continous.
% 	$\nabla g$ is invertable and $br(x_1) = \exp_{x_1} \nabla g^c(x_1)$.
% 	It is not surprised that
% 	\[
% g^c(x_1) = -c(x_1 ,z) - \frac{1}{\lambda_1} \sum_{i=2}^{n} \lambda_i\, c(x_i, z) =
% - \frac{1}{2 \lambda_1} W^2( \sum_{i=1}^n \lambda_i \delta_{x_i}, M),
% - f_1/\lambda_i.
% \]
% we can then take derivative with respect to $x_1$.
% \end{rmk}

\subsubsection{To a more general case by conditional probability}
% \label{discussion_conditional_prob}
To attack general case when \textcolor{cyan}{$\mu_i, i \geq 2$ are not assumed Dirac measures},
we should consider conditonal measures of optimal multi-marginal transfer plan $\gamma$:
\[
	\diff \gamma(x_1, x^\prime)= \gamma(\diff x_1 \mid x^\prime)\, \diff \pi(x^\prime) ,
\]
where $\pi$ is the projection of $\gamma$ from $x = (x_1, x^\prime)$ to $x^\prime$,
by abuse of language we also denote by it the push-forward measure from $\gamma$.
For $\pi$-a.e $x^\prime$., $\gamma(\cdot \mid x^\prime)$ is a probability measure
concentrated on $M \times \{x^\prime\}$.

One has from definition of conditional measures that,
as function on Borel measurable sets,
\[
	\bar{\mu} = B_{\#} \gamma = \int_{M^{n-1}} B_{\#} \gamma(\cdot \mid x^\prime)\, \diff \pi(x^\prime).
\]


If we have that for $\gamma$-a.e. $x^\prime$,
$B_{\#} \gamma(\cdot \mid x^\prime)$
is absolutely continous,
then $\bar{\mu}(E)=0$ for volume measure zero set $E$ by integration above.
Note that \cref{equa:absolutely_continuity_estimation} also holds by integration.
% \[
% 	\bar{\mu}(E)
% =\int_{M^{n-1}}
% \frac{ \mu_1( br^{-1}(E) \cap A_{x^\prime} )}{\mu_1(A_{x^\prime})}
% \diff \pi(x^\prime)
% =\int_{M^{n-1}}
% \bar{\mu}^{x^\prime}(E)
% \diff \pi(x^\prime)
% \leq \int_{M^{n-1}} C\,\mu_1^{\prime}(E) \diff \pi(x^\prime) \leq C\, \mu_1(E).
% = 0.
% \]

Generally speaking, computation of $\gamma(\cdot \mid x^\prime)$ is only possible when
\begin{itemize}
	\item $\pi$ is has countable support.
	      This is equivalent to that $\mu_i, i \geq 2$ has countable support.
	      Fix any $x^\prime$ in the support of $\pi$, by direct verification
	      \[
		      \gamma(\cdot \mid x^\prime) =
		      \frac{
			      \mathbbm{1}_{M \times \{x^\prime\}}
			      \gamma
		      }{\pi(x^\prime)}.
	      \]
	\item $\gamma$ is absolutely continous with density function $f: M^n \rightarrow \mathbb{R} $.
	      \[
		      \gamma(\diff x_1 \mid x^\prime) =
		      \frac{
			      f(x_1, x^\prime) \diff \text{Vol}(x_1)
			      % \diff \gamma (x_1, x^\prime)
		      }
		      {\int_{M} f(x_1, x^\prime) \diff \text{Vol}(x_1)
		      }
	      \]
	      where we set the right hand side zero if it is undeterminated
	      and we remove this $x^\prime$ from consideration.
\end{itemize}
Note in both cases,
$\gamma(\cdot \mid x^\prime)$
has absolutely continous push-forward measure
$\mu_1^{x^\prime} := \text{proj}^1_{\#}\gamma(\cdot \mid x^\prime)$
to the first coordinate.
But only in the first case that
measure $\gamma(\cdot \mid x^\prime) \leq \pi(x^\prime) \, \gamma$ is controled by $\gamma$.
Or we hope $f$ has positive lower bound,
for example \textcolor{cyan}{continous and strictly positive}, like volume measure.
In the case $\mu_1=\text{Vol}$? Every point in $\text{spt}\mu_1$ runs over all $\times_i \text{spt}\mu_i$.

\textcolor{red}{Unfortunately, this control is required to apply conditioning optimal plan.}
One may consider the case when at least one marginal is discrete,
but no absolutely continuity can be derived without barycenter push-forward.

It is true that we have $\gamma$ is an optimal plan for all its marginals
% \[
% 	B_{\#} \gamma              = \bar{\mu} \text{ and }
% 	\text{proj}^1_{\#} \gamma  = \mu_1,
% \]
and conditioning of (multi-marginal) optimal plan is still optimal from \cref{thm:restriction_optimal_plan}.
Hence, $\gamma(\cdot \mid x^\prime)$ is an optimal plan of its marginals.
% \textcolor{cyan}{in the case of discrete measure}.
Apply previous discussion, we get absolutely continous measure
\[
	\exp(-\frac{1}{\lambda_1}\nabla f_1)_{\#}  \mu_1^{x^\prime}=
	B_{\#}\gamma(\cdot, x^\prime) =
	\bar{\mu}^{x^\prime},
\]
where $\bar{\mu}^{x^\prime}$ is a barycenter measure.

% Since conditioning of optimal maps are still optimal,
% barycenter formula \cref{formula_barycenter} still holds.
% We already know $T_1 := \exp(-\nabla g_1)$ is the optimal transfer map from $\bar{\mu}^{x^\prime}$ to $\mu_1^{x^\prime}$,
% then $T_1 = \exp(- \nabla g_1)$ for $\mu_1^{x^\prime}$ almost everyhere and
% \begin{align*}
% 	\bar{\mu} = B_{\#} \gamma & =
% 	\int_{M^{n-1}}\bar{\mu}^{x^\prime} \, \diff \pi(x^\prime)                                            \\
% 	                          & = \int_{M^{n-1}} \gamma( T_1(\cdot) \mid x^\prime) \diff \pi(x^\prime)   \\
% 														& = \gamma \circ \exp(-\nabla g_1)
% 														% &= \exp (- \frac{1}{\lambda_1} \nabla_1 f)_{\#} \gamma
% 	.
% \end{align*}
% Again, $\bar{\mu}$ has compact support as $T_1$ is Vol-a.e. continous.

% One possible \textbf{investigation}: Conditional optimal plan, not just the simple case of restriction.
% Hence, the push-forward measure
% $\bar{\mu}^{x^\prime} = br_{\#} \diff \mu_1^{x^\prime}$ is absolutely continous as well.
% We thus have
% \[
% 	B_{\#} \diff \gamma = \int_{M^{n-1}} f(\cdot, x^\prime)  \diff \pi(x^\prime) \, \diff \text{Vol}(\cdot).
% \]
% Moreover, the previous integral is in fact a finite sum.
% For a measurable set $E \subset \text{spt}(\bar{\mu})$ with volume measure $0$,

% So $\bar{\mu} = B_{\#} \diff \gamma$ is absolutely continous with respect to $\mu_1$.
% \begin{rmk}
% 	\textcolor{red}{NEED further investigation}, maybe one can use \cref{formula_barycenter}.

% 	We may guess that	the density of $\bar{\mu}$,
% 	$\int_{M^{n-1}} f(\cdot, x^\prime)  \diff \pi(x^\prime)$
% 	is dominated by density of $\mu_1$ up to a constant coefficient.
% 	% Lipschitz constant of \( \exp _ { z } \nabla g ( z ) \).
% \end{rmk}

\subsubsection{To a more general case by consistency of barycenter}

Now we consider measure $\mathbb{P} = \lambda_1 \delta_{\mu_1} + \lambda_2 \delta_{\mu_2}$
without assuming $\mu_2$ discrete.
Note here it is for simplicity that we only consider the case $n=2$.
Approxiamte $\mu_2$ in Wasserstein metric by a sequence of measures $\mu_2^{m}$,
then $\mathbb{P}_m := \lambda_1 \delta_{\mu_1} + \lambda_2 \delta_{\mu_2^m}$ converges to $\mathbb{P}$
in $P_2(P_2(M))$.

By the consistency of barycenters, the unique barycenter $\bar{\mu}_m$ of $\mathbb{P}_m$
converges in Wasserstein metric (or just weakly) to the unique barycenter $\bar{\mu}$ of $\mathbb{P}$.
Recall that there is no duality between $L^{\infty}(\bar{\mu})$ (possible non-separable) and $L^1 (\bar{\mu})$
in functional analysis even when $M$ is compact with Lebesgue measure.
According to Proposition 4.4.2 in \cite{Bogachev2007} below, we need to show that $\bar{\mu}$ is a linear functional on $L^{\infty}(\bar{\mu})$.
\begin{prop}
	Let \( \mu \) be a finite nonnegative measure.
	A continuous linear function \( \Psi \) on \( L ^ { \infty } ( \mu ) \) has the form
	\( \Psi ( f ) = \int _ { X } f g d \mu \)
	, where \( g \in L ^ { 1 } ( \mu ) \),
	precisely when the set function \( A \mapsto \Psi \left( I _ { A } \right) \) is countably additive.
\end{prop}

We define $\Psi(f):= \int_M f \diff \bar{\mu} = \lim_m \int_M f \diff \bar{\mu}_m$ for $f$ continous.
Extend $\Psi$ to be a continous linear functional on $L^{\infty}(\bar{\mu})$ by Hahn-Banach theorem.
Do we still have $\Psi(I_A) = \bar{\mu}(A)$?

A related discussion is available on
\href{https://math.stackexchange.com/questions/574130/does-weak-convergence-with-uniformly-bounded-densities-imply-absolute-continuity/574888#574888}{math Stack Exchange}.
Think about this example carefully.
Let $\lambda$ be the arglength measure and $\phi_n \ge 0$ a continuous function on $\mathbb T$ with $\int_{\mathbb T} \phi_n\, d\lambda = 1$ and $\phi_n(x) = 0$ if $|x-1| \ge \frac 1n$. Then for each continuous function $f\colon \mathbb T \to \mathbb R$ we have $\int_{\mathbb T} f\phi_n d\lambda \to f(1)$, that is $\phi_n \lambda \to \delta_1$ weakly. But $\delta_1$ is not $\lambda$-continuous.

A digression, generally for question if density of $\bar{\mu}_m$ converges to density of $\bar{\mu}$,
we possibly need asymptotically equicontinuous in \cite{Sweeting1986Converse}.

We don't need to find out density explicitly.
Instead, let prove the absolutely continuity by showing that
for any $\epsilon > 0$ there is a $\delta > 0$ such that
\begin{equation}
	\label{equa:absolutely_continous}
	\forall E \subset M \, \text{measurable, } \text{Vol}(E) < \delta
	\implies \bar{\mu}(E) < \epsilon
\end{equation}
In our situation, a \textcolor{red}{uniform $\delta$} can be chosen for all $\bar{\mu}_m$
% because of a discrete situation of previous conditioning optimal plan calculation.
% See the now ``trivial'' \cref{formula_barycenter},
% we can use it to conclude uniform Lipschitz constant for $m$.
because $\bar{\mu}_m$ is pushed by $T_1^{m}$ from $\mu_1$ with a Lipschitz constant bound
independent of $m$.
Recall \cref{equa:absolutely_continuity_estimation} for detail.

Recall for open set $E \subset M$, $\bar{\mu}(E) \leq \liminf \bar{\mu}_m(E)$,
as indicator function of open set is lower semi-continuous.
Hence \cref{equa:absolutely_continous} holds for all open sets.
As Borel measure is outer regular, for general measurable $E$ with $\text{Vol}(E) < \frac{\delta}{2}$,
we select an open set $E^\prime$ such that
$ E \subset E^\prime$ and $ \text{Vol}(E^\prime) < \delta$,
then $\bar{\mu}(E) \leq \bar{\mu}(E^\prime) < \epsilon$.

Previous argument works for general $\lambda_1 \delta_{\mu_1} + \lambda_2 \mathbb{P}$ for any $\mathbb{P} \in P_2(P_2(M))$
if we approxiamte $\mathbb{P}$ by finite supported measures.

To attack even more general case,
we should be able to single out a part of $\mathbb{P}$ in the form $\lambda_1 \delta_{\mu_1}$
with lower bound on $\lambda_1$ and dominated $\mu_1$.
One way to do so is to assume $\mu_1$ has bounded density.
\begin{defn}
	[The set $ \mathcal { A } _ { L }$]
	For \( 0 < L < \infty \), let \( \mathcal { A } _ { L } \subset W_2(M) \) be the set of Borel probability
	measures with compact support on \( M \), absolutely continuous with respect to volume, whose densities have \( L ^ { \infty } \)
	norm less than or equal to \( L\).
\end{defn}

Note that, since the bound on the \( L ^ { \infty } \) norm is preserved under weak-* convergence,
\( \mathcal { A } _ { L } \) is a countable union of weakly-* closed set, and thus Borel measurable, subset of \( W_2( M ) \).

% If $\mu_1 \in \mathcal{A}_L$, then by duality between $L^p(\bar{\mu})$ and $L^q(\bar{\mu})$
% and that they are separable spaces,
% we have that $\bar{\mu}$ has density in $L^p(\bar{\mu})$ for
% $p < \infty$ with a upper norm bound depending only on constat $L$ (, $\lambda_1$ and $M$).
% We pass $p$ to infity to get $\bar{\mu} \in \mathcal{A}_L$.
% Hence, our conclusion holds for any $\mathbb{P}$ that is not atomless on $\mathcal{A}_L$.

Finally, if we are given $\mathbb{P}$ with only assumption that $\mathbb{P}(\mathcal{A}_L(M)) > 0$,
we need more control on the density function of $\bar{\mu}_m$
to ``remove'' the dependency of its upper bound on a singled-out coefficient $\lambda_1$.
One hope is that $\lambda_1$ is replaced by $\mathbb{P}(\mathcal{A}_L(M))$.


\subsection{Calculate density function}

Every measure $\mu$ on $M$ in following discussion is absolutely continous and has compact support.
Here we use the fact that all possible values of $B(x_1, \ldots, x_n)$ are contained in a bounded set if each $x_i$ varies in a bounded set.
To conculde it, since $f$ is continous, there is a upper bound of $f$ and thus a upper bound of distance between $x_i$ and $B(x_1, \ldots, x_n)$.
% We always apply continuities of tranfer maps to get compact support of barycenter measure.
By convention, we denote by $g$ the density function for absolutely measure $\mu$.
One principle of differential geometry is to differentiate everything once we could.

% If we only apply general bound on $c$-concave function
Recall change of variable in \cref{thm:jacobian_identity} (Theorem 4.2 in \cite{cordero2001riemannian}),
\[
	\bar{g} = g_i \circ T_i \det D T_i :
	= g_i \circ T_i \det[Y(H-\text{Hess} u_i)]
\]
where $T_i = \exp(-\nabla u_i)$ is the unique optimal maps from $\bar{\mu}$ to $\mu_i$.
Note that this is why we need that all measures have compact support.

For general $c$-concave function $\mu_i$, we have only upper hessian bound.
And this is not enough to get an estimation on the absolute value of Jacobian determinants.
% For instance, in Euclidean space, $DT_i = (\lambda_i -1)/\lambda_i \leq \text{Id}$.

We differentiate the equality $B(T_1(x), \ldots, T_n(x))=x$ for $\bar{\mu}$-a.e. $x$,
\begin{align*}
	\text{Id} =\sum_{i=1}^n \partial_i B\, DT_i
	 & =\sum_{i=1}^n D \exp(-\frac{1}{\lambda_i}\nabla f_i) \, D \exp(-\nabla u_i)                     \\
	 & =\sum_{i=1}^n D \exp^{-1}(-\nabla \left( \frac{f_i}{\lambda_i}\right)^c) \, D \exp(-\nabla u_i) \\
	 & =\sum_{i=1}^n(H-\text{Hess}(f_i / \lambda_i)^c)^{-1}\,Y_i^{-1}\,
	Y_i\,(H-\text{Hess} u_i)                                                                           \\
	 & =\sum_{i=1}^n(H-\text{Hess}(f_i / \lambda_i)^c)^{-1}\,
	(H-\text{Hess}u_i) .
\end{align*}

Then by Minkowski's determinant inequality, we get
\begin{align*}
	1 & \geq \sum_{i=1}^{n} \det [H-\text{Hess}(f_i/\lambda_i)^c]^{-1/n}\,\det[H-\text{Hess}u_i]^{1/n} \\
	  & =\sum_{i=1}^n \det[\partial_i B]^{1/n}\,\det[DT_i]^{1/n}
\end{align*}
Observe that in our discussion $(f_i / \lambda_i)^c$ is calculated at the barycenter $x$ of $\sum_{j=1}^{n} \delta_{T_j x}$,
and patial derivative means all $T_j x$ for $ j \ne i$ are fixed in calculation.
Hence, we can have $(f_i / \lambda_i)^c = g_i$.
Recall that we know $(f_i / \lambda_i)^c$ has hessian bound from both sides, see \cref{equa:hessian_bound_f}.
From it we get $\det[\partial_i B]^{1/n} \geq \min \{1, \lambda_i / (1 - \lambda_i)\} > \lambda_i$,
where $C > 0$ depends only on hessian bound of square distance function and Lipschitz constant of exponential map.

Combine these two inequalities, and we then apply Jensen inequlity
\[
	\bar{g} \leq
	\left[ \sum_{i=1}^n \frac{\det[\partial_i B]^{1/n}}
	{g_i^{1/n} \circ T_i}\right]^{-n}
	< \left[ \sum_{i=1}^n \frac{C \, \lambda_i}
	{g_i^{1/n} \circ T_i}\right]^{-n}
	\leq C^{-n} \sum_{i=1}^n \lambda_i g_i \circ T_i.
\]
With this estimation in hand,
one shows easily that if measure $\mathbb{P} \in \mathcal{W}_2(\mathcal{W}_2(M))$ on $W_2(M)$ give mass to the measurable set $\mathcal{A}_L$,
then it has a unique absolutely continous barycenter.

\subsubsection{Jacobian determinant inequality for the Wasserstein barycenter}

This is done by Kim and Pass.
\begin{defn}[Volume distortion]
	Let \( \lambda \) be a Borel probability measure on \( M \) with a
	unique barycenter \( \bar { x } \) (that is, such that \( B C ( \lambda ) \) is a singleton). We define the generalized,
	or barycentric, volume distortion coefficients at \( y \notin \operatorname { cut } ( \bar { x } ) \)

	\[ \alpha _ { \lambda } ( y ) : = \frac { \operatorname { det } \left[ - \left. D _ { y z } ^ { 2 } \right| _ { z = \bar { x } } c ( y , z ) \right] } { \operatorname { det } \left[ \left. \int _ { M } D _ { z z } ^ { 2 } \right| _ { z = \bar { x } } c ( x , z ) d \lambda ( x ) \right] } \]
	where \( D _ { z z } ^ { 2 } c ( x , z ) \) denotes the Hessian of the function \( z \mapsto c ( x , z ) \), and the determinants
	are computed in exponential local coordinates at \( \bar { x } \) and \( y . \)
\end{defn}

\begin{thm}
	[Jacobian determinant inequality for the Wasserstein barycenter]
	Assume that the Wasserstein barycenter \( \bar { \mu } \) of the measure \( \Omega \) on \( P ( M ) \) is absolutely continuous.
	Letting \( T _ { \mu } \) denote the optimal map from \( \bar { \mu } \) to \( \mu \), consider the measure on \( M \) given by
	\[ \lambda _ { x } : = \int _ { P ( M ) } \delta _ { T _ { \mu } ( x ) } d \Omega ( \mu ) \]
	which is defined with respect to a.e. $x$.
	Then, for \( \bar { \mu } \)-a.e. \(x\),
	\[ 1 \geq \int _ { P ( M ) } \alpha _ { \lambda _ { x } } ^ { 1 / n } \left( T _ { \mu } ( x ) \right) \operatorname { det } ^ { 1 / n } D T _ { \mu } ( x ) d \Omega ( \mu ) \]
\end{thm}

% \subsubsection{Use local coordinate}
% Need to work on it.

\subsubsection{Use Skorohod representation}
\textcolor{red}{This is not possible!}
One may consider to construct absolutely continous random variables.
For example, to use the Skorohod representation (see section 8.5 in \cite{Bogachev2007}),
\begin{defn}
	We shall say that a topological space \( X \) has the strong
	Skorohod property for Radon measures if to every Radon probability measure
	\( \mu \) on \( X \),
	one can associate a Borel mapping \( \xi _ { \mu } : [ 0,1 ] \rightarrow X \) such that \( \mu \) is
	the image of Lebesgue measure under the mapping \( \xi _ { \mu } \) and \( \xi _ { \mu _ { n } } ( t ) \rightarrow \xi _ { \mu } ( t ) \) a.e.
	whenever the measures \( \mu _ { n } \) converge weakly to \( \mu . \)
\end{defn}

However, even if $\mu$ is absolutely continous respect to Lebesgue measure on $\mathbb{R}$,
we don't have necessarily that $\xi_{\mu}$ is a absolutely continous function.
In fact, the Housdorff dimension of the image of $\xi_{\mu}$ is not likely to be greater than 2.
See \cite{Besicov1937Sets} for discussions on $\delta$-Lipschitz curves,
their Housdorff dimensions are bounded by $2-\delta$.

% \subsection{Control on density function}





\printbibliography
\end{document}
