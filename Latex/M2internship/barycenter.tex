\documentclass{report}

\usepackage{amsthm, amsmath, amsfonts, mathrsfs, amssymb}
\newtheorem{thm}{Theorem}[section]
\newtheorem{prop}[thm]{Proposition}
\newtheorem{coro}[thm]{Corollary}
\newtheorem{lem}[thm]{Lemma}
\theoremstyle{remark}
\newtheorem{rmk}{Remark}[chapter]
\theoremstyle{definition}
\newtheorem{defn}{Definition}[chapter]

\usepackage{biblatex}
\addbibresource{Bibliography.bib}

\usepackage{hyperref}
\hypersetup{
    colorlinks=true,
    linkcolor=blue,
    filecolor=magenta,
    urlcolor=cyan,
}

\newcommand{\diff}{\operatorname{d}}

\title{Existence and Uniqueness of Barycenters in various settings}
\author{Jianyu MA}
\date{\today}

\begin{document}
\pagenumbering{Alph}
% \begin{titlepage}
\maketitle

\begin{abstract}
	Barycenter, as a geometric concept, has easy generalization in various abstract spaces. Further applications of such a concept rely heavily on its existence and uniqueness. In the report, we study this problem in geometrical and measure-theoretical settings.

	Current working plan is:
	\begin{itemize}
		\item Verify example of non-uniqueness in relation of non-branching: four points by M.Bertrand
		\item Hilbert manifold geometric properties
		\item Counter examples of non-uniqueness and reasons
	\end{itemize}
\end{abstract}

\tableofcontents
\pagenumbering{arabic}
%! TEX root = ../barycenter.tex
\chapter{Papers Review}
% In this chapter, I shall review in order papers: \cite{ohta2012barycenters}, \cite{le2017existence}.
\section{Barycenters in Alexandrov spaces of curvature bounded below}
\begin{prop}[Example 3.1 a]
	Let \( X \) be the infinite dimensional ellipsoid of axes of lengths \( c _ { n } = ( n + 1 ) / 2 n \) with \( n \in \mathbb { N } , \) namely 
	\[
	X = \left\{ \left( x _ { 1 } , x _ { 2 } , \ldots \right) \in \mathbb { R } ^ { \infty } \mid \sum _ { n \in \mathbb { N } } \frac { x _ { n } ^ { 2 } } { c _ { n } ^ { 2 } } = 1 \right\} 
\]

	Then \( X \) is complete, but \( \mu = \left( \delta _ { ( 1,0,0 , \ldots ) } + \delta _ { ( - 1,0,0 , \ldots ) } \right) / 2 \) has no barycenter in \( X \).
\end{prop}

\begin{proof}
	As $ 1 / 2 \leq c_{n} \leq 1$, we know $X$ is closed subspace of Hilbert space $\ell^2$. Hence we can calculate distance through inner product. Pick $ x \in X$ and set $ e=(1,0,0\ldots)$, we have:
	\begin{align*}
		W_2(\delta_x, \mu)^2 & = \int_{X} d(x, \cdot)^2 \diff \mu = \frac{d(x,e)^2+d(x,-e)^2 }{2} \\
		                     & =\frac{\Vert x - e \Vert^2 + \Vert x + e \Vert^2}{2}               \\
		                     & =\Vert x \Vert^2 + 1
	\end{align*}

	Hence a barycenter of $\mu$ in $X$ should minimize its length. For a vector $x$ in $X$ to attain minimum length, if restricting to the first $n$ coordinate components, $x$ cannot have nonzero components except the last one (minimum length axes), i.e., $x_i =0$ for $i<n$. Otherwise we can keep other not considered coordinate components, $x_j$ for $j>n$,  unchanged but vanish first $n-1$ coordinate components to get a strictly shorter vector in $X$. This indicates that no such barycenter $x$ could exist in $X$.
\end{proof}

\begin{rmk}
	\begin{enumerate}
		% \item Hilbert space is not locally compact, as unit closed ball in infinite dimensional normed space is not sequential compact. Recall that the last claim can be proved using \href{https://en.wikipedia.org/wiki/Riesz%27s_lemma}{Wikipedia: Riesz's lemma}.
		% \item  $X$ is the unit sphere of $\ell^2$ with a different inner product: \[\langle x, y \rangle:=\sum_{i \in \mathbb{N}^*}\frac{x_i}{c_i}\cdot \frac{y_i}{c_i}.\]
		% We see that $X$ is homeomorphic to the unit sphere in $\ell^2$. From previous point, $X$ is then not compact. However, this conclusion is almost trivial as we don't have barycenter for $\mu$ in $X$.
		\item $X$ is in fact not locally compact as a subspace of $\ell^2$. Consider the sequence with only one nonzero coordinate component, it has no converging subsequence.
		      % as proved in \cite{bessaga1975selected} Chapter IV §2 Theorem 2.1, that unit sphere in $\ell^2$ is homeomorphic to $\ell^2$.
		\item $X$ with metric inherited form $\ell^2$ is apparently not a length space. To explore more of this example, we should start to consider induced length metric $\hat{d}$ on $X$ by inherited metric from $\ell^2$.
		\item $X$ is not locally compact with respect to the induced length metric topology. By compatibility of induced length structure with the topology of base space $X$ (see Exercise 2.1.5 in \cite{burago2001course}), open set in original topology is again open in the induced length metric topology. Cover an otherwise compact set of $X$ in new topology by open sets in original topology, we get a contradiction.
		      % \item Unit sphere in $\ell^2$ is geodesic with respect to induced length metric. By Theorem 2.4.16 in \cite{burago2001course}, we only need to show that midpoint in $X$ always exists, i.e., $\forall x,y \in X$, $\exists z \in X$ such that $d(x,z)=d(y,z)=\frac{1}{2}d(x,y)$. For such two points $x$ and $y$, $z$ can be selected from the intersection of $X$ and the hypersurface passing the origin in $\ell^2$ that evenly separates $x$ and $y$.

		      % \item  $X$ is geodesic with respect to induced length metric $\hat{d}$.
		      % Induced length structure in $X$.
		      %Recall that shortest paths are closed under point-wise convergence (Proposition 2.5.17 in \cite{burago2001course}, proved by lower semi-continuity of length function).
		      % We show that distance in $X$ is approximate by curves with finite coordinate components. To begin with, any two points in $X$ with finite coordinate components can be connected by a shortest path. Then pass a point to have infinite coordinate components and later do the same for another point.
		      % We use the fact that points with finite coordinate components are dense in $\ell^2$ and so are they in $X$.
	\end{enumerate}
\end{rmk}


\begin{rmk}[Example 2.1 c]
	% This example is in contradiction with \href{https://en.wikipedia.org/wiki/Hadamard_space}{Wikipedia: Hadamard space}. The author claims Hilbert space is of non-negative curvature, while Wikipedia says ``a normed space is an Hadamard space if and only if it is a Hilbert space".
	Hilbert space satisfies equality in triangle comparison with Euclidean space, hence can be regarded ``flat".
\end{rmk}

Then we discuss the infinitesimal structure of an Alexandrov space \( ( X , d )\).

\begin{defn}[space of directions]
	Fix \( z \in X \) and let \( \hat { \Sigma } _ { z } \) be the set of all unit speed geodesics emanating from \( z \). For \( \gamma , \eta \in \hat { \Sigma } _ { z } , \) by virtue of the curvature bound, the limit
	\begin{equation}
		\label{angle_def}
		\angle _ { z } ( \gamma , \eta ): = \arccos \left( \lim _ { s , t \downarrow 0 } \frac { s ^ { 2 } + t ^ { 2 } - d _ { X }\left( \gamma ( s ) , \eta ( t ) \right) ^ { 2 } } { 2 s t } \right)
	\end{equation}
	exists and is regarded as the angle (pseudo-)distance of \( \hat { \Sigma } _ { z } . \) We define the space of directions \( \left( \Sigma _ { z } , \angle _ { z } \right) \) at \( z \) as the completion of \( \Sigma _ { z } / \sim \) with respect to \( \angle _ { z } , \) where \( \gamma \sim \eta \) if \( \angle _ { z } ( \gamma , \eta ) = 0 . \)
\end{defn}

\begin{defn}[tangent cone]
	The tangent cone \( \left( C _ { z } , d _ { C _ { z } } \right) \) is defined as the Euclidean cone over \( \left( \Sigma _ { z } , \angle _ { z } \right) , \)
	that is to say,
	\begin{align*}
		C _ { z }:                                           & = \Sigma _ { z } \times [ 0 , \infty ) / \Sigma _ { z } \times \{ 0 \}          \\
		d _ { C _ { z } } ( ( \gamma , s ) , ( \eta , t ) ): & = \sqrt { s ^ { 2 } + t ^ { 2 } - 2 s t \cos \angle _ { z } ( \gamma , \eta ) }
	\end{align*}
\end{defn}

\begin{thm}[Computation of distance in tangent cone]
	Let $\gamma$ and  $\eta$ be two geodesics from unit interval into $X$ starting at $z$. Then $\gamma^\prime(0)$ and $\eta^\prime(0)$ are naturally elements in $C_z$. We can calculate their distance as:
	\[  d_{C_z}(\gamma^\prime(0),\eta^\prime(0))=\lim _ { t \downarrow 0 } \frac  {d ( \eta ( t ) , \gamma ( t ) )  }{ t }\]
\end{thm}

\begin{proof}
	This is a calculation of angle between $\gamma^\prime(0)$ and $\eta^\prime(0)$ with a specific selected converging process. To apply angle definition in \cref{angle_def}, normalize $\gamma$ and $\eta$ as $\gamma\left(\vert\gamma^\prime(0)\vert \cdot\right)$ and $\eta\left(\vert\eta^\prime(0)\vert \cdot\right)$
	\[
		\cos\angle _ { z } ( \gamma , \eta ): = \left( \lim _ { s , t \downarrow 0 } \frac { s ^ { 2 } + t ^ { 2 } - d _ { X }\left( \gamma (\vert \gamma^\prime(0)\vert s ) , \eta (\vert \eta^\prime(0)\vert  t ) \right) ^ { 2 } } { 2 s t } \right)
	\]
	To get desired formula, put $ x := \vert  \gamma^\prime(0)\vert s =\vert \eta^\prime(0)\vert t $ and pass $x \downarrow 0$.
\end{proof}

\begin{rmk}[Lemma 4.5, Lang and Schroeder's inequality]
	When passing from finite support measures to general ones, the author should use Wasserstein metric $W_p$ convergence but not only weak convergence. Otherwise, he cannot justify convergence of integrals of distance function $d_{C_z}$. Here we use the fact that finite supported measures is a dense subspace of Wasserstein space $\mathcal{W}_2(X)$ if $X$ is a Polish space (see \cite{villani2008optimal} Theorem 6.18), recalled in this report as \cref{topology_Wasserstein}.
\end{rmk}


%! TEX root = ../barycenter.tex
\chapter{Introduction and preparation}
\section{Definition of barycenter}
To begin with, recall in physics we define barycenter (or written as barycentre), i.e.,  center inertial mass, as a geometic point where we can image gravity acts at. In the case of a system of particles $\boldsymbol{P}_\nu$, $\nu = 1, \ldots , N$, each with mass $m_\nu$ that are located in space with coordinates $\boldsymbol{r}_\nu$, $\nu = 1, \ldots , N$, the coordinate of the center of mass $C$ satisfies condition:
\[
	\sum _{\nu=1}^{N} m_{\nu}\boldsymbol {r}_{\nu} -	\sum _{\nu=1}^{N}m_{\nu}\boldsymbol {r}_C = \sum _{\nu=1}^{N}m_{\nu}(\boldsymbol {r} _{\nu}-\boldsymbol {r}_C )=\boldsymbol {0}.
\]
Then set \( M := \sum _ { \nu = 1 } ^ { N } m _ { \nu } \), we should have
\[
	\boldsymbol { r } _ { C } = \left( \sum _ { \nu = 1 } ^ { N } m _ { \nu } \boldsymbol { r } _ { \nu } \right) / M .
\]
We can characterize barycenter as the unique solution of minimizing problem
\begin{equation}
	\label{barycenter_def}
	\boldsymbol{r}_C = \operatorname{arg} \min_{\boldsymbol{r}} \sum_{\nu= 1}^N \frac{m_\nu}{M} \Vert \boldsymbol{r}_{\nu} - \boldsymbol{r}\Vert^2
\end{equation}
as norm is calculated as inner product on Eculidean space.

To generalize \ref{barycenter_def}, there are two core concepts to keep: metric distance and measure integral.

\section{Geometry and probability background}
We should introduce necessary concepts and recall classic results to help our discussion; for the sake of instruction, we present some classic proofs.
\subsection{Metric geometry}
\subsection{Measure theory}

Let \( X \) be a metric space and \( \mathscr { M } ( X ) \) the space of all measures defined on \( \mathscr { B } _ { X } . \) An element \( \mu \in \mathscr { H } ( X ) \) is a nonnegative, countably additive set function defined on \( \mathscr { B } _ { X } \) with the property \( \mu ( X ) = 1 . \quad C ( X ) \) stands for the space of all bounded real valued continuous functions on \( X \).  We shall now topologize the space \( \mathscr { M } ( X ) \) by defining a base of open neighborhoods for any point \( \mu . \) Consider the family of sets of the form
\( V _ { \mu } \left( f _ { 1 } , f _ { 2 } , \ldots , f _ { k } ; \varepsilon _ { 1 } , \ldots , \varepsilon _ { k } \right) \)
\[ \left\{\nu \in \mathscr { M } ( X ) \, \mid  \left| \int f _ { i } d v - \int f _ { i } d \mu \right| < \varepsilon _ { i } , \quad i = 1,2 , \ldots , k \right\} \]

where \( f _ { 1 } , f _ { 2 } , \ldots , f _ { k } \) are elements from \( C ( X ) \) and \( \varepsilon _ { 1 } , \varepsilon _ { 2 } , \ldots , \varepsilon _ { k } , \) are positive numbers. It is easy to verify that the family of sets obtained by varying \( k , f _ { 1 } , f _ { 2 } , \ldots , f _ { k } , \varepsilon _ { 1 } , \ldots , \varepsilon _ { k } \) satisfies the axioms of a basis for a topology.

\begin{defn}
	We shall refer to this as the weak topology  in \( \mathscr { M } ( X ) \).
\end{defn}

It is then clear that a net \( \left\{ \mu _ { \alpha } \right\} \) of measures converges in the weak topology to a measure \( \mu \) if and only if \( \int f d \mu _ { \alpha } \rightarrow \int f d \mu \) for every \( f \in C ( X ) . \) In such a case we shall say that \( \mu _ { \alpha } \) converges "weakly" to \( \mu \) or \( \mu _ { \alpha } \Rightarrow \mu \) in symbols. Unless otherwise stated, \( \mathscr { M } ( X ) \) will always be considered as a topological space with the weak topology.

We shall first recall a theorem which yields several useful equivalent definitions of the weak topology.

\begin{thm}
	Let \( \mu _ { \alpha } \) be a net in \( \mathscr { M } ( X ) . \) Then the following statements are equivalent:
	\begin{itemize}
		\label{weak_convergence_thm}
		\item \( \mu _ { \alpha } \Rightarrow \mu \)
		\item \( \lim _ { \alpha } \int g d \mu _ { \alpha } = \int g d \mu \) for all \( g \in U ( X ) \) where \( U ( X ) \) is the space of all bounded real valued uniformly continuous functions;
		\item \( \overline { \lim } _ { \alpha } \mu _ { \alpha } ( C ) \leqslant \mu ( C ) \) for every closed set \( C \)
		\item \( \lim _ { \alpha } \mu _ { \alpha } ( G ) \geqslant \mu ( G ) \) for every open set \( G \)
		\item \( \lim _ { \alpha } \mu _ { \alpha } ( A ) = \mu ( A ) \) for every Borel set \( A \) whose boundary has \( \mu \) -measure \( 0 . \)
	\end{itemize}
\end{thm}

For each point \( x \in X \) we shall denote by \( p _ { x } \) the measure degenerate at the point \( x \). Denote \( D = \left\{ p _ { x }: x \in X \right\} \).

\begin{lem}
	\label{dirac_measure_weak_homeomeorphic}
	\( X \) is homeomorphic to the (topological) subset $D$.
\end{lem}

\begin{proof}
	For any point \( x \) and \( g \in C ( X ) , \) we have \( \int g d p _ { x } = g ( x ) \). If \( x _ { \alpha } \rightarrow x _ { 0 } \) then \( g \left( x _ { \alpha } \right) \rightarrow g \left( x _ { 0 } \right) . \) Hence \( p _ { x _ { \alpha } } \Rightarrow p _ { x _ { 0 } } \). Conversely, let \( p _ { x _ { \alpha } } \Rightarrow p _ { x _ { 0 } } \) If \( x _ { \alpha } \) does not converge to \( x _ { 0 } \), there is an open set \( G \) and a subnet \( \left\{ x _ { \beta } \right\} \) such that \( x _ { 0 } \in G \) and \( x _ { \beta } \in X - G \) for all \( \beta . \) Let \( g \) be a continuous function such that \( 0 \leqslant g \leqslant 1 , g \left( x _ { 0 } \right) = 0 \) and \( g ( x ) = 1 \) for \( x \in X - G \). Then \( \int g d p _ { x _ { \beta } } = 1 , \) while \( \int g d p _ { x _ { 0 } } = 0 . \) This is a contradiction. This completes the proof.
\end{proof}

\begin{lem}
	\( D \) is a sequentially closed subset of \( \mathscr { M } ( X ) \).
\end{lem}

\begin{proof}
	Let \( \left\{ x _ { n } \right\} \) be a sequence of points in \( X \) such that \( p _ { x _ { n } } \Rightarrow q \) Suppose \( \left\{ x _ { n } \right\} \) does not have any convergent subsequence. Then the set \( S = \left\{ x _ { 1 } , x _ { 2 } , \ldots \right\} \) is closed and thus is any subset \( C \) of \( S . \) Since \( p _ { x _ { n } } \Rightarrow q \) we have by Theorem \ref{weak_convergence_thm}, \( q ( C ) \geqslant \overline { \lim } p _ { x _ { n } } ( C ) \) for \( C \subseteq S \). It follows that for every infinite subset \( S _ { 1 } \subseteq S , q \left( S _ { 1 } \right) = 1 \). This is a contradiction since \( q \) is a measure.

	Thus there is a subsequence \( \left\{ x _ { n _ { k } } \right\} , x _ { n _ { k } } \rightarrow x . \) By Lemma \ref{dirac_measure_weak_homeomeorphic}, \( q = p _ { x } \). Hence \( D \) is sequentially closed.
\end{proof}


\begin{thm}
	\label{finite_support_approximation}
	Let \( X \) be a separable metric space and \( E \subseteq X \) dense in \( X  \). Then the set of all measures whose supports are finite subsets of \( E \) is dense in \( \mathscr{ M } ( X ) \), the set of all Borel probability measures on $X$.
\end{thm}

\begin{proof}
	This proof is copy from Theorem 6.3 in \cite{parthasarathy2005probability}.

	It is obviously enough to prove that the set of all measures whose supports are finite subsets of \( X \) is dense in \( \mathscr { M } ( X ) . \) Let us denote the class of such measures by \( \mathscr { F } ( X ) . \) It is clear that any measure concentrated in a countable subset of \( X \) is a weak limit of measures from \( \mathscr { F } ( X ) . \) Thus it is sufficient to prove that any measure is a weak limit of measures vanishing outside countable subsets of \( X \).

	Choose and fix \( \mu \in \mathscr { M } ( X ) \). Since \( X \) is separable we can, for each integer \( n , \) write \( X \) as \( \bigcup _ { j } A _ { n j } , A _ { n j } \cap A _ { n k } = \phi \) if \( j \neq k , A _ { n j } \in \mathscr { B } _ { X } \) for all \( n \) and \( j \) and the diameter of \( A _ { n j } \) is \( \leq 1 / n \) for all \( j \). Let \( x _ { n j } \in A _ { n j } \) be arbitrary. Let \( \mu _ { n } \) be the measure with masses \( \mu \left( A _ { n j } \right) \) at the points \( x _ { n j } , \) respectively. Let $g$ be an arbitrary uniformly continuous (we can even assume Lipschitz continous here) bounded function, and let \[ \alpha _ { n j } = \inf _ { x \in A _ { n j } } g ( x ) , \quad \beta _ { n j } = \sup _ { x \in A _ { n j } } g ( x ) \]
	Since \( g \) is uniformly continuous and since the diameter of \( A _ { n j } \rightarrow 0 \) as \( n \rightarrow \infty \) uniformly in \( j , \sup _ { j } \left( \beta _ { n j } - \alpha _ { n j } \right) \rightarrow 0 \) as \( n \rightarrow \infty \). Now
	\begin{align*}
		\left| \int g d \mu _ { n } - \int g d \mu \right| & = \left| \sum \int _ { A _ { n j } } \left( g - g \left( x _ { n j } \right) \right) d \mu \right|     \\
		                                                   & \leq \sup_{j} \left( \beta _ { n j } - \alpha _ { n j } \right) \xrightarrow{ n \rightarrow \infty} 0.
	\end{align*}
\end{proof}

\begin{thm}[Prokhorov's theorem]
	If \( \mathcal { X } \) is a Polish space, then a set \( \mathcal { P } \subset P ( \mathcal { X } ) \) is precompact for the weak topology if and only if it is tight, i.e. for any \( \varepsilon > 0 \) there is a compact set \( K _ { \varepsilon } \) such that \( \mu \left[ \mathcal { X } \backslash K _ { \varepsilon } \right] \leq \varepsilon \) for all \( \mu \in \mathcal { P } \).
\end{thm}

\section{Optimal transportation}
We start with a basic definition in probability theory.

\begin{defn}[Coupling]
	Let \( ( \mathcal { X } , \mu ) \) and \( ( \mathcal { Y } , \nu ) \) be two probability spaces. Coupling \( \mu \) and \( \nu \) means constructing two random variables \( X \) and \( Y \) on some probability space \( ( \Omega , \mathbb { P } ) , \) such that law \( ( X ) = \mu \), law \( ( Y ) = \nu . \) The couple \( ( X , Y ) \) is called a coupling of \( ( \mu , \nu ) . \) By abuse of language, the law of \( ( X , Y ) \) is also called a coupling of \( ( \mu , \nu ) \).
\end{defn}

If \( \mu \) and \( \nu \) are the only laws in the problem, then without loss of generality one may choose \( \Omega = \mathcal { X } \times \mathcal { Y } . \) In a more measure-theoretical formulation, coupling \( \mu \) and \( \nu \) means constructing a measure \( \pi \) on \( \mathcal { X } \times \mathcal { Y } \) such that \( \pi \) admits \( \mu \) and \( \nu \) as marginals on \( \mathcal { X } \) and \( \mathcal { Y } \) respectively.


We then define the optimal coupling or optimal transport. Here one introduces a cost function \( c ( x , y ) \) on \( \mathcal { X } \times \mathcal { Y } , \) that can be interpreted as the work needed to move one unit of mass from location \( x \) to location \( y . \) Then one considers the Monge-Kantorovich minimization problem
\[
	\operatorname{inf}  \mathbb { E } c ( X , Y )
\]
where the pair \( ( X , Y ) \) runs over all possible couplings of \( ( \mu , \nu ) ; \) or equivalently, in terms of measures,
\[ \inf \int _ { \mathcal { X } \times \mathcal { Y } } c ( x , y ) d \pi ( x , y ) \]
where the infimum runs over all joint probability measures \( \pi \) on \( \mathcal { X } \times \mathcal { Y } \) with marginals \( \mu \) and \( \nu . \) Such joint measures are called trans- ference plans (or transport plans, or transportation plans); those achieving the infimum are called optimal transference plans.

The first good thing about optimal couplings is that they exist:
\begin{thm}[Existence of an optimal coupling]
	Let \( ( \mathcal { X } , \mu ) \) and \( ( \mathcal { Y } , \nu ) \) be two Polish probability spaces; let \( a: \mathcal { X } \rightarrow \mathbb { R } \cup \{ - \infty \} \) and \( b: \mathcal { Y } \rightarrow \mathbb { R } \cup \{ - \infty \} \) be two upper semicontinuous functions such that \( a \in L ^ { 1 } ( \mu ) , b \in L ^ { 1 } ( \nu ) . \) Let \( c: \mathcal { X } \times \mathcal { Y } \rightarrow \mathbb { R } \cup \{ + \infty \} \) be a lower semicontinuous cost function, such that \( c ( x , y ) \geq a ( x ) + b ( y ) \) for all \( x , y . \) Then there is a coupling of \( ( \mu , \nu ) \) which minimizes the total cost \( \mathbb { E } c ( X , Y ) \) among all possible couplings \( ( X , Y ) \).
\end{thm}
The proof relies on basic variational arguments involving the topology of weak convergence (i.e. imposed by bounded continuous test functions). There are two key properties to check: (a) lower semicontinuity, (b) compactness. These issues are taken care of respectively in two lemmas below.

\begin{lem}[Lower semicontinuity of the cost functional]
	\label{lower_semicontinuity_of_the_cost_functional}
	Let \( \mathcal { X } \) and \( \mathcal { Y } \) be two Polish spaces, and \( c: \mathcal { X } \times \mathcal { Y } \rightarrow \mathbb { R } \cup \{ + \infty \} \) a lower semicontinuous cost function. Let \( h: \mathcal { X } \times \mathcal { Y } \rightarrow \mathbb { R } \cup \{ - \infty \} \) be an upper semicontinuous function such that \( c \geq h . \) Let \( \left( \pi _ { k } \right) _ { k \in \mathbb { N } } \) be a sequence of
	probability measures on \( \mathcal { X } \times \mathcal { Y } , \) converging weakly to some \( \pi \in P ( \mathcal { X } \times \mathcal { Y } ) \), in such a way that \( h \in L ^ { 1 } \left( \pi _ { k } \right) , h \in L ^ { 1 } ( \pi ) , \) and
	\[ \int _ { \mathcal { X } \times \mathcal { Y } } h d \pi _ { k } \underset { k \rightarrow \infty } { \longrightarrow } \int _ { \mathcal { X } \times \mathcal { Y } } h d \pi \]
	\[ \int _ { \mathcal { X } \times \mathcal { Y } } c d \pi \leq \liminf _ { k \rightarrow \infty } \int _ { \mathcal { X } \times \mathcal { Y } } c d \pi _ { k } \]
	In particular, if \( c \) is nonnegative, then \( F: \pi \rightarrow \int c d \pi \) is lower semicontinuous on \( P ( \mathcal { X } \times \mathcal { Y } ) , \) equipped with the topology of weak convergence.
\end{lem}

\begin{proof}
 Replacing \( c \) by \( c - h , \) we may assume that \( c \) is a nonnegative lower semicontinuous function. Then \( c \) can be written as the pointwise limit of a nondecreasing family \( \left( c _ { \ell } \right) _ { \ell \in \mathbb { N } } \) of continuous real-valued functions. By monotone convergence,
	\[ \int c d \pi = \lim _ { \ell \rightarrow \infty } \int c _ { \ell } d \pi = \lim _ { \ell \rightarrow \infty } \lim _ { k \rightarrow \infty } \int c _ { \ell } d \pi _ { k } \leq \liminf _ { k \rightarrow \infty } \int c d \pi _ { k } \]
\end{proof}
\begin{lem}[Tightness of transference plans]
	Let \( \mathcal { X } \) and \( \mathcal { Y } \) be two Polish spaces. Let \( \mathcal { P } \subset P ( \mathcal { X } ) \) and \( \mathcal { Q } \subset P ( \mathcal { Y } ) \) be tight subsets of \( P ( \mathcal { X } ) \) and \( P ( \mathcal { Y } ) \) respectively. Then the set \( \Pi ( \mathcal { P } , \mathcal { Q } ) \) of all transference plans whose marginals lie in \( \mathcal { P } \) and \( \mathcal { Q } \) respectively, is itself tight in \( P ( \mathcal { X } \times \mathcal { Y } ) \).
\end{lem}

\begin{proof}
 Let \( \mu \in \mathcal { P } , \nu \in \mathcal { Q } , \) and \( \pi \in \Pi ( \mu , \nu ) . \) By assump- tion, for any \( \varepsilon > 0 \) there is a compact set \( K _ { \varepsilon } \subset \mathcal { X } , \) independent of the choice of \( \mu \) in \( \mathcal { P } \), such that \( \mu \left[ \mathcal { X } \backslash K _ { \varepsilon } \right] \leq \varepsilon ; \) and similarly there is a compact set \( L _ { \varepsilon } \subset \mathcal { Y } , \) independent of the choice of \( \nu \) in \( \mathcal { Q } , \) such that \( \nu \left[ \mathcal { Y } \backslash L _ { \varepsilon } \right] \leq \varepsilon . \) Then for any coupling \( ( X , Y ) \) of \( ( \mu , \nu ) \)
	\[ \mathbb { P } \left[ ( X , Y ) \notin K _ { \varepsilon } \times L _ { \varepsilon } \right] \leq \mathbb { P } \left[ X \notin K _ { \varepsilon } \right] + \mathbb { P } \left[ Y \notin L _ { \varepsilon } \right] \leq 2 \varepsilon. \]

	The desired result follows since this bound is independent of the coupling, and \( K _ { \varepsilon } \times L _ { \varepsilon } \) is compact in \( \mathcal { X } \times \mathcal { Y }  \).
\end{proof}

By passing to the limit in the equation for marginals, we see that \( \Pi ( \mu , \nu ) \) is closed, so it is in fact compact. Then let \( \left( \pi _ { k } \right) _ { k \in \mathbb { N } } \) be a sequence of probability measures on \( \mathcal { X } \times \mathcal { Y } \), such that \( \int c d \pi _ { k } \) converges to the infimum transport cost. Extracting a subsequence if necessary, we may assume that \( \pi _ { k } \) converges to some \( \pi \in \Pi ( \mu , \nu ) . \) The function \( h: ( x , y ) \longmapsto a ( x ) + b ( y ) \) lies in \( L ^ { 1 } \left( \pi _ { k } \right) \) and in \( L ^ { 1 } ( \pi ) \), and \( c \geq h \) by assumption; moreover, \( \int h d \pi _ { k } = \int h d \pi = \) \( \int a d \mu + \int b d \nu \); so Lemma 4.3 implies
\[ \int c d \pi \leq \liminf _ { k \rightarrow \infty } \int c d \pi _ { k }. \]

Thus $\pi$ is minimizing.
\subsection{Wasserstein space}

\begin{defn}[Wasserstein distances]
	\label{Wasserstein_distance}
	Let  \(( \mathcal { X } , d ) \) be a Polish metric space, and let \( p \in [ 1 , \infty ) . \) For any two probability measures \( \mu , \nu \) on \( \mathcal { X } , \) the Wasserstein distance of order \( p \) between \( \mu \) and \( \nu \) is defined by the formula
	\begin{align*}
		W _ { p } ( \mu , \nu ) & = \left( \inf _ { \pi \in \Pi ( \mu , \nu ) } \int _ { \mathcal { X } } d ( x , y ) ^ { p } d \pi ( x , y ) \right) ^ { 1 / p }                                                      \\
		                        & = \inf \left\{ \left[ \mathbb { E } d ( X , Y ) ^ { p } \right] ^ { \frac { 1 } { p } } , \quad \operatorname { law } ( X ) = \mu , \quad \operatorname { law } ( Y ) = \nu \right\}
	\end{align*}
\end{defn}

\begin{lem}
	\label{lower_semicontinous_Wasserstein_distance}
	Wasserstein distance is lower semi-continous with respect to weakly convergence of measures on $P(\mathscr{X})$.
\end{lem}

\begin{proof}
	This is a direct consequence of lower semicontinuity of the cost functional, see lemma \ref{lower_semicontinuity_of_the_cost_functional}.
\end{proof}

\begin{defn}[Wasserstein space]
	\label{Wasserstein_space}
	With the same conventions as in Definition \ref{Wasserstein_distance} , the Wasserstein space of order \( p \) is defined as
	\[
		P _ { p } ( \mathcal { X } ): = \left\{ \mu \in P ( \mathcal { X } ) ; \quad \int _ { \mathcal { X } } d \left( x _ { 0 } , x \right) ^ { p } \mu ( d x ) < + \infty \right\}
	\]
	where \( x _ { 0 } \in \mathcal { X } \) is arbitrary. This space does not depend on the choice of the point \( x _ { 0 } \). Then \( W _ { p } \) defines a (finite) distance on \( P _ { p } ( \mathcal { X } ) \).
\end{defn}

\begin{defn}[Weak convergence in \( P _ { p } \) ]
	Let \( ( \mathcal { X } , d ) \) be a Polish space, and \( p \in [ 1 , \infty ) \). Let \( \left( \mu _ { k } \right) _ { k \in \mathbb { N } } \) be a sequence of probability measures in \( P _ { p } ( X ) \) and let \( \mu \) be another element of \( P _ { p } ( \mathcal { X } ) . \) Then \( \left( \mu _ { k } \right) \) is said to converge weakly in \( P _ { p } ( \mathcal { X } ) \) if any one of the following equivalent properties is satisfied for some (and then any) \( x _ { 0 } \in \mathcal { X }: \)
	\begin{enumerate}
		\item \( \mu _ { k } \longrightarrow \mu \) and \( \int d \left( x _ { 0 } , x \right) ^ { p } d \mu _ { k } ( x ) \longrightarrow \int d \left( x _ { 0 } , x \right) ^ { p } d \mu ( x ) \)
		\item \( \mu _ { k } \longrightarrow \mu \) and \( \limsup _ { k \rightarrow \infty } \int d \left( x _ { 0 } , x \right) ^ { p } d \mu _ { k } ( x ) \leq \int d \left( x _ { 0 } , x \right) ^ { p } d \mu ( x ) \)
		\item \( \mu _ { k } \longrightarrow \mu \) and \( \lim _ { R \rightarrow \infty } \limsup _ { k \rightarrow \infty } \int _ { d \left( x _ { 0 } , x \right) \geq R } d \left( x _ { 0 } , x \right) ^ { p } d \mu _ { k } ( x ) = 0 \)
		\item For all continuous functions \( \varphi \) with \( | \varphi ( x ) | \leq C \left( 1 + d \left( x _ { 0 } , x \right) ^ { p } \right) \)
		      \( C \in \mathbb { R } \), one has
		      \[ \int \varphi ( x ) d \mu _ { k } ( x ) \longrightarrow \int \varphi ( x ) d \mu ( x ) \]
	\end{enumerate}
\end{defn}

\begin{thm}[$W _ { p }$  metrizes \( P _ { p } \)]
	Let \( ( \mathcal { X } , d ) \) be a Polish space, and \( p \in [ 1 , \infty ) ; \) then the Wasserstein distance \( W _ { p } \) metrizes the weak convergence in \( P _ { p } ( \mathcal { X } ) . \) In other words, if \( \left( \mu _ { k } \right) _ { k \in \mathbb { N } } \) is a sequence of measures in \( P _ { p } ( \mathcal { X } ) \) and \( \mu \) is another measure in \( P ( \mathcal { X } ) , \) then the statements
	\[ \mu _ { k } \text { converges weakly in } P _ { p } ( \mathcal { X } ) \text { to } \mu \]
	and
	\[ W _ { p } \left( \mu _ { k } , \mu \right) \longrightarrow 0 \]
	are equivalent.
\end{thm}

For topology property of Wasserstein space, we refer to Theorem 6.18 in \cite{villani2008optimal}. This result is also a nature extension of general finite support measure approximation theorem \ref{finite_support_approximation}.
\begin{thm}[Toplogy of Wasserstein space]
	\label{topology_Wasserstein}
	Let \( \mathcal { X } \) be a complete separable metric space and \( p \in [ 1 , \infty ) \). Then the Wasserstein space \( P _ { p } ( \mathcal { X } ) , \) metrized by the Wasserstein distance \( W _ { p } , \) is also a complete separable metric space. In short: The Wasserstein space over a Polish space is itself a Polish space. Moreover, any probability measure can be approximated by a sequence of probability measures with finite support.
\end{thm}

\begin{thm}[Displacement interpolation as geodesics]
	\label{geodesic_Wasserstein_space}
	Let \( ( \mathcal { X } , d ) \) be a complete separable, locally compact length space. Let \( p > 1 \) and let \( P _ { p } ( \mathcal { X } ) \) be the space of probability measures on \( \mathcal { X } \) with finite moment of order \( p , \) metrized by the Wasserstein distance \( W _ { p } . \) Then, given any two \( \mu _ { 0 } , \mu _ { 1 } \in P _ { p } ( \mathcal { X } ) , \) and a continuous curve \( \left( \mu _ { t } \right) _ { 0 \leq t \leq 1 } , \) valued in \( P ( \mathcal { X } ) , \) the following properties are equivalent:
	\begin{enumerate}
		\item \( \mu _ { t } \) is the law of \(\gamma _ { t } \) where \( \gamma \) is a random (minimizing, constantspeed) geodesic such that \( \left( \gamma _ { 0 } , \gamma _ { 1 } \right) \) is an optimal coupling;
		\item \( \left( \mu _ { t } \right) _ { 0 \leq t \leq 1 } \) is a geodesic curve in the space \( P _ { p } ( \mathcal { X } ) \)
	\end{enumerate}
	Moreover, if \( \mu _ { 0 } \) and \( \mu _ { 1 } \) are given, there exists at least one such curve.
	More generally, if \( \mathcal { K } _ { 0 } \subset P _ { p } ( \mathcal { X } ) \) and \( \mathcal { K } _ { 1 } \subset \operatorname { P } _ { p } ( \mathcal { X } ) \) are compact subsets of
	\( P ( \mathcal { X } ) , \) then the set of geodesic curves \( \left( \mu _ { t } \right) _ { 0 < t < 1 } \) such that \( \mu _ { 0 } \in \mathcal { K } _ { 0 } \) and
	\( \mu _ { 1 } \in \mathcal { K } _ { 1 } \) is compact and nonempty; and also the set of dynamical opti-
	mal transference plans \( \Pi \) with \( \left( e _ { 0 } \right) _ { \# } \Pi \in \mathcal { K } _ { 0 } , \left( e _ { 1 } \right) _ { \# } \Pi \in \mathcal { K } _ { 1 } \) is compact
	and nonempty.
\end{thm}

\part{Barycenters in Geometrical Spaces}

\chapter{Locally compact complete metric space}

\section{Riemannian manifold}
\subsection{Euclidean spaces}

\chapter{Alexandrov spaces}
\section{Curvature bounded above}
\subsection{Hilbert spaces}
\section{Curvature bounded below}

\chapter{Hilbert Riemannian manifold}

\part{Barycenters in Wasserstein Spaces}

\printbibliography
\end{document}
