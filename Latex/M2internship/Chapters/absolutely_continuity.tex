%! TEX root = ../barycenter.tex
\section{Barycenter in Wasserstein space}

In the previous chapter,
we used consistency to build existence of barycenter in the Wasserstein space
$\mathcal{W}_p(E)$ over a proper space $E$.
And we proved existence for measures of shape $\sum_{j=1}^{n} \lambda_j \delta_{\mu_j}$;
the barycenter we constructed is $B_{\#}\boldsymbol{\gamma} \in \mathcal{W}_p(E)$,
where $B$ is a measurable function from $n$ points $x_j \in E$ to a barycenter of
$\sum_{j=1}^{n} \lambda_j \delta_{x_j}$
and $\boldsymbol{\gamma}$ is an optimal plan with marginals $\mu_j$ with respect to a continuous cost function $f$.

We shall firstly show that barycenter of $\sum_{j=1}^n \lambda_j \delta_{\mu_j}$
is unique if some $\mu_i$
is absolutely continuous with respect to volume measure on the manifold $M$.

\subsection{Uniqueness of barycenter measure}

We start with a description (\cite[Theorem 10.41]{villani2008optimal})
for optimal transportation on Riemannian manifold.
\begin{thm}[Uniqueness of optimal transference map for the square distance]
	\label{thm:uniquness_monge_problem_manifold}
	Let \( M \) be a Riemannian manifold, and \( c ( x , y ) = \frac{1}{2} d ( x , y ) ^ { 2 } \).
	Let \( \mu , \nu \) be two probability measures on \( M \), such that the optimal cost
	between \( \mu \) and \( \nu \) is finite.
	If \( \mu \) is absolutely continuous (with respect to volume measure $\operatorname{Vol}$ on $M$),
	then the Monge-Kantorovich mass transportation
	problem between \( \mu \) and \( \nu \) admits a unique optimal transference plan, and it
	has the form \( \diff \pi ( x , y ) = \diff \mu ( x ) \delta [ y = T ( x ) ] \),
	or equivalently \[ \pi = ( \mathrm { Id } \times T )_{\#} \mu , \]
	where \( T \) is uniquely determined, \( \mu \)-almost everywhere,
	by the requirements that \( T_{\#} \mu = \nu \).
\end{thm}

The map \( T \) may be referred to either as
the optimal map or optimal transference map between \( \mu \) and \( v \).
With the help of this theorem, we are ready to prove
\begin{prop}
	\label{prop:uniquness_barycenter_Wasserstein}
	For measure $\mathbb{P} \in \mathcal{W}_2(\mathcal{W}_2(M))$ on $\mathcal{W}_2(M)$,
	assume $\mathbb{P}$ gives mass to the set of absolutely continous measures $P_{ac}(M) \subset \mathcal{W}_2(M)$,
	i.e.,	$\mathbb{P}(P_{ac}(M)) > 0 $.
	Then $\mathbb{P}$ has a unique barycenter measure on $M$.
\end{prop}

We remark that $P_{ac}(M)$ is a Borel measurable subset of $(\mathcal{W}_2(M), W_2)$,
see \cite[Proposition 2.1]{KIM2017640} for a proof.

\begin{proof}
	We consider space $\mathcal{W}_2(\mathcal{W}_2(M))$, probability measures on $\mathcal{W}_2(M)$,
	any convex combination of elements in this space is still inside it.
	The squared Wasserstein distance function $W^2_2(\mu, \cdot)$ is convex by definition of optimal plan,
	for $0 \leq \lambda_1, \lambda_2 \leq 1, \gamma_1 + \gamma_2 =1 $ and $ \nu_1,\nu_2 \in \mathcal{W}_2(M)$,
	\begin{equation}
		\label{equa:convexity_Wassersetein_distance}
		W_2^2(\mu, \lambda_1 \nu_2 + \lambda_2 \nu_2) \leq \lambda_1 W_2^2(\mu, \nu_1) + \lambda_2 W_2^2(\mu, \nu_2).
	\end{equation}

	And when $\mu \in \mathcal{W}_2(M)$ is absolutely continuous, convexity above becomes strict convexity.
	It means inequality \cref{equa:convexity_Wassersetein_distance} becomes equality only when
	$\lambda_1=0$ or $\lambda_2=0$.
	To prove this claim,
	through \cref{thm:uniquness_monge_problem_manifold} we write
	$\gamma_1 : = (\operatorname{Id}  \times T_1)_{\#}\mu$ the optimal plan from $\mu$ to $\nu_1$ and
	$\gamma_2 : = (\operatorname{Id}  \times T_2)_{\#}\mu$ the optimal plan from $\mu$ to $\nu_2$.

	Set $\gamma := \lambda_1 \gamma_1 + \lambda_2 \gamma_2$ for $\gamma_1, \gamma_2$ that turn
	\cref{equa:convexity_Wassersetein_distance} into equality,
	we have
	\begin{align*}
		\lambda_1 W_2^2(\mu, \nu_1) + \lambda_2 W_2^2(\mu, \nu_2) & = W_2^2(\mu, \lambda_1 \nu_2 + \lambda_2 \nu_2)            \\
		                                                          & \leq \int_{M \times M} d(x,y)^2 \diff \gamma(x,y)          \\
		                                                          & =	\lambda_1 W_2^2(\mu, \nu_1) + \lambda_2 W_2^2(\mu, \nu_2)
	\end{align*}
	Then $\gamma$ is an optimal plan from $ \mu$ to $\lambda_1 \nu_1 + \lambda_2 \nu_2$,
	but it is not in the form of transform map unless $\lambda_1 =0$ or $\lambda_2 =0$.

	After integration with respect to $\mathbb{P} \in \mathcal{W}_2(\mathcal{W}_2(M))$,
	we get a convex function $\int_{\mathcal{W}_2(M)} W_2^2(\mu, \cdot) \diff \mathbb{P}(\mu)$.
	And it is strictly convex if $\mathbb{P}$ gives mass to absolutely continuous measures.
	Hence, barycenter of $\mathbb{P}$ is unique.
\end{proof}

\subsection{Absolutely continuity of barycenter for measures with finite support}

All main ideas in proving absolutely continuity come from \cite{KIM2017640},
we basically re-write their proof in following sections.

We start with measure
$\sum_{j}^{n} \lambda_{j} \delta_{\mu_j}$ on $\mathcal{W}_{2}(M)$.
% for $M$ a complete Riemannian manifold.
To use results and notation in \cref{section:barycenter_manifold},
we fix $i = 1$ there.
Recall that we could define $f_1 \in \mathcal{I}( \bar{ \mathcal{X} }, Y)$
as a $c$-concave function.
Assume from now on \textcolor{cyan}{
	$ \lambda_1 \neq 0$ and that $\mu_1$ is absolutely continuous and
	concentrated in the compact set $\bar{ \mathcal{X} }$
}.
The barycenter measure $\bar{\mu}$ of $\sum_{i}^{n} \lambda_{i} \delta_{\mu_i}$ is unique by \cref{prop:uniquness_barycenter_Wasserstein}.
% because this measure gives mass to element in $P_{ac}(M)$.

% \subsubsection{One measure absolutely continous and others Dirac}
% We firstly consider the case when \textcolor{cyan}
\subsubsection{$\mu_i = \delta_{x_i}, i \geq 2$ are Dirac measures}

In this case, there is only one measure $\boldsymbol{\gamma}$ with marginals $\mu_i$.
As a consequence, $\exp(-\nabla f_1 / \lambda_1)$ pushes $\mu_1$ to $\bar{\mu}$.
As a left inverse to $\exp(-\nabla f_1/\lambda_1)$,
\(\exp(-\nabla g_1)\) pushes $\bar{\mu}$ to $\mu_1$.
Then $\bar{\mu}$ is concentrated on the union of barycenters, compact set $Y$.
% We shall show soon $\exp(- \nabla g_1)$ could be almost everywhere Lipschitz
% with respect to volume measure if $M$ has lower bound for Ricci curvature.
% Now assume that is the case.
% This long inequality implies that any $w$ in the equality
% $c(x_i, w) = f_i / \lambda_i (x_i) + (f_i / \lambda_i)^c(w)$ must be a barycenter of $\sum_{i=1}^{n} \lambda_i \delta_{x_i}$.
% Moreover, we have following equality holds whenever one of gradients exits,
% \[\exp(-\nabla g_i) = \exp(-\nabla (f_i / \lambda_i)^c) = \exp^{-1}(-\nabla f_i / \lambda_i).\]
% Note cut-locus are excluded from consideration in above equality,
% and as a corollary $\nabla g_i = \nabla (f_i / \lambda_i)^c$.
% Therefore, two locally Lipschitz function $g_i$ and $f_i /\lambda_i$ coincide almost everywhere.
% If follows $g_i=f_i /\lambda_i$ on $M$.
% We discuss following in the sense of $\bar{\mu}$ almost everywhere.
% The function $g_1$ is by definition just a sum of squared distance functions.
% Now we aim to find out when $ \exp(- \nabla g_1 )$ is a Lipschitz function.
% with Lipschitz constant depending only on $M$ and $\lambda_1$.
% As $M$ is compact, though tangent bundle $TM$ is not compact, local Lipschitz plus bounded diameter of $M$ implies global Lipschitz of $\exp$.
% Function $g_1$ has hessian upper bound by \cref{prop:differentiate_optimal_transport},
% here we treat it as an infimal convolution over a fixed sigleton set $\{x_1\}$.
% Note that from $ \nabla g_1 = \nabla (f_1 / \lambda_1)^c$,
% $g_1$ and $(f_1 / \lambda_1)^c$ have the same hessian.
% Assume squared distance function has upper bound,
% then  $g_1$ has hessian upper bound as a $c$-conjugate function;
% function $g_1$ has hessian lower bound as a negative linear combination of square distance function.
% We then have
% \begin{equation}
% 	\label{equa:hessian_bound_f}
% 	-\frac{1-\lambda_1}{\lambda_1} H \leq \nabla^2 g_1, \qquad \nabla^2 (f_1 / \lambda_1)^c \leq H,
% \end{equation}
% where we denote by $H$ a possible bound from above for the hessian of square distance funtion.
% Recall that $H$ is bounded from above.
% Squared distance function $c$ has bounded Hessian from above.
% Hence $ \| \nabla^2 g \|$ is bounded from above by taking the second derivative of definition and also applying minimallity of cost at barycenter.
% Hence, we need two properties to have Lipschitzness of $\exp(-\nabla (f_1 / \lambda_1)^c)$:
% \begin{itemize}
% 	\item $ \exp $ is Lipschitz on the domain we are interested in.
% 	      % This would be the case when we are in the support of a compact supported function.
% 	      For example, in a compact domain.
% 	\item The hessian of square distance function has upper bound $H$.
% 	      This is the case when we have lower bound of sectional curvature.
% \end{itemize}
% We always take the second condition in following discussion.
% When $x_1$ runs through a bounded set,
% all possible barycenters are included in a bounded set.
% Hence $\bar{\mu}$ has compact support.
% As we actually only consider measures with compact support, the first assumption is fullfilled as well.
% because $br$ pushes $\mu_1$ to $\bar{\mu}$ by construction of $\bar{\mu}$.
For $y \in Y$, $g_1$ is smooth by \cref{lem:barycenter_out_of_cut_locus},
so $\exp( - \nabla g_1)$ is Lipschitz around a ball of $y$.
Choose a finte cover for $Y$ from these balls,
and denote by $C$ the global Lipschitz constant of map \(\exp(-\nabla g_1)\) on $Y$.
Absolutely continuity of $\mu_1$ means that given $\epsilon > 0$,
we can find a $\delta >0$ such that $ \operatorname{Vol}(E) < \delta \implies \mu_1(E) < \epsilon$.
On the other hand, we have for measurable set $E \subset Y$,
$\operatorname{Vol}(B^{-1}( E\times \{x^\prime\})) < C^n \text{Vol}(E)$,
where we write $x^\prime = (x_2, \ldots, x_n)$.
Finally,
\begin{equation}
	\label{equa:absolutely_continuity_estimation}
	E \subset Y, \operatorname{Vol}(E) < \delta / C^n \implies \bar{\mu}(E)=\mu_1(B^{-1}(E \times \{x^\prime\})) < \epsilon.
\end{equation}
Thus $\bar{\mu}$ is absolutely continous.
% Here the Lipschitz constant $C$ depends on the support of $\mu_1$.
% Since $\exp(-\nabla(f_1 / \lambda_1)^c) = \exp(-\nabla g_1) $ is continous for Vol-a.e. $x_1$
% (outside of cut-locus of $x_j, j \ne 1$),
% we deduce from absolutely continuity of $\bar{\mu}$ and compact support of $\mu_1$ that
% $\bar{\mu}$ has also compact support.
% We have in general that the $c$-conjugate of $f_1 / \lambda_1$ satisfies
% $(f_1 / \lambda_1)^c = g_1^{cc} \geq g_1$,
% and thus $c(w, x_1) \geq (f_1 / \lambda_1)^c(w) + f_1/\lambda_1 (x_1) \geq g_1(w) + f_1/\lambda_1 (x_1)$.
% When $w$ is barycenter for some $x_1$,
% we have $g_1(w) = (f_1 / \lambda_1)^c(w)$.
% Hence, from the definition of $\bar{\mu}$ we have
% $g_1 = (f_1 /\lambda_1)^c$ for $\bar{\mu}$ alomost everywhere.
% And we have by minimallity that,
% $x_1 = \exp_w(- \nabla g_1(w)) = \exp_w(- \nabla (f_1 / \lambda_1)^c(w))$ once gradients exist.
% Therefore, we have $\nabla g_1 = \nabla (f_1 / \lambda_1)^c$ for volume measure almost everywhere.
% \begin{rmk}
% 	Once after we prove that $\bar{\mu}$ is indeed absolutely continous.
% 	$\nabla g$ is invertable and $br(x_1) = \exp_{x_1} \nabla g^c(x_1)$.
% 	It is not surprised that
% 	\[
% g^c(x_1) = -c(x_1 ,z) - \frac{1}{\lambda_1} \sum_{i=2}^{n} \lambda_i\, c(x_i, z) =
% - \frac{1}{2 \lambda_1} W^2( \sum_{i=1}^n \lambda_i \delta_{x_i}, M),
% - f_1/\lambda_i.
% \]
% we can then take derivative with respect to $x_1$.
% \end{rmk}
% \subsubsection{To a more general case by conditional probability}
% \label{discussion_conditional_prob}
% To attack general case when \textcolor{cyan}{
\subsubsection{$\mu_i, i \geq 2$ are discrete measures}
To proceed with this slightly general case,
we should consider conditonal measures of an optimal multi-marginal transfer plan $\gamma$
with marginals $\mu_i$:
\[
	\diff \gamma(x_1, x^\prime)= \gamma(\diff x_1 \mid x^\prime)\, \diff \pi(x^\prime) ,
\]
where $\pi$ is the projection of $\gamma$ from $x = (x_1, x^\prime)$ to $x^\prime$,
by abuse of language we also denote by it the push-forward measure from $\gamma$.
For $\pi$-a.e $x^\prime$., $\gamma(\cdot \mid x^\prime)$ is a probability measure
concentrated on $M \times \{x^\prime\}$.

One has from the definition of conditional measures that,
as function on Borel measurable sets,
\[
	\bar{\mu} = B_{\#} \gamma = \int_{M^{n-1}} B_{\#} \gamma(\cdot \mid x^\prime)\, \diff \pi(x^\prime).
\]

If we have that for $\gamma$ almost everywhere $x^\prime$ that
$B_{\#} \gamma(\cdot \mid x^\prime)$
is absolutely continous,
then $\bar{\mu}(E)=0$ for volume measure zero set $E$ by integration above.
Note that \cref{equa:absolutely_continuity_estimation} also holds by integration
for some global Lipschitz constant $C$ of $\exp( - \nabla g_1)$ on the compact set of all possible barycenters.
% \[
% 	\bar{\mu}(E)
% =\int_{M^{n-1}}
% \frac{ \mu_1( br^{-1}(E) \cap A_{x^\prime} )}{\mu_1(A_{x^\prime})}
% \diff \pi(x^\prime)
% =\int_{M^{n-1}}
% \bar{\mu}^{x^\prime}(E)
% \diff \pi(x^\prime)
% \leq \int_{M^{n-1}} C\,\mu_1^{\prime}(E) \diff \pi(x^\prime) \leq C\, \mu_1(E).
% = 0.
% \]

Generally speaking, computation of $\gamma(\cdot \mid x^\prime)$ is only possible when
\begin{itemize}
	\item $\pi$ is has countable support.
	      This is equivalent to that all $\mu_i, i \geq 2$ have countable support.
	      Fix any $x^\prime$ in the support of $\pi$, by direct verification
	      \[
		      \gamma(\cdot \mid x^\prime) =
		      \frac{
			      \mathbbm{1}_{M \times \{x^\prime\}}
			      \gamma
		      }{\pi(x^\prime)}.
	      \]
	\item $\gamma$ is absolutely continous with density function $f: M^n \rightarrow \mathbb{R} $.
	      \[
		      \gamma(\diff x_1 \mid x^\prime) =
		      \frac{
			      f(x_1, x^\prime) \diff \operatorname{Vol}(x_1)
			      % \diff \gamma (x_1, x^\prime)
		      }
		      {\int_{M} f(x_1, x^\prime) \diff \operatorname{Vol}(x_1)
		      }
	      \]
	      where we set the right hand side zero if it is undeterminated
	      and we remove this $x^\prime$ from consideration.
\end{itemize}
% Note in both cases, $\gamma(\cdot \mid x^\prime)$
% has absolutely continous push-forward measure
% $\mu_1^{x^\prime} := \text{proj}^1_{\#}\gamma(\cdot \mid x^\prime)$
% to the first coordinate.
But only in the first case that the measure $\gamma(\cdot \mid x^\prime) \leq \pi(x^\prime) \, \gamma$
is controled by $\gamma$.
% Or we hope $f$ has positive lower bound,
% for example \textcolor{cyan}{continous and strictly positive}, like volume measure.
% In the case $\mu_1=\operatorname{Vol}$? Every point in $\text{spt}\mu_1$ runs over all $\times_i \text{spt}\mu_i$.
% \textcolor{red}{Unfortunately, this control is required to apply conditioning optimal plan.}
% One may consider the case when at least one marginal is discrete,
% but no absolutely continuity can be derived without barycenter push-forward.
% It is true that we have $\gamma$ is an optimal plan for all its marginals
% \[
% 	B_{\#} \gamma              = \bar{\mu} \text{ and }
% 	\text{proj}^1_{\#} \gamma  = \mu_1,
% \]
In this case, conditioning of (multi-marginal) optimal plan is still optimal by \cref{thm:restriction_optimal_plan}.
That is to say, $\gamma(\cdot \mid x^\prime)$ is an optimal plan of its marginals.
% \textcolor{cyan}{in the case of discrete measure}.
Apply previous discussion, we know that $B_{\#}\gamma(\cdot, x^\prime)$
is absolutely continous.
% \[
	% \exp(-\frac{1}{\lambda_1}\nabla f_1)_{\#}  \mu_1^{x^\prime}=
% \]
% where $\bar{\mu}^{x^\prime}$ is a barycenter measure.
% Since conditioning of optimal maps are still optimal,
% barycenter formula \cref{formula_barycenter} still holds.
% We already know $T_1 := \exp(-\nabla g_1)$ is the optimal transfer map from $\bar{\mu}^{x^\prime}$ to $\mu_1^{x^\prime}$,
% then $T_1 = \exp(- \nabla g_1)$ for $\mu_1^{x^\prime}$ almost everyhere and
% \begin{align*}
% 	\bar{\mu} = B_{\#} \gamma & =
% 	\int_{M^{n-1}}\bar{\mu}^{x^\prime} \, \diff \pi(x^\prime)                                            \\
% 	                          & = \int_{M^{n-1}} \gamma( T_1(\cdot) \mid x^\prime) \diff \pi(x^\prime)   \\
% 														& = \gamma \circ \exp(-\nabla g_1)
% 														% &= \exp (- \frac{1}{\lambda_1} \nabla_1 f)_{\#} \gamma
% 	.
% \end{align*}
% Again, $\bar{\mu}$ has compact support as $T_1$ is Vol-a.e. continous.
% One possible \textbf{investigation}: Conditional optimal plan, not just the simple case of restriction.
% Hence, the push-forward measure
% $\bar{\mu}^{x^\prime} = br_{\#} \diff \mu_1^{x^\prime}$ is absolutely continous as well.
% We thus have
% \[
% 	B_{\#} \diff \gamma = \int_{M^{n-1}} f(\cdot, x^\prime)  \diff \pi(x^\prime) \, \diff \operatorname{Vol}(\cdot).
% \]
% Moreover, the previous integral is in fact a finite sum.
% For a measurable set $E \subset \text{spt}(\bar{\mu})$ with volume measure $0$,
% So $\bar{\mu} = B_{\#} \diff \gamma$ is absolutely continous with respect to $\mu_1$.
% \begin{rmk}
% 	\textcolor{red}{NEED further investigation}, maybe one can use \cref{formula_barycenter}.
% 	We may guess that	the density of $\bar{\mu}$,
% 	$\int_{M^{n-1}} f(\cdot, x^\prime)  \diff \pi(x^\prime)$
% 	is dominated by density of $\mu_1$ up to a constant coefficient.
% 	% Lipschitz constant of \( \exp _ { z } \nabla g ( z ) \).
% \end{rmk}

\subsubsection{$\mu_i, i \geq 2$ have compact supports}

For simplicity that we only consider the case $n=2$.
Approximate $\mu_2$ in Wasserstein metric by a sequence of measures $\mu_2^{m}$,
then $\mathbb{P}_m := \lambda_1 \delta_{\mu_1} + \lambda_2 \delta_{\mu_2^m}$ converges to $\mathbb{P}$
in $\mathcal{W}_2(\mathcal{W}_2(M))$.

By the consistency of barycenters, the unique barycenter $\bar{\mu}_m$ of $\mathbb{P}_m$
converges in Wasserstein metric to the unique barycenter $\bar{\mu}$ of $\mathbb{P}$.
% We don't need to find out density explicitly.
We prove the absolutely continuity by showing that
for any $\epsilon > 0$ there is a $\delta > 0$ such that
\begin{equation}
	\label{equa:absolutely_continous}
	\forall E \subset M \, \text{measurable, } \operatorname{Vol}(E) < \delta
	\implies \bar{\mu}(E) < \epsilon
\end{equation}
We argue the same way as \cref{equa:absolutely_continuity_estimation}.
In our situation, a uniform $\delta$ can be chosen for all $\bar{\mu}_m$
because a global Lipschitz constant (independent of $m$)
of $\exp( - \nabla g_1)$ exists on the compact set of all possible barycenters.
Note that $\bar{\mu}$ is concentrated on this compact set.
% $\bar{\mu}_m$ is pushed by $T_1^{m}$ from $\mu_1$ with a Lipschitz constant bound
% independent of $m$.

Recall for open set $E \subset M$, $\bar{\mu}(E) \leq \liminf \bar{\mu}_m(E)$,
because indicator function of open set is lower semi-continuous.
Hence, \cref{equa:absolutely_continous} holds for all open sets.
As Borel measure is outer regular, for general measurable $E$ with $\operatorname{Vol}(E) < \frac{\delta}{2}$,
we select an open set $E^\prime$ such that
$ E \subset E^\prime$ and $ \operatorname{Vol}(E^\prime) < \delta$,
then $\bar{\mu}(E) \leq \bar{\mu}(E^\prime) < \epsilon$.
