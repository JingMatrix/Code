%! TEX root = ../barycenter.tex
\chapter{Barycenter in Metric spaces}
\label{chapter:metric_spaces}

In this chapter, we prove barycenter's existence for proper spaces\cite[Lemma 3.2]{ohta2012barycenters}.
Recall that we call $z$ a barycenter of $\mu$ if $ z \in \arg\min_{x}\int_{E} d(x, y)^2 \diff \mu (y)$.
And a metric space is proper if every bounded closed set is compact.
Then we construct some counter-examples in the rest of this chapter.


\section{Barycenter's existence for proper spaces}

A proper (metric) space is complete and separable,
because we can write a proper metric space as a countable union of compact closed balls.
This assumption on compactness ensures the existence of barycenter.

\begin{prop}
	\label{lem:existence_proper_spaces}
	If \( ( E , d ) \) is a proper metric space, then any \( \mu \in \mathcal{W}_ { 2 } ( E ) \) has a barycenter.
\end{prop}

\begin{proof}
	Fix \( z _ { 0 } \in E \) and take \( r > 1 \) large enough to satisfy
	\[ \mu \left( B \left( z _ { 0 } , r \right) \right) \geq \frac { 1 } { 2 } , \quad \int _ { E \backslash B \left( z _ { 0 } , r \right) } d \left( z _ { 0 } , y \right) ^ { 2 } \diff \mu ( y ) \leq 1 \]
	Then we have
	\[ \int _ { E } d \left( z _ { 0 } , y \right) ^ { 2 } \diff \mu ( y ) \leq r ^ { 2 } \cdot \mu \left( B \left( z _ { 0 } , r \right) \right) + 1 \leq r ^ { 2 } + 1 \]
	while for every \( x \in E \backslash B \left( z _ { 0 } , 3 r \right) \)
	\[ \int _ { E } d ( x , y ) ^ { 2 } \diff \mu ( y ) \geq \int _ { B \left( z _ { 0 } , r \right) } d ( x , y ) ^ { 2 } \diff \mu ( y ) > ( 2 r ) ^ { 2 } \cdot \mu \left( B \left( z _ { 0 } , r \right) \right) \geq 2 r ^ { 2 } \]
	holds.
	Therefore, it is sufficient to consider the infimum of
	\[ x \longmapsto \int _ { E } d ( x , y ) ^ { 2 } \diff \mu ( y ) \]
	only for \( x \in B \left( z _ { 0 } , 3 r \right) , \) and it is achieved at some point due to the compactness of the closure of \( B \left( z _ { 0 } , 3 r \right) . \)
\end{proof}

Here is a simple example \cite[Example 3.1 (a)]{ohta2012barycenters}
showing that lack of properness violates the existence of barycenter.

\begin{example}[A complete space with a measure without barycenter]
	\label{example:ellipsoid_subspaces}
	Let \( E \) be the infinite dimensional ellipsoid of axes of length \( c _ { n } = ( n + 1 ) / 2 n \) with \( n \in \mathbb { N}^* , \) namely
	\[
		E = \left\{ \left( x _ { 1 } , x _ { 2 } , \ldots \right) \in \ell^2 | \sum _ { n \in \mathbb { N }^* } \frac { x _ { n } ^ { 2 } } { c _ { n } ^ { 2 } } = 1 \right\}
	\]

	Then \( E \) is complete as a closed subspace of Hilbert space $\ell^2,$
	but measure $\mu = \frac{1}{2}\delta _ { ( 1,0,0 , \ldots ) } + \frac{1}{2} \delta _ { ( - 1,0,0 , \ldots ) } $  has no barycenter in \( E \).
\end{example}

\begin{proof}
	$E$ is a closed subspace of Hilbert space $\ell^2$ since $ 1 / 2 \leq c_{n} \leq 1$.
	Linear Hilbert structure enables us to calculate distance through inner product.
	Pick $ x \in E$ and set $ e=(1,0,0\ldots)$, by definition,
	\begin{align*}
		\int_{E} d(x, \cdot)^2 \diff \mu & = \frac{d(x,e)^2+d(x,-e)^2 }{2}                         \\
		                                 & =\frac{\| x - e \|^2 + \| x + e \|^2}{2} =\| x \|^2 + 1
	\end{align*}

	So a barycenter $z$ of $\mu$ in $E$ should minimize its length $\| z \|$,
	but this is impossible in the space $E$.
	We prove it by contradiction.
	Assume now a vector $z \in E$ attains minimal length.
	Consider the first $n$ coordinate components of $z$,
	there could not exist any non-zero components except the last one, i.e., $z_i =0$ for $i<n$.
	Otherwise, we can keep its coordinate components $z_j$ for $j>n$,
	which are out of consideration,
	unchanged but vanish its first $n-1$ coordinate components to get a strictly shorter vector in $E$.
	This contradiction implies that no such a barycenter $z$ could exist in $E$.
\end{proof}

\begin{rmk}
	$E$ is not locally compact thus not a proper space.
	Indeed, consider the sequence with only one non-zero coordinate component, it has no converging subsequence.
\end{rmk}

We shall present some counter examples for more geometric metric spaces.
A discussion on length spaces might give some insight on the existence of barycenter.

\section{Non-proper length spaces}

Recall that on smooth manifolds, we define Riemannian metric through length of curves.
If we remove the assumptions concerning manifold structure,
we arrive at the definition of a general length space.

We start with induced length structures, which define rectifiable curves.
\begin{defn}[Induced length structures]
	\label{defn:length_structure}
	Let \( ( E , d ) \) be a path-connected metric space and \( \gamma \) be a curve in \( E \),
	i.e., a continuous map \( \gamma: [ a , b ] \rightarrow E \).
	Consider a partition \( Y \) of \( [ a , b ] \), that is,
	a finite collection of points \( Y = \left\{ y _ { 0 } , \ldots , y _ { N } \right\} \)
	such that \( a = y _ { 0 } \leq y _ { 1 } \leq y _ { 2 } \leq \ldots \leq y _ { N } = b \).
	The supremum of the sums
	\[
		\Sigma ( Y ) = \sum _ { i = 1 } ^ { N } d \left( \gamma \left( y _ { i - 1 } \right) , \gamma \left( y _ { i } \right) \right)
	\]
	over all the partitions $Y$ is called the length of $\gamma$
	with respect to the metric $d$ and denoted by $L_d (\gamma)$.
	A curve is said to be rectifiable if its length is finite.
\end{defn}

\begin{rmk}
	\label{rmk:curve_continuity_endpoints}
	For $\gamma : [a, b] \rightarrow E$ a curve and a subinterval $[x , y] \subset [a,b]$,
	we write $\gamma_{[x , y]}$ the restriction of $\gamma$ to the subinterval $[x , y]$.
	If \( \gamma \) is rectifiable,
	the function $ L ( \gamma , x , y ) = L \left( \gamma | _ { [ x , y ] } \right)$ is continuous
	in \( x \) and \( y \).
	We prove the continuity of \( L \) in \( y , a < y \leq b \),
	from the left for illustration.
	Take \( \varepsilon > 0 \) and
	consider a partition \( Y \) such that \( L ( \gamma ) - \Sigma ( Y ) < \varepsilon \). One may suppose that
	\( y _ { j - 1 } < y = y _ { j } \). Then
	\[
		L \left( \gamma , y _ { j - 1 } , y \right) - d \left( \gamma \left( y _ { j - 1 } \right) , \gamma ( y ) \right) < \varepsilon
	\]
	and the same inequality takes place for each $y^\prime$ such that $y_{j-1} \leq y^\prime \leq y$.
	Hence, we have shown that $L(\gamma, z, y) \rightarrow 0$ as $z \rightarrow y$ with $z<y$.
	Note that for $x < z <y$, we have
	$L(\gamma, x, z) + L(\gamma, z, y) = L(\gamma, x, y)$.
	So we have proved the left continuity of $L(\gamma, x,y)$ in $y$.
\end{rmk}

Once we have a length structure,
for two points \( x , y \in E \) we set the associated distance \( d_L ( x , y ) \) between them to be the infimum of lengths of rectifiable curves connecting these points:
\[
	d _ { L } ( x , y ) = \inf \{ L ( \gamma ) \,|\, \gamma: [ a , b ] \rightarrow E , \gamma \text{ rectifiable }, \gamma ( a ) = x , \gamma ( b ) = y \}.
\]

\begin{defn}[Length space]
	\label{defn:length_spaces}
	A metric space $(E,d)$ is a length space if the distance $\hat{d} := d_{L_{d}}$ defined by
	the induced length structure $L_d$ coincides with $d$.

	% We also call it an intrinsic metric space.
	We call $\hat{d}$ the intrinsic metric induced by $d$.
\end{defn}

We can justify this definition by \cite[Proposition 2.3.12]{burago2001course}
\begin{itemize}
	\item $L_{\hat{d}} = L_{d}$ on rectifiable curves.
	\item The intrinsic metric induced by \( \hat { d } \) coincides with \( \hat { d }\).
\end{itemize}
% Proofs could be found in \cite[Section 2.3]{burago2001course}.
This is important for understanding the definition,
so we state it as a theorem.

\begin{thm}[Alternative definition of length space]
	\( ( X , d ) \) is a length space if and only if for any points
	\( x , y \in X \) and any \( \varepsilon > 0 \) there exists a curve \( \gamma \) connecting \( x \) and \( y \) such
	that \( L _ { d } ( \gamma ) < d ( x , y ) + \varepsilon \).
\end{thm}

A good thing is that we still have classic Hopf-Rinow theorem
in Riemannian geometry for general length spaces.
That is
\begin{prop}[Proposition 2.5.22 in \cite{burago2001course}]
	A locally compact complete length space is proper.
\end{prop}
