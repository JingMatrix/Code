%! TEX root = ../barycenter.tex
\section{Existence for proper space}

A metric space is proper if every bounded closed set is compact.
A proper (metric) space is complete and separable,
because we can write a proper metric space as a countable union of compact closed balls.
This assumption on compactness ensures the existence of barycenter.

\begin{prop}
	\label{lem:existence_proper_space}
	If \( ( E , d ) \) is a proper metric space, then any \( \mu \in \mathcal{W}_ { 2 } ( E ) \) has a barycenter.
\end{prop}

\begin{proof}
	Fix \( z _ { 0 } \in E \) and take \( r > 1 \) large enough to satisfy
	\[ \mu \left( B \left( z _ { 0 } , r \right) \right) \geq \frac { 1 } { 2 } , \quad \int _ { E \backslash B \left( z _ { 0 } , r \right) } d \left( z _ { 0 } , y \right) ^ { 2 } \diff \mu ( y ) \leq 1 \]
	Then we have
	\[ \int _ { E } d \left( z _ { 0 } , y \right) ^ { 2 } \diff \mu ( y ) \leq r ^ { 2 } \cdot \mu \left( B \left( z _ { 0 } , r \right) \right) + 1 \leq r ^ { 2 } + 1 \]
	while for every \( x \in E \backslash B \left( z _ { 0 } , 3 r \right) \)
	\[ \int _ { E } d ( x , y ) ^ { 2 } \diff \mu ( y ) \geq \int _ { B \left( z _ { 0 } , r \right) } d ( x , y ) ^ { 2 } \diff \mu ( y ) > ( 2 r ) ^ { 2 } \cdot \mu \left( B \left( z _ { 0 } , r \right) \right) \geq 2 r ^ { 2 } \]
	holds.
	Therefore, it is sufficient to consider the infimum of
	\[ x \longmapsto \int _ { E } d ( x , y ) ^ { 2 } \diff \mu ( y ) \]
	only for \( x \in B \left( z _ { 0 } , 3 r \right) , \) and it is achieved at some point due to the compactness of the closure of \( B \left( z _ { 0 } , 3 r \right) . \)
\end{proof}

Here is a simple example that shows lack of properness
violates the existence of barycenter given by \cite[Example 3.1 (a)]{ohta2012barycenters}.

\begin{example}
	\label{example:ellipsoid_subspace}
	Let \( E \) be the infinite dimensional ellipsoid of axes of lengths \( c _ { n } = ( n + 1 ) / 2 n \) with \( n \in \mathbb { N } , \) namely
	\[
		E = \left\{ \left( x _ { 1 } , x _ { 2 } , \ldots \right) \in \ell^2 | \sum _ { n \in \mathbb { N }^* } \frac { x _ { n } ^ { 2 } } { c _ { n } ^ { 2 } } = 1 \right\}
	\]

	Then \( E \) is complete as a closed subspace of Hilbert space $\ell^2,$
	but measure $\mu = \frac{1}{2}\delta _ { ( 1,0,0 , \ldots ) } + \frac{1}{2} \delta _ { ( - 1,0,0 , \ldots ) } $  has no barycenter in \( E \).
\end{example}

\begin{proof}
	$E$ is closed subspace of Hilbert space $\ell^2$ since $ 1 / 2 \leq c_{n} \leq 1$.
	Linear Hilbert structure enables us to calculate distance through inner product.
	Pick $ x \in E$ and set $ e=(1,0,0\ldots)$, by definition,
	\begin{align*}
		\int_{E} d(x, \cdot)^2 \diff \mu & = \frac{d(x,e)^2+d(x,-e)^2 }{2}                         \\
		                                 & =\frac{\| x - e \|^2 + \| x + e \|^2}{2} =\| x \|^2 + 1
	\end{align*}

	So a barycenter $z$ of $\mu$ in $E$ should minimize its length $\| z \|$,
	but it is impossible in this space.
	Assume now vector $z \in E$ attains minimal length.
	Consider first $n$ coordinate components of $z$,
	there could not exist any non-zero components except the last one, i.e., $z_i =0$ for $i<n$.
	Otherwise, we can keep coordinate components $z_j$ for $j>n$, which are out of consideration, unchanged but vanish first $n-1$ coordinate components to get a strictly shorter vector in $E$.
	This indicates that no such barycenter $z$ could exist in $E$.
\end{proof}

\begin{rmk}
	$E$ is not locally compact as a subspace of $\ell^2$, thus not a proper space.
	Consider the sequence with only one non-zero coordinate component, it has no converging subsequence.
\end{rmk}

We shall present some counter examples for more geometric metric spaces.
A discussion on length might give some insight on the existence of barycenter.

\section{Length space but not proper}

Recall that on smooth manifolds, we define Riemannian metric through length of curves.
Remove assumptions for manifold structure, we arrive at the definition of general length space.

We start with induced length structure, which defines rectifiable curve.
\begin{defn}[Induced length structure]
	\label{defn:length_structure}
	Let \( ( E , d ) \) be a path-connected metric space and \( \gamma \) be a curve in \( E \),
	i.e., a continuous map \( \gamma: [ a , b ] \rightarrow E \).
	Consider a partition \( Y \) of \( [ a , b ] \), that is,
	a finite collection of points \( Y = \left\{ y _ { 0 } , \ldots , y _ { N } \right\} \)
	such that \( a = y _ { 0 } \leq y _ { 1 } \leq y _ { 2 } \leq \ldots \leq y _ { N } = b \).
	The supremum of the sums
	\[
		\Sigma ( Y ) = \sum _ { i = 1 } ^ { N } d \left( \gamma \left( y _ { i - 1 } \right) , \gamma \left( y _ { i } \right) \right)
	\]
	over all the partitions $Y$ is called the length of $\gamma$ with respect to the metric $d$ and denoted $L_d (\gamma)$.
	A curve is said to be rectifiable if its length is finite.
\end{defn}

\begin{rmk}
	For $\gamma : [a, b] \rightarrow E$ a curve and a subinterval $[x , y] \subset [a,b]$,
	we write $\gamma_{[x , y]}$ the restriction of $\gamma$ to the subinterval $[x , y]$.
	If \( \gamma \) is rectifiable,
	the function \( L \left( \gamma | _ { [ x , y ] } \right) = L ( \gamma , x , y ) \) is continuous
	in \( x \) and \( y \).
	We prove the continuity of \( L \) in \( y , a < y \leq b \),
	from the left for illustration purpose.
	Take \( \varepsilon > 0 \) and
	consider a partition \( Y \) such that \( L ( \gamma ) - \Sigma ( Y ) < \varepsilon \). One may suppose that
	\( y _ { j - 1 } < y = y _ { j } \). Then
	\[
		L \left( \gamma , y _ { j - 1 } , y \right) - d \left( \gamma \left( y _ { j - 1 } \right) , \gamma ( y ) \right) < \varepsilon
	\]
	and the same inequality takes place for each $y^\prime$ such that $y_{j-1} \leq y^\prime \leq y$.
\end{rmk}

Once we are disposal with a length structure,
for two points \( x , y \in E \) we set the associated distance \( d_L ( x , y ) \) between them to be the infimum of lengths of rectifiable curves connecting these points:
\[
	d _ { L } ( x , y ) = \inf \{ L ( \gamma ) \,|\, \gamma: [ a , b ] \rightarrow E , \gamma \text{ rectifiable }, \gamma ( a ) = x , \gamma ( b ) = y \}.
\]

\begin{defn}[Length space]
	\label{defn:length_space}
	A metric space $(E,d)$ is a length space if the distance $\hat{d} := d_{L_{d}}$ induced
	from the length structure $L_d$ coincides with $d$. We also call it an intrinsic metric space.
\end{defn}

We justify this definition by the fact that
twice induced distance $\hat{\hat{d}}$ of metric $d$ coincides with the induced distance $\hat{d}$ of metric $d$.
Also, we have $L_{\hat{d}} = L_{d}$ on rectifiable curves.
Proofs could be found in \cite[Section 2.3]{burago2001course}.
This is important for understanding the definition,
so we state it as a theorem.

\begin{thm}[Alternative definition of length space]
	\( ( X , d ) \) is a length space if and only if for any points
	\( x , y \in X \) and any \( \varepsilon > 0 \) there exists a curve \( \gamma \) connecting \( x \) and \( y \) such
	that \( L _ { d } ( \gamma ) < d ( x , y ) + \varepsilon \).
\end{thm}

Good news is that we get classic Hopf-Rinow theorem in Riemannian geometry for general length spaces.
This is \cite[Proposition 2.5.22]{burago2001course}.

\begin{prop}
	A locally compact complete length space is proper.
\end{prop}
