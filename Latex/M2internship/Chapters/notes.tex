%! TEX root = ../barycenter.tex
\chapter{Notes, out of report}
\section{Metric geometry}

The induced length space topology of $(E, \hat{d})$ is not locally compact.
By compatibility of induced length structure (\cite[Exercise 2.1.5]{burago2001course}) with the topology of base space $E$,
open set in original topology $(E, d)$ is again open in the induced length space topology $(E, \hat{d})$.

For non-smooth analysis, we use the definition of speed in \cite[Section 2.7]{burago2001course} for a curve.

\begin{defn}[Speed of curve]
	Let \( ( E , d ) \) be a metric space and \( \gamma : I \rightarrow X \) a curve.
	The speed of \( \gamma \) at \( t \in I \), denoted by \( v _ { \gamma } ( t ) \),
	is defined by
	\[ v _ { \gamma } ( t ) : = \lim _ { \varepsilon \rightarrow 0 } \frac { d ( \gamma ( t ) , \gamma ( t + \varepsilon ) ) } { | \varepsilon | } \]
	if the limit exists.
\end{defn}

We can take arc-length proportional parametrization on $[0,1]$ for any rectifiable curves.
They are then Lipschitz curves, and we recover classic length integral formula.

\begin{thm}
	Let \( E \) be a metric space and \( \gamma : [ a , b ] \rightarrow X \) be a Lipschitz curve.
	Then the speed \( v _ { \gamma } ( t ) \) exists for almost all \( t \in [ a , b ] \) and \( L ( \gamma ) = \)
	\( \int _ { a } ^ { b } v _ { \gamma } ( t ) \diff \lambda \) where $\lambda$ is the Lebesgue measure.
\end{thm}

\begin{rmk}
	Here we should discuss two different length metrics appeared, one is induced from $\ell^2$ norm and the other one is smooth Riemannian length structure. Are they two coincided (on common admissible curves)? Two ways to think about this relation
	\begin{enumerate}
		\item Apply Theorem 2.4.3 in \cite{burago2001course}, we then need to show Riemannian length structure is lower semi-continuous with respect to $\ell^2$ metric.
		\item Consider $E$ as a Hilbert sub-manifold of $\ell^2$, the norm for tangent space is the restriction of canonical norm of $\ell^2$.
	\end{enumerate}
\end{rmk}
\section{Measure theory}

Let \( X \) be a metric space and \( \mathscr { M } ( X ) \) the space of all measures defined on \( \mathscr { B } _ { X } . \) An element \( \mu \in \mathscr { H } ( X ) \) is a nonnegative, countably additive set function defined on \( \mathscr { B } _ { X } \) with the property \( \mu ( X ) = 1 . \quad C ( X ) \) stands for the space of all bounded real valued continuous functions on \( X \). We shall now topologize the space \( \mathscr { M } ( X ) \) by defining a base of open neighborhoods for any point \( \mu . \) Consider the family of sets of the form
\( V _ { \mu } \left( f _ { 1 } , f _ { 2 } , \ldots , f _ { k } ; \varepsilon _ { 1 } , \ldots , \varepsilon _ { k } \right) \)
\[ \left\{\nu \in \mathscr { M } ( X ) \, \mid  \left| \int f _ { i } d v - \int f _ { i } d \mu \right| < \varepsilon _ { i } , \quad i = 1,2 , \ldots , k \right\} \]

where \( f _ { 1 } , f _ { 2 } , \ldots , f _ { k } \) are elements from \( C ( X ) \) and \( \varepsilon _ { 1 } , \varepsilon _ { 2 } , \ldots , \varepsilon _ { k } , \) are positive numbers. It is easy to verify that the family of sets obtained by varying \( k , f _ { 1 } , f _ { 2 } , \ldots , f _ { k } , \varepsilon _ { 1 } , \ldots , \varepsilon _ { k } \) satisfies the axioms of a basis for a topology.

\begin{defn}
	We shall refer to this as the weak topology  in \( \mathscr { M } ( X ) \).
\end{defn}

It is then clear that a net \( \left\{ \mu _ { \alpha } \right\} \) of measures converges in the weak topology to a measure \( \mu \) if and only if \( \int f d \mu _ { \alpha } \rightarrow \int f d \mu \) for every \( f \in C ( X ) . \) In such a case we shall say that \( \mu _ { \alpha } \) converges ``weakly" to \( \mu \) or \( \mu _ { \alpha } \Rightarrow \mu \) in symbols. Unless otherwise stated, \( \mathscr { M } ( X ) \) will always be considered as a topological space with the weak topology.

We shall first recall a theorem which yields several useful equivalent definitions of the weak topology.

\begin{thm}
	Let \( \mu _ { \alpha } \) be a net in \( \mathscr { M } ( X ) . \) Then the following statements are equivalent:
	\begin{itemize}
		\label{thm:weak_convergence}
		\item \( \mu _ { \alpha } \Rightarrow \mu \)
		\item \( \lim _ { \alpha } \int g d \mu _ { \alpha } = \int g d \mu \) for all \( g \in U ( X ) \) where \( U ( X ) \) is the space of all bounded real valued uniformly continuous functions;
		\item \( \overline { \lim } _ { \alpha } \mu _ { \alpha } ( C ) \leqslant \mu ( C ) \) for every closed set \( C \)
		\item \( \lim _ { \alpha } \mu _ { \alpha } ( G ) \geqslant \mu ( G ) \) for every open set \( G \)
		\item \( \lim _ { \alpha } \mu _ { \alpha } ( A ) = \mu ( A ) \) for every Borel set \( A \) whose boundary has \( \mu \) -measure \( 0 . \)
	\end{itemize}
\end{thm}

For each point \( x \in X \) we shall denote by \( p _ { x } \) the measure degenerate at the point \( x \). Denote \( D = \left\{ p _ { x }: x \in X \right\} \).

\begin{lem}
	\label{dirac_measure_weak_homeomorphic}
	\( X \) is homeomorphic to the (topological) subset $D$.
\end{lem}

\begin{proof}
	For any point \( x \) and \( g \in C ( X ) , \) we have \( \int g d p _ { x } = g ( x ) \). If \( x _ { \alpha } \rightarrow x _ { 0 } \) then \( g \left( x _ { \alpha } \right) \rightarrow g \left( x _ { 0 } \right) . \) Hence \( p _ { x _ { \alpha } } \Rightarrow p _ { x _ { 0 } } \). Conversely, let \( p _ { x _ { \alpha } } \Rightarrow p _ { x _ { 0 } } \) If \( x _ { \alpha } \) does not converge to \( x _ { 0 } \), there is an open set \( G \) and a subnet \( \left\{ x _ { \beta } \right\} \) such that \( x _ { 0 } \in G \) and \( x _ { \beta } \in X - G \) for all \( \beta . \) Let \( g \) be a continuous function such that \( 0 \leqslant g \leqslant 1 , g \left( x _ { 0 } \right) = 0 \) and \( g ( x ) = 1 \) for \( x \in X - G \). Then \( \int g d p _ { x _ { \beta } } = 1 , \) while \( \int g d p _ { x _ { 0 } } = 0 . \) This is a contradiction. This completes the proof.
\end{proof}

\begin{lem}
	\( D \) is a sequentially closed subset of \( \mathscr { M } ( X ) \).
\end{lem}

\begin{proof}
	Let \( \left\{ x _ { n } \right\} \) be a sequence of points in \( X \) such that \( p _ { x _ { n } } \Rightarrow q \). Suppose \( \left\{ x _ { n } \right\} \) does not have any convergent subsequence. Then the set \( S = \left\{ x _ { 1 } , x _ { 2 } , \ldots \right\} \) is closed and thus is any subset \( C \) of \( S . \) Since \( p _ { x _ { n } } \Rightarrow q \) we have by Theorem \Cref{thm:weak_convergence}, \( q ( C ) \geqslant \overline { \lim } p _ { x _ { n } } ( C ) \) for \( C \subseteq S \). It follows that for every infinite subset \( S _ { 1 } \subseteq S , q \left( S _ { 1 } \right) = 1 \). This is a contradiction since \( q \) is a measure.

	Thus there is a subsequence \( \left\{ x _ { n _ { k } } \right\} , x _ { n _ { k } } \rightarrow x . \) By \Cref{dirac_measure_weak_homeomorphic}, \( q = p _ { x } \). Hence \( D \) is sequentially closed.
\end{proof}

\begin{thm}
	\label{finite_support_approximation}
	Let \( X \) be a separable metric space and \( E \subseteq X \) dense in \( X  \). Then the set of all measures whose supports are finite subsets of \( E \) is dense in \( \mathscr{ M } ( X ) \), the set of all Borel probability measures on $X$.
\end{thm}

\begin{proof}
	This proof is copy from Theorem 6.3 in \cite{parthasarathy2005probability}.

	It is obviously enough to prove that the set of all measures whose supports are finite subsets of \( X \) is dense in \( \mathscr { M } ( X ) . \) Let us denote the class of such measures by \( \mathscr { F } ( X ) . \) It is clear that any measure concentrated in a countable subset of \( X \) is a weakly limit of measures from \( \mathscr { F } ( X ) . \) Thus it is sufficient to prove that any measure is a weakly limit of measures vanishing outside countable subsets of \( X \).

	Choose and fix \( \mu \in \mathscr { M } ( X ) \). Since \( X \) is separable we can, for each integer \( n , \) write \( X \) as \( \bigcup _ { j } A _ { n j } , A _ { n j } \cap A _ { n k } = \phi \) if \( j \neq k , A _ { n j } \in \mathscr { B } _ { X } \) for all \( n \) and \( j \) and the diameter of \( A _ { n j } \) is \( \leq 1 / n \) for all \( j \). Let \( x _ { n j } \in A _ { n j } \) be arbitrary. Let \( \mu _ { n } \) be the measure with masses \( \mu \left( A _ { n j } \right) \) at the points \( x _ { n j } , \) respectively. Let $g$ be an arbitrary uniformly continuous (we can even assume Lipchitz continuous here) bounded function, and let \[ \alpha _ { n j } = \inf _ { x \in A _ { n j } } g ( x ) , \quad \beta _ { n j } = \sup _ { x \in A _ { n j } } g ( x ) \]
	Since \( g \) is uniformly continuous and since the diameter of \( A _ { n j } \rightarrow 0 \) as \( n \rightarrow \infty \) uniformly in \( j , \sup _ { j } \left( \beta _ { n j } - \alpha _ { n j } \right) \rightarrow 0 \) as \( n \rightarrow \infty \). Now
	\begin{align*}
		\left| \int g d \mu _ { n } - \int g d \mu \right| & = \left| \sum \int _ { A _ { n j } } \left( g - g \left( x _ { n j } \right) \right) d \mu \right|     \\
		                                                   & \leq \sup_{j} \left( \beta _ { n j } - \alpha _ { n j } \right) \xrightarrow{ n \rightarrow \infty} 0.
	\end{align*}
\end{proof}

\begin{thm}[Prokhorov's theorem]
	If \( \mathcal { X } \) is a Polish space, then a set \( \mathcal { P } \subset P ( \mathcal { X } ) \) is pre-compact for the weak topology if and only if it is tight, i.e. for any \( \varepsilon > 0 \) there is a compact set \( K _ { \varepsilon } \) such that \( \mu \left[ \mathcal { X } \backslash K _ { \varepsilon } \right] \leq \varepsilon \) for all \( \mu \in \mathcal { P } \).
\end{thm}

Later we need to have a measurable selection of barycenter when given measure varies. We refer to Theorem 6.9.6 in \cite{Bogachev2007} for following statement, proof is given as Theorem 35.43 in \cite{kechris1995}.

\begin{thm}
	\label{thm:measurale_selection}
	Let \( X \) and \( Y \) be Polish spaces and let \( \Gamma \in \mathcal { B } ( X \times Y ) \). Suppose, additionally, that the set \( \Gamma _ { x }: = \{ y \in Y: ( x , y ) \in \Gamma \} \) is nonempty and \( \sigma \)-compact for all \( x \in X \). Then \( \Gamma \) contains the graph of some Borel mapping \( f: X \rightarrow Y\).
\end{thm}


\begin{rmk} [Similarity between Wasserstein distance and length structure]
	% This comparison may be used in the discussion of geometric properties of Wasserstein space.
	Following comparison seems mysterious and interesting.
	\begin{itemize}
		\item Wasserstein distance
		      \begin{enumerate}
			      \item lower semi-continuous with respect to weak convergence
			      \item definition involves infimum
			      \item isometric embedding between base space and Wasserstein space
		      \end{enumerate}
		\item Length structure
		      \begin{enumerate}
			      \item lower semi-continuous with repect to point-wise converge of curves
			      \item definition involves infimum
			      \item isometric embedding between base space and space of geodesics
		      \end{enumerate}
	\end{itemize}
\end{rmk}

\section{Functional analysis}

\begin{thm}[Hahn-Banach Theorem]
	Given two disjoint, nonempty, convex sets $A$ and $B$ in a topological vector space $X$. If \( A \) is compact, \( B \) is closed, and \( X \) is locally convex, then there exist \( \Lambda \in X ^ { * } , \gamma _ { 1 } \in R , \gamma _ { 2 } \in R , \) such that
	\[
		\operatorname { Re } \Lambda x < \gamma _ { 1 } < \gamma _ { 2 } < \operatorname { Re } \Lambda y
	\]
\end{thm}

\begin{thm}[Banach-Alaoglu Theorem]
	% If \( V \) is a neighborhood of 0 in a separable topological vector space \( X , \) and if \( \left\{ \Lambda _ { n } \right\} \) is a sequence in \( X ^ { * } \) such that
	% \[ \left| \Lambda _ { n } x \right| \leq 1 \quad ( x \in V , n = 1,2,3 , \ldots ) \]
	% then there is a subsequence \( \left\{ \Lambda _ { n _ { i } } \right\} \) and there is a \( \Lambda \in X ^ { * } \) such that
	% \[ \Lambda x = \lim _ { i \rightarrow \infty } \Lambda _ { n _ { i } } x \quad ( x \in X ) \]
	If \( V \) is a neighborhood of 0 in a topological vector space \( X \) and if
	\[ K = \left\{ \Lambda \in X ^ { * }: | \Lambda x | \leq 1, \forall x \in V \right\} \]
	then $K$ is weak* compact.
\end{thm}

\begin{rmk}
	\begin{itemize}
		\item Weak convergence of measures coincides with weak* topology on the dual space of bounded continous function $C_b(\mathcal{X})$.
		\item Wasserstein metric toplogy of measures coincides with weak* topology on the (topological) dual space of $(1 + d(x_0, \cdot)^p) C_b(\mathcal{X})$.
		\item $C_b(\mathbb{R})$ is not separable, we can consider the uncountable set of bounded functions taking value $0$ or $1$ on all integers.
		\item The reason why Banach-Alaoglu theorem is not applicable is that the space of measures $P(\mathcal{X})$ is not closed in the weak* topolgy. Otherwise, this space will be weak* compact by Banach-Alaoglu theorem. And since weak convergence is metriziable, compactness is equivalent to sequential compactness. Then there is a classic counterexample of ``escape of mass". Consider Dirac measures on integers, this set has no accumulation points as measures on $\mathbb{R}$. Tightness is a control on the property of $P(\mathcal{X})$ being closed in weak* topology of the dual of $C_b(\mathcal{X})$.
		\item We can apply Hahn-Banach theorem here:
		      \begin{equation}
			      \label{hahn_banach_application}
			      \mu \in \overline { \operatorname { Conv } } ( \mathcal { K } ) \Longleftrightarrow \mu(f) \leq \sup _ { \nu \in \mathcal { K } } \int _ { \mathcal{X} } f \diff \nu, \quad \forall f \in C _ { b }( \mathcal{X} ).
		      \end{equation}
		      For instance we can prove the separability of \( \mathscr { P } ( \mathcal{X} ) \) by choosing \( \mathcal { K }: = \{ \delta _ { x }: x \in D \} \), where \( D \) is a countable dense subset of \( \mathcal{X}\). By \Cref{hahn_banach_application} we easily check that \( \operatorname { Conv } (\mathcal { K }) \subset \mathscr { P } ( \mathcal{X} ) \subset \overline { \operatorname { Conv } } (\mathcal { K } \)) and therefore the subset of all the convex combinations with rational coefficients of \( \delta \)-measures concentrated in \( D \) is weakly dense in \( \mathscr { P } ( \mathcal{X} ) \). The same holds for Wasserstein metric topology.

	\end{itemize}
\end{rmk}

Hence, convex analysis may not help us in the problem of Wasserstein barycenter, as the set of Dirac measures is not convex, not closed in the weak* topology of $C_b(\mathcal{X})$. The main application should optimal transport theory.

\section{Work in progress}

\subsection{Countinous selection of barycenter}

From proof of \Cref{thm:barycenter_finite_support_measure},
we can define almost everywhere a
selection of barycenter by barycenter formula \cref{equa:formula_barycenter}
and then extend it by countinuity to a bigger domain.

However,
even the barycenter formula \cref{equa:formula_barycenter} is not defined everywhere.

\subsection{Calculate density function}

Recall that there is no duality between $L^{\infty}(\widebar{\mu})$ (possible non-separable) and $L^1 (\widebar{\mu})$
in functional analysis even when $M$ is compact with Lebesgue measure.
According to Proposition 4.4.2 in \cite{Bogachev2007} below, we need to show that $\widebar{\mu}$ is a linear functional on $L^{\infty}(\widebar{\mu})$.
\begin{prop}
	Let \( \mu \) be a finite nonnegative measure.
	A continuous linear function \( \Psi \) on \( L ^ { \infty } ( \mu ) \) has the form
	\( \Psi ( f ) = \int _ { X } f g d \mu \)
	, where \( g \in L ^ { 1 } ( \mu ) \),
	precisely when the set function \( A \mapsto \Psi \left( I _ { A } \right) \) is countably additive.
\end{prop}
We define $\Psi(f):= \int_M f \diff \widebar{\mu} = \lim_m \int_M f \diff \widebar{\mu}_m$ for $f$ continous.
Extend $\Psi$ to be a continous linear functional on $L^{\infty}(\widebar{\mu})$ by Hahn-Banach theorem.
Do we still have $\Psi(I_A) = \widebar{\mu}(A)$?

A related discussion is available on
\href{https://math.stackexchange.com/questions/574130/does-weak-convergence-with-uniformly-bounded-densities-imply-absolute-continuity/574888#574888}{math Stack Exchange}.
Think about this example carefully.
Let $\lambda$ be the arglength measure and $\phi_n \ge 0$ a continuous function on $\mathbb T$ with $\int_{\mathbb T} \phi_n\, d\lambda = 1$ and $\phi_n(x) = 0$ if $|x-1| \ge \frac 1n$. Then for each continuous function $f\colon \mathbb T \to \mathbb R$ we have $\int_{\mathbb T} f\phi_n d\lambda \to f(1)$, that is $\phi_n \lambda \to \delta_1$ weakly. But $\delta_1$ is not $\lambda$-continuous.

A digression, generally for question if density of $\widebar{\mu}_m$ converges to density of $\widebar{\mu}$,
we possibly need asymptotically equicontinuous in \cite{Sweeting1986Converse}.

To attack even more general case,
we should be able to single out a part of $\mathbb{P}$ in the form $\lambda_1 \delta_{\mu_1}$
with lower bound on $\lambda_1$ and dominated $\mu_1$.
One way to do so is to assume $\mu_1$ has bounded density.
\begin{defn}
	[The set $ \mathcal { A } _ { L }$]
	For \( 0 < L < \infty \), let \( \mathcal { A } _ { L } \subset W_2(M) \) be the set of Borel probability
	measures with compact support on \( M \), absolutely continuous with respect to volume, whose densities have \( L ^ { \infty } \)
	norm less than or equal to \( L\).
\end{defn}

Note that, since the bound on the \( L ^ { \infty } \) norm is preserved under weak-* convergence,
\( \mathcal { A } _ { L } \) is a countable union of weakly-* closed set, and thus Borel measurable, subset of \( W_2( M ) \).

% If $\mu_1 \in \mathcal{A}_L$, then by duality between $L^p(\widebar{\mu})$ and $L^q(\widebar{\mu})$
% and that they are separable spaces,
% we have that $\widebar{\mu}$ has density in $L^p(\widebar{\mu})$ for
% $p < \infty$ with a upper norm bound depending only on constat $L$ (, $\lambda_1$ and $M$).
% We pass $p$ to infity to get $\widebar{\mu} \in \mathcal{A}_L$.
% Hence, our conclusion holds for any $\mathbb{P}$ that is not atomless on $\mathcal{A}_L$.

Finally, if we are given $\mathbb{P}$ with only assumption that $\mathbb{P}(\mathcal{A}_L(M)) > 0$,
we need more control on the density function of $\widebar{\mu}_m$
to ``remove'' the dependency of its upper bound on a singled-out coefficient $\lambda_1$.
One hope is that $\lambda_1$ is replaced by $\mathbb{P}(\mathcal{A}_L(M))$.


Every measure $\mu$ on $M$ in following discussion is absolutely continous and has compact support.
Here we use the fact that all possible values of $B(x_1, \ldots, x_n)$ are contained in a bounded set if each $x_i$ varies in a bounded set.
To conculde it, since $f$ is continous, there is a upper bound of $f$ and thus a upper bound of distance between $x_i$ and $B(x_1, \ldots, x_n)$.
% We always apply continuities of tranfer maps to get compact support of barycenter measure.
By convention, we denote by $g$ the density function for absolutely measure $\mu$.
One principle of differential geometry is to differentiate everything once we could.

% If we only apply general bound on $c$-concave function
Recall change of variable in \Cref{thm:jacobian_identity} (Theorem 4.2 in \cite{cordero2001riemannian}),
\[
	\widebar{g} = g_i \circ T_i \det D T_i :
	= g_i \circ T_i \det[Y(H-\text{Hess} u_i)]
\]
where $T_i = \exp(-\nabla u_i)$ is the unique optimal maps from $\widebar{\mu}$ to $\mu_i$.
Note that this is why we need that all measures have compact support.

For general $c$-concave function $\mu_i$, we have only upper hessian bound.
And this is not enough to get an estimation on the absolute value of Jacobian determinants.
% For instance, in Euclidean space, $DT_i = (\lambda_i -1)/\lambda_i \leq \text{Id}$.


There is only one element $\gamma$ in the set $\Gamma$ by uniqueness of optimal plans
since coupling between $\widebar{\mu}$ and $\mu_i$ is optimal.
% We can get every $\mu_i$ from $\mu_1$ if first push $\mu_1$ to $\widebar{\mu}$ through $T_1^{-1}$ then push $\widebar{\mu}$ to $\mu_i$ through $T_i$. Hence,
% \[B \circ (T_1, T_2, \ldots, T_n) \circ T_1^{-1} = T_1^{-1}\]
% is the unique transfer map from $\mu_1$ to $\widebar{\mu}$. That is to say, $\widebar{\mu}$-a.e., $B \circ (T_1, T_2, \ldots, T_n) $ is the identity map.
Hence, $\gamma = (T_1, T_2, \ldots, T_n)_{\#}\widebar{\mu}$
	is concentrated on the image of $(T_1, T_2, \ldots, T_n)$.
We have $B_{\#}(T_1, T_2, \ldots, T_n)_{\#}\widebar{\mu}=\widebar{\mu}$,
thus $B(T_1, T_2, \ldots, T_n) = \text{Id}$ for $\widebar{\mu}$ almost everywhere
by uniqueness of optimal plans from $\widebar{\mu}$ to $\widebar{\mu}$.
This is already included in \Cref{lem:inverse_barycenter}.
We differentiate the equality $B(T_1(x), \ldots, T_n(x))=x$ for $\widebar{\mu}$-a.e. $x$,
\begin{align*}
	\text{Id} =\sum_{i=1}^n \partial_i B\, DT_i
	 & =\sum_{i=1}^n D \exp(-\frac{1}{\lambda_i}\nabla f_i) \, D \exp(-\nabla u_i)                     \\
	 & =\sum_{i=1}^n D \exp^{-1}(-\nabla \left( \frac{f_i}{\lambda_i}\right)^c) \, D \exp(-\nabla u_i) \\
	 & =\sum_{i=1}^n(H-\text{Hess}(f_i / \lambda_i)^c)^{-1}\,Y_i^{-1}\,
	Y_i\,(H-\text{Hess} u_i)                                                                           \\
	 & =\sum_{i=1}^n(H-\text{Hess}(f_i / \lambda_i)^c)^{-1}\,
	(H-\text{Hess}u_i) .
\end{align*}

Then by Minkowski's determinant inequality, we get
\begin{align*}
	1 & \geq \sum_{i=1}^{n} \det [H-\text{Hess}(f_i/\lambda_i)^c]^{-1/n}\,\det[H-\text{Hess}u_i]^{1/n} \\
	  & =\sum_{i=1}^n \det[\partial_i B]^{1/n}\,\det[DT_i]^{1/n}
\end{align*}
Observe that in our discussion $(f_i / \lambda_i)^c$ is calculated at the barycenter $x$ of $\sum_{j=1}^{n} \delta_{T_j x}$,
and patial derivative means all $T_j x$ for $ j \ne i$ are fixed in calculation.
Hence, we can have $(f_i / \lambda_i)^c = g_i$.
% Recall that we know $(f_i / \lambda_i)^c$ has hessian bound from both sides, see \cref{equa:hessian_bound_f}.
From it we get $\det[\partial_i B]^{1/n} \geq \min \{1, \lambda_i / (1 - \lambda_i)\} > \lambda_i$,
where $C > 0$ depends only on hessian bound of square distance function and Lipschitz constant of exponential map.

Combine these two inequalities, and we then apply Jensen inequlity
\[
	\widebar{g} \leq
	\left[ \sum_{i=1}^n \frac{\det[\partial_i B]^{1/n}}
	{g_i^{1/n} \circ T_i}\right]^{-n}
	< \left[ \sum_{i=1}^n \frac{C \, \lambda_i}
	{g_i^{1/n} \circ T_i}\right]^{-n}
	\leq C^{-n} \sum_{i=1}^n \lambda_i g_i \circ T_i.
\]
With this estimation in hand,
one shows easily that if measure $\mathbb{P} \in \mathcal{W}_2(\mathcal{W}_2(M))$ on $W_2(M)$ give mass to the measurable set $\mathcal{A}_L$,
then it has a unique absolutely continous barycenter.

\subsubsection{Jacobian determinant inequality for the Wasserstein barycenter}

This is done by Kim and Pass.
\begin{defn}[Volume distortion]
	Let \( \lambda \) be a Borel probability measure on \( M \) with a
	unique barycenter \( \widebar { x } \) (that is, such that \( B C ( \lambda ) \) is a singleton). We define the generalized,
	or barycentric, volume distortion coefficients at \( y \notin \operatorname { cut } ( \widebar { x } ) \)

	\[ \alpha _ { \lambda } ( y ) : = \frac { \operatorname { det } \left[ - \left. D _ { y z } ^ { 2 } \right| _ { z = \widebar { x } } c ( y , z ) \right] } { \operatorname { det } \left[ \left. \int _ { M } D _ { z z } ^ { 2 } \right| _ { z = \widebar { x } } c ( x , z ) d \lambda ( x ) \right] } \]
	where \( D _ { z z } ^ { 2 } c ( x , z ) \) denotes the Hessian of the function \( z \mapsto c ( x , z ) \), and the determinants
	are computed in exponential local coordinates at \( \widebar { x } \) and \( y . \)
\end{defn}

\begin{thm}
	[Jacobian determinant inequality for the Wasserstein barycenter]
	Assume that the Wasserstein barycenter \( \widebar { \mu } \) of the measure \( \Omega \) on \( P ( M ) \) is absolutely continuous.
	Letting \( T _ { \mu } \) denote the optimal map from \( \widebar { \mu } \) to \( \mu \), consider the measure on \( M \) given by
	\[ \lambda _ { x } : = \int _ { P ( M ) } \delta _ { T _ { \mu } ( x ) } d \Omega ( \mu ) \]
	which is defined with respect to a.e. $x$.
	Then, for \( \widebar { \mu } \)-a.e. \(x\),
	\[ 1 \geq \int _ { P ( M ) } \alpha _ { \lambda _ { x } } ^ { 1 / n } \left( T _ { \mu } ( x ) \right) \operatorname { det } ^ { 1 / n } D T _ { \mu } ( x ) d \Omega ( \mu ) \]
\end{thm}

% \subsubsection{Use local coordinate}
% Need to work on it.

\subsubsection{Use Skorohod representation}
\textcolor{red}{This is not possible!}
One may consider to construct absolutely continous random variables.
For example, to use the Skorohod representation (see section 8.5 in \cite{Bogachev2007}),
\begin{defn}
	We shall say that a topological space \( X \) has the strong
	Skorohod property for Radon measures if to every Radon probability measure
	\( \mu \) on \( X \),
	one can associate a Borel mapping \( \xi _ { \mu } : [ 0,1 ] \rightarrow X \) such that \( \mu \) is
	the image of Lebesgue measure under the mapping \( \xi _ { \mu } \) and \( \xi _ { \mu _ { n } } ( t ) \rightarrow \xi _ { \mu } ( t ) \) a.e.
	whenever the measures \( \mu _ { n } \) converge weakly to \( \mu . \)
\end{defn}

However, even if $\mu$ is absolutely continous respect to Lebesgue measure on $\mathbb{R}$,
we don't have necessarily that $\xi_{\mu}$ is a absolutely continous function.
In fact, the Housdorff dimension of the image of $\xi_{\mu}$ is not likely to be greater than 2.
See \cite{Besicov1937Sets} for discussions on $\delta$-Lipschitz curves,
their Housdorff dimensions are bounded by $2-\delta$.

% \subsection{Control on density function}
