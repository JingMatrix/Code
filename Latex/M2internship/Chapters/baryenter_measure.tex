%! TEX root = ../barycenter.tex
\section{Uniqueness of barycenter over Wasserstein space}

For optimal transportation on Riemannian manifold,
\cref{thm:optimal_transport_manifold} gives a detailed description.
However, this theorem requires measures in consideration to have compact supports.

In our discussion,
we have no information in advance
whether a barycenter measure has compact support.
We can get following uniqueness on transfer map
without explicit infomation of it.
See Theorem 10.41 in \cite{villani2008optimal} for full details.
\begin{thm}[Uniqueness of Solution of the Monge problem for the square distance]
	\label{thm:uniquness_monge_problem_manifold}
	Let \( M \) be a Riemannian manifold, and \( c ( x , y ) = d ( x , y ) ^ { 2 } \).
	Let \( \mu , \nu \) be two probability measures on \( M \), such that the optimal cost
	between \( \mu \) and \( \nu \) is finite. If \( \mu \) is absolutely continuous,
	% then there is a
	% unique solution of the Monge problem between \( \mu \) and \( \nu \).
	% volume measure on \( M . \) 
	then the Monge-Kantorovich mass transportation
	problem between \( \mu \) and \( \nu \) admits a unique optimal transference plan, and it
	has the form \( d \pi ( x , y ) = d \mu ( x ) \delta [ y = T ( x ) ] \), or equivalently
	\[ \pi = ( \mathrm { Id } \times T ) \# \mu , \]
	where \( T \) is uniquely determined, \( \mu \)-almost everywhere, by the requirements
	that \( T \# \mu = \nu \).
\end{thm}

With the help of this theorem, we are ready to prove
\begin{prop}
	For measure $\mathbb{P} \in \mathcal{W}_2(\mathcal{W}_2(M))$ on $\mathcal{W}_2(M)$,
	assume $\mathbb{P}$ gives mass to the set of absolutely continous measures $P_{ac}(M)$, $\mathbb{P}(P_{ac}(M)) \geq 0 $.
	Then $\mathbb{P}$ has a unique barycenter measure on $M$.
\end{prop}

\begin{proof}
	We consider space $\mathcal{W}_2(\mathcal{W}_2(M))$, probability measures on $\mathcal{W}_2(M)$,
	any convex combination of elements in this space is still inside it.
	The squared Wasserstein distance function $W^2_2(\mu, \cdot)$ is convex by definition of optimal plan,
	for $0 \leq \lambda_1, \lambda_2 \leq 1, \gamma_1 + \gamma_2 =1 $ and $ \nu_1,\nu_2 \in \mathcal{W}_2(M)$,
	\begin{equation}
		\label{equa:convexity_Wassersetein_distance}
		W_2^2(\mu, \lambda_1 \nu_2 + \lambda_2 \nu_2) \leq \lambda_1 W_2^2(\mu, \nu_1) + \lambda_2 W_2^2(\mu, \nu_2).
	\end{equation}

	And when $\mu \in \mathcal{W}_2(M)$ is absolutely continuous, convexity above becomes strict convexity.
	It means inequality \cref{equa:convexity_Wassersetein_distance} becomes equality only when
	$\lambda_1=0$ or $\lambda_2=0$.
	To prove this claim,
	through \cref{thm:uniquness_monge_problem_manifold} we write
	$\gamma_1 : = (\text{Id}  \times T_1)_{\#}\mu$ the optimal plan from $\mu$ to $\nu_1$ and
	$\gamma_2 : = (\text{Id}  \times T_2)_{\#}\mu$ the optimal plan from $\mu$ to $\nu_2$.

	Set $\gamma := \lambda_1 \gamma_1 + \lambda_2 \gamma_2$ for $\gamma_1, \gamma_2$ that turn
	\cref{equa:convexity_Wassersetein_distance} into equality,
	we have
	\begin{align*}
		\lambda_1 W_2^2(\mu, \nu_1) + \lambda_2 W_2^2(\mu, \nu_2) & = W_2^2(\mu, \lambda_1 \nu_2 + \lambda_2 \nu_2)            \\
																															& \leq \int_{M \times M} d(x,y)^2 \diff \gamma(x,y)                    \\
		                                                          & =	\lambda_1 W_2^2(\mu, \nu_1) + \lambda_2 W_2^2(\mu, \nu_2)
	\end{align*}
	Then $\gamma$ is an optimal plan from $ \mu$ to $\lambda_1 \nu_1 + \lambda_2 \nu_2$,
	but it is not in the form of transform map unless $\lambda_1 =0$ or $\lambda_2 =0$.

	After integration with respect to $\mathbb{P} \in \mathcal{W}_2(\mathcal{W}_2(M))$,
	we get a convex function $\int_{\mathcal{W}_2(M)} W_2^2(\mu, \cdot) \diff \mathbb{P}(\mu)$.
	And it is strictly convex if $\mathbb{P}$ gives mass to absolutely continuous measures.
	Hence, barycenter of $\mathbb{P}$ is unique.
\end{proof}

\section{Absolutely continuity of barycenter}

We discuss in the context of proof of \cref{thm:barycenter_finite_support_measure}.

\subsection{Barycenter of finite supported measure}

Consider a measure $\sum_{i}^{n} \lambda_{i} \delta_{\mu_i}$ on $P_{2}(M)$ for $M$ a complete Riemannian manifold.
Assume from now on \textcolor{cyan}{$ \lambda_1 \neq 0$ and
	that $\mu_1$ is absolutely continous with compact support}.

% \textcolor{red}{Later the assumption of compact support could be removed}?

The barycenter $\bar{\mu}$ of measure $\sum_{i}^{n} \lambda_{i} \delta_{\mu_i}$ is unique,
because this meause gives mass to element in $P_{ac}(M)$.

In the case that \textcolor{cyan}{$\bar{\mu}$ is absolutely continous},
there is only one element in the set $\Gamma$ by uniqueness of optimal plans.
We can get every $\mu_i$ from $\mu_1$ if first push $\mu_1$ to $\bar{\mu}$ through $T_1^{-1}$ then push $\bar{\mu}$ to $\mu_i$ through $T_i$. Hence,
\[B \circ (T_1, T_2, \ldots, T_n) \circ T_1^{-1} = T_1^{-1}\]
is the unique transfer map from $\mu_1$ to $\bar{\mu}$. That is to say, $\bar{\mu}$-a.e., $B \circ (T_1, T_2, \ldots, T_n) $ is the identity map.
This is already included in \cref{lem:inverse_barycenter}.

\subsubsection{One measure absolutely continous and others Dirac}
We firstly consider the case when \textcolor{cyan}{$\mu_i = \delta_{x_i}, i \geq 2$ are Dirac measures}.

In this case, there is only one element in the set $\Gamma$.
As a consequence, $\exp(-\nabla f_1 / \lambda_1)$ pushes $\mu_1$ to $\bar{\mu}$.
As a left inverse to $\exp(-\nabla f_1/\lambda_1)$,
\(\exp(-\nabla g_1)\) pushes $\bar{\mu}$ to $\mu_1$.

We have in general that the $c$-conjugate of $f_1 / \lambda_1$ satisfies
$(f_1 / \lambda_1)^c = g_1^{cc} \geq g_1$,
and thus $c(w, x_1) \geq (f_1 / \lambda_1)^c(w) + f_1/\lambda_1 (x_1) \geq g_1(w) + f_1/\lambda_1 (x_1)$.
When $w$ is barycenter for some $x_1$,
we have $g_1(w) = (f_1 / \lambda_1)^c(w)$.
Hence, from the definition of $\bar{\mu}$ we have
$g_1 = (f_1 /\lambda_1)^c$ for $\bar{\mu}$ alomost everywhere.
% This long inequality implies that any $w$ in the equality
% $c(x_i, w) = f_i / \lambda_i (x_i) + (f_i / \lambda_i)^c(w)$ must be a barycenter of $\sum_{i=1}^{n} \lambda_i \delta_{x_i}$.
% Moreover, we have following equality holds whenever one of gradients exits,
% \[\exp(-\nabla g_i) = \exp(-\nabla (f_i / \lambda_i)^c) = \exp^{-1}(-\nabla f_i / \lambda_i).\]
% Note cut-locus are excluded from consideration in above equality,
% and as a corollary $\nabla g_i = \nabla (f_i / \lambda_i)^c$.
% Therefore, two locally Lipschitz function $g_i$ and $f_i /\lambda_i$ coincide almost everywhere.
% If follows $g_i=f_i /\lambda_i$ on $M$.

We discuss following in the sense of $\bar{\mu}$ almost everywhere. The function $g_1 = (f_1 / \lambda_1)^c $ is by definition just a sum of squared distance functions.
Now we aim to find out when \( \exp(- \nabla g_1 ) = \exp(- \nabla (f_1 / \lambda_1)^c) \) is a Lipschitz function.
% with Lipschitz constant depending only on $M$ and $\lambda_1$.
% As $M$ is compact, though tangent bundle $TM$ is not compact, local Lipschitz plus bounded diameter of $M$ implies global Lipschitz of $\exp$.
% Function $g_1$ has hessian upper bound by \cref{prop:differentiate_optimal_transport},
% here we treat it as an infimal convolution over a fixed sigleton set $\{x_1\}$.
% Note that from $ \nabla g_1 = \nabla (f_1 / \lambda_1)^c$,
% $g_1$ and $(f_1 / \lambda_1)^c$ have the same hessian.
Function $(f_1 / \lambda_1)^c$ has hessian upper bound as a $c$-conjugate function;
function $g_1$ has hessian lower bound as a negative linear combination of square distance function.
We then have
\begin{equation}
	\label{equa:hessian_bound_f}
	-\frac{1-\lambda_1}{\lambda_1} H \leq \nabla^2 (f_1 / \lambda_1)^c \leq H,
\end{equation}
where we denote by $H$ a possible bound from above for the hessian of square distance funtion.
% Recall that $H$ is bounded from above.
% Squared distance function $c$ has bounded Hessian from above.
% Hence $ \| \nabla^2 g \|$ is bounded from above by taking the second derivative of definition and also applying minimallity of cost at barycenter.

Hence, we need two properties to have Lipschitzness of $\exp(-\nabla (f_1 / \lambda_1)^c)$:
\begin{itemize}
	\item $ \exp $ is Lipschitz on the domain we are interested in.
	      % This would be the case when we are in the support of a compact supported function.
	      For example, in a compact domain.
	\item The hessian of square distance function has upper bound $H$.
	      This is the case when we have lower bound of sectional curvature.
\end{itemize}

We always take the second condition in following discussion.
When $x_1$ runs through a bounded set,
all possible barycenters are included in a bounded set.
Hence $\bar{\mu}$ has compact support.
As we actually only consider measures with compact support, the first assumption is fullfilled as well.

% because $br$ pushes $\mu_1$ to $\bar{\mu}$ by construction of $\bar{\mu}$.

Denote by $C$ the Lipschitz constant of map \(\exp(-\nabla(f_1/\lambda_1)^c)\).
If we have $ \text{Vol}(E) < \delta \implies \mu_1(E) < \epsilon$,
then by $\text{Vol}(B^{-1}( E\times \{x^\prime\})) < C^n \text{Vol}(E)$,
where we write $x^\prime = (x_2, \ldots, x_n)$,
we have
\begin{equation}
	\label{equa:absolutely_continuity_estimation}
	\text{Vol}(E) < \delta / C^n \implies \bar{\mu}(E)=\mu_1(B^{-1}(E \times \{x^\prime\})) < \epsilon.
\end{equation}
Thus $\bar{\mu}$ is absolutely continous.
Here the Lipschitz constant $C$ depends on the support of $\mu_1$.
% Since $\exp(-\nabla(f_1 / \lambda_1)^c) = \exp(-\nabla g_1) $ is continous for Vol-a.e. $x_1$
% (outside of cut-locus of $x_j, j \ne 1$),
% we deduce from absolutely continuity of $\bar{\mu}$ and compact support of $\mu_1$ that
% $\bar{\mu}$ has also compact support.

% \begin{rmk}
% 	Once after we prove that $\bar{\mu}$ is indeed absolutely continous.
% 	$\nabla g$ is invertable and $br(x_1) = \exp_{x_1} \nabla g^c(x_1)$.
% 	It is not surprised that
% 	\[
% g^c(x_1) = -c(x_1 ,z) - \frac{1}{\lambda_1} \sum_{i=2}^{n} \lambda_i\, c(x_i, z) =
% - \frac{1}{2 \lambda_1} W^2( \sum_{i=1}^n \lambda_i \delta_{x_i}, M),
% - f_1/\lambda_i.
% \]
% we can then take derivative with respect to $x_1$.
% \end{rmk}

\subsubsection{To a more general case by conditional probability}
% \label{discussion_conditional_prob}
To attack general case when \textcolor{cyan}{$\mu_i, i \geq 2$ are not assumed Dirac measures},
we should consider conditonal measures of optimal multi-marginal transfer plan $\gamma$:
\[
	\diff \gamma(x_1, x^\prime)= \gamma(\diff x_1 \mid x^\prime)\, \diff \pi(x^\prime) ,
\]
where $\pi$ is the projection of $\gamma$ from $x = (x_1, x^\prime)$ to $x^\prime$,
by abuse of language we also denote by it the push-forward measure from $\gamma$.
For $\pi$-a.e $x^\prime$., $\gamma(\cdot \mid x^\prime)$ is a probability measure
concentrated on $M \times \{x^\prime\}$.

One has from definition of conditional measures that,
as function on Borel measurable sets,
\[
	\bar{\mu} = B_{\#} \gamma = \int_{M^{n-1}} B_{\#} \gamma(\cdot \mid x^\prime)\, \diff \pi(x^\prime).
\]


If we have that for $\gamma$-a.e. $x^\prime$,
$B_{\#} \gamma(\cdot \mid x^\prime)$
is absolutely continous,
then $\bar{\mu}(E)=0$ for volume measure zero set $E$ by integration above.
Note that \cref{equa:absolutely_continuity_estimation} also holds by integration.
% \[
% 	\bar{\mu}(E)
% =\int_{M^{n-1}}
% \frac{ \mu_1( br^{-1}(E) \cap A_{x^\prime} )}{\mu_1(A_{x^\prime})}
% \diff \pi(x^\prime)
% =\int_{M^{n-1}}
% \bar{\mu}^{x^\prime}(E)
% \diff \pi(x^\prime)
% \leq \int_{M^{n-1}} C\,\mu_1^{\prime}(E) \diff \pi(x^\prime) \leq C\, \mu_1(E).
% = 0.
% \]

Generally speaking, computation of $\gamma(\cdot \mid x^\prime)$ is only possible when
\begin{itemize}
	\item $\pi$ is has countable support.
	      This is equivalent to that $\mu_i, i \geq 2$ has countable support.
	      Fix any $x^\prime$ in the support of $\pi$, by direct verification
	      \[
		      \gamma(\cdot \mid x^\prime) =
		      \frac{
			      \mathbbm{1}_{M \times \{x^\prime\}}
			      \gamma
		      }{\pi(x^\prime)}.
	      \]
	\item $\gamma$ is absolutely continous with density function $f: M^n \rightarrow \mathbb{R} $.
	      \[
		      \gamma(\diff x_1 \mid x^\prime) =
		      \frac{
			      f(x_1, x^\prime) \diff \text{Vol}(x_1)
			      % \diff \gamma (x_1, x^\prime)
		      }
		      {\int_{M} f(x_1, x^\prime) \diff \text{Vol}(x_1)
		      }
	      \]
	      where we set the right hand side zero if it is undeterminated
	      and we remove this $x^\prime$ from consideration.
\end{itemize}
Note in both cases,
$\gamma(\cdot \mid x^\prime)$
has absolutely continous push-forward measure
$\mu_1^{x^\prime} := \text{proj}^1_{\#}\gamma(\cdot \mid x^\prime)$
to the first coordinate.
But only in the first case that
measure $\gamma(\cdot \mid x^\prime) \leq \pi(x^\prime) \, \gamma$ is controled by $\gamma$.
Or we hope $f$ has positive lower bound,
for example \textcolor{cyan}{continous and strictly positive}, like volume measure.
In the case $\mu_1=\text{Vol}$? Every point in $\text{spt}\mu_1$ runs over all $\times_i \text{spt}\mu_i$.

\textcolor{red}{Unfortunately, this control is required to apply conditioning optimal plan.}
One may consider the case when at least one marginal is discrete,
but no absolutely continuity can be derived without barycenter push-forward.

It is true that we have $\gamma$ is an optimal plan for all its marginals
% \[
% 	B_{\#} \gamma              = \bar{\mu} \text{ and }
% 	\text{proj}^1_{\#} \gamma  = \mu_1,
% \]
and conditioning of (multi-marginal) optimal plan is still optimal from \cref{thm:restriction_optimal_plan}.
Hence, $\gamma(\cdot \mid x^\prime)$ is an optimal plan of its marginals.
% \textcolor{cyan}{in the case of discrete measure}.
Apply previous discussion, we get absolutely continous measure
\[
	\exp(-\frac{1}{\lambda_1}\nabla f_1)_{\#}  \mu_1^{x^\prime}=
	B_{\#}\gamma(\cdot, x^\prime) =
	\bar{\mu}^{x^\prime},
\]
where $\bar{\mu}^{x^\prime}$ is a barycenter measure.

% Since conditioning of optimal maps are still optimal,
% barycenter formula \cref{formula_barycenter} still holds.
% We already know $T_1 := \exp(-\nabla g_1)$ is the optimal transfer map from $\bar{\mu}^{x^\prime}$ to $\mu_1^{x^\prime}$,
% then $T_1 = \exp(- \nabla g_1)$ for $\mu_1^{x^\prime}$ almost everyhere and
% \begin{align*}
% 	\bar{\mu} = B_{\#} \gamma & =
% 	\int_{M^{n-1}}\bar{\mu}^{x^\prime} \, \diff \pi(x^\prime)                                            \\
% 	                          & = \int_{M^{n-1}} \gamma( T_1(\cdot) \mid x^\prime) \diff \pi(x^\prime)   \\
% 														& = \gamma \circ \exp(-\nabla g_1)
% 														% &= \exp (- \frac{1}{\lambda_1} \nabla_1 f)_{\#} \gamma
% 	.
% \end{align*}
% Again, $\bar{\mu}$ has compact support as $T_1$ is Vol-a.e. continous.

% One possible \textbf{investigation}: Conditional optimal plan, not just the simple case of restriction.
% Hence, the push-forward measure
% $\bar{\mu}^{x^\prime} = br_{\#} \diff \mu_1^{x^\prime}$ is absolutely continous as well.
% We thus have
% \[
% 	B_{\#} \diff \gamma = \int_{M^{n-1}} f(\cdot, x^\prime)  \diff \pi(x^\prime) \, \diff \text{Vol}(\cdot).
% \]
% Moreover, the previous integral is in fact a finite sum.
% For a measurable set $E \subset \text{spt}(\bar{\mu})$ with volume measure $0$,

% So $\bar{\mu} = B_{\#} \diff \gamma$ is absolutely continous with respect to $\mu_1$.
% \begin{rmk}
% 	\textcolor{red}{NEED further investigation}, maybe one can use \cref{formula_barycenter}.

% 	We may guess that	the density of $\bar{\mu}$,
% 	$\int_{M^{n-1}} f(\cdot, x^\prime)  \diff \pi(x^\prime)$
% 	is dominated by density of $\mu_1$ up to a constant coefficient.
% 	% Lipschitz constant of \( \exp _ { z } \nabla g ( z ) \).
% \end{rmk}

\subsubsection{To a more general case by consistency of barycenter}

Now we consider measure $\mathbb{P} = \lambda_1 \delta_{\mu_1} + \lambda_2 \delta_{\mu_2}$
without assuming $\mu_2$ discrete.
Note here it is for simplicity that we only consider the case $n=2$.
Approxiamte $\mu_2$ in Wasserstein metric by a sequence of measures $\mu_2^{m}$,
then $\mathbb{P}_m := \lambda_1 \delta_{\mu_1} + \lambda_2 \delta_{\mu_2^m}$ converges to $\mathbb{P}$
in $P_2(P_2(M))$.

By the consistency of barycenters, the unique barycenter $\bar{\mu}_m$ of $\mathbb{P}_m$
converges in Wasserstein metric (or just weakly) to the unique barycenter $\bar{\mu}$ of $\mathbb{P}$.
Recall that there is no duality between $L^{\infty}(\bar{\mu})$ (possible non-separable) and $L^1 (\bar{\mu})$
in functional analysis even when $M$ is compact with Lebesgue measure.
According to Proposition 4.4.2 in \cite{Bogachev2007} below, we need to show that $\bar{\mu}$ is a linear functional on $L^{\infty}(\bar{\mu})$.
\begin{prop}
	Let \( \mu \) be a finite nonnegative measure.
	A continuous linear function \( \Psi \) on \( L ^ { \infty } ( \mu ) \) has the form
	\( \Psi ( f ) = \int _ { X } f g d \mu \)
	, where \( g \in L ^ { 1 } ( \mu ) \),
	precisely when the set function \( A \mapsto \Psi \left( I _ { A } \right) \) is countably additive.
\end{prop}

We define $\Psi(f):= \int_M f \diff \bar{\mu} = \lim_m \int_M f \diff \bar{\mu}_m$ for $f$ continous.
Extend $\Psi$ to be a continous linear functional on $L^{\infty}(\bar{\mu})$ by Hahn-Banach theorem.
Do we still have $\Psi(I_A) = \bar{\mu}(A)$?

A related discussion is available on
\href{https://math.stackexchange.com/questions/574130/does-weak-convergence-with-uniformly-bounded-densities-imply-absolute-continuity/574888#574888}{math Stack Exchange}.
Think about this example carefully.
Let $\lambda$ be the arglength measure and $\phi_n \ge 0$ a continuous function on $\mathbb T$ with $\int_{\mathbb T} \phi_n\, d\lambda = 1$ and $\phi_n(x) = 0$ if $|x-1| \ge \frac 1n$. Then for each continuous function $f\colon \mathbb T \to \mathbb R$ we have $\int_{\mathbb T} f\phi_n d\lambda \to f(1)$, that is $\phi_n \lambda \to \delta_1$ weakly. But $\delta_1$ is not $\lambda$-continuous.

A digression, generally for question if density of $\bar{\mu}_m$ converges to density of $\bar{\mu}$,
we possibly need asymptotically equicontinuous in \cite{Sweeting1986Converse}.

We don't need to find out density explicitly.
Instead, let prove the absolutely continuity by showing that
for any $\epsilon > 0$ there is a $\delta > 0$ such that
\begin{equation}
	\label{equa:absolutely_continous}
	\forall E \subset M \, \text{measurable, } \text{Vol}(E) < \delta
	\implies \bar{\mu}(E) < \epsilon
\end{equation}
In our situation, a \textcolor{red}{uniform $\delta$} can be chosen for all $\bar{\mu}_m$
% because of a discrete situation of previous conditioning optimal plan calculation.
% See the now ``trivial'' \cref{formula_barycenter},
% we can use it to conclude uniform Lipschitz constant for $m$.
because $\bar{\mu}_m$ is pushed by $T_1^{m}$ from $\mu_1$ with a Lipschitz constant bound
independent of $m$.
Recall \cref{equa:absolutely_continuity_estimation} for detail.

Recall for open set $E \subset M$, $\bar{\mu}(E) \leq \liminf \bar{\mu}_m(E)$,
as indicator function of open set is lower semi-continuous.
Hence \cref{equa:absolutely_continous} holds for all open sets.
As Borel measure is outer regular, for general measurable $E$ with $\text{Vol}(E) < \frac{\delta}{2}$,
we select an open set $E^\prime$ such that
$ E \subset E^\prime$ and $ \text{Vol}(E^\prime) < \delta$,
then $\bar{\mu}(E) \leq \bar{\mu}(E^\prime) < \epsilon$.

Previous argument works for general $\lambda_1 \delta_{\mu_1} + \lambda_2 \mathbb{P}$ for any $\mathbb{P} \in P_2(P_2(M))$
if we approxiamte $\mathbb{P}$ by finite supported measures.

To attack even more general case,
we should be able to single out a part of $\mathbb{P}$ in the form $\lambda_1 \delta_{\mu_1}$
with lower bound on $\lambda_1$ and dominated $\mu_1$.
One way to do so is to assume $\mu_1$ has bounded density.
\begin{defn}
	[The set $ \mathcal { A } _ { L }$]
	For \( 0 < L < \infty \), let \( \mathcal { A } _ { L } \subset W_2(M) \) be the set of Borel probability
	measures with compact support on \( M \), absolutely continuous with respect to volume, whose densities have \( L ^ { \infty } \)
	norm less than or equal to \( L\).
\end{defn}

Note that, since the bound on the \( L ^ { \infty } \) norm is preserved under weak-* convergence,
\( \mathcal { A } _ { L } \) is a countable union of weakly-* closed set, and thus Borel measurable, subset of \( W_2( M ) \).

% If $\mu_1 \in \mathcal{A}_L$, then by duality between $L^p(\bar{\mu})$ and $L^q(\bar{\mu})$
% and that they are separable spaces,
% we have that $\bar{\mu}$ has density in $L^p(\bar{\mu})$ for
% $p < \infty$ with a upper norm bound depending only on constat $L$ (, $\lambda_1$ and $M$).
% We pass $p$ to infity to get $\bar{\mu} \in \mathcal{A}_L$.
% Hence, our conclusion holds for any $\mathbb{P}$ that is not atomless on $\mathcal{A}_L$.

Finally, if we are given $\mathbb{P}$ with only assumption that $\mathbb{P}(\mathcal{A}_L(M)) > 0$,
we need more control on the density function of $\bar{\mu}_m$
to ``remove'' the dependency of its upper bound on a singled-out coefficient $\lambda_1$.
One hope is that $\lambda_1$ is replaced by $\mathbb{P}(\mathcal{A}_L(M))$.


\subsection{Calculate density function}

Every measure $\mu$ on $M$ in following discussion is absolutely continous and has compact support.
Here we use the fact that all possible values of $B(x_1, \ldots, x_n)$ are contained in a bounded set if each $x_i$ varies in a bounded set.
To conculde it, since $f$ is continous, there is a upper bound of $f$ and thus a upper bound of distance between $x_i$ and $B(x_1, \ldots, x_n)$.
% We always apply continuities of tranfer maps to get compact support of barycenter measure.
By convention, we denote by $g$ the density function for absolutely measure $\mu$.
One principle of differential geometry is to differentiate everything once we could.

% If we only apply general bound on $c$-concave function
Recall change of variable in \cref{thm:jacobian_identity} (Theorem 4.2 in \cite{cordero2001riemannian}),
\[
	\bar{g} = g_i \circ T_i \det D T_i :
	= g_i \circ T_i \det[Y(H-\text{Hess} u_i)]
\]
where $T_i = \exp(-\nabla u_i)$ is the unique optimal maps from $\bar{\mu}$ to $\mu_i$.
Note that this is why we need that all measures have compact support.

For general $c$-concave function $\mu_i$, we have only upper hessian bound.
And this is not enough to get an estimation on the absolute value of Jacobian determinants.
% For instance, in Euclidean space, $DT_i = (\lambda_i -1)/\lambda_i \leq \text{Id}$.

We differentiate the equality $B(T_1(x), \ldots, T_n(x))=x$ for $\bar{\mu}$-a.e. $x$,
\begin{align*}
	\text{Id} =\sum_{i=1}^n \partial_i B\, DT_i
	 & =\sum_{i=1}^n D \exp(-\frac{1}{\lambda_i}\nabla f_i) \, D \exp(-\nabla u_i)                     \\
	 & =\sum_{i=1}^n D \exp^{-1}(-\nabla \left( \frac{f_i}{\lambda_i}\right)^c) \, D \exp(-\nabla u_i) \\
	 & =\sum_{i=1}^n(H-\text{Hess}(f_i / \lambda_i)^c)^{-1}\,Y_i^{-1}\,
	Y_i\,(H-\text{Hess} u_i)                                                                           \\
	 & =\sum_{i=1}^n(H-\text{Hess}(f_i / \lambda_i)^c)^{-1}\,
	(H-\text{Hess}u_i) .
\end{align*}

Then by Minkowski's determinant inequality, we get
\begin{align*}
	1 & \geq \sum_{i=1}^{n} \det [H-\text{Hess}(f_i/\lambda_i)^c]^{-1/n}\,\det[H-\text{Hess}u_i]^{1/n} \\
	  & =\sum_{i=1}^n \det[\partial_i B]^{1/n}\,\det[DT_i]^{1/n}
\end{align*}
Observe that in our discussion $(f_i / \lambda_i)^c$ is calculated at the barycenter $x$ of $\sum_{j=1}^{n} \delta_{T_j x}$,
and patial derivative means all $T_j x$ for $ j \ne i$ are fixed in calculation.
Hence, we can have $(f_i / \lambda_i)^c = g_i$.
Recall that we know $(f_i / \lambda_i)^c$ has hessian bound from both sides, see \cref{equa:hessian_bound_f}.
From it we get $\det[\partial_i B]^{1/n} \geq \min \{1, \lambda_i / (1 - \lambda_i)\} > \lambda_i$,
where $C > 0$ depends only on hessian bound of square distance function and Lipschitz constant of exponential map.

Combine these two inequalities, and we then apply Jensen inequlity
\[
	\bar{g} \leq
	\left[ \sum_{i=1}^n \frac{\det[\partial_i B]^{1/n}}
	{g_i^{1/n} \circ T_i}\right]^{-n}
	< \left[ \sum_{i=1}^n \frac{C \, \lambda_i}
	{g_i^{1/n} \circ T_i}\right]^{-n}
	\leq C^{-n} \sum_{i=1}^n \lambda_i g_i \circ T_i.
\]
With this estimation in hand,
one shows easily that if measure $\mathbb{P} \in \mathcal{W}_2(\mathcal{W}_2(M))$ on $W_2(M)$ give mass to the measurable set $\mathcal{A}_L$,
then it has a unique absolutely continous barycenter.

\subsubsection{Jacobian determinant inequality for the Wasserstein barycenter}

This is done by Kim and Pass.
\begin{defn}[Volume distortion]
	Let \( \lambda \) be a Borel probability measure on \( M \) with a
	unique barycenter \( \bar { x } \) (that is, such that \( B C ( \lambda ) \) is a singleton). We define the generalized,
	or barycentric, volume distortion coefficients at \( y \notin \operatorname { cut } ( \bar { x } ) \)

	\[ \alpha _ { \lambda } ( y ) : = \frac { \operatorname { det } \left[ - \left. D _ { y z } ^ { 2 } \right| _ { z = \bar { x } } c ( y , z ) \right] } { \operatorname { det } \left[ \left. \int _ { M } D _ { z z } ^ { 2 } \right| _ { z = \bar { x } } c ( x , z ) d \lambda ( x ) \right] } \]
	where \( D _ { z z } ^ { 2 } c ( x , z ) \) denotes the Hessian of the function \( z \mapsto c ( x , z ) \), and the determinants
	are computed in exponential local coordinates at \( \bar { x } \) and \( y . \)
\end{defn}

\begin{thm}
	[Jacobian determinant inequality for the Wasserstein barycenter]
	Assume that the Wasserstein barycenter \( \bar { \mu } \) of the measure \( \Omega \) on \( P ( M ) \) is absolutely continuous.
	Letting \( T _ { \mu } \) denote the optimal map from \( \bar { \mu } \) to \( \mu \), consider the measure on \( M \) given by
	\[ \lambda _ { x } : = \int _ { P ( M ) } \delta _ { T _ { \mu } ( x ) } d \Omega ( \mu ) \]
	which is defined with respect to a.e. $x$.
	Then, for \( \bar { \mu } \)-a.e. \(x\),
	\[ 1 \geq \int _ { P ( M ) } \alpha _ { \lambda _ { x } } ^ { 1 / n } \left( T _ { \mu } ( x ) \right) \operatorname { det } ^ { 1 / n } D T _ { \mu } ( x ) d \Omega ( \mu ) \]
\end{thm}

% \subsubsection{Use local coordinate}
% Need to work on it.

\subsubsection{Use Skorohod representation}
\textcolor{red}{This is not possible!}
One may consider to construct absolutely continous random variables.
For example, to use the Skorohod representation (see section 8.5 in \cite{Bogachev2007}),
\begin{defn}
	We shall say that a topological space \( X \) has the strong
	Skorohod property for Radon measures if to every Radon probability measure
	\( \mu \) on \( X \),
	one can associate a Borel mapping \( \xi _ { \mu } : [ 0,1 ] \rightarrow X \) such that \( \mu \) is
	the image of Lebesgue measure under the mapping \( \xi _ { \mu } \) and \( \xi _ { \mu _ { n } } ( t ) \rightarrow \xi _ { \mu } ( t ) \) a.e.
	whenever the measures \( \mu _ { n } \) converge weakly to \( \mu . \)
\end{defn}

However, even if $\mu$ is absolutely continous respect to Lebesgue measure on $\mathbb{R}$,
we don't have necessarily that $\xi_{\mu}$ is a absolutely continous function.
In fact, the Housdorff dimension of the image of $\xi_{\mu}$ is not likely to be greater than 2.
See \cite{Besicov1937Sets} for discussions on $\delta$-Lipschitz curves,
their Housdorff dimensions are bounded by $2-\delta$.

% \subsection{Control on density function}
