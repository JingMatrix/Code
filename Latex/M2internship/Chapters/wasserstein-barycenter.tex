%! TEX root = ../barycenter.tex
\chapter{Wasserstein space}

Paper review for \emph{Existence and consistency of Wasserstein barycenters}\cite{le2017existence}.

By \cref{topology_Wasserstein}, if $(E,d)$ is a Polish space, then so is $\mathcal{W}_p(E)$. Recall that we only need Polish space to define Wasserstein metric, hence we can talk about $\mathcal{W}_p(\mathcal{W}_p(E))$, and hence study barycenter in Wasserstein space.

% If assume further $X$ is a locally compact length space, then $\mathcal{W}_p(E)$ is a strictly intrinsic length space (see Corollary 7.22 in \cite{villani2008optimal}, recalled in this report as \cref{geodesic_Wasserstein_space}) with respect to $\mathcal{W}_p(E)$. That is the case in this paper, we assume in this section that $X$ is geodesic locally compact space. Now $\mathscr{W}_p(E)$ is a Polish geodesic space, however it is not necessarily a locally compact space.

To take care of notation, we use following convention; we will recall them when necessary,
\begin{itemize}
	\item $x$ and $y$, points in Polish space $(E,d)$.
	\item $\mu$ and $\nu$, Borel probability measures on $(E,d)$ with finite $p$-moment, i.e., elements in $\mathcal{W}_p(X)$.
	\item $X$ and $Y$, random variables in $(E, d)$, i.e., Borel measurable functions with value in $E$.
	\item $\mathbb{P}$, Borel measures on Wasserstein space $\mathcal{W}_p(E)$.
	\item \( \hat{\mu} \), random probability measure in $\mathcal{W}_p(E)$, i.e., Borel measurable functions with value in $\mathcal{W}_p(E)$.
	\item $(\Omega, \mathcal{B}, P)$, an abstract measure space which we use as base space for random variables or measures. That is to say, we may write $X_{\#}(P)=\mu$ and $\hat{\mu}_{\#}(P)=\mathbb{P}$, where \# means push-forward of measures.
	\item $W_p$, Wasserstein distance of $\mathcal{W}_p(E)$ and $\mathcal{W}_p(\mathcal{W}_p(E))$.
	\item $W_p(\mu, E)$, distance between $\mu$ and $E$ in $\mathcal{W}_p(E)$.

	      If barycenter $\delta_x$ of $\mu$ exists, then one has $W_p(\mu, \delta_x)=W_p(\mu, E)$.
	\item $W_p(\mathbb{P}, \mathcal{W}_p(E))$, distance between $\mathbb{P}$ and $\mathcal{W}_p(E)$ in $\mathcal{W}_p(\mathcal{W}_p(E))$.

	      If barycenter $\delta_\mu$ of $\mathbb{P}$ exists, then one has $W_p(\mathbb{P}, \delta_\mu)=W_p(\mathbb{P}, \mathcal{W}_p(E))$.
	\item $\mu_n \Rightarrow \mu$ stands for weak convergence of measures.
	\item $\mu_n \rightarrow \mu$ stands for convergence in Wasserstein metric.

\end{itemize}

Compactness is crucial for the existence of barycenter. Bad news is that Wasserstein space is not locally compact unless the base space is compact (see Remark 7.1.9 in \cite{Ambrosio2005}). Good news is that weak convergence behaves well with Wasserstein metric and we can get weak compactness more easily.

Properness of base space gives properness of Wasserstein space for weak convergence topology.

\begin{prop}
	For a proper Polish space $(E,d)$, every bounded set in $\mathcal{W}(E)$ is tight, hence pre-compact in weak convergence topology.
\end{prop}
\begin{proof}
	By Markov inequality, $\mu(E\backslash B(x,r)) \leq W_p(\mu, \delta_x)^p/r^p$, measure of closed ball (compact by assumption) of sufficient large radius can be arbitrarily close to $1$.
\end{proof}

% Apply the definition of barycenter, condition in this proposition will be satisfied by lower semi-continuity of Wasserstein distance.
This simple proposition is enough to build existence of barycenter.

By abuse of language, we drop $\delta$ symbol in following discussion when there is no ambiguity.
\begin{thm}
	Let $(E,d)$ be a proper Polish space. Assume $\mathbb{P}_n$ is a sequence with barycenter $\mu_n$ and $\mathbb{P}_n \rightarrow \mathbb{P}$ with respect to Wasserstein metric. Then $\mu_n$ is tight and any weak limit $\mu$ of $\mu_n$ will be a barycenter of $\mathbb{P}$.
\end{thm}

\begin{proof}
	Existence of $\mu$ as a weak limit comes from boundness of $\mu_n$.

	By continuity of Wasserstein distance, we have
	\[
		W_p(\mathbb{P}, \mathcal{W}_p(E)) = \lim_{n \rightarrow \infty}W_p(\mathbb{P}_n, \mathcal{W}_p(E))=\lim_{n \rightarrow \infty}W_p(\mathbb{P}_n, \mu_n)
	\]
	By lower semi-continuity of Wasserstein distance, we have
	\[
		\lim_{n \rightarrow \infty}W_p(\mathbb{P}_n, \mu_n)=\lim_{n \rightarrow \infty}W_p(\mathbb{P}, \mu_n)\geq W_p(\mathbb{P}, \mu)
	\]
	Hence, $W_p(\mathbb{P}, \mathcal{W}_p(E)) =W_p(\mathbb{P}, \mu)$.
\end{proof}

As weak convergence is near to Wasserstein convergence, we can see from following proposition that previous weak convergence is in fact convergence in Wasserstein metric.

\begin{prop}
	For a weak convergence sequence $\mu_n \Rightarrow \mu$, the following condition are sufficient and necessary (hence equivalent) to have $\mu_n \rightarrow \mu$ convergence in Wasserstein metric.
	\begin{enumerate}
		\item $W_p(\mu_n, \delta_x) \rightarrow W_p(\mu, \delta_x)$ for an element $x$ in $E$.
		\item $W_p(\mu_n, \nu) \rightarrow W_p(\mu, \nu)$ for an element $\nu$ in $\mathcal{W}_p(E)$.
		\item $W_p(\delta_{\mu_n}, \mathbb{P}) \rightarrow W_p(\delta_\mu, \mathbb{P})$ for an element $\mathbb{P}$ in $\mathcal{W}_p(\mathcal{W}_p(E))$.
	\end{enumerate}
\end{prop}

That is say, what we need in addition is the convergence of Wasserstein distance with respect to a point either in $E$, $\mathcal{W}_p(E)$ or $\mathcal{W}_p(\mathcal{W}_p(E))$.

\begin{proof}
	The first point is stated in \cref{thm:Wp_metrizes_weak_convergence}. The second point is proved in this paper as Lemma 14. The last one is from the second one.

	Take $\hat{\mu}$ a random probability measure with law $\mathbb{P}$, by lower semi-continuity of Wasserstein metric with respect to weak convergence we have
	\begin{align*}
		\mathbb{E}W_p(\mu, \hat{\mu}) & =W_p(\mu, \mathbb{P})=\lim_{n \rightarrow \infty} W_p(\mu_n, \mathbb{P}) \\
		                              & = \mathbb{E}\lim_{n \rightarrow \infty} W_p(\mu_n, \hat{\mu})           \\
		                              & \geq \mathbb{E}W_p(\mu, \hat{\mu}).
	\end{align*}
	Therefore, we have almost everywhere that $\lim_{n \rightarrow \infty} W_p(\mu_n, \hat{\mu})=W_p(\mu, \hat{\mu})$.
\end{proof}

As finite support measure is dense in Wasserstein space, the final goal should be the existence of barycenter for finite support measure in $\mathcal{W}_p(E)$.

\begin{thm}
	Let $(E,d)$ be a locally compact Polish space. Then barycenter of finite support measure on $\mathcal{W}_p(E)$ always exists.
\end{thm}

This is a standard convex optimization problem as $\mathcal{W}_p(E)$ is a convex space and distance function is convex.

\begin{proof}
	Let $\lambda_i$ be $n$ given positive coefficients with sum $1$, $\mu_i$ be $n$ given measures on $(E,d)$. We aims to find solution for optimization problem
	\[
		\min_{\nu} \sum_{i=1}^{n}\lambda_i W_p(\nu, \mu_i)^p.
	\]
	For a given $\nu$, let $(X, X_0, X_1,\ldots,X_n)$ be a choice of random variable with value in $E^{n+1}$ such that $(X,X_i)$ is an optimal coupling for $\nu$ and $\mu_i$. We thus have $\mathbb{E}d(X,X_i)^p = W_p(\nu, \mu_i)^p$.

	Define $f(x_1, x_2, \ldots, x_n):=\min_{x \in E} \sum_{i=1}^{n} \lambda_i d(x, x_i)^p$, we use ``min" as $E$ is locally compact then minimum always exists. Denote $\Gamma$ the set of all possible choice of $(X_1, \ldots, X_n)$, we have
	\begin{align*}
		\sum_{i=1}^{n}\lambda_i W_p(\nu, \mu_i)^p &= \mathbb{E} \sum_{i=1}^{n}\lambda_i d(X,X_i)^p \\
																						&\geq \mathbb{E} f(X_1, \ldots, X_n)\\
																						&\geq \min_{\Gamma}\mathbb{E} f(X_1, \ldots, X_n)
	\end{align*}
	These two inequality can be in fact equality, as
	\begin{itemize}
		\item For the first inequality, we need that the existence of a measurable function $T(x_1, x_2, \ldots, x_n):= \arg \min_{x \in E} \sum_{i=1}^{n} \lambda_i d(x, x_i)^p$. This is guaranteed by measurable selection theorem.
		\item For the second one, as $f$ is continuous on $E^n$. The existence of solution to this minimization problem on $\Gamma$ then reduces to a multi-marginal optimal transportation problem with cost funtion $f$.
	\end{itemize}
To be concrete, let $\boldsymbol \gamma$ be a solution to the (generalizes) optimal transportation problem of marginal distribution $(\mu_1, \ldots, \mu_n)$ with respect to cost funtion $f$. Then $\mu:= T_{\#}\boldsymbol \gamma$ will be solution to our optimization problem, i.e., barycenter of $\sum_{i=1}^{n}\lambda_i \delta_{\mu_i}$.

\end{proof}

