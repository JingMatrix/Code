%! TEX root = ../barycenter.tex
\chapter{Wasserstein space over proper space}

Paper review for \emph{Existence and consistency of Wasserstein barycenters}\cite{le2017existence}.

By \cref{topology_Wasserstein}, if $(E,d)$ is a Polish space, then so is $\mathcal{W}_p(E)$. Recall that we only need Polish space to define Wasserstein metric, hence we can talk about $\mathcal{W}_p(\mathcal{W}_p(E))$, and hence study barycenter in Wasserstein space.

% If assume further $X$ is a locally compact length space, then $\mathcal{W}_p(E)$ is a strictly intrinsic length space (see Corollary 7.22 in \cite{villani2008optimal}, recalled in this report as \cref{geodesic_Wasserstein_space}) with respect to $\mathcal{W}_p(E)$. That is the case in this paper, we assume in this section that $X$ is geodesic locally compact space. Now $\mathscr{W}_p(E)$ is a Polish geodesic space, however it is not necessarily a locally compact space.

To take care of notation, we use following convention; we will recall them when necessary,
\begin{itemize}
	\item $x$ and $y$, ponts in Polish space $(E,d)$.
	\item $\mu$ and $\nu$, Borel probability measures on $(E,d)$ with finite $p$-moment, i.e., elements in $\mathcal{W}_p(X)$.
	\item $X$ and $Y$, random variables in $(E, d)$, i.e., Borel measurable functions with value in $E$.
	\item $\mathbb{P}$, Borel measures on Wasserstein space $\mathcal{W}_p(E)$.
	\item \( \hat{\mu} \), random probability measure in $\mathcal{W}_p(E)$, i.e., Borel measurable functions with value in $\mathcal{W}_p(E)$.
	\item $(\Omega, \mathcal{B}, P)$, an abstract measure space which we use as base space for random variables or measures. That is to say, we may write $X_{\#}(P)=\mu$ and $\hat{\mu}_{\#}(P)=\mathbb{P}$, where \# means push-forward of measures.
	\item $W_p$, Wasserstein distance of $\mathcal{W}_p(E)$ and $\mathcal{W}_p(\mathcal{W}_p(E))$.
	\item $W_p(\mu, E)$, distance between $\mu$ and $E$ in $\mathcal{W}_p(E)$.

	      If barycenter $\delta_x$ of $\mu$ exists, then one has $W_p(\mu, \delta_x)=W_p(\mu, E)$.
	\item $W_p(\mathbb{P}, \mathcal{W}_p(E))$, distance between $\mathbb{P}$ and $\mathcal{W}_p(E)$ in $\mathcal{W}_p(\mathcal{W}_p(E))$.

	      If barycenter $\delta_\mu$ of $\mathbb{P}$ exists, then one has $W_p(\mathbb{P}, \delta_\mu)=W_p(\mathbb{P}, \mathcal{W}_p(E))$.
	\item $\mu_n \Rightarrow \mu$ stands for weak convergence of measures.
	\item $\mu_n \rightarrow \mu$ stands for convergence in Wasserstein metric.

\end{itemize}

Compactness is crucial for the existence of barycenter. Bad news is that Wasserstein space is not locally compact unless the base space is compact (see Remark 7.1.9 in\cite{Ambrosio2005}). Good news is that weak convergence behaves well with Wasserstein metric and we can get weak compactness more easily.

Properness of base space gives properness of Wasserstein space for weak convergence topology. We remark that a proper metric space is Polish.

\begin{prop}
	For a proper space $(E,d)$, every bounded set in $\mathcal{W}(E)$ is tight, hence pre-compact in weak convergence topology.
\end{prop}
\begin{proof}
	By Markov inequality, $\mu(E\backslash B(x,r)) \leq W_p(\mu, \delta_x)^p/r^p$, measure of closed ball (compact by assumption) of sufficient large radius can be arbitrarily close to $1$.
\end{proof}

% Apply the definition of barycenter, condition in this proposition will be satisfied by lower semi-continuity of Wasserstein distance.
This simple proposition is enough to build existence of barycenter.

By abuse of language, we drop $\delta$ symbol in following discussion when there is no ambiguity.
\begin{thm}
	Let $(E,d)$ be a proper space. Assume $\mathbb{P}_n \in \mathcal{W}_p(\mathcal{W}_p(E))$ is a sequence with barycenter $\mu_n$ and $\mathbb{P}_n \rightarrow \mathbb{P}$ with respect to Wasserstein metric. Then $\mu_n$ is tight and any weak limit $\mu$ of $\mu_n$ will be a barycenter of $\mathbb{P}$.
\end{thm}

\begin{proof}
	Existence of $\mu$ as a weak limit comes from boundness of $\mu_n$.
	To show that barycenters of bounded set is bounded, by abuse of notation, for a $x$ in $E$,
	\[
		W_p(\mu_n, x) \leq W_p(\mu_n, \mathbb{P}_n)  + W_p(\mathbb{P}_n, x) \leq 2 W_p(\mathbb{P}_n , x),
	\]
	we used definition of barycenter in the second inequality.
	And the last item is bounded as $\mathbb{P}_n$ is a converging sequence.

	By continuity of Wasserstein distance, we have
	\[
		W_p(\mathbb{P}, \mathcal{W}_p(E)) = \lim_{n \rightarrow \infty}W_p(\mathbb{P}_n, \mathcal{W}_p(E))=\lim_{n \rightarrow \infty}W_p(\mathbb{P}_n, \mu_n)
	\]
	By lower semi-continuity of Wasserstein distance, we have
	\[
		\lim_{n \rightarrow \infty}W_p(\mathbb{P}_n, \mu_n)=\lim_{n \rightarrow \infty}W_p(\mathbb{P}, \mu_n)\geq W_p(\mathbb{P}, \mu)
	\]
	Hence, $W_p(\mathbb{P}, \mathcal{W}_p(E)) =W_p(\mathbb{P}, \mu)$.
\end{proof}

As weak convergence is near to Wasserstein convergence, we can see from following proposition that previous weak convergence is in fact convergence in Wasserstein metric.

\begin{prop}
	For a weak convergence sequence $\mu_n \Rightarrow \mu$, the following condition are sufficient and necessary (hence equivalent) to have $\mu_n \rightarrow \mu$ convergence in Wasserstein metric.
	\begin{enumerate}
		\item $W_p(\mu_n, \delta_x) \rightarrow W_p(\mu, \delta_x)$ for an element $x$ in $E$.
		\item $W_p(\mu_n, \nu) \rightarrow W_p(\mu, \nu)$ for an element $\nu$ in $\mathcal{W}_p(E)$.
		\item $W_p(\delta_{\mu_n}, \mathbb{P}) \rightarrow W_p(\delta_\mu, \mathbb{P})$ for an element $\mathbb{P}$ in $\mathcal{W}_p(\mathcal{W}_p(E))$.
	\end{enumerate}
\end{prop}

That is say, what we need in addition is the convergence of Wasserstein distance with respect to a point either in $E$, $\mathcal{W}_p(E)$ or $\mathcal{W}_p(\mathcal{W}_p(E))$.

\begin{proof}
	The first point is stated in \cref{thm:Wp_metrizes_weak_convergence}. The second point is proved in this paper as Lemma 14. The last one is from the second one.

	Take $\hat{\mu}$ a random probability measure with law $\mathbb{P}$, by lower semi-continuity of Wasserstein metric with respect to weak convergence we have
	\begin{align*}
		\mathbb{E}W_p(\mu, \hat{\mu}) & =W_p(\mu, \mathbb{P})=\lim_{n \rightarrow \infty} W_p(\mu_n, \mathbb{P}) \\
		                              & \geq \mathbb{E}\liminf_{n \rightarrow \infty} W_p(\mu_n, \hat{\mu})      \\
		                              & \geq \mathbb{E}W_p(\mu, \hat{\mu}).
	\end{align*}
	The first inequality comes from Fatou's lemma. Now these two inequalities are in fact equalities. An subsequence of $\mu_n$ will satisfy this equality as well. Hence we can prove by contradiction that $\liminf$ above can be safely substitued by $\lim$.
	Therefore, we have almost everywhere that $\lim_{n \rightarrow \infty} W_p(\mu_n, \hat{\mu})=W_p(\mu, \hat{\mu})$.
\end{proof}

As finite support measure is dense in Wasserstein space, the final goal should be the existence of barycenter for finite support measure in $\mathcal{W}_p(E)$.

\begin{thm}
	\label{thm:barycenter_finite_support_measure}
	Let $(E,d)$ be a proper space. Then barycenter of finite support measure on $\mathcal{W}_p(E)$ always exists.
\end{thm}

This is a standard convex optimization problem as $\mathcal{W}_p(E)$ is a convex space (with linear convex structure) and distance function is convex.

\begin{proof}
	Let $\lambda_i$ be $n$ given positive coefficients with sum $1$, $\mu_i$ be $n$ given measures on $(E,d)$. We aims to find solution for optimization problem
	\[
		\min_{\nu} \sum_{i=1}^{n}\lambda_i W_p(\nu, \mu_i)^p.
	\]
	For a given $\nu$, let $(X, X_1,\ldots,X_n)$ be a choice of random variable with value in $E^{n+1}$ such that $(X,X_i)$ is an optimal coupling for $\nu$ and $\mu_i$. We thus have $\mathbb{E}d(X,X_i)^p = W_p(\nu, \mu_i)^p$.

	Denote $\boldsymbol{x}=(x_1, x_2, \ldots, x_n) \in E^n$ and define $f(\boldsymbol{x})= W_p(\eta, E)^p$, where $\eta = \sum_{i=1}^{n} \lambda_i \delta_{x_i}$. We use ``min'' in this definition as barycenter of measures in $E$ always exists. Denote $\Gamma$ the set of all possible choice of $(X_1, \ldots, X_n)$, we have
	\begin{align*}
		\sum_{i=1}^{n}\lambda_i W_p(\nu, \mu_i)^p & = \mathbb{E} \sum_{i=1}^{n}\lambda_i d(X,X_i)^p \\
		                                          & \geq \mathbb{E} f(X_1, \ldots, X_n)             \\
		                                          & \geq \inf_\Gamma \mathbb{E} f(X_1, \ldots, X_n)
	\end{align*}
	These two inequalities can be in fact equalities.

	For the second one, the existence of solution to this minimization problem on $\Gamma$ is a multi-marginal optimal transportation problem with cost funtion $f$.

	Inspired by the proof of existence of optimal plan, we should first show that $\Gamma$ is weakly compact. $\Gamma$ is tight as elements in it have pre-fixed marginal distriutions. $\Gamma$ is weakly closed by stability of optimal plans, which is guaranteed by lower semi-continuity of transport cost with respect to weak convergence.
	% To prove stability, we need the dual formulation of optimal transportation and the fact that a plan is optimal if and only if it is concentrated on a cyclical monotone set, see Theorem 5.12 in \cite{villani2008optimal}.

	What we need to show in addition is that $f$ is indeed lower semi-continous. Set $\eta:= \sum_i^{n}\lambda_i \delta_{x_i} \in \mathcal{W}_p(E)$, observe that sequence convergence of $\boldsymbol{x}$ in $E^n$ corresponds to weak convergence of $\eta$. And we see that $f(\boldsymbol{x}) = W_p(\eta, E)^p=W_p(\eta, y)^p = \sum_{1}^{n} \lambda_i d(y, x_i)^p$ for some barycenter $y$ of $\eta$ in $E$. Let $\eta_j$ be a sequence of measures of such form corresponding to $\boldsymbol{x}_j$ and let $y_j$ be a barycenter of $\eta_j$.

	If we can have $y_j \rightarrow y$ up to a subsequence for some $y$, then by continuity of distance fucntion and $\boldsymbol{x}_j \rightarrow \boldsymbol{x}$
	\[
		\lim W_p(\eta_j, E) = \lim W_p(\eta_j, y_j) =  W_p(\eta, y).
	\]

	It is necessary that $y$ is a barycenter of $\eta$ as a limit of $y_j$, since for a barycenter $z$ of $\eta$ we can pass to the limit in $W_p(\eta_i, z) \geq W_p(\eta_i, E)$.

	% As $E$ is proper, our primary aim is to show that $y_i$ is bounded in $E$.

	Recall that bounded set in Wasserstein space is weakly pre-compact and weakly convergence of Dirac measures has the same topology as base space. Hence we only need to show that $y_i$ is bounded is the Wasserstein space $\mathcal{W}_p(E)$. This comes from that $\eta_j$ is bounded and barycenters of bounded set is bounded as we proved before.
	This shows that $f(\boldsymbol{x})$ is continous with respect to $\boldsymbol{x}$ since
	any sequence of $f(\boldsymbol{x_i})$ has a subsequence converging to $f(\boldsymbol{x})$.

	For the first inequality, we need to show the existence of a measurable function $B(\boldsymbol{x}) \in \arg \min_{x \in E} \sum_{i=1}^{n} \lambda_i d(x, x_i)^p$. This is guaranteed by measurable selection theorem. The set
	\[
	\Gamma:=	\{
		(y,\boldsymbol{x}) \in E^{n+1}\mid  f(\boldsymbol{x}) - \sum_{i=1}^{n} \lambda_i d(y,x_i)^p = 0
		\}
	\]
	is closed as a pre-image of continous function.
	% is measurable as lower semi-continous function $f$ is a limit of continous function hence Borel measurable.
% In fact, this set is closed because lower level sets for lower semi-continous function are closed
% and all arguments that attains minimun vaule is a level set.
We claim that the sliced set $\Gamma_{\boldsymbol{x}}:=\{y \mid (y, \boldsymbol{x}) \in \Gamma\}$ is compact,
% is a closed subset of $E$, so it is $\sigma$-compact.
because barycenters are located in the union of $n$ bounded balls with centers $x_i$.
We conculde the existence of measurable seletion function $B$ by \cref{thm:measurale_selection}.

	For our proof, to be concrete, let $\boldsymbol \gamma$ be a solution to the second inequality.
	Then $\mu:= B_{\#}\boldsymbol \gamma$ will be a solution to our optimization problem, i.e., barycenter of $\sum_{i=1}^{n}\lambda_i \delta_{\mu_i}$.
\end{proof}

\begin{rmk}
	If one $\mu_i$ is absolutely continous respect to volume measure on complete manifold $E$, then the barycenter $\nu$ is unqiue.
	We can argue the uniquness of barycenter $\mu:= B_{\#}\boldsymbol \gamma$ by classical Brenier theorem in the case of $\mathbb{R}^n$.
	We need to use duality argument, see Definition 3.6 in \cite{agueh2011barycenters} and this is done in \cite{kim2015multi} for the case of compact Riemannian manifold.
\end{rmk}
