%! TEX root = ../barycenter.tex
\chapter{Papers Review}
In this chapter, I shall review in order papers: \cite{ohta2012barycenters}, \cite{le2017existence}.
\section{Barycenters in Alexandrov spaces of curvature bounded below}
\begin{prop}[Example 3.1 a]
	Let \( X \) be the infinite dimensional ellipsoid of axes of lengths \( c _ { n } = ( n + 1 ) / 2 n \) with \( n \in \mathbb { N } , \) namely \[ X = \left\{ \left( x _ { 1 } , x _ { 2 } , \ldots \right) \in \mathbb { R } ^ { \infty } \mid \sum _ { n \in \mathbb { N } } \frac { x _ { n } ^ { 2 } } { c _ { n } ^ { 2 } } = 1 \right\} \]

	Then \( X \) is complete, but \( \mu = \left( \delta _ { ( 1,0,0 , \ldots ) } + \delta _ { ( - 1,0,0 , \ldots ) } \right) / 2 \) has no barycenter in \( X \).
\end{prop}

\begin{proof}
	As $ 1 / 2 \leq c_{n} \leq 1$, we know $X$ is closed subspace of Hilbert space $\ell^2$. Hence we can calculate distance through inner product. Pick $ x \in X$ and set $ e=(1,0,0\ldots)$, we have:
	\begin{align*}
		W_2(\delta_x, \mu)^2 & = \int_{X} d(x, \cdot)^2 \diff \mu = \frac{d(x,e)^2+d(x,-e)^2 }{2} \\
		                     & =\frac{\Vert x - e \Vert^2 + \Vert x + e \Vert^2}{2}               \\
		                     & =\Vert x \Vert^2 + 1
	\end{align*}

	Hence a barycenter of $\mu$ in $X$ should minimize its length. For a vector $x$ in $X$ to attain minimum length, if restricting to the first $n$ coordinate components, $x$ cannot have nonzero components except the last one (minimum length axes), i.e., $x_i =0$ for $i<n$. Otherwise we can keep other not considered coordinate components, $x_j$ for $j>n$,  unchanged but vanish first $n-1$ coordinate components to get a strictly shorter vector in $X$. This indicates that no such barycenter $x$ could exist in $X$.
\end{proof}

\begin{rmk}
	\begin{enumerate}
		\item Hilbert space is not locally compact, as unit closed ball in infinite dimensional normed space is not sequential compact. Recall that the last claim can be proved using \href{https://en.wikipedia.org/wiki/Riesz%27s_lemma}{Wikipedia: Riesz's lemma}.
		\item % $X$ is the unit sphere of $\ell^2$ with a different inner product: \[\langle x, y \rangle:=\sum_{i \in \mathbb{N}^*}\frac{x_i}{c_i}\cdot \frac{y_i}{c_i}.\]
		      We see that $X$ is homeomorphic to the unit sphere in $\ell^2$. From previous point, $X$ is then not compact. However, this conclusion is almost trivial as we don't have barycenter for $\mu$ in $X$.
		\item $X$ is in fact not locally compact as a subspace of $\ell^2$, as proved in \cite{bessaga1975selected} Chapter IV §2 Theorem 2.1, that unit sphere in $\ell^2$ is homeomorphic to $\ell^2$.
		\item $X$ with metric inherited form $\ell^2$ is apparently not a length space. To explore more of this example, we should start to consider induced length metric $\hat{d}$ on $X$ by inherited metric from $\ell^2$.
		\item $X$ is not locally compact with respect to the induced length metric topology. By compatibility of induced length structure with the topology of base space $X$ (see Exercise 2.1.5 in \cite{burago2001course}), open set in original topology is again open in the induced length metric topology. Cover an otherwise compact set of $X$ in new topology by open sets in original topology, we get a contradiction.
		      % \item Unit sphere in $\ell^2$ is geodesic with respect to induced length metric. By Theorem 2.4.16 in \cite{burago2001course}, we only need to show that midpoint in $X$ always exists, i.e., $\forall x,y \in X$, $\exists z \in X$ such that $d(x,z)=d(y,z)=\frac{1}{2}d(x,y)$. For such two points $x$ and $y$, $z$ can be selected from the intersection of $X$ and the hypersurface passing the origin in $\ell^2$ that evenly separates $x$ and $y$.
		      % \item  $X$ is geodesic with respect to induced length metric $\hat{d}$.
		      % Induced length structure in $X$. 
		      %Recall that shortest paths are closed under point-wise convergence (Proposition 2.5.17 in \cite{burago2001course}, proved by lower semi-continuity of length function).
		      % We show that distance in $X$ is approximate by curves with finite coordinate components. To begin with, any two points in $X$ with finite coordinate components can be connected by a shortest path. Then pass a point to have infinite coordinate components and later do the same for another point.
		      % We use the fact that points with finite coordinate components are dense in $\ell^2$ and so are they in $X$.
		\item A geodesic space $(E, \hat{d})$ always admits midpoint $z$ of two points $x$ and $y$ in $E$ as the barycenter of $\mu : = \frac{1}{2} (\delta_x + \delta_y)$, since $z$ attain equality in following general inequality:
		      \[
			      \hat{d}(x,y)^2 \leq \left(\hat{d}(x,z) + \hat{d}(z,y)\right)^2 \leq 2\left(\hat{d}(x,z)^2+ \hat{d}(z,y)^2\right)
		      \]
		\item An interesting question could be: consider $X$ with induced length metric $\hat{d}$, does the barycenter of $\mu$ exist?
		      Basically, we need to answer whether $\hat{d}(e,-e)$ is realized by a rectifiable curve in $X$.

		      Assume now we have $L_{d}(\gamma)=\hat{d}(e,-e)$.
		      Restricting to the first $n$ coordinate components, barycenters always exist and are exactly endponits of shortest axes. Existence is a priori guaranteed by compactness of finite dimensional ellipse.
		      % And those two endponits attain equality in following inequality:

		      % 		            \[
		      % 		      	      \hat{d}(e,-e)^2 \leq \left(\hat{d}(e,z) + \hat{d}(z,-e)\right)^2 \leq 2\left(\hat{d}(e,z)^2+ \hat{d}(z,-e)^2\right)
		      % 		            \]
		      The restricted induced length metric $\hat{d}_n(e,-e)$ depends on $n$ as we are considering minimal geodesics in $n-1$ dimensional ellipse. Actually, set $k=\sqrt{1-c_n^2}$ as the eccentricity, we have
		      \[
			      \hat{d}_n(e,-e)= 2\int _{0}^{\tfrac {\pi }{2}}{\sqrt {1-k^{2}\sin ^{2}\theta }}\diff \theta = 2\int _{0}^{1}{\frac {\sqrt {1-k^{2}t^{2}}}{\sqrt {1-t^{2}}}}\diff t
		      \]
		      This is a decreasing function in $n$.	To apply the same idea in the proof of previous proposition, we should decompose distance $d$ of $\ell^2$ into two parts, involving first $n$ coordinate components or not, $d^2=d_n^2+d_r^2$, i.e., $\Vert \cdot \Vert^2 = \Vert \cdot \Vert_n^2 + \Vert \cdot \Vert_r^2$ with $d_n=\Vert \cdot \Vert_n$ the distance in $\mathbb{R} ^n $. For $\Delta=[\delta_0=0,\delta_1, \ldots, \delta_m=1]$ all possible finite partition of $\gamma$, we have:
		      \[
			      \hat{d}(e,-e)=\sup_{\Delta} \sum_{i=1}^{m} \Vert\gamma(\delta_{i-1}) - \gamma(\delta_{i})\Vert
		      \]
		      % We use that square function is continuous in the second equality, and $\gamma_i$ is the value of $\gamma$ at $i$th partition.:w
		      We \textbf{would like to} have following energy variation hold:
		      \begin{equation}
			      \label{energy_variation_in_X}
			      \gamma = \operatorname{arg} \inf_{\eta} \sup_{\Delta} \sum_{i=1}^{m} \Vert\eta(\delta_{i-1})- \eta(\delta_{i})\Vert^2
		      \end{equation}
		      Then this formula \ref{energy_variation_in_X} will contradict our assumption. Write $\gamma$ in coordinate as $(\gamma_1, \gamma_2\ldots\gamma_n\ldots)$, to keep $\gamma_k$ for $ k > n$ unchanged. Up to choose a bigger integer $n$ we can assume $ \exists t \in [0,1]$, such that $\gamma_n(t) \neq 0$. Consider $\tilde{\gamma}^n = (\gamma_1, \gamma_2 \ldots \gamma_n)$ a function with $\tilde\gamma^n(t)$ in $E_{\gamma(t)}$ where
		      \begin{align*}
			      E_{\gamma(t)} : & = \left\{ (x_1,x_2 \ldots x_n) \in \mathbb{R}^n \mid \sum_{i=1}^n \frac{x_i^2}{ c_i^2} = c_{\gamma(t)}^2\right\} \\
			      c_{\gamma(t)}:  & =\sqrt{ 1- \sum_{i > n} \frac{\gamma_j(t)^2}{c_j^2}} \geq 0
		      \end{align*}
		      Finally $\eta := (0,0,\ldots, c_n c_{\gamma(\cdot)},\gamma_{n+1},\ldots)$ will violate \ref{energy_variation_in_X} since energy variation coincide with arc-length variation in complete Riemannian manifold and we have $\int_{[0,1]} c_n c_{\gamma(\cdot)} \diff \lambda < L_d(\tilde{\gamma}^n)$.

		      This problem should be more related to Hilbert Riemannian manifold, we put it aside for now.
		\item As Example 5.1 in \cite{grossman1965hilbert} (I suspect this is the origin of our example), we can actually show that $e$ and $-e$ cannot be connected by minimal geodesic.

		      Hilbert Riemannian maindfold  theory is needed to use this term smooth (the original author used it).
		      % we can safely replace it with rectifiable in this example.

		      Define \( T: X \rightarrow X \) by \( T x = y \), where
		      \[
			      y _ { 1 } = x _ { 1 } , y _ { 2 } = 0 , y _ { i } = \frac{c_i}{c_{i-1}} x_{i-1} \text { for } i \geq 3 . \]

		      Then \( T \) is a smooth map with only \( e \) and \(- e \) as fixed points. Any smooth curve from \( e\) to \( -e \) is taken by \( T \) into another such curve which is strictly shorter than the original, since length structure is induced from $\ell^2$ norm and $T$ is strictly decreasing with respect to $\ell^2$ norm as $c_i$ is a decreasing sequence. Therefore, there is no minimal geodesic from \( e \) to \( -e . \)

		      There I should discuss two different length metric appeared, one is induced from $\ell^2$ norm and the other one is smooth Riemannian length structure. Are they two coincided (on common admissible curves)? Two ways to think about this relation
		      \begin{enumerate}
			      \item Apply Theorem 2.4.3 in \cite{burago2001course}, we then need to show Riemannian length structure is lower semi-continuous with respect to $\ell^2$ metric.
			      \item Consider $X$ as a Hilbert submainifold of $\ell^2$, the norm for tangent space is the restriction of canonical norm of $\ell^2$.
		      \end{enumerate}

		\item Now there is another way to show that $X$ with metric from $\ell^2$ is not locally compact. $X$ is complete and there is rectifiable curve connecting $e$ and $-e$, by Exercise 2.5.25 in \cite{burago2001course} there is shortest path connecting $e$ and $-e$ if $X$ is locally compact. This proposition is proofed by showing that balls of induced length metric are pre-compact in the topology of original topology.

		\item We have answered partially the existence of barycenter of $\mu$ in $X$ with induced length metric. However, the barycenter of two points might exist without bothering minimal geodesics. nor bothering midpoints.


	\end{enumerate}
\end{rmk}
\begin{lem}[Lemma 3.2]
	If \( ( X , d ) \) is a proper metric space, then any \( \mu \in \mathcal { P } _ { 2 } ( X ) \) has a barycenter.
\end{lem}

\begin{proof}
	Fix \( z _ { 0 } \in X \) and take \( r > 1 \) large enough to satisfy
	\[ \mu \left( B \left( z _ { 0 } , r \right) \right) \geq \frac { 1 } { 2 } , \quad \int _ { X \backslash B \left( z _ { 0 } , r \right) } d \left( z _ { 0 } , x \right) ^ { 2 } d \mu ( x ) \leq 1 \]
	Then we have
	\[ \int _ { X } d \left( z _ { 0 } , x \right) ^ { 2 } d \mu ( x ) \leq r ^ { 2 } \cdot \mu \left( B \left( z _ { 0 } , r \right) \right) + 1 \leq r ^ { 2 } + 1 \]
	while for every \( w \in X \backslash B \left( z _ { 0 } , 3 r \right) \)
	\[ \int _ { X } d ( w , x ) ^ { 2 } d \mu ( x ) \geq \int _ { B \left( z _ { 0 } , r \right) } d ( w , x ) ^ { 2 } d \mu ( x ) > ( 2 r ) ^ { 2 } \cdot \mu \left( B \left( z _ { 0 } , r \right) \right) \geq 2 r ^ { 2 } \]
	holds. Therefore it is sufficient to consider the infimum of
	\[ w \longmapsto \int _ { X } d ( w , x ) ^ { 2 } d \mu ( x ) \]
	only for \( w \in B \left( z _ { 0 } , 3 r \right) , \) and it is achieved at some point due to the compactness of the closure of \( B \left( z _ { 0 } , 3 r \right) . \)
\end{proof}

\begin{rmk}[Counter-examples]
	\begin{enumerate}
		\item Locally compact length space is not sufficient, consider unit disk without origin and uniform measure on it.
		\item Complete metric space is not sufficient, consider last example.
		\item Complete length space is sufficient? That is the interest of last remark.
	\end{enumerate}
\end{rmk}
\begin{rmk}[Exmaple 2.1 c]
	% This example is in contradiction with \href{https://en.wikipedia.org/wiki/Hadamard_space}{Wikipedia: Hadamard space}. The author claims Hilbert space is of non-negative curvature, while Wikipedia says ``a normed space is an Hadamard space if and only if it is a Hilbert space".
	Hilbert space satisfies equality in triangle comparison with Eculidean space, hence can be regarded ``flat".
\end{rmk}

Then we discuss the infinitesimal structure of an Alexandrov space \( ( X , d ) . \)

\begin{defn}[space of directions]
	Fix \( z \in X \) and let \( \hat { \Sigma } _ { z } \) be the set of all unit speed geodesics emanating from \( z \). For \( \gamma , \eta \in \hat { \Sigma } _ { z } , \) by virtue of the curvature bound, the limit
	\begin{equation}
		\label{angle_def}
		\angle _ { z } ( \gamma , \eta ): = \arccos \left( \lim _ { s , t \downarrow 0 } \frac { s ^ { 2 } + t ^ { 2 } - d _ { X }\left( \gamma ( s ) , \eta ( t ) \right) ^ { 2 } } { 2 s t } \right)
	\end{equation}
	exists and is regarded as the angle (pseudo-)distance of \( \hat { \Sigma } _ { z } . \) We define the space of directions \( \left( \Sigma _ { z } , \angle _ { z } \right) \) at \( z \) as the completion of \( \Sigma _ { z } / \sim \) with respect to \( \angle _ { z } , \) where \( \gamma \sim \eta \) if \( \angle _ { z } ( \gamma , \eta ) = 0 . \)
\end{defn}

\begin{defn}[tangent cone]
	The tangent cone \( \left( C _ { z } , d _ { C _ { z } } \right) \) is defined as the Euclidean cone over \( \left( \Sigma _ { z } , \angle _ { z } \right) , \)
	that is to say,
	\begin{align*}
		C _ { z }:                                           & = \Sigma _ { z } \times [ 0 , \infty ) / \Sigma _ { z } \times \{ 0 \}          \\
		d _ { C _ { z } } ( ( \gamma , s ) , ( \eta , t ) ): & = \sqrt { s ^ { 2 } + t ^ { 2 } - 2 s t \cos \angle _ { z } ( \gamma , \eta ) }
	\end{align*}
\end{defn}

\begin{thm}[Computation of distance in tangent cone]
	Let $\gamma$ and  $\eta$ be two geodesics from unit interval into $X$ starting at $z$.  Then $\gamma^\prime(0)$ and $\eta^\prime(0)$ are naturally elements in $C_z$.  We can calculate their distance as:
	\[  d_{C_z}(\gamma^\prime(0),\eta^\prime(0))=\lim _ { t \downarrow 0 } \frac  {d ( \eta ( t ) , \gamma ( t ) )  }{ t }\]
\end{thm}

\begin{proof}
	This is a calculation of angle between $\gamma^\prime(0)$ and $\eta^\prime(0)$ with a specific selected converging process. To apply angle definition in \ref{angle_def}, normalsize $\gamma$ and $\eta$ as $\gamma\left(\vert\gamma^\prime(0)\vert \cdot\right)$ and $\eta\left(\vert\eta^\prime(0)\vert \cdot\right)$
	\[
		\cos\angle _ { z } ( \gamma , \eta ): = \left( \lim _ { s , t \downarrow 0 } \frac { s ^ { 2 } + t ^ { 2 } - d _ { X }\left( \gamma (\vert \gamma^\prime(0)\vert s ) , \eta (\vert \eta^\prime(0)\vert  t ) \right) ^ { 2 } } { 2 s t } \right)
	\]
	To get desired formula, put $ x := \vert  \gamma^\prime(0)\vert s =\vert \eta^\prime(0)\vert t $ and pass $x \downarrow 0$.
\end{proof}

\begin{rmk}[Lemma4.5, Lang and Schroeder's inequality]
	When passing from finit support measures to general ones, the author should use Wasserstein metric $W_p$ convergence but not only weak convergence. Otherwise, he cannot justify convergence of integrals of distance function $d_{C_z}$. Here we use the fact that finite supported measures is a dense subspace of Wasserstein space $\mathscr{W}_2(X)$ if $X$ is a Polish space (see \cite{villani2008optimal} Theorem 6.18), recalled in this report as Theorem \ref{topology_Wasserstein}.
\end{rmk}
\section{Existence and consistency of Wasserstein...}
By Theorem \ref{topology_Wasserstein}, if $X$ is a Polish space, then so is $\mathscr{W}_p(X)$. If assume further $X$ is a locally compact length space, then $\mathscr{W}_p(X)$ is a strictly intrinsic length space (see Corollary 7.22 in \cite{villani2008optimal}, recalled in this report as Theorem \ref{geodesic_Wasserstein_space}) with respect to $W_p(X)$. That is the case in this paper, we assume in this section that $X$ is geodesic locally compact space. Now $\mathscr{W}_p(X)$ is a Polish geodesic space, however it is not necessarily a locally compact space.

We use following notation convention:
\begin{itemize}
	\item \( \hat{\mu} \), random probability measure in
\end{itemize}
