%! TEX root = ../barycenter.tex
\chapter{Papers Review}
% In this chapter, I shall review in order papers: \cite{ohta2012barycenters}, \cite{le2017existence}.
\section{Barycenters in Alexandrov spaces of curvature bounded below}
\begin{prop}[Example 3.1 a]
	Let \( X \) be the infinite dimensional ellipsoid of axes of lengths \( c _ { n } = ( n + 1 ) / 2 n \) with \( n \in \mathbb { N } , \) namely 
	\[
	X = \left\{ \left( x _ { 1 } , x _ { 2 } , \ldots \right) \in \mathbb { R } ^ { \infty } \mid \sum _ { n \in \mathbb { N } } \frac { x _ { n } ^ { 2 } } { c _ { n } ^ { 2 } } = 1 \right\} 
\]

	Then \( X \) is complete, but \( \mu = \left( \delta _ { ( 1,0,0 , \ldots ) } + \delta _ { ( - 1,0,0 , \ldots ) } \right) / 2 \) has no barycenter in \( X \).
\end{prop}

\begin{proof}
	As $ 1 / 2 \leq c_{n} \leq 1$, we know $X$ is closed subspace of Hilbert space $\ell^2$. Hence we can calculate distance through inner product. Pick $ x \in X$ and set $ e=(1,0,0\ldots)$, we have:
	\begin{align*}
		W_2(\delta_x, \mu)^2 & = \int_{X} d(x, \cdot)^2 \diff \mu = \frac{d(x,e)^2+d(x,-e)^2 }{2} \\
		                     & =\frac{\Vert x - e \Vert^2 + \Vert x + e \Vert^2}{2}               \\
		                     & =\Vert x \Vert^2 + 1
	\end{align*}

	Hence a barycenter of $\mu$ in $X$ should minimize its length. For a vector $x$ in $X$ to attain minimum length, if restricting to the first $n$ coordinate components, $x$ cannot have nonzero components except the last one (minimum length axes), i.e., $x_i =0$ for $i<n$. Otherwise we can keep other not considered coordinate components, $x_j$ for $j>n$,  unchanged but vanish first $n-1$ coordinate components to get a strictly shorter vector in $X$. This indicates that no such barycenter $x$ could exist in $X$.
\end{proof}

\begin{rmk}
	\begin{enumerate}
		% \item Hilbert space is not locally compact, as unit closed ball in infinite dimensional normed space is not sequential compact. Recall that the last claim can be proved using \href{https://en.wikipedia.org/wiki/Riesz%27s_lemma}{Wikipedia: Riesz's lemma}.
		% \item  $X$ is the unit sphere of $\ell^2$ with a different inner product: \[\langle x, y \rangle:=\sum_{i \in \mathbb{N}^*}\frac{x_i}{c_i}\cdot \frac{y_i}{c_i}.\]
		% We see that $X$ is homeomorphic to the unit sphere in $\ell^2$. From previous point, $X$ is then not compact. However, this conclusion is almost trivial as we don't have barycenter for $\mu$ in $X$.
		\item $X$ is in fact not locally compact as a subspace of $\ell^2$. Consider the sequence with only one nonzero coordinate component, it has no converging subsequence.
		      % as proved in \cite{bessaga1975selected} Chapter IV §2 Theorem 2.1, that unit sphere in $\ell^2$ is homeomorphic to $\ell^2$.
		\item $X$ with metric inherited form $\ell^2$ is apparently not a length space. To explore more of this example, we should start to consider induced length metric $\hat{d}$ on $X$ by inherited metric from $\ell^2$.
		\item $X$ is not locally compact with respect to the induced length metric topology. By compatibility of induced length structure with the topology of base space $X$ (see Exercise 2.1.5 in \cite{burago2001course}), open set in original topology is again open in the induced length metric topology. Cover an otherwise compact set of $X$ in new topology by open sets in original topology, we get a contradiction.
		      % \item Unit sphere in $\ell^2$ is geodesic with respect to induced length metric. By Theorem 2.4.16 in \cite{burago2001course}, we only need to show that midpoint in $X$ always exists, i.e., $\forall x,y \in X$, $\exists z \in X$ such that $d(x,z)=d(y,z)=\frac{1}{2}d(x,y)$. For such two points $x$ and $y$, $z$ can be selected from the intersection of $X$ and the hypersurface passing the origin in $\ell^2$ that evenly separates $x$ and $y$.

		      % \item  $X$ is geodesic with respect to induced length metric $\hat{d}$.
		      % Induced length structure in $X$.
		      %Recall that shortest paths are closed under point-wise convergence (Proposition 2.5.17 in \cite{burago2001course}, proved by lower semi-continuity of length function).
		      % We show that distance in $X$ is approximate by curves with finite coordinate components. To begin with, any two points in $X$ with finite coordinate components can be connected by a shortest path. Then pass a point to have infinite coordinate components and later do the same for another point.
		      % We use the fact that points with finite coordinate components are dense in $\ell^2$ and so are they in $X$.
	\end{enumerate}
\end{rmk}


\begin{rmk}[Example 2.1 c]
	% This example is in contradiction with \href{https://en.wikipedia.org/wiki/Hadamard_space}{Wikipedia: Hadamard space}. The author claims Hilbert space is of non-negative curvature, while Wikipedia says ``a normed space is an Hadamard space if and only if it is a Hilbert space".
	Hilbert space satisfies equality in triangle comparison with Euclidean space, hence can be regarded ``flat".
\end{rmk}

Then we discuss the infinitesimal structure of an Alexandrov space \( ( X , d )\).

\begin{defn}[space of directions]
	Fix \( z \in X \) and let \( \hat { \Sigma } _ { z } \) be the set of all unit speed geodesics emanating from \( z \). For \( \gamma , \eta \in \hat { \Sigma } _ { z } , \) by virtue of the curvature bound, the limit
	\begin{equation}
		\label{angle_def}
		\angle _ { z } ( \gamma , \eta ): = \arccos \left( \lim _ { s , t \downarrow 0 } \frac { s ^ { 2 } + t ^ { 2 } - d _ { X }\left( \gamma ( s ) , \eta ( t ) \right) ^ { 2 } } { 2 s t } \right)
	\end{equation}
	exists and is regarded as the angle (pseudo-)distance of \( \hat { \Sigma } _ { z } . \) We define the space of directions \( \left( \Sigma _ { z } , \angle _ { z } \right) \) at \( z \) as the completion of \( \Sigma _ { z } / \sim \) with respect to \( \angle _ { z } , \) where \( \gamma \sim \eta \) if \( \angle _ { z } ( \gamma , \eta ) = 0 . \)
\end{defn}

\begin{defn}[tangent cone]
	The tangent cone \( \left( C _ { z } , d _ { C _ { z } } \right) \) is defined as the Euclidean cone over \( \left( \Sigma _ { z } , \angle _ { z } \right) , \)
	that is to say,
	\begin{align*}
		C _ { z }:                                           & = \Sigma _ { z } \times [ 0 , \infty ) / \Sigma _ { z } \times \{ 0 \}          \\
		d _ { C _ { z } } ( ( \gamma , s ) , ( \eta , t ) ): & = \sqrt { s ^ { 2 } + t ^ { 2 } - 2 s t \cos \angle _ { z } ( \gamma , \eta ) }
	\end{align*}
\end{defn}

\begin{thm}[Computation of distance in tangent cone]
	Let $\gamma$ and  $\eta$ be two geodesics from unit interval into $X$ starting at $z$. Then $\gamma^\prime(0)$ and $\eta^\prime(0)$ are naturally elements in $C_z$. We can calculate their distance as:
	\[  d_{C_z}(\gamma^\prime(0),\eta^\prime(0))=\lim _ { t \downarrow 0 } \frac  {d ( \eta ( t ) , \gamma ( t ) )  }{ t }\]
\end{thm}

\begin{proof}
	This is a calculation of angle between $\gamma^\prime(0)$ and $\eta^\prime(0)$ with a specific selected converging process. To apply angle definition in \cref{angle_def}, normalize $\gamma$ and $\eta$ as $\gamma\left(\vert\gamma^\prime(0)\vert \cdot\right)$ and $\eta\left(\vert\eta^\prime(0)\vert \cdot\right)$
	\[
		\cos\angle _ { z } ( \gamma , \eta ): = \left( \lim _ { s , t \downarrow 0 } \frac { s ^ { 2 } + t ^ { 2 } - d _ { X }\left( \gamma (\vert \gamma^\prime(0)\vert s ) , \eta (\vert \eta^\prime(0)\vert  t ) \right) ^ { 2 } } { 2 s t } \right)
	\]
	To get desired formula, put $ x := \vert  \gamma^\prime(0)\vert s =\vert \eta^\prime(0)\vert t $ and pass $x \downarrow 0$.
\end{proof}

\begin{rmk}[Lemma 4.5, Lang and Schroeder's inequality]
	When passing from finite support measures to general ones, the author should use Wasserstein metric $W_p$ convergence but not only weak convergence. Otherwise, he cannot justify convergence of integrals of distance function $d_{C_z}$. Here we use the fact that finite supported measures is a dense subspace of Wasserstein space $\mathcal{W}_2(X)$ if $X$ is a Polish space (see \cite{villani2008optimal} Theorem 6.18), recalled in this report as \cref{topology_Wasserstein}.
\end{rmk}

