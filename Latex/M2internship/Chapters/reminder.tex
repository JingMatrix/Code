%! TEX root = ../barycenter.tex
\section{Convex analysis on Riemannian manifold}
\subsection{Polar factorization of maps on manifolds}

\begin{lem} [Lipschitz cost]
	\label{lem:Lipschitz_cost}
	Let \( ( M , d ) \) be a metric space whose diameter \( | M | : = \sup \{ d ( x , z ) \mid x , z \in M \} \) is finite.
	For each \( y \in M \), the function \( \psi ( x ) = d ^ { 2 } ( x , y ) / 2 \) is Lipschitz continuous:
	\begin{equation}
		\label{equa:Lipschitz}
		| \psi ( x ) - \psi ( z ) | \leq | M | d ( x , z ).
	\end{equation}
\end{lem}

\begin{proof}
	The triangle inequality shows that \( \phi ( x ) : = d ( x , y ) \) has Lipschitz constant one:
	\[ \phi ( x ) - \phi ( z ) = d ( x , y ) - d ( z , y ) \leq d ( x , z ) \]
	for all \( x , z \in M\).
	Also, \( \phi ( x ) = d ( x , y ) \leq | M | < \infty \) is bounded.
	The desired estimate \cref{equa:Lipschitz} then follows easily for \( \psi ( x ) = \phi ^ { 2 } ( x ) / 2 \) :
	\begin{align*}
		2 | \psi ( x ) - \psi ( z ) | & = | \phi ( x ) ( \phi ( x ) - \phi ( z ) ) + \phi ( z ) ( \phi ( x ) - \phi ( z ) ) | \\
		                              & \leq | M | d ( x , z ) + | M | d ( x , z )
	\end{align*}
\end{proof}

\begin{lem} [Infimal convolutions are Lipschitz]
	\label{lem:infimal_convolution_Lipschitz}
	Fix a metric space \( ( M , d ) \) having finite diameter.
	Any \( \psi : M \rightarrow \mathbf { R } \cup \{ \pm \infty \} \) given by an infimal convolution \( \psi = \psi ^ { c c } \) with \( c ( x , y ) = d ^ { 2 } ( x , y ) / 2 \) is either identically infinite \( \psi = \pm \infty \) or Lipschitz continuous throughout \( M\).
	Indeed, it satisfies \cref{equa:Lipschitz}.
\end{lem}

\begin{proof}
	More generally, suppose \( \psi = \phi ^ { c } \) for some \( \phi : M \rightarrow \mathbf { R } \cup \{ \pm \infty \} \) meaning
	\begin{equation}
		\label{equa:infimal_convolution}
		\psi ( x ) = \inf _ { y \in M } c ( x , y ) - \phi ( y )
	\end{equation}
	Observe \( 0 \leq c ( x , y ) \leq | M | ^ { 2 } / 2 \) is bounded.
	Either \( \phi \) is unbounded above, in which case \( ( 11 ) \) yields \( \psi = - \infty \) and the lemma holds trivially, or else \( \psi \) is bounded below.
	Fix \( z \in M \), and note \( \psi ( z ) = + \infty \) in \cref{equa:infimal_convolution} occurs only if \( \phi : = - \infty \) everywhere, in which case \( \psi = + \infty \) again holds trivially.
	Thus we may assume that \( \psi \) is finite everywhere.
	Given any \( \epsilon > 0 \), there exists \( y \in M \)
	such that \( \psi ( z ) + \epsilon \geq c ( z , y ) - \phi ( y ) \), while \( \psi ( x ) \leq c ( x , y ) - \phi ( y ) \) holds because of \cref{equa:infimal_convolution}.
	Subtracting these two inequalities yields
	\begin{align*}
		\psi ( x ) - \psi ( z ) & \leq c ( x , y ) - c ( z , y ) + \epsilon \\
		                        & \leq | M | d ( x , z ) + \epsilon
	\end{align*}
	by \Cref{lem:Lipschitz_cost}. Since the last inequality holds for all \( \epsilon > 0 \), the Lipschitz
	estimate \cref{equa:Lipschitz} has been proved.
\end{proof}

One reason to study square distance on manifold could be that under Taylor expansion, it is directlt related to inner product of Riemannian metric.
An example is to prove super-differentiability of geodesic distance squared in \cite{mccann2001polar}.

\begin{prop}
	[Superdifferentiability of geodesic distance squared]
	Let \( ( M , g ) \) be a \( C ^ { 3 } \)-smooth Riemannian manifold, possibly with boundary.
	Suppose \( \sigma : [ 0,1 ] \rightarrow M \) has minimal length among piecewise \( C ^ { 1 } \) curves joining \( y = \sigma ( 0 ) \) to \( x = \sigma ( 1 ) \notin \partial M \), parameterized with constant speed.
	Then \( \psi ( \cdot ) = d ^ { 2 } ( \cdot , y ) / 2 \) has super-gradient \( \dot { \sigma } ( 1 ) \in \widebar { \partial } \psi _ { x } \) at \( x \).
\end{prop}

\begin{proof}
	Since \( x \) lies in the interior of \( M \), there is some \( \epsilon > 0 \) and neighbourhood \( X \subset M \) of \( x \) such that: at each \( z \in X \), the exponential map exp \( _ { z } \) maps
	the ball \( \mathbf { B } ( \mathbf { 0 } , \epsilon ) \subset T M _ { z } \) diffeomorphically onto some open set \( U _ { z } \supset X \), as in Milnor.
	The proposition will first be established when
	\( y = \sigma ( 0 ) \in X \), in which case \( \psi \) is actually differentiable at \( \exp _ { y } \dot { \sigma } ( 0 ) = x \)
	We compute its derivative by linearizing exp \( _ { x } \mathbf { v } \in X \) around the origin and $\exp _ { y }$ around $ \dot { \sigma } ( 0 )$:
	\begin{align*}
		\psi \left( \exp _ { x } \mathbf { v } \right) & = d ^ { 2 } \left( y , \exp _ { y } \left( \exp _ { y } ^ { - 1 } \exp _ { x } \mathbf { v } \right) \right) / 2 \\ & = \left| \exp _ { y } ^ { - 1 } \left( \exp _ { x } \mathbf { v } \right) \right| _ { y } ^ { 2 } / 2 \\ & = \left| \dot { \sigma } ( 0 ) + D \left( \exp _ { y } ^ { - 1 } \right) _ { x } D \left( \exp _ { x } \right) _ { 0 } \mathbf { v } + o \left( | \mathbf { v } | _ { x } \right) \right| _ { y } ^ { 2 } / 2 \\ & = | \dot { \sigma } ( 0 ) | _ { y } ^ { 2 } / 2 + g \left\langle \dot { \sigma } ( 0 ) , \left( D \exp _ { y } \right) _ { \dot { \sigma } ( 0 ) } ^ { - 1 } I \mathbf { v } \right\rangle _ { y } + o \left( | \mathbf { v } | _ { x } \right) \\ & = d ^ { 2 } ( x , y ) / 2 + g \langle \dot { \sigma } ( 1 ) , \mathbf { v } \rangle _ { x } + o \left( | \mathbf { v } | _ { x } \right)
	\end{align*}
	so that \( \nabla \psi ( x ) = \dot { \sigma } ( 1 )\).

	Here the last equation follows from \( \dot { \sigma } ( 1 ) = \)
	\( D \left( \exp _ { y } \right) _ { \dot { \sigma } ( 0 ) } \dot { \sigma } ( 0 ) \) and Gauss' lemma.
\end{proof}

% \section{Multi-marginal optimal transport on Riemannian manifold}

\subsection{A Riemannian interpolation inequality à la Borell, Brascamp and Lieb}

We copy following results (tex code) from \cite{cordero2001riemannian}.

\begin{proof}[Proof of \Cref{lem:c-super-gradients_imply_super-gradients}]
	Fix \( ( x , y ) \in \mathcal{X} \times Y \) such that \( y \in \partial ^ { c } \phi ( x ) \).
	For every \( z \in \mathcal{X} \), \cref{equa:c-super-differential} implies
	\[ \phi ( z ) \leq \phi ( x ) + c ( z , y ) - c ( x , y ) .\]
	Any shortest vector \( v \in T _ { x } M \) satisfying \( \exp _ { x } ( - v ) = y \) has length \( | v | = \)
	\( d ( x , y )  \).
	Now  \cite[Proposition 6]{mccann2001polar} generalizes
	\( y = \exp _ { x } \left[ - \nabla d _ { y } ^ { 2 } ( x ) / 2 \right] \)
	by asserting \( v \in \partial \left( d _ { y } ^ { 2 } / 2 \right) ( x ) \).
	Thus for \( z = \exp _ { x } u \),
	\[ c \left( \exp _ { x } u , y \right) - c ( x , y ) \leq \langle v , u \rangle + o ( | u | ) \]
	as \( u \rightarrow 0 \).
	Combining the two inequalities yields the desired result \cref{equa:super-differential}.
\end{proof}
As we have seen in \Cref{lem:barycenter_out_of_cut_locus},
it is shown by \cite[Lemma 3.1]{kim2015multi} that this proposition implies
that barycenters on $M$ are not in the cut-locus considered.

First recall that a geodesic ball \( B _ { r } ( x ) \) of radius \( r \) around \( x \in M \) is said to be embedded
if the exponential map \( \exp _ { x } : \tilde { B } _ { r } ^ { x } ( 0 ) \rightarrow B _ { r } ( x ) \) defines
a diffeomorphism from the open ball \( \tilde { B } _ { r } ^ { x } ( 0 ) \subset T _ { x } M \) onto \( B _ { r } ( x ) \subset M \).
A geodesic ball \( B _ { r } ( x ) \) around \( x \) is a convex embedded ball
if it is embedded and geodesically convex---meaning every pair of points \( y , z \in B _ { r } ( x ) \)
are joined by a unique geodesic of length less than \( 2 r \),
and this geodesic is contained in \( B _ { r } ( x )\).
Small enough balls are always convex embedded balls.

\begin{defn}[Semi-concavity]
	\label{defn:semi-concavity}
	Fix \( \Omega \subset M \) open.
	A function \( \phi : \Omega \rightarrow \mathbb { R } \) is semi-concave at \( x _ { 0 } \in \Omega \)
	if there exists a convex embedded ball \( B _ { r } \left( x _ { 0 } \right) \) and
	a smooth function \( V : B _ { r } \left( x _ { 0 } \right) \rightarrow \mathbb { R } \)
	such that \( \phi + V \) is geodesically concave throughout \( B _ { r } \left( x _ { 0 } \right)\).
	The function \( \phi \) is semi-concave on \( \Omega \) if
	it is semi-concave at each point of \( \Omega \).
\end{defn}

It is remarkable that the converse also holds true: if \( \psi \) is semi-concave around \( x \) then \cref{defn:hessian} follows from \cref{equa:hessian_expan}.


\begin{thm}[Aleksandrov-Bangert]
	Let \( \phi : \Omega \rightarrow M \) be semi-concave function on an open set \( \Omega \subset M . \) Then \( \phi \) admits a Hessian almost everywhere on \( \Omega \).
\end{thm}
% A proof from \cite{bangert1979analytiche} is in German.
% This theorem is also proved in \cite[Theorem 14.1]{villani2008optimal}.
% It asserts that notion of local semi-convexity
% with quadratic modulus is invariant by $C^2$ diffeomorphism, so it suffices to prove for $\mathbb{R}^n$.

\begin{lem}[Local characterization of semi-concavity]
	\label{lem:sufficient_condition_semi-concavity}
	Let \( \phi : \Omega \rightarrow \mathbb { R } \) be a continuous function and fix \( x _ { 0 } \in \Omega \).
	Assume that there exists a neighborhood \( U \) of \( x _ { 0 } \) and a positive constant \( C \) such that for every \( x \in U \) and
	\( u \in T _ { x } M \) one has,
	\[ \limsup _ { r \rightarrow 0 } \frac { \phi \left( \exp _ { x } r u \right) + \phi \left( \exp _ { x } - r u \right) - 2 \phi ( x ) } { r ^ { 2 } } \leq C \]
	Then \( \phi \) is semi-concave around \( x _ { 0 } \).
\end{lem}
Following lemma (\cite[Lemma 3.12]{cordero2001riemannian}) implies
the semi-convexity of squared distance function $d_y^2$ at point $x$.
% Because it is shown in \cite[Lemma 3.11]{cordero2001riemannian} that
% \cref{equa:hessian_bound_distance_squared} is a sufficient condition to be semi-convex at a point.
Moreover, one can actually choose the local smooth function in \cref{defn:semi-concavity} as squared distance function
if we dig the proof of \Cref{lem:sufficient_condition_semi-concavity} in \cite[Lemma 3.11]{cordero2001riemannian}.


Combining with the definition of $c$-concave,
\Cref{lem:hessian_bound_distance_squared} implies following proposition,
where we use compactness to ensure a global lower bound for sectional curvatures.

\begin{prop}[\( c \)-concave functions are semi-concave]
	Fix \(\mathcal{X} \subset \subset M \) open and \( Y \subset M \) compact.
	A \( c \)-concave function \( \phi \in \mathcal{I} ^ { c } ( \widebar { \mathcal{X} } , Y ) \) is semi-concave on \( X \) (and hence admits a Hessian \cref{defn:hessian} almost everywhere in \( X \)).
\end{prop}

\begin{proof}[Proof for the first point in \Cref{prop:differentiate_optimal_transport}]
	Suppose \( \phi \) admits a Hessian \cref{defn:hessian} at \( x \in X \).
	Then \( \phi \) is differentiable at \( x \) and
	\Cref{example:minimizer_differentiable} shows that \( \partial ^ { c } \phi ( x ) = \{ F ( x ) \} = \{ y \} \).
	Thus for every \( z \in \mathcal { X } \), \cref{equa:c-super-differential} yields
	\begin{equation}
		\label{equa:c-concave_distance_compare}
		\phi ( z ) \leq \phi ( x ) + d _ { y } ^ { 2 } ( z ) / 2 - d _ { y } ^ { 2 } ( x ) / 2
	\end{equation}
	Taking \( z = \exp _ { x } ( \pm u ) \) and \( \psi : = d _ { y } ^ { 2 } / 2 \) gives
	\[ \frac { \phi \left( \exp _ { x } u \right) + \phi \left( \exp _ { x } - u \right) - 2 \phi ( x ) } { | u | ^ { 2 } } \leq \frac { \psi \left( \exp _ { x } u \right) + \psi \left( \exp _ { x } - u \right) - 2 \psi ( x ) } { | u | ^ { 2 } } \]
	As \( | u | \rightarrow 0 \) the left hand side tends to \( \left\langle \operatorname { Hess } _ { x } \phi ( u ) , u \right\rangle \) by hypothesis, so the right hand side is bounded below.
	Then \Cref{prop:distance_cut_locus} ensures that
	\( x \notin \operatorname { cut } ( y ) \), or equivalently \( y \notin c u t ( x ) \).
	From \cref{equa:c-concave_distance_compare} we also observe that the function
	\begin{equation}
		\label{proof:h_definition}
		h ( z ) : = d _ { y } ^ { 2 } ( z ) / 2 - \phi ( z )
	\end{equation}
	has a minimum at \( z = x \). The Taylor expansion \cref{equa:hessian_expan} then implies the
	existence and non-negativity of its Hessian: Hess\(_{ x } h = H - \) Hess\(_ { x } \phi \geq 0\).
\end{proof}

\begin{proof}[Proof for the seconde point in \Cref{prop:differentiate_optimal_transport}]
	Fix a unit tangent vector \( u \in T _ { x } M \) and set \( x _ { s } = \exp _ { x } ( s u ) \).
	For \( y _ { s } \in \partial ^ { c } \phi \left( x _ { s } \right) \) we want
	to establish the estimate \( \left| v _ { s } - \diff F _ { x } ( s u ) \right| = o ( s ) \),
	where \( v _ { s } \in T _ { y } M \) is the shortest vector such that
	\( y _ { s } = \exp _ { y } v _ { s } \),
	and the error term is independent of \( u \).

	Introduce \( u _ { s } \in T _ { x _ { s } } M \) such that \( y _ { s } = \exp _ { x _ { s } } u _ { s } \) with \( \left| u _ { s } \right| = d \left( x _ { s } , y _ { s } \right) \)
	and let \( w _ { s } : = u _ { s } + \nabla d _ { y } ^ { 2 } \left( x _ { s } \right) / 2 \).
	Then
	\begin{equation}
		\label{proof:y_s_definition}
		y _ { s } = \exp _ { x _ { s } }
		\left( - \nabla d _ { y } ^ { 2 } \left( x _ { s } \right) / 2 + w _ { s } \right).
	\end{equation}

	Applying  \Cref{lem:c-super-gradients_imply_super-gradients} to \( y _ { s } \in \partial ^ { c } \phi \left( x _ { s } \right) \) yields \( - u _ { s } \in \partial \phi \left( x _ { s } \right) \)
	and hence
	\( w _ { s } \in \partial h \left( x _ { s } \right) \),
	where \( h \) is as in \Cref{proof:h_definition}.
	Recall that \( h \) has a Hessian at \( x \) satisfying \( \diff F _ { x } = Y \operatorname { Hess } _ { x } h \),
	which by definition \cref{defn:hessian} means
	\begin{equation}
		\label{proof:hessian_definition}
		\Pi _ { x , s u } w _ { s } = s \operatorname{ Hess } _ { x } h ( u ) + o ( s ).
	\end{equation}

	Thus the curve \( w _ { s } \) through \( ( x , 0 ) \in T M \) is differentiable at \( s = 0 \)
	and we can
	identify the vertical component of its tangent vector as
	\( \dot { w } _ { s = 0 } = \operatorname { Hess } _ { x } h ( u ) . \)
	Recall that the function \( z \rightarrow \exp _ { z } \left( - \nabla d _ { y } ^ { 2 } ( z ) / 2 \right) = y \)
	is constant outside \( \operatorname{cut}( y ) \).
	Differentiating \Cref{proof:y_s_definition} using the chain rule then yields
	\begin{align*}
		\dot { y } _ { s = 0 } & = \diff \left( \exp _ { x _ { 0 } } \right) _ { - \nabla d _ { y } ^ { 2 } \left( x _ { 0 } \right) / 2 } \left( \dot { w } _ { 0 } \right) \\ & = Y \dot { w } _ { 0 } \\ & = \diff F _ { x } ( u )
	\end{align*}

	The appearance of \( Y = \diff ( \exp ) _ { - \nabla \phi ( x ) } \) in the second identity follows from
	\( \nabla h \left( x _ { 0 } \right) = 0 \) in \Cref{proof:h_definition}.
	Finally, since \( y _ { 0 } = y \) we have
	\begin{align*}
		y _ { s } & = \exp _ { y } \left( s \dot { y } _ { 0 } + o ( s ) \right)                             \\
		          & = \exp _ { y } \left( s \diff F _ { x } ( u ) + o ( s ) \right) = \exp _ { y } v _ { s }
	\end{align*}
	The size of the error term \( o ( s ) \) here does not depend on the unit vector \( u \)
	since it did not depend on \( u \) in \Cref{proof:hessian_definition}.
	Comparison with \cref{equa:differentiate_optimal_transport} ends the proof
	of the proposition.
\end{proof}

\begin{lem}[Equivalence of algebraic and geometric Jacobians]
	Let \( x \in \Omega \).
	Then \( \partial ^ { c } \phi \left( B _ { r } ( x ) \right) \) shrinks nicely to \( y : = F ( x ) \) when \( r \rightarrow 0 \) and
	\[ \lim _ { r \rightarrow 0 } \frac { \operatorname { vol } \left[ \partial ^ { c } \phi \left( B _ { r } ( x ) \right) \right] } { \operatorname { vol } \left[ B _ { r } ( x ) \right] } = \operatorname { det } d F _ { x } .\]
\end{lem}

Here shrinks nicely means there exists \( R ( r ) \rightarrow 0 \) as \( r \rightarrow 0 \) such that
\( \partial ^ { c } \phi \left( B _ { r } ( x ) \right) \subset B _ { R ( r ) } ( y ) \)
with  \( \operatorname{Vol} \left[\partial^c \phi (B_r (x)) \right] \geq
\alpha \operatorname{Vol} \left[ B _ { R ( r ) } ( y ) \right]\)
for some $\alpha >0$ ; see \cite[140]{Rudin1987real}.
