%! TEX root = ../barycenter.tex
\chapter{Barycenter in Wasserstein space}

In this chapter,
we focus on spaces of measures with Wasserstein metric, namely Wasserstein space.
Study on these spaces has application in statistics,
as summarized in \cite{le2017existence}.
It is also possible to have visualization if we plot uniform measure as their support;
jump to \cref{example:barycenter_sphere} for a quick view.

It is possible to define Wasserstein metric for measures on a Polish space,
i.e., a complete and separable metric space.
Wasserstein spaces turn out to be Polish as well.
Moreover, if the base space $(E,d)$ of measures is geodesic,
then the Wasserstein space $(\mathcal{W}_p(E), W_p)$ over it is also geodesic.
In previous chapter, we saw existence of barycenter in a proper (metric) space.
However, for $\mathcal{W}_p(E)$ to be locally compact we have necessarily that
$E$ must be compact.
Having seen some counter-examples,
we should be cautious for the existence of barycenter in a Wasserstein space.
A sufficient condition is to assume that $(E,d)$ is proper,
and we shall cover the proof of this claim.

The definition of Wasserstein metric starts with optimal transportation theory.
We introduce it with slight abstraction for later application in our proofs.

\section{Optimal transportation}

\begin{defn}[Coupling]
	Let \(  \mu \) and \(  \nu  \) be two probability measures on $E$.
	Coupling \( \mu \) and \( \nu \) means constructing two random variables \( X \) and \( Y \) on some probability space \( ( \Omega , P )\),
	such that law \( ( X ) = \mu \), law \( ( Y ) = \nu .
	\) The couple \( ( X , Y ) \) is called a coupling of \( ( \mu , \nu ) .
	\) By abuse of language,
	the law of \( ( X , Y ) \) is also called a coupling of \( ( \mu , \nu ) \).
\end{defn}

If \( \mu \) and \( \nu \) are the only laws in the problem,
then without loss of generality one may choose \( \Omega = E \times E .
\) In a more measure-theoretical formulation,
coupling \( \mu \) and \( \nu \) means constructing a measure \( \pi \) on \( E \times E \) such that \( \pi \) admits \( \mu \) and \( \nu \) as marginals on \( E \) and \( E \) respectively.

We then define the optimal coupling or optimal transport.
Here one introduces a cost function \( c ( x , y ) \) on \( E \times E \),
which can be interpreted as the work needed to move one unit of mass from location \( x \) to location \( y \).
Then one considers the Monge-Kantorovich minimization problem
\[
	\inf  \mathbb{ E } c ( X , Y )
\]
where the pair \( ( X , Y ) \) runs over all possible couplings of \( ( \mu , \nu ) \);
or equivalently, in terms of measures,
\[ \inf \int _ { E \times E } c ( x , y ) \diff \pi ( x , y ) \]
where the infimum runs over all joint probability measures \( \pi \) on \( E \times E \) with marginals \( \mu \) and \( \nu \).
Such joint measures are called transference plans (or transport plans, or transportation plans);
those achieving the infimum are called optimal transference plans.

The first good thing about optimal couplings is that they exist (\cite[Theorem 4.1]{villani2008optimal}):
\begin{thm}[Existence of an optimal coupling]
	\label{thm:existence_optimal_coupling}
	Let \(  \mu \) and \(  \nu  \) be two probability measures on Polish space $E$;
	let \( c: E \times E \rightarrow \mathbb{ R }  \) be
	a non-negative continuous cost function.
	Then there is a coupling of \( ( \mu , \nu ) \) which minimizes the total cost \( \mathbb{ E } c ( X , Y ) \) among all possible couplings \( ( X , Y ) \).
\end{thm}

The proof relies on basic variational arguments involving the topology of weakly convergence,
which is imposed by bounded continuous test functions.
There are two key properties to check:
\begin{enumerate}
	\item lower semi-continuity with respect to weakly convergence,
	\item compactness in weak topology.
\end{enumerate}
These issues are taken care of respectively in two lemmas below.

\begin{lem}[Lower semi-continuity of the cost functional]
	\label{lem:lower_semi-continuity_of_the_cost_functional}
	Let \( E \) be a Borel measurable space
	and \( f : E  \rightarrow \mathbb{ R }\) be a non-negative continuous function.
	Let \( \left( \pi _ { k } \right) _ { k \in \mathbb{ N } } \) be a sequence of
	probability measures on \( E \),
	converging weakly to a probability measure \( \pi \),
	then we have
	\[
		\int _ { E } f \diff \pi \leq \liminf _ { k \rightarrow \infty } \int _ { E } f \diff \pi _ { k }.
	\]
	In other words, \( F: \pi \mapsto \int f \diff \pi \) is lower semi-continuous
	with respect to weakly convergence of probability measures on $E$.
\end{lem}

\begin{proof}
	We write \( f \)  as the point-wise limit of a non-decreasing family \( \left( f _ { \ell } \right) _ { \ell \in \mathbb{ N } } \) of  bounded continuous real-valued functions.
	By monotone convergence,
	\[
		\int f \diff \pi = \lim _ { \ell \rightarrow \infty } \int f _ { \ell } \diff \pi = \lim _ { \ell \rightarrow \infty } \lim _ { k \rightarrow \infty } \int f _ { \ell } \diff \pi _ { k } \leq \liminf _ { k \rightarrow \infty } \int f \diff \pi _ { k },
	\]
	where we use $c_\ell \leq c$ and pass to the infimum limit in the last inequality.
\end{proof}

\begin{lem}[Tightness of transference plans]
	\label{lem:tightness_transference_plan}
	Let $E$ be a Polish space, denote by $P(E)$ the space of probability measures on $E$.
	Let \( \mathcal{ P }, \mathcal{ Q } \subset P ( E ) \) be two tight subsets of \( P ( E ) \).
	Then the set \( \Pi ( \mathcal{ P } , \mathcal{ Q } ) \) of all transference plans whose marginals lie in \( \mathcal{ P } \) and \( \mathcal{ Q } \) respectively, is itself tight in \( P ( E \times E ) \).
\end{lem}

\begin{proof}
	Let \( \mu \in \mathcal{ P } , \nu \in \mathcal{ Q } , \) and \( \pi \in \Pi ( \mu , \nu ) . \) By assumption, for any \( \varepsilon > 0 \) there is a compact set \( K _ { \varepsilon } \subset E , \) independent of the choice of \( \mu \) in \( \mathcal{ P } \), such that \( \mu \left[ E \backslash K _ { \varepsilon } \right] \leq \varepsilon ; \) and similarly there is a compact set \( L _ { \varepsilon } \subset E , \) independent of the choice of \( \nu \) in \( \mathcal{ Q } , \) such that \( \nu \left[ E \backslash L _ { \varepsilon } \right] \leq \varepsilon . \) Then for any coupling \( ( X , Y ) \) of \( ( \mu , \nu ) \)
	\[ \mathbb{ P } \left[ ( X , Y ) \notin K _ { \varepsilon } \times L _ { \varepsilon } \right] \leq \mathbb{ P } \left[ X \notin K _ { \varepsilon } \right] + \mathbb{ P } \left[ Y \notin L _ { \varepsilon } \right] \leq 2 \varepsilon. \]

	The desired result follows since this bound is independent of the coupling, and \( K _ { \varepsilon } \times L _ { \varepsilon } \) is compact in \( E \times E  \).
\end{proof}

\begin{proof}[Proof of existence of optimal plan]
	By standard passing to the limit argument for the definition of marginals,
	we see that \( \Pi ( \mu , \nu ) \) is closed with respect to weakly convergence,
	so it is in fact compact.
	Then let \( \left( \pi _ { k } \right) _ { k \in \mathbb{ N } } \) be a sequence of probability measures on \( E \times E \),
	such that \( \int c \diff \pi _ { k } \) converges to the infimum transport cost.
	Extracting a subsequence if necessary, we may assume that \( \pi _ { k } \) converges to some \( \pi \in \Pi ( \mu , \nu ) \).
	Then \cref{lem:lower_semi-continuity_of_the_cost_functional} gives
	\[ \int c \diff \pi \leq \liminf _ { k \rightarrow \infty } \int c \diff \pi _ { k }. \]

	Thus $\pi$ is minimizing.
\end{proof}

We shall talk about conditional transference plans in the next chapter,
for which we prove that the restriction of optimal plan is still optimal (\cite[Theorem 4.6]{villani2008optimal}).

\begin{thm}[Optimality is inherited by restriction]
	\label{thm:restriction_optimal_plan}
	Let \(  \mu  \) and \(  \nu \) be two probability measures on Polish space $E$;
	let \( c : E \times E \rightarrow \mathbb{ R } \) be a positive continuous cost function.
	Assume that the optimal transport cost from \( \mu \) to \( \nu \) is finite
	and let \( \pi \in \Pi ( \mu , \nu ) \) be an optimal transport plan.
	Let \( \widetilde { \pi } \) be a non-negative measure on \( E \times E \),
	such that \( \widetilde { \pi } \leq \pi \)
	and \( \widetilde { \pi } [ E \times E ] > 0\).
	Then the probability measure
	\[ \pi ^ { \prime } : = \frac { \widetilde { \pi } } { \widetilde { \pi } [ E \times E ] } \]
	is an optimal transference plan between its marginals \( \mu ^ { \prime } \) and \( \nu ^ { \prime } \).

	Moreover, if \( \pi \) is the unique optimal transference plan between \( \mu \)
	and \( \nu \),
	then also \( \pi ^ { \prime } \) is the unique optimal transference plan between \( \mu ^ { \prime } \)
	and \( \nu ^ { \prime } \)
\end{thm}

\begin{proof}
	Assume that \( \pi ^ { \prime } \) is not optimal;
	then there exists
	\( \pi ^ { \prime \prime } \) such that
	\begin{equation}
		\label{proof:restriction_project}
		\left( \operatorname { proj } _ { 1 } \right) _ { \# } \pi ^ { \prime \prime } = \left( \operatorname { proj } _ { 1 } \right) _ { \# } \pi ^ { \prime } = \mu ^ { \prime },
		\quad \left( \operatorname { proj } _ { 2 } \right) _ { \# } \pi ^ { \prime \prime } = \left( \operatorname { proj } _ { 2 } \right) _ { \# } \pi ^ { \prime } = \nu ^ { \prime }
	\end{equation}
	yet
	\begin{equation}
		\label{proof:restriction_not_optimal}
		\int c ( x , y ) \diff \pi ^ { \prime \prime } ( x , y ) < \int c ( x , y ) \diff \pi ^ { \prime } ( x , y )
	\end{equation}
	Then consider
	\[ \widehat { \pi } : = ( \pi - \widetilde { \pi } ) + \widetilde { Z } \pi ^ { \prime \prime } \]
	where \( \widetilde { Z } = \widetilde { \pi } [ E \times E ] > 0 . \) Clearly, \( \widehat { \pi } \) is a non-negative measure. On the
	other hand, it can be written as
	\[ \widehat { \pi } = \pi + \widetilde { Z } \left( \pi ^ { \prime \prime } - \pi ^ { \prime } \right) \]
	then \cref{proof:restriction_project} shows that \( \widehat { \pi } \) has the same marginals as \( \pi \),
	while \cref{proof:restriction_not_optimal} implies that it has a lower transport cost than \( \pi \).
	Here we use the fact that the total cost is finite.
	This contradicts the optimality of \( \pi \). The conclusion
	is that \( \pi ^ { \prime } \) is in fact optimal.
\end{proof}

\subsection{Wasserstein metric}

\begin{defn}[Wasserstein distances]
	\label{Wasserstein_distance}
	Let  \(( E , d ) \) be a Polish metric space, and let \( p \in [ 1 , \infty ) . \) For any two probability measures \( \mu , \nu \) on \( E , \) the Wasserstein distance of order \( p \) between \( \mu \) and \( \nu \) is defined by the formula
	\begin{align*}
		W _ { p } ( \mu , \nu ) & = \left( \inf _ { \pi \in \Pi ( \mu , \nu ) } \int _ { E } d ( x , y ) ^ { p } \diff \pi ( x , y ) \right) ^ { 1 / p }                                                              \\
		                        & = \inf \left\{ \left[ \mathbb{ E } d ( X , Y ) ^ { p } \right] ^ { \frac { 1 } { p } } , \quad \operatorname { law } ( X ) = \mu , \quad \operatorname { law } ( Y ) = \nu \right\}
	\end{align*}
\end{defn}

Abstract calculation is possible if $\Pi(\mu,\nu)$ is effectively a singleton.
\begin{example}[Wasserstein distance of Delta measures]
	\label{example:delta_measure_Wasserstein_distance}
	If $\mu=\delta_x$ for some $x \in E$,
	then $\Pi(\delta_x, \nu)$ consists of only one element since $\delta_x$ is always independent with $\nu$.
	Hence,
	\[
		W_p(\delta_x, \nu)^p = \int_{E} d(x, y)^p \diff \nu(y).
	\]
	Especially, we have $W_p(\delta_x, \delta_y) = d(x,y)$ for $x,y \in E$.
	This means that $x \mapsto \delta_x$ gives an isometric embedding from $E$ to $\mathcal{W}_p(E)$.
\end{example}

\begin{lem}
	\label{lem:lower_semicontinous_Wasserstein_distance}
	Wasserstein distance is lower semi-continuous with respect to weakly convergence of measures on $E$.
\end{lem}

\begin{proof}
	This is a direct consequence of lower semi-continuity of the cost functional, see \cref{lem:lower_semi-continuity_of_the_cost_functional}.
\end{proof}

\begin{defn}[Wasserstein space]
	\label{Wasserstein_space}
	With the same conventions as in \cref{Wasserstein_distance} , the Wasserstein space of order \( p \) is defined as
	\[
		\mathcal{W}_p(E): = \left\{ \mu \text{ Borel  probability measure on } E \,|
		\, \int _ { E } d \left( x _ { 0 } , x \right) ^ { p } \diff \mu (x) < + \infty \right\}
	\]
	where \( x _ { 0 } \in E \) is arbitrary. This space does not depend on the choice of the point \( x _ { 0 } \). Then \( W _ { p } \) defines a (finite) distance on \( \mathcal{W}_p(E) \).
\end{defn}

$W_p$ metric convergence is closely related to weakly convergence,
as one may guess from \cref{lem:lower_semicontinous_Wasserstein_distance}.
Following theorem is a more precise description of it (\cite[Theorem 6.9]{villani2008optimal}).
We use $\rightharpoonup$ to indicate weakly convergence of measures.

\begin{thm}[$W_p$  metric and weakly convergence]
	\label{thm:Wp_metricizes_weak_convergence}
	Let \( ( E , d ) \) be a Polish space, and \( p \in [ 1 , \infty ) \).
	Let \( \left( \mu _ { k } \right) _ { k \in \mathbb{ N } } \) be a
	sequence of probability measures in \( \mathcal{W}_p ( E ) \)
	converges weakly to \( \mu \in \mathcal{W}_p(E)	\);
	we denote it by $\mu_k \rightharpoonup \mu$.

	Then \( W_p \left( \mu _ { k }, \mu \right) \rightarrow 0 \)
	if and only if we have convergence of $p$ moment for
	some (and then any) \( x _ { 0 } \in E\):

	\[
		\int d ( x _ { 0 } , x ) ^ { p } \diff \mu _ { k } ( x ) \rightarrow \int d ( x _ { 0 } , x ) ^ { p } \diff \mu ( x ).
	\]

\end{thm}

Metric properties of $(\mathcal{W}_p, W_p)$ are also of interest.
In fact, we shall use following theorem \cite[Theorem 6.18]{villani2008optimal} throughout remained discussion.
It is also an extension of general finite support measure approximation theorem.
\begin{thm}[Topology of Wasserstein space]
	\label{thm:topology_Wasserstein}
	Let \( E \) be a complete separable metric space and \( p \in [ 1 , \infty ) \).
	Then the Wasserstein space \( \mathcal{W}_p(E) \),
	metricized by the Wasserstein distance \( W _ { p }\),
	is also a complete separable metric space.
	In short: The Wasserstein space over a Polish space is itself a Polish space.
	Moreover, any probability measure can be approximated by a sequence of probability measures with finite support.
\end{thm}
