%! TEX root = ../barycenter.tex
\YYCleverefInput{/var/tmp/latex/barycenter.sed}
\section{Barycenter in manifolds}

\label{section:barycenter_manifold}
Recall that we define $c(x, y):=\frac{1}{2} d(x, y)^2$.
Now for $n$ points $x_j \in M, j=1,2,\ldots,n$, we re-define $f$ on $M^n$ with a coefficient $\frac{1}{2}$
to make later notations easier,
\[
	f(x_1, x_2, \ldots, x_n) := \min_{w} \sum_{j=1}^n \lambda_j c(w, x_j) =: \frac{1}{2}W_2(\sum_{j=1}^n \lambda_j \delta_{x_j}, M)^2.
\]
We remark that this re-definition affects nothing of what we have discussed before.

Pick an $i, 1 \leq i \leq n$ such that $\lambda_i \neq 0$, then we fix $x_j$ for $j \neq i$ and
define $f_i (x_i) : = f(x_1, \ldots, x_n)$.
Take an arbitrary open pre-compact neighbourhood $\mathcal{X} \subset \subset M$ of $x_i$;
let $Y$ the set of all barycenters of $\sum_{j=1}^n \lambda_j \delta_{x_j}$
when $x_i$ runs through in $\widebar{ \mathcal{X} }$.
Then $Y$ is compact by \Cref{rmk:barycenter_compact}.
% Consider a $c$-concave function $ g_i \in \mathcal{I}(Y, \widebar{ \mathcal{X} })$
% \[
% 				g_i(w) := - \frac{1}{\lambda_i} \sum_{j\neq i}^n \lambda_j c( x_j, w)  =
% 				\inf_{x_i \in \widebar{ \mathcal{X} }}c( x_i, w ) -
% 				\frac{1}{\lambda_i} \sum_{j=1}^n \lambda_j c( x_j, w),
% \]

We define $g_i(w): = - 1/ \lambda_i \sum_{j \neq i} \lambda_j c(w, x_j) $ on $M$,
then by definition $f_i /\lambda_i \in \mathcal{I}^c (\widebar{ \mathcal{X}}, Y)$
is the $c$-transform of
$g_i$ restricted to $Y$.
The $c$-transform inequality
$ c(w, x_i) \geq g_i(w) + f_i / \lambda_i (x_i)$
becomes equality when $z$ is a barycenter of $\sum_{i=1}^{n} \lambda_i \delta_{x_i}$,
\begin{equation}
	\label{equa:g_i_conjugate}
	c(z, x_i) = g_i(z) + \frac{1}{\lambda_i} f_i(x_i).
\end{equation}

\begin{rmk}[$g_i$ on $Y$ is the $c$-transform of $f_i / \lambda_i$]
	In general, we have by the definition of $c$-transform that
	$g_i(w) \leq (f_i / \lambda_i)^c(w) := \inf_{x_i \in \widebar{ \mathcal{X} } } c(x_i, w) - f_i / \lambda_i (x_i)$.
	If $ c(z, x_i) = g_i (z) + f_i / \lambda_i (x_i)$,
	then
	$g_i(z) = (f_i / \lambda_i)^c (z)$ since $(f_i / \lambda_i)^c(z) + f_i / \lambda_i(x_i) \leq c (z, x_i)$.
	This implies that $g_i = (f_i / \lambda_i)^c$ on $Y$.
	% If $g_i$ is differentiable then $f_i / \lambda_i \geq g_i$ is sub-differentiable
	% and thus differentiable with the same gradient as $g_i$.
	% So $(f_i / \lambda_i)^c$ is differentiable on $Y$.
	% By \Cref{example:minimizer_differentiable}, for each barycenter $z \in Y$,
	% there exists a unique $x_i = \exp_z (- \nabla g_i (z))$ such that $z$ is the barycenter
	% of $\sum_{j=1}^n \lambda_j \delta_{x_j}$.
	% However, it says nothing on the uniqueness for barycenter of $\sum_{j=1}^n \lambda_j \delta_{x_j}$.
	% Both $g_i$ and $(f_i / \lambda_i)^c$ are continous;
	% $g_i$ is differentiable alomst everywhere with respect to volume measure.
	% because in local normal coordinate $g_i$ is differentiable almost everywhere
	% with respect to Lesbegue measure.
\end{rmk}

Moreover, \Cref{lem:minimizer_differentiable} tells us two things,
\begin{itemize}
	\item $f_i$ is Lipschitz on $\widebar{ \mathcal{X} }$;
	\item If $ \nabla f_i $ exists, then $z: = \exp( - \frac{1}{\lambda_i}\nabla f_i)$ is the
	      unique barycenter of $\sum_{i=1}^n \lambda_i \delta_{x_i}$.
\end{itemize}
Recall that $B$ is a measurable selection of barycenter of
measure $\sum_{j=1}^{n} \lambda_j \delta_{x_j}$ on $M$.
% that is to say,
% \[
% B(\boldsymbol{x}) \in \arg \min_{w \in M} \sum_{i=1}^{n} \lambda_i d(w, x_i)^2.
% \]
We have the following \textbf{barycenter formula}
if $f_i$ is differentiable at $x_i$,
\begin{equation}
	\label{equa:formula_barycenter}
	B(x_1, x_2, \ldots, x_n) = \exp_{x_i} (- \frac{1}{\lambda_i} \nabla f_i(x_i)).
\end{equation}

The barycenter formula \cref{equa:formula_barycenter} expresses the fact that,
from point $x_i$,
we can reach the unique barycenter of $\sum_{j=1}^n \lambda_i \delta_{x_j}$ if we follow the direction $-\nabla f_i$
and advance $\| \nabla f_i / \lambda_i \| $.
In the case of Euclidean space,
\[
	\frac{1}{\lambda_i} \nabla f_i(x_i) = x_i - \sum_{j=1}^n \lambda_j x_j;                              \quad
	\nabla g_i(w)  = \frac{1}{\lambda_i} (w - \sum_{j=1}^n \lambda_j x_j) -(w-x_i).
\]

The following fine property of barycenter is stated in \cite[Lemma 3.1]{kim2015multi}.
We emphasize the fact that this lemma connects the notion of barycenter with geometry of our manifold $M$.
\begin{lem}
	\label{lem:barycenter_out_of_cut_locus}
	Fix \( \left( x _ { 1 } , \ldots , x _ { n } \right) \in M^n\).
	Let $\lambda_j >0, 1 \leq j \leq n$ be $n$ positive numbers such that
	$\sum_{j=1}^n \lambda_j =1$.
	Then any barycenter $z$ of $\sum_{j=1}^n \lambda_j \delta_{x_j}$
	is not in the cut locus of \( x _ { j } \) for any \( j . \)
\end{lem}

\begin{proof}
	Choose a barycenter \( z \) in the cut locus of \( x _ { i } \) for some \( i \);
	we shall show that \( z \) cannot minimize
	\( y \mapsto \sum _ { j = 1 } ^ { n } \lambda_j d ^ { 2 } \left( x _ { j } , y \right) \).
	By \Cref{lem:hessian_bound_distance_squared} (for the sake of later convenience, we state it the next section) and compactness,
	we can find a constant \( K \) such that, for all \( u \in T _ { z } M \),
	and \( j = 1,2 , \ldots n \), we have
	\[
		\frac { d ^ { 2 }_{x _ { j } } \left(  \exp _ { z } u \right) + d ^ { 2 }_{x _ { j } } \left( \exp _ { z } ( - u ) \right) - 2 d ^ { 2 }_{ x_j} \left( z \right) }
		{ | u | ^ { 2 } } \leq K
	\]
	On the other hand, by \Cref{prop:distance_cut_locus},
	we can find some non
	zero \( u \in T _ { z } M \) such that
	\[
		\frac { d ^ { 2 }_{x _ { i } } \left(  \exp _ { z } u \right) + d ^ { 2 }_{x _ { i } } \left( \exp _ { z } ( - u ) \right) - 2 d ^ { 2 }_{ x_i} \left( z \right) }
		{ | u | ^ { 2 } } \leq - \frac{1}{\lambda_i} K
	\]
	Therefore, we have
	\begin{align*}
	\sum _ { j = 1 } ^ { n } \lambda_j d ^ { 2 }_{ x _ { j } }( z ) & =
\sum _ { j \neq i } ^ { n } \lambda_j d ^ { 2 }_{ x _ { j } }(z ) + \lambda_i d ^ { 2 } _ { x _ { i } } \left( z \right) \\
											  & \geq \frac { - ( 1 - \lambda_i ) K | u | ^ { 2 } } { 2 } +
											  \frac { 1 } { 2 } \sum _ { j \neq i } ^ { n } \lambda_j \left( d ^ { 2 } _{ x _ { j } }( \exp _ { z } u ) +
									  d ^ { 2 } _{ x _ { j }} \left( \exp _ { z } ( - u ) \right) \right) \\
											  & + \frac { K | u | ^ { 2 } } { 2 } + \frac { \lambda_i } { 2 }
											  \left( d ^ { 2 } _{ x _ { i }} \left( \exp _ { z } u \right) +
											  d ^ { 2 } _{ x _ { i }} \left( \exp _ { z } ( - u ) \right)  \right)                                                       \\
											  & > \frac { 1 } { 2 } \sum _ { j = 1 } ^ { n } \lambda_j d ^ { 2 } _{ x _ { j } } \left( \exp _ { z } u \right)
											  + \frac { 1 } { 2 } \sum _ { j = 1 } ^ { n } \lambda_j d ^ { 2 }_{ x _ { j }} \left( \exp _ { z } ( - u ) \right)
	\end{align*}
	Therefore, either
	\[ \sum _ { j = 1 } ^ { n } \lambda_j d ^ { 2 } _{ x _ { j } }\left( \exp _ { z } (u) \right)
	< \sum _ { j = 1 } ^ { n } d ^ { 2 } _{ x _ { j }} \left( z \right) \]
	or
	\[ \sum _ { j = 1 } ^ { n } \lambda_j d ^ { 2 } _{ x _ { j } }\left( \exp _ { z } (-u) \right)
	< \sum _ { j = 1 } ^ { n } d ^ { 2 } _{ x _ { j }} \left( z \right) \]
	in either case, \( z \) cannot minimize \( \sum _ { j = 1 } ^ { n } \lambda_j d ^ { 2 } \left( x _ { j } , z \right) \)
\end{proof}

\begin{rmk}[Idea behind the proof of \Cref{lem:barycenter_out_of_cut_locus}]
The barycenter $z$ achieves the minimum of function $y \mapsto \sum_{i=1}^n \lambda_i d^2_{x_i}(y)$.
And assume we already know that, just for the purpose of illustration in this remark,
the Hessian of this function exists at point $z$,
then this Hessian must be positive as $z$ achieves minimum.
That is to say, $\sum_{i=1}^n \lambda_i \operatorname{Hess}_z d^2_{x_i} \geq 0$.
\Cref{lem:hessian_bound_distance_squared} tells that each $\operatorname{Hess}_z d^2_{x_i}$
has a upper bound constant $K$, so it must have a lower bound constant $(1-\lambda_i)/\lambda_i K$.
While \Cref{prop:distance_cut_locus} says that at cut locus
the Hessian diverges to $-\infty$,
hence the barycenter $z$ could not locate at the cut locus of any $x_i$.
\end{rmk}

\Cref{lem:barycenter_out_of_cut_locus} implies that $g_i$ is differentiable at
any barycenter $z$ of $\sum_{j=1}^n \lambda_j \delta_{x_j}$.
% The \Cref{lem:barycenter_out_of_cut_locus} above tells that $g_i$ is differentiable at $z$.
% Since $z$ is not in the cut locus of $x_i$,
% $x_i$ is then not in the cut locus of $z$ and
% thus $c(z, x_i)$ is differentiable.
% If in addition, $f_i / \lambda_i$ is differentiable at $x_i$,
% then both hand side of \cref{equa:g_i_conjugate} are differentiable at $x_i$ and $z$.
Differentiating \cref{equa:g_i_conjugate} with respect to % $x_i$ and
$z$ and using the minimality property,
applying \cref{equa:exponential_map_and_squared_distance_function} we obtain,
\begin{align*}
	% z   & = \exp_{x_i}( - \frac{1}{\lambda_i} \nabla f_i), \\
	x_i & = \exp_{z}( - \nabla g_i).
\end{align*}
Therefore, $ \exp(- \nabla g_i) $ is a left inverse to $\exp( - \nabla f_i / \lambda_i)$,
\begin{equation}
	\label{equa:left_inverse_barycenter_map}
	\exp( - \nabla g_i) \circ \exp( - \frac{1}{\lambda_i} \nabla f_i) = \operatorname{Id},
	\quad \text{if } \nabla f_i \text{ exists}.
\end{equation}

\subsection{Uniqueness and curvature conditions}

\label{section:uniqueness_and_curvature}

Whenever the barycenter formula \cref{equa:formula_barycenter} holds,
the barycenter of $\sum_{j=1}^n \lambda_j \delta_{x_j}$ is unique.
It means that non-uniqueness reveals the fact that the Lipschitz function
$f_i$ on $\widebar{ \mathcal{X} }$ is not \emph{a priori} differentiable.
In addition, on the set where $\nabla f_i$ exists, the map
$\exp( -\nabla f_i / \lambda_i )$ is injective by \cref{equa:left_inverse_barycenter_map}.

In general, sufficient condition for uniqueness of barycenters of a probability measure $\mu$ on $M$
involves convex properties of distance function.
For example, in \Cref{prop:uniquness_barycenter_Wasserstein} we prove uniqueness of barycenter in Wasserstein space
from this perspective.
In \cite[Proposition IX.7.1]{chavel2006riemannian},
Ricci curvature bound from above plus bounded support for measure $\mu$ are used
to impose uniqueness of barycenter of $\mu$.
We could also use curvature condition for general length space as done by Ohta in \cite{ohta2012barycenters},
which also summarizes related works.
